\id{IRSTI 50.01.85}{}

\begin{articleheader}
\sectionwithauthors{U. Kayumova, A. Mussabekov, A. Brener, Razali Bin Yaakob, A. Yegenova}{CONTROL EQUATIONS FOR THE PROPAGATION OF NONLINEAR WAVES IN
VISCOUS FILM FLOWS WITH A MASS SOURCE}

{\bfseries
\textsuperscript{1}U. Kayumova\textsuperscript{\envelope },
\textsuperscript{1}A. Mussabekov,
\textsuperscript{1}A. Brener,
\textsuperscript{2}Razali Bin Yaakob,
\textsuperscript{1}A. Yegenova
}
\end{articleheader}

\begin{affiliation}
\textsuperscript{1}M.Auezov South Kazakhstan University, Shymkent, Kazakhstan,

\textsuperscript{2}University Putra Malaysia, Kuala-Lumpur, Malaysia

\raggedright \textsuperscript{\envelope }Correspondent-author: dreams\_dream@mail.ru
\end{affiliation}

The paper devotes to theoretical description and mathematical modelling
the nonlinear modes of liquid film flows in the presence of possible
mass sources, in the case of film condensation as a sample. The main new
scientific result of the study is a theoretical description of the
influence of the mass source on the characteristics of nonlinear waves
arising during the flow of a viscous condensate film over a
non-isothermal surface. The study was carried out within the framework
of the long-wave approximation. The conditions for the existence of
solutions describing the propagation of single nonlinear waves in
condensate films are determined, and equations are derived for
calculating the evolution of their wave characteristics. Relationships
are obtained for estimating the scale of the propagation length of
nonlinear waves in liquid films with variable flow rate. The conclusions
of the work can be useful in design work and in developing process and
apparatus control systems in the chemical and pharmaceutical industries.

{\bfseries Keywords:} liquid films flows, mass sources, non-linear wave
propagation, control parameters, govern evolution equation.

\begin{articleheader}
{\bfseries МАССА ДЕРЕККӨЗІ БАР ТҰТҚЫР ПЛЕНКАЛЫ АҒЫНДАРДА СЫЗЫҚТЫ ЕМЕС
ТОЛҚЫНДАРДЫҢ ТАРАЛУЫН БАҚЫЛАУ ТЕҢДЕУЛЕРІ}

{\bfseries
\textsuperscript{1}У. Каюмова\textsuperscript{\envelope },
\textsuperscript{1}А. Мұсабеков,
\textsuperscript{1}А. Бренер,
\textsuperscript{2}Разали Бин Якуб,
\textsuperscript{1}А.Егенова
}
\end{articleheader}

\begin{affiliation}
\textsuperscript{1} М.Әуезов атындағы Оңтүстік Қазақстан университеті, Шымкент, Қазақстан,

\textsuperscript{2} Путра Малайзия университеті, Куала-Лумпур, Малайзия,

e-mail: dreams\_dream@mail.ru
\end{affiliation}

Бұл зерттеу жұмысы сұйық пленка ағындарының сызықтық емес режимдерін
теориялық сипаттау және математикалық модельдеуге арналған, үлгі ретінде
пленка конденсациясы жағдайында ықтимал масса көздері. Зерттеудің
негізгі жаңа ғылыми нәтижесі-тұтқыр конденсат қабықшасының изотермиялық
емес бетке ағуы кезінде пайда болатын сызықты емес толқындардың
сипаттамаларына масса көзінің әсерінің теориялық сипаттамасы. Зерттеу
ұзын толқынды жуықтау шеңберінде жүргізілді. Конденсат қабықшаларында
бір сызықты емес толқындардың таралуын сипаттайтын ерітінділердің болу
шарттары анықталып, олардың толқындық сипаттамаларының эволюциясын
есептеу үшін теңдеулер алынады. Қатынастар ағынның өзгермелі жылдамдығы
бар сұйық қабықшалардағы сызықты емес толқындардың таралу ұзындығының
шкаласын бағалау үшін алынады. Жұмыстың нәтижелері химия және
фармацевтика өнеркәсібіндегі аппаратураларды есептеуде және
технологиялық процестерді басқаруда пайдаланылуы мүмкін.

{\bfseries Түйін сөздер}: сұйық пленка ағындары, массалық көздер, сызықты
емес толқындардың таралуы, бақылау параметрлері, эволюция теңдеуін
басқару.

\begin{articleheader}
{\bfseries УПРАВЛЯЮЩИЕ УРАВНЕНИЯ ДЛЯ РАСПРОСТРАНЕНИЯ НЕЛИНЕЙНЫХ ВОЛН В
ТЕЧЕНИЯХ ВЯЗКОЙ ПЛЕНКИ С ИСТОЧНИКОМ МАССЫ}

{\bfseries \textsuperscript{1}У. Каюмова\textsuperscript{\envelope },
\textsuperscript{1}А. Мұсабеков, \textsuperscript{1}А. Бренер,
\textsuperscript{2}Разали Бин Якуб, \textsuperscript{1}А.Егенова}
\end{articleheader}

\begin{affiliation}
\textsuperscript{1} Южно-Казахстанский университет им. М. Ауэзова, Шымкент, Казахстан,

\textsuperscript{2}Университет Путра Малайзия, Куала-Лумпур, Малайзия,

e-mail: dreams\_dream@mail.ru
\end{affiliation}

Статья посвящена теоретическому описанию и математическому моделированию
нелинейных режимов пленочных течений жидкости при наличии возможных
источников массы, в случае пленочной конденсации в качестве образца.
Основным новым научным результатом исследования является теоретическое
описание влияния источника массы на характеристики нелинейных волн,
возникающих при обтекании пленки вязкого конденсата неизотермической
поверхностью. Исследование проводилось в рамках длинноволнового
приближения. Определены условия существования решений, описывающих
распространение одиночных нелинейных волн в пленках конденсата, и
выведены уравнения для расчета эволюции их волновых характеристик.
Получены соотношения для оценки масштаба длины распространения
нелинейных волн в пленках жидкости с переменным расходом. Выводы этой
работы могут быть полезны при проектировании и разработке систем
управления процессами и аппаратурой в химической промышленности.

{\bfseries Ключевые слова:} течения пленок жидкости, источники массы,
нелинейное распространение волн, управляющие параметры, уравнение
эволюции управляющих волн.

\begin{multicols}{2}
{\bfseries Introduction.} The problem of theoretical description of wave
processes in condensate films, despite the well-known works {[}1-4{]},
is far from exhausted, which is explained by a wide variety of
manifestations of nonlinearity and dispersion effects during wave
motions of films, especially in processes, the course of which is
complicated by heat and mass transfer, or various phase transitions
(solid phase - liquid or liquid - vapor) {[}2, 5, 6{]}. When modeling
such processes, it is fundamentally important to take into account the
presence of heat and mass sources, changes in the physical
characteristics of the interacting media due to non-isothermality or
variable concentration of individual components.

In a number of works it was established, for example, that the
stationary Nusselt problem may not have a solution for non-isothermal
film flows even for small Reynolds numbers. This circumstance leads to
the emergence and propagation of non-linear waves of complex
configuration in the dissipative system {[}7{]}. Another problem
complicating the process of mathematical modeling is the complexity of
the linkage of the equations that form the system {[}8, 9{]}. The main
difficulty is that since in the presence of mass sources the flow rate
of film flows changes in time and space, it is difficult to isolate a
component constant of the solution that could then be subjected to an
asymptotic analysis taking into account disturbances {[}10, 11{]}.

Known works in this area {[}11{]} do not allow us to answer questions
about how the intensity of the mass source affects the conditions for
the occurrence of wave flows and the characteristics of the resulting
nonlinear waves. At the same time, this question is of fundamental
importance, since the stability of the wave-free flow regime depends
significantly on the changing fluid flow rate, which leads to a changing
film thickness {[}12, 13, 14{]}.

The purpose of this theoretical work and its main scientific
contribution is the derivation of general control equations for the flow
of a viscous liquid film in the presence of a mass source. The
derivation is specified using the example of the process of film
condensation on a flat surface {[}15, 16{]}. The novelty of the work
also lies in the fact that the developed mathematical model is subjected
to asymptotic analysis {[}17, 18, 19{]}, which allows us to identify the
main control parameters and study the features and dependencies of the
wave characteristics of the resulting nonlinear waves on the identified
control parameters. The results obtained as a result of the theoretical
studies carried out can find wide application in the calculation of
technological processes, design of corresponding schemes and
installations, as well as in methods and schemes for controlling
technological film processes operating in modes of variable flow rate
and the presence of mass sources.

{\bfseries Materials and methods.} The main research method in this work
was theoretical research, construction of mathematical models and their
asymptotic analysis.

Previously, the method of integral relations was used to obtain a basic
system of equations for film thickness and flow rate in flows with mass
sources, in particular during film condensation {[}12{]}.

The general form of the evolutionary equation in this case can be
represented in the following form {[}12, 20{]}
\end{multicols}
\begin{equation}
\frac{\partial j}{\partial t}
+ \frac{\partial}{\partial x} \left(
    \frac{f_2}{f_1^2} \cdot \frac{j^2}{h}
\right)
+ \frac{j}{f_1 h^2} \left(
    f_3 \nu_w - \mathit{Ih}
\right)
= g_{\mathit{ef}} h + \frac{\sigma h}{\rho} \cdot \frac{\mathit{dK}_S}{\mathit{dx}}
\end{equation}

\begin{multicols}{2}
It is important to note that in the studied problem there is a parameter $I$,
which is a control parameter characterizing the intensity of the mass
source in the system.

The intensity of the condensation process decreases with increasing
thickness of the condensate film. Therefore, at a sufficiently large
distance from the initial point, the intensity of the mass source can be
considered small {[}21{]}. At the same time, since we are talking about
films of small thickness compared to the characteristic length of
surface waves, we can introduce a small parameter
$\varepsilon=Ij/\hat{j_0}$,
into the model, where is the average velocity of the condensate flow
in the unperturbed film in the area under consideration. With the help
of such a small parameter, it becomes possible to scale the model by
introducing slow variables $t=et$ and $z=ex$, as well as a fast phase
variable $h=q(z,t)/e$.

The following system of evolutionary equations reads
\end{multicols}

\begin{multline*}
\varepsilon \left(
    \frac{\partial j}{\partial \eta}
    + \frac{1}{\varepsilon} \frac{\partial j}{\partial \eta} \cdot \frac{\partial \theta}{\partial \tau}
\right)
+ \varepsilon A \left[
    \frac{\partial}{\partial z} \left( \frac{j^2}{h} \right)
    + \frac{1}{\varepsilon} \frac{\partial}{\partial \eta} \left( \frac{j^2}{h} \right) \cdot \frac{\partial \theta}{\partial z}
\right]
+ \left( B + \varepsilon B_1 \right) \cdot \frac{j}{h^2}
- g_{\mathit{ef}} h =\\
= \varepsilon^3 K_1 h \left[
    \frac{\partial^3 h}{\partial z^3}
    + \frac{3}{\varepsilon} \frac{\partial^3 h}{\partial z^2 \partial \eta} \cdot \frac{\partial \theta}{\partial z}
    + \frac{3}{\varepsilon^2} \frac{\partial^3 h}{\partial z \partial \eta^2} \cdot \left( \frac{\partial \theta}{\partial z} \right)^2
    + \frac{3}{\varepsilon} \frac{\partial^2 h}{\partial z\, \partial \eta} \frac{\partial^2 \theta}{\partial z^2}
+ \frac{3}{\varepsilon^2} \frac{\partial^2 h}{\partial \eta^2} \frac{\partial \theta}{\partial z} \frac{\partial^2 \theta}{\partial z^2}\right.\\\left.
+ \frac{1}{\varepsilon^3} \frac{\partial^3 h}{\partial \eta^3} \left( \frac{\partial \theta}{\partial z} \right)^3
+ \frac{1}{\varepsilon} \frac{\partial h}{\partial \eta} \frac{\partial^3 \theta}{\partial z^3}
\right],
\end{multline*}

\begin{equation}
\varepsilon \left[ \frac{\partial h}{\partial \tau}
+ \frac{1}{\varepsilon} \frac{\partial h}{\partial \eta} \cdot \frac{\partial \theta}{\partial \tau} \right]
+ \varepsilon \left[ \frac{\partial j}{\partial z}
+ \frac{1}{\varepsilon} \frac{\partial j}{\partial \eta} \cdot \frac{\partial \theta}{\partial z} \right]
= \varepsilon \cdot \frac{\hat{j}_0}{h}.
\end{equation}

A solution to the basic system can be presented in the form of
expansions in powers of a small parameter

\begin{equation}
j = \sum_{i=0}^{N} \varepsilon^i J_i(\tau, z, \eta) \bigg|_{\eta = \frac{\theta}{\varepsilon}} + \varepsilon^{N+1} R_{1N}(\tau, z, \eta, \varepsilon),
\end{equation}

\begin{equation}
h = \sum_{i=0}^{N} \varepsilon^i H_i(\tau, z, \eta) \bigg|_{\eta = \frac{\theta}{\varepsilon}} + \varepsilon^{N+1} R_{2N}(\tau, z, \eta, \varepsilon).
\end{equation}

The zeroth order of the system reads


\begin{equation}
\varepsilon^0 \Rightarrow 
\left\{
\begin{aligned}
&\frac{\partial \theta}{\partial \tau} \cdot \frac{\partial J_0}{\partial \eta}
+ A \frac{\partial \theta}{\partial z} \cdot \frac{\partial}{\partial \eta} \left( \frac{J_0^2}{H_0} \right)
+ B \cdot \frac{J_0}{H_0^2}
= g_{\mathit{ef}} H_0 + K_1 \left( \frac{\partial \theta}{\partial z} \right)^3 H_0 \cdot \frac{\partial^3 H_0}{\partial \eta^3},
\\[1ex]
&\frac{\partial \theta}{\partial \tau} \cdot \frac{\partial H_0}{\partial \eta}
+ \frac{\partial \theta}{\partial z} \cdot \frac{\partial J_0}{\partial \eta}
= 0.
\end{aligned}
\right.
\end{equation}

The firth order reads:

\begin{equation}
\varepsilon^1 \Rightarrow
\left\{
\begin{aligned}
&\frac{\partial \theta}{\partial \tau} \cdot \frac{\partial J_1}{\partial \eta}
+ A \frac{\partial \theta}{\partial z} \cdot \frac{\partial}{\partial \eta} \left( \frac{2J_0}{H_0} - J_1 \right)
- A \frac{\partial \theta}{\partial z} \cdot \frac{\partial}{\partial \eta} \left( \frac{J_0^2}{H_0^2} - H_1 \right) \\
&\quad + B \cdot \frac{J_1}{H_0^2}
- 2B \cdot \frac{J_0}{H_0^3} H_1
- g_{\mathit{ef}} H_1
- K_1 H_0 \cdot \frac{\partial^3 H_1}{\partial \eta^3} \left( \frac{\partial \theta}{\partial z} \right)^3
+ K_1 H_1 \left( \frac{\partial \theta}{\partial z} \right)^3 \cdot \frac{\partial^3 H_0}{\partial \eta^3} \\
&\quad = - A \cdot \frac{\partial}{\partial z} \left( \frac{J_0^2}{H_0} \right)
- B_1 \cdot \frac{J_0}{H_0^2}
+ 3K_1 H_0 \cdot \frac{\partial^3 H_0}{\partial z\, \partial \eta^2} \left( \frac{\partial \theta}{\partial z} \right)^2 \\
&\qquad + 3K_1 H_0 \cdot \frac{\partial^2 H_0}{\partial \eta^2} \cdot \frac{\partial \theta}{\partial z} \cdot \frac{\partial^2 \theta}{\partial z^2}, \\[1.5ex]
&\frac{\partial \theta}{\partial \tau} \cdot \frac{\partial H_1}{\partial \eta}
+ \frac{\partial \theta}{\partial z} \cdot \frac{\partial J_1}{\partial \eta}
= \frac{\langle j_0 \rangle}{H_0}
- \frac{\partial H_0}{\partial \tau}
- \frac{\partial J_0}{\partial z}.
\end{aligned}
\right.
\end{equation}

\begin{multicols}{2}
Next, we sequentially obtain recurrence relations for the following
orders of expansion. Note that all systems, except the first, are linear
with respect to the sought functions and are decoupled.

{\bfseries Results and discussion.} Proceeding to a more detailed analysis,
it is important to note that, unlike {[}22, 23{]}, the systems can be
decoupled without the additional assumption of a constant phase
velocity. It follows from relations (5)

\begin{equation}
J_0 = - \frac{\dfrac{\partial \theta}{\partial \tau}}{\dfrac{\partial \theta}{\partial z}} H_0 + \Psi(\tau, z).
\end{equation}

From the physical meaning of the problem under consideration,
$Y(t,z)=0$, which leads to the following equation for $H_0$
\end{multicols}

\begin{equation}
\frac{\left( \dfrac{\partial \theta}{\partial \tau} \right)^2}{\dfrac{\partial \theta}{\partial z}} (A - 1) \cdot \frac{\partial H_0}{\partial \eta}
- K_1 \left( \frac{\partial \theta}{\partial z} \right)^3 H_0 \cdot \frac{\partial^3 H_0}{\partial \eta^3}
= g_{\mathit{ef}} H_0 + \frac{\dfrac{\partial \theta}{\partial \tau}}{\dfrac{\partial \theta}{\partial z}} \cdot \frac{B}{H_0}.
\end{equation}

\begin{multicols}{2}
The systems for subsequent approximations can be similarly decoupled.
Moreover, there is a quasi-linear relationship between the flow rates
and thicknesses as functions of fast and slow variables.

\begin{equation}
J_i = - \frac{\dfrac{\partial \theta}{\partial \tau}}{\dfrac{\partial \theta}{\partial z}} H_i + F\left(H_{i-1}, J_{i-1} \, ; \, \tau, z \right).
\end{equation}

The analysis of the model allows us to conclude that the surface
pressure gradient along the film flow, caused by the variable
curvature of the surface, contributes to the dispersion of waves only
in the zero order, since waves of higher orders evolve under the
influence of the effects of variability of physical
characteristics. Literature analysis shows that the value of the
coefficient $K_1$ is different, namely: $\varepsilon$ of or
$\varepsilon^2$, for different liquids depending on the type of liquid
{[}24{]}.  Therefore, in this paper, it is proposed to introduce a
correction $H_{02}$ into the solution of equation (8) to evaluate the
effect of surface tension, taking into account the effect of surface
tension: $H=H_{01}+H_{02}$.  In this case, the inequality
$H_{02}<<H_{01}$ is valid.

As a result, we obtain the following equation for $H_{01}$
\end{multicols}

\begin{equation}
\frac{\left( \dfrac{\partial \theta}{\partial \tau} \right)^2}{\dfrac{\partial \theta}{\partial z}} (A - 1) \cdot \frac{\partial H_{01}}{\partial \eta}
= \frac{g H_{01}^2 + B \cdot \dfrac{\partial \theta / \partial \tau}{\partial \theta / \partial z}}{H_{01}}.
\end{equation}

The resulting equation has an obvious traveling wave solution

\begin{equation}
H_{01} = \sqrt{
\left( H_{02}^2(\eta_0) + S_1 \right)
\exp\left( 2g S_2 (\eta - \eta_0) \right)
- S_1
}.
\end{equation}

Here

\begin{equation}
S_1 = \frac{B \left( \dfrac{\partial \theta}{\partial \tau} \right)}{g \left( \dfrac{\partial \theta}{\partial z} \right)}, \quad
S_2 = \frac{\left( \dfrac{\partial \theta}{\partial z} \right)}{\left( \dfrac{\partial \theta}{\partial \tau} \right)^2 (A - 1)}.
\end{equation}

Discarding terms of the second and higher orders of smallness with
respect to the correction, $H_{02}$ we obtain the equation

\begin{equation}
\frac{\left( \dfrac{\partial \theta}{\partial \tau} \right)^2}{\dfrac{\partial \theta}{\partial z}} (A - 1) \cdot \frac{\partial H_{01}}{\partial \eta}
= H_{02} \left[ g - \frac{B \left( \dfrac{\partial \theta}{\partial \tau} \right) / \left( \dfrac{\partial \theta}{\partial z} \right)}{H_{01}^2} \right]
+ K \left( \frac{\partial \theta}{\partial z} \right)^3 H_{01} \cdot \frac{\partial^3 H_{01}}{\partial \eta^3}
+ K \left( \frac{\partial \theta}{\partial z} \right)^3 H_{01} \cdot \frac{\partial^3 H_{02}}{\partial \eta^3}
\end{equation}

\begin{multicols}{2}
Let' s supposed the first term is of higher order of smallness in the
right-hand side, that leads to homogeneous equation. As it is shown in
{[}12{]} the Wronskian of a homogeneous equation of such type is zero,
because of the absence of second derivative $\frac{\partial
H_{02}}{\partial \eta^2}$ {[}12{]}. From this it follows that
resulting equation does not contain monotonically increasing
solutions. So, for the possibility of the appearance of oscillating
solutions in the form of a distortion of the wave profile of the
ripple type, the following condition must be satisfied

\begin{equation}
(A - 1) \cdot \frac{\partial \theta}{\partial z} < 0
\end{equation}

The last inequality relates the integral characteristics of the
velocity profile in the film, depending on the varying temperature
across the film thickness, and the wave number of the carrier wave. If
inequality (14) is not satisfied, then we can expect that a small
disturbance of the carrier wave profile will be smoothed out by
capillary forces {[}12{]}.

From inequality (14) it follows the condition $A<1$.

Since the wave number and frequency in a film of variable flow rate
change with time, at least two terms in their expansion in a Taylor
series should be saved.
\end{multicols}

\begin{equation}
\theta(z, \tau, \varepsilon) = \beta(\tau, \varepsilon) \left(z + \varphi(\tau, \varepsilon)\right)
+ \beta_1(\tau, \varepsilon) \left(z + \varphi(\tau, \varepsilon)\right)^2
\end{equation}

Thus, the following evolution equation reads

\begin{equation}
\beta \left( \frac{\partial \varphi}{\partial \tau} \right)^2 (A - 1) \cdot \frac{\partial H_0}{\partial \eta}
- K_1 \beta^3 H_0 \cdot \frac{\partial^3 H_0}{\partial \eta^3}
= g_{\mathit{ef}} H_0 + \frac{\partial \varphi}{\partial \tau} \cdot \frac{B}{H_0}.
\end{equation}

\begin{multicols}{2}
In order for building mathematical models for the evolution of wave
disturbances of the condensate film profile, the methods of the
century-old perturbation theory should be used {[}12{]}.

With the help of supposition of the stationary flow regime nearby the
stability boundary, the process can be considered as quasi-stationary.

Under the condition of the slowness of the functions $j_0$ and $h_0$,
describing stationary solutions, it should be valid a certain relation
to $j_1=L(h_1)$ be valid, where $j_1<<j_0$ and $h_1<<h_0$ are
disturbances of the stationary solution of the film condensation
problem, and is also a slow function. Unlike {[}24{]}, to describe the
evolution of the wave packet in the weakly nonlinear approximation, we
leave the second-order terms.

As a result, the following relation is correct
\end{multicols}

\begin{equation}
\frac{\partial j_1}{\partial t}
+ \alpha_1 \frac{\partial j_1}{\partial x}
+ \alpha_2 \frac{\partial h_1}{\partial x}
+ \alpha_3 \frac{\partial^3 h_1}{\partial x^3}
+ \alpha_4 j_1
+ \alpha_5 h_1
=
\beta_1 j_1 \frac{\partial j_1}{\partial x}
+ \beta_2 h_1 \frac{\partial h_1}{\partial x}
+ \beta_3 j_1^2
+ \beta_4 h_1^2,
\end{equation}

\begin{equation}
\frac{\partial h_1}{\partial t} + \frac{\partial j_1}{\partial x}
= \mathit{zh}_1 + z_2 h_1^2.
\end{equation}

The coefficients of resulting system (17), (18) of equations play the
role of control parameters:

\begin{equation*}
\alpha_1 = \frac{2 f_2}{f_1^2} \cdot \frac{j_0}{h_0};\quad
\alpha_2 = - \frac{f_2}{f_1^2} \cdot \frac{j_0^2}{h_0^2};\quad
\alpha_3 = - \frac{\sigma}{\rho} \, h_0;\quad
\alpha_4 =
\frac{2 f_2}{f_1^2} \cdot \frac{\partial}{\partial x} \left( \frac{j_0}{h_0} \right)
+ \frac{1}{h_0} \left( \frac{\nu_w f_3}{f_1} - \frac{\lambda \Delta T}{r \rho f_1} \right);
\end{equation*}

\begin{equation}
\alpha_5 =
- \frac{f_2}{f_1^2} \cdot \frac{\partial}{\partial x} \left( \frac{j_0^2}{h_0^2} \right)
- \left( \frac{\nu_w f_3}{f_1} - \frac{\lambda \Delta T}{r \rho f_1} \right) \cdot \frac{2 j_0}{h_0^3}
- g
- \frac{\sigma}{\rho} \cdot \frac{\partial^3 h_0}{\partial x^3};
\end{equation}

\begin{equation}
\beta_1 = \frac{2 f_2}{f_1^2} \cdot \frac{1}{h_0};\quad
\beta_2 = \frac{2 f_2}{f_1^2} \cdot \frac{j_0^2}{h_0^3};\quad
\beta_3 = - \frac{f_2}{f_1^2 h_0^2} \cdot \frac{\partial h_0}{\partial x};\quad
\beta_4 = \frac{f_2}{f_1^2} \cdot \frac{\partial}{\partial x} \left( \frac{j_0^2}{h_0^3} \right)
+ \left( \frac{\nu_w f_3}{f_1} - \frac{\lambda \Delta T}{r \rho f_1} \right) \cdot \frac{3 j_0}{h_0^4};\quad
\end{equation}

\begin{equation}
z_1 = - \frac{\lambda \Delta T}{r \rho h_0^2};\quad
z_2 = \frac{\lambda \Delta T}{r \rho h_0^3}.
\end{equation}

Let us introduce new stretched variables $X=\varepsilon x,\quad T=\varepsilon t$
and $\eta=\frac{\theta(X,T)}{\varepsilon}$
a fast variable
.

Then we can look for a solution to the system (17), (18) in the form

\begin{equation}
h_1 = H \exp(\eta), \quad j_1 = J \exp(\eta).
\end{equation}

Then, dividing the terms of the equations by powers of the small
parameter and discarding rapidly oscillating components of small
amplitude of the type $H^2\exp(2\eta)$ and $J^2\exp(2\eta)$, we arrive
at an approximate linear system for the amplitudes

\begin{equation}
J \left[ \frac{\partial \theta}{\partial T} + \alpha_1(X) \frac{\partial \theta}{\partial X} + \alpha_4(X) \right]
+ H \left[ \alpha_2(X) \frac{\partial \theta}{\partial X} + \alpha_3(X) \left( \frac{\partial \theta}{\partial X} \right)^3 + \alpha_5(X) \right] = 0,
\end{equation}

\begin{equation}
J \frac{\partial \theta}{\partial X} + H \left( \frac{\partial \theta}{\partial T} - z_1(X) \right) = 0.
\end{equation}

For the solvability of system (23), (24), the dispersion relation must
be satisfied

\begin{equation}
\left[
\begin{matrix}
\frac{\partial \theta}{\partial T} + \alpha_1 \frac{\partial \theta}{\partial X} + \alpha_4 \\
\alpha_2 \frac{\partial \theta}{\partial X} + \alpha_3 \left( \frac{\partial \theta}{\partial X} \right)^3 + \alpha_5 \\
\frac{\partial \theta}{\partial X} \\
\frac{\partial \theta}{\partial T} - z_1
\end{matrix}
\right]
= 0.
\end{equation}

From this it follows the desired representation for $J=L(X,T)H$, where

\begin{equation}
L(X, T) = \frac{\frac{\partial \theta}{\partial T - z_1}}{\frac{\partial \theta}{\partial X}},
\end{equation}

Considering $L(X,T)$ that is a function of slow variables, and
substituting the last relation into the original system, we arrive at
a zero-order equation for the function that is the film thickness
simulation:

\begin{equation}
\frac{\partial h_1}{\partial t}
+ \left( \alpha_1 + \frac{\alpha_2}{L} \right) \frac{\partial h_1}{\partial X}
+ \alpha_3 \frac{\partial^3 h_1}{\partial X^3}
+ \left( \alpha_4 + \frac{\alpha_5}{L} \right) h_1
=
\left( \beta_1 L + \frac{\beta_2}{L} \right) h_1 \frac{\partial h_1}{\partial X}
+ \left( \beta_3 L + \frac{\beta_4}{L} \right) h_1^2.
\end{equation}

In the resulting equation, it is convenient to switch to a moving
coordinate system:

\begin{equation}
t; \quad \xi = t - \int \frac{dX}{\alpha_1 + \frac{\alpha_2}{L}}.
\end{equation}

As a result, the following govern equation has been derived

\begin{equation}
\frac{\partial h_1}{\partial t}
+ \frac{\beta_1 L + \frac{\beta_2}{L}}{\alpha_1 + \frac{\alpha_2}{L}} h_1 \frac{\partial h_1}{\partial \xi}
- \frac{\alpha_3}{\alpha_1 + \frac{\alpha_2}{L}} \frac{\partial^3 h_1}{\partial \xi^3}
=
- \left( \alpha_4 + \frac{\alpha_5}{L} \right) h_1
+ \left( \beta_3 L + \frac{\beta_4}{L} \right) h_1^2.
\end{equation}

\begin{multicols}{2}
The resulting equation is close in structure to the Korteweg-de-Vries
equation {[}12, 25{]} with a nonlinear perturbation of the right-hand
side and slowly changing coefficients. The presence of such a
perturbation leads to the fact that the dispersion relation of the last
equation (29) contains a nonzero imaginary part, and an undamped wave
solution can exist only on the neutral line and in the region of
increasing amplitudes.

{\bfseries Conclusion.} A mathematical model of propagation of long-wave
nonlinear surface waves with small amplitude in flowing condensate films
has been developed. When deriving the flow equations, the dependence of
the liquid viscosity on the temperature of the supporting surface has
been taken into account. Asymptotic analysis of the model has made it
possible to estimate the degree of influence of variable viscosity and
disturbances of the mass source intensity on the wave characteristics.
The paper presents the general structure of the perturbed wave equation,
which can be used for solving problems of propagation of nonlinear waves
in condensate films under various boundary conditions.
\end{multicols}

\begin{center}
{\bfseries References}
\end{center}

\begin{references}
1. Katsiavria A., Papageorgiou D. T. Nonlinear waves in viscous
multilayer shear flows in the presence of interfacial slip//Wave
Motion.-2022.-Vol.114.103018. DOI 10.1016/j.wavemoti.2022.103018

2. Mukhopadhyay S., Mukhopadhyay A. Waves and instabilities of
viscoelastic fluid film flowing down an inclined wavy bottom//Physical
Review E.- 2020-Vol.102(2): 023117.
DOI \\10.1103/ Physl RevE.102.023117

3. Ogrosky H. R. Linear stability and nonlinear dynamics in a long-wave
model of film flows inside a tube in the presence of surfactant//Journal
of Fluid Mechanics.-2021.- Vol.908.- P.1-16. DOI\\
\href{https://doi.org/10.1017/jfm.2020.878}{10.1017/jfm.2020.878}

4. Zakaria K., Sirwah M. A. Nonlinear dynamics of a liquid film flow
over a solid substrate in the presence of external shear stress and
electric field//The European Physical Journal
Plus-2022.-Vol.137(9):1087. DOI 10.1140/epjp/s13360-022-03249-7

5. Beloglazkin A. N., Shkadov V. Y. Nonlinear Waves in Film Viscous
Liquid Flows at Arbitrary Kapitsa Numbers //Fluid Dynamics.-2021.-Vol
56(4).- P.539-551.
DOI\href{http://dx.doi.org/10.1134/S0015462821040029}{10.1134/S0015462821040029}

6. Mukhopadhyay S., Mukhopadhyay A. Hydrodynamic instability and wave
formation of a viscous film flowing down a slippery inclined substrate:
Effect of odd-viscosity // European Journal of
Mechanics-B / Fluids.-2021.-Vol.89.- P.161-170.
DOI 10.1016/j.euromechflu.2021.05.013

7. Duruk S. Nonlinear dynamics of thin liquid films subjected to
mixed-frequency electrical field//Physics of Fluids.-2020.Vol.32(5). P.
DOI \href{http://dx.doi.org/10.1063/5.0008220}{10.1063/5.0008220}

8. Lerisson G., Ledda P. G., Balestra G., Gallaire F. Instability of a
thin viscous film flowing under an inclined substrate: steady
patterns//Journal of Fluid Mechanics.-2020.Vol.898.
DOI \href{http://dx.doi.org/10.1017/jfm.2020.396}{10.1017/jfm.2020.396}

9. Mukhopadhyay S., Cellier N., Mukhopadhyay A. Long-wave instabilities
of evaporating/condensing viscous film flowing down a wavy inclined
wall: Interfacial phase change effect of uniform layers//Physics of
Fluids.-2022.-Vol.34(4). DOI
\href{http://dx.doi.org/10.1063/5.0089068}{10.1063/5.0089068}

10. Witelski T. P. Nonlinear dynamics of dewetting thin films//AIMS
Mathematics.-2020.-Vol. V.5(5). - P.4229-4259. DOI
\href{http://dx.doi.org/10.3934/math.2020270}{10.3934/math.2020270}

11. Camassa R., Marzuola J. L., Ogrosky H. R., Swygert S. On the
stability of traveling wave solutions to thin-film and long-wave models
for film flows inside a tube//Physica D: Nonlinear
Phenomena.--2021.-Vol.415(3): 132750.
DOI:\href{http://dx.doi.org/10.1016/j.physd.2020.132750}{10.1016/j.physd.2020.132750}

12. Brener A., Yegenova A., Botayeva S. Equations of Nonlinear Waves in
Thin Film Flows with Mass Sources and Surface Activity at the Moving
Boundary//WSEAS Transactions on Fluid
Mechanics.-2020.-Vol.15.-P.149-163. DOI
\href{http://dx.doi.org/10.37394/232013.2020.15.15}{10.37394/232013.2020.15.15}

13. Kim D. J., Kim D. Low-order modelling of three-dimensional surface
waves in liquid film flow on a rotating disk//Journal of Fluid
Mechanics. - 2024.-Vol.985. DOI
\href{http://dx.doi.org/10.1017/jfm.2024.274}{10.1017/jfm.2024.274}

14. Hu T., Fu Q., Yang L. Falling film with insoluble surfactants:
effects of surface elasticity and surface viscosities//Journal of Fluid
Mechanics.-2020.-Vol.889. DOI
\href{http://dx.doi.org/10.1017/jfm.2020.89}{10.1017/jfm.2020.89}

15. Maksymov I. S., Pototsky A. Solitary-like wave dynamics in thin
liquid films over a vibrated inclined plane//Applied
Sciences.-2020.-Vol.13(3):1888.2023.
\href{https://doi.org/10.3390/app13031888}{}

16. Ranganathan U., Chattopadhyay G., Tiwari N. Evolution of a thin film
down an incline: A new \\perspective // Physics of Fluids .-2020.-Vol.32(1).
DOI \href{http://dx.doi.org/10.1063/1.5127815}{10.1063/1.5127815}

17. Zhou G., Prosperetti A. Cаpillary waves on a falling film // Physical
Review Fluids .-2020.-Vol.5(11) : 114005. DOI
\href{http://dx.doi.org/10.1103/PhysRevFluids.5.114005}{10.1103/PhysRevFluids.5.114005}

18. Akylas T. R. David J. Benney: Nonlinear wave and instability
processes in fluid flows//Annual Review of Fluid
Mechanics.-2020.-Vol.52(1).-P.21-36.DOI
\href{http://dx.doi.org/10.1146/annurev-fluid-010518-040240}{10.1146/annurev-fluid-010518-040240}

19. Khusnutdinova K., Gavrilyuk S., Ostrovsky L. Nonlinear dispersive
waves in fluids and solids //
\href{https://www.researchgate.net/journal/Wave-Motion-0165-2125?_tp=eyJjb250ZXh0Ijp7ImZpcnN0UGFnZSI6InB1YmxpY2F0aW9uIiwicGFnZSI6InB1YmxpY2F0aW9uIn19}{Wave
Motion}.- 2023.- Vol.118:103123. DOI
\href{http://dx.doi.org/10.1016/j.wavemoti.2023.103123}{10.1016/j.wavemoti.2023.103123}

20. Singh G., Tiwari N. Stability of traveling waves of a thermoviscous
liquid film down the outer surface of a cylinder//Physical Review
Fluids.-2024.-Vol.9(2): 024002.
DOI 10.1103/PhysRevFluids.9.024002

21. Sojahrood A. J., Haghi H., Karshafian R., Kolios M. C. Classification
of the major nonlinear regimes of oscillations, oscillation properties,
and mechanisms of wave energy dissipation in the nonlinear oscillations
of coated and uncoated bubbles//Physics of Fluids.-2021.-V.33(1):16105
DOI \href{http://dx.doi.org/10.1063/5.0032766}{10.1063/5.0032766}

22. Helal M. A., Badawi S. E., Mahmoud W. Wave propagation over a beach
within a nonlinear theory//Inf. Sci. Lett.-2022.-V.11(05).-P.1741-1755.
DOI 10.18576/isl/110531

23. Dong X., Chen W. Numerical analysis of wave effect on steam--air
condensation on a vertical surface // Annals of Nuclear
Energy.-2022.-Vol.168.108872.
DOI 10.1016/j.anucene.2021.108872

24. Choi Y., Son G., Lee G. Numerical simulation of wavy film
condensation in a vertical channel with non-condensable
gas//International Journal of Heat and Mass
Transfer.-2022.-~Vol.149:119173. DOI\\
10.1016/j.ijheatmasstransfer.2019.119173

25. Iqbal M., Faridi W. A., Algethamie R., Alomari F. A., Murad M. A. S.,
Alsubaie N. E., Seadawy A. R. Extraction of newly soliton wave structure
to the nonlinear damped Korteweg--de Vries dynamical equation through a
computational technique//Optical and Quantum
Electronics.-2024.-Vol.56(7):1189. ~DOI
\href{https://ui.adsabs.harvard.edu/link_gateway/2024OQEle..56.1189I/doi:10.1007/s11082-024-06880-z}{10.1007/s11082-024-06880-z}~
\end{references}

\begin{authorinfo}
\emph{{\bfseries Information about the authors}}

Kayumova U.- doctoral student, South Kazakhstan University. M. Auezova,
Shymkent, Kazakhstan, e-mail: \\dreams\_dream@mail.ru;

Musabekov A.- PhD., Associate Professor, South Kazakhstan University
named after M. Auezov, Shymkent, Kazakhstan,
e-mail: musabekov\_a@rambler.ru;

Brener A.- Doctor of Technical Sciences, Professor, South Kazakhstan
University. M. Auezova, Shymkent, Kazakhstan, e-mail: amb\_52@mail.ru;

Razaliy Bin Yaakob- PhD, Associate Professor, Universiti Putra Malaysia,
Kuala Lumpur, Malaysia, e-mail: \\razaliy@upm.edu.my;

Egenova A.-PhD, South Kazakhstan University. M. Auezova, Shymkent,
Kazakhstan, e-mail: aegenova@mail.ru

\emph{{\bfseries Сведения об авторах}}

Каюмова У.- докторант, Южно-Казахстанский университет им. М. Ауэзова, г.
Шымкент, Казахстан, e-mail: \\dreams\_dream@mail.ru;

Мусабеков А. - к.т.н., ассоциированный профессор, Южно-Казахстанский
университет имени М. Ауэзова, Шымкент, Казахстан, e-mail: musabekov\_a@rambler.ru;

Бренер A.- д.т.н., профессор, Южно-Казахстанский университет им. М.
Ауэзова, г. Шымкент, Казахстан, e-mail: \\amb\_52@mail.ru;

Разали Бин Якуб- PhD, ассоциированный профессор, Университет Путра
Малайзия, Куала-Лумпур, Малайзия, e-mail: razaliy@upm.edu.my;

Егенова A.- доктор PhD Южно-Казахстанский университет им. М. Ауэзова, г.
Шымкент, Казахстан, e-mail: \\aegenova@mail.ru
\end{authorinfo}
