\id{ҒТАМР 20.53.19}{}

{\bfseries ДИАБЕТТІК РЕТИНОПАТИЯНЫ ЕМДЕУДЕ КӨЗ ТҮБІ ТІНДЕРІНЕ}

{\bfseries ЛАЗЕРЛІК ӘСЕРДІ МАТЕМАТИКАЛЫҚ МОДЕЛЬДЕУ}

{\bfseries \textsuperscript{1}С.З. Сапақова}

\begin{figure}[H]
	\centering
	\includegraphics[width=0.8\textwidth]{media/ict/image16}
	\caption*{}
\end{figure}
{\bfseries \textsuperscript{\envelope },
\textsuperscript{2}Д.
Даниярова}
\begin{figure}[H]
	\centering
	\includegraphics[width=0.8\textwidth]{media/ict/image16}
	\caption*{}
\end{figure}
{\bfseries ,
\textsuperscript{1}Н.
Есмұхамедов}
\begin{figure}[H]
	\centering
	\includegraphics[width=0.8\textwidth]{media/ict/image16}
	\caption*{}
\end{figure}
{\bfseries ,
\textsuperscript{1}Р.
Арманқызы}
\begin{figure}[H]
	\centering
	\includegraphics[width=0.8\textwidth]{media/ict/image16}
	\caption*{}
\end{figure}
,

{\bfseries \textsuperscript{1}А.Б.
Ембердиева}
\begin{figure}[H]
	\centering
	\includegraphics[width=0.8\textwidth]{media/ict/image16}
	\caption*{}
\end{figure}
{\bfseries ,\textsuperscript{3}А.
Қалдыбаева}
\begin{figure}[H]
	\centering
	\includegraphics[width=0.8\textwidth]{media/ict/image16}
	\caption*{}
\end{figure}


\textsuperscript{1}Халықаралық Ақпараттық Технологиялар Университеті,
Алматы, Қазақстан,

\textsuperscript{2}Халықаралық білім беру корпорациясы, Алматы,
Қазақстан,

\textsuperscript{3}Қазақ ұлттық қыздар педагогикалық университеті,
Алматы, Қазақстан

\textsuperscript{\envelope }Автор-корреспондент: sapakovasz@gmail.com

Бұл жұмыс диабеттік ретинопатияны емдеуде лазерлік терапияны
оңтайландыру мақсатында көздің түбі тіндеріне лазерлік әсерді
математикалық модельдеуге арналған. Мақалада тіндердегі температураның
таралуын модельдеу әдістері және лазердің параметрлерін пайдалана
отырып, көз торына (сетчаткаға) тиімді әсер ету жолдары қарастырылады.
Зерттеу барысында лазерлік әсер кезінде жылу өткізгіштігін сипаттайтын
дифференциалды теңдеулер және жылу әсерлерін болжауға арналған арнайы
математикалық модельдер қолданылды. Моделдер лазерлік энергияның терең
тіндерге таралуын, сонымен қатар лазердің әсерін бақылауды
оңтайландыруды қамтамасыз етеді. Зерттеуде уақыт пен тереңдік бойынша
температураның таралуын модельдеуге ерекше көңіл бөлінді, бұл лазердің
параметрлерін (қуаты, импульс ұзақтығы, қарқындылығы) дәл баптауға
мүмкіндік береді. Моделдеу нәтижелері тіндердің беткі қабаттарында
температураның айтарлықтай тез көтерілуі және терең қабаттарда баяу
таралуы байқалғанын көрсетті, бұл лазерлік әсерді дәл бақылаудың
маңыздылығын растайды. Мақалада көз ауруларын емдеуде лазердің
параметрлерін дұрыс таңдаудың маңыздылығы талқыланады, әсіресе диабеттік
ретинопатияны емдеуде. Ұсынылған модельдер офтальмологиядағы лазерлік
технологияларды әрі қарай дамытуға және лазерлік терапияның қауіпсіздігі
мен тиімділігін арттыруға мүмкіндік береді. Сонымен қатар, лазерлік
әсерді тиімді қолдану үшін қосымша зерттеулер мен эксперименттер қажет
екені көрсетілген.

{\bfseries Түйін сөздер:} лазерлік терапия, диабеттік ретинопатия,
математикалық модельдеу, температура, лазер параметрлері,
дифференциалдық теңдеулер.

{\bfseries МАТЕМАТИЧЕСКОЕ МОДЕЛИРОВАНИЕ ЛАЗЕРНОГО ВОЗДЕЙСТВИЯ НА ГЛАЗНОЕ
ДНО}

{\bfseries ПРИ ЛЕЧЕНИИ ДИАБЕТИЧЕСКОЙ РЕТИНОПАТИИ}

{\bfseries \textsuperscript{1}С.З. Сапакова\textsuperscript{\envelope },
\textsuperscript{2}Д. Даниярова, \textsuperscript{1}Н. Есмұхамедов,
\textsuperscript{1}Р. Арманқызы,}

{\bfseries \textsuperscript{1}А.Б. Ембердиева, \textsuperscript{3}А.
Қалдыбаева}

\textsuperscript{1} Международный университет информационных технологий,
Алматы, Казахстан\\
\textsuperscript{2} Международная образовательная корпорация, Алматы,
Казахстан,\\
\textsuperscript{3} Казахский национальный женский педагогический
университет, Алматы, Казахстан,

e-mail: sapakovasz@gmail.com

Данная работа посвящена математическому моделированию лазерного
воздействия на ткани глазного дна с целью оптимизации лазерной терапии
при лечении диабетической ретинопатии. В статье рассматриваются методы
моделирования распределения температуры в тканях и способы эффективного
воздействия на сетчатку с использованием лазерных параметров. В процессе
исследования применялись дифференциальные уравнения, описывающие
теплопроводность, а также специальные математические модели для
прогнозирования тепловых эффектов при лазерном воздействии. Модели
обеспечивают оптимизацию распространения лазерной энергии в глубокие
такни и контроль воздействия лазера. Особое внимание уделено
моделированию распределения температуры во времени и по глубине, что
позволяет точно настроить параметры лазеры (мощность, продолжительность
импульса, интенсивность). Результаты моделирования показали, что
температура в поверхностных слоях повышается значительно быстрее, чем в
глубоких, что подтверждает необходимость точного контроля лазерного
воздействия. В статье обсуждается важность правильного выбора лазерных
параметров для лечения глазных заболеваний, особенно диабетической
ретинопатии. Предложенные модели могут способствовать дальнейшему
развитию лазерных технологий в офтальмологии и улучшению безопасности и
эффективности лазерной терапии. Также указано, что для эффективного
применения лазерного воздействия необходимы дополнительные исследования
и эксперименты.

{\bfseries Ключевые слова:} лазерная терапия, диабетическая ретинопатия,
математическое моделирование, температура, лазерные параметры,
дифференциальные уравнения.

{\bfseries MATHEMATICAL MODELING OF LASER EXPOSURE ON THE FUNDUS IN THE
TREATMENT OF DIABETIC RETINOPATHY}

{\bfseries \textsuperscript{1}S.Z. Sapakova\textsuperscript{\envelope },
\textsuperscript{2}D. Daniyarova, \textsuperscript{1}N.Yesmukhamedov,
\textsuperscript{1} R.Armankyzy,}

{\bfseries \textsuperscript{1}A.B.Emberdieva, \textsuperscript{3}A
Kaldybaeva}

\textsuperscript{1} International University of Information
Technologies, Almaty, Kazakhstan,\\
\textsuperscript{2} International Educational Corporation, Almaty,
Kazakhstan,\\
\textsuperscript{3} Kazakh National Women' s Pedagogical
University, Almaty, Kazakhstan,

e-mail: sapakovasz@gmail.com

This paper is dedicated to the mathematical modeling of laser exposure
on the fundus tissues for optimizing laser therapy in the treatment of
diabetic retinopathy. The article discusses methods of modeling the
distribution of temperature in tissues and ways to effectively impact
the retina using laser parameters. Differential equations describing
heat conduction and specialized mathematical models for predicting
thermal effects during laser exposure were used in the study. The models
ensure the optimization of laser energy distribution to deep tissues and
control of laser effects. Special attention was paid to modeling the
temperature distribution over time and depth, which allows for the
precise adjustment of laser parameters (power, pulse duration,
intensity). The modeling results showed that the temperature in surface
layers increases significantly faster than in deeper layers, confirming
the importance of precise laser impact control. The paper discusses the
importance of choosing the correct laser parameters in the treatment of
eye diseases, especially diabetic retinopathy. The proposed models can
contribute to the further development of laser technologies in
ophthalmology and enhance the safety and effectiveness of laser therapy.
Additionally, it is noted that further research and experiments are
required for the effective application of laser exposure.

{\bfseries Keywords:} laser therapy, diabetic retinopathy, mathematical
modeling, temperature, laser parameters, differential equations.

{\bfseries Кіріспе.} Диабеттік ретинопатия -- бұл қант диабетінің ұзаққа
созылған ағымына байланысты дамитын көздің ретиналды ауруы. Глюкоза
алмасуының бұзылуы салдарынан көздің тор қабығын қоректендіретін қан
тамырлары зақымданып, көрудің нашарлауына әкеледі және уақытында
емделмеген жағдайда соқырлыққа себеп болуы мүмкін. Лазерлік терапия
диабеттік ретинопатияны емдеудің тиімді әдістерінің бірі болып табылады,
ол тор қабығының зақымдануын болдырмауға және оның одан әрі дамуын
тоқтатуға бағытталған. Лазерлік сәулелену тор қабығына әсер етіп,
ақуыздардың коагуляциясын және аномальды тамырлардың жойылуын тудырады,
бұл аурудың дамуын тоқтатуға көмектеседі. Дегенмен, лазерлік
технологияларды офтальмологияда тиімді пайдалану үшін лазерлік
сәулеленудің тор қабығына әсерін дәл моделдеу қажет. Лазерлік коагуляция
жергілікті температураның көтерілуіне әкеледі, бұл зақымданған және сау
тіндерге зиян келтіруі мүмкін, егер лазердің параметрлері дұрыс
таңдалмаса. Лазерлік сәуленің тиімділігін модельдеу үшін көз тіндерінің
бірнеше қасиеттерін, оның ішінде олардың біртекті еместігін, лазерлік
сәулеленумен жылу процесін, лазердің параметрлерін (толқын ұзындығы,
қуат, импульс ұзақтығы және жиілік) және тор қабығы мен қоршаған
тіндердің биомеханикалық қасиеттерін ескеру қажет. Лазерлік сәулеленудің
көз түбіне әсерін математикалық моделі көптеген факторлардың әсерін
ескеру қажет болғандықтан күрделі болып табылады. Әрбір жаңа модель
лазерлік сәуленің тор қабығына әсерін дәл болжауға және оны емдеу үшін
ең тиімді параметрлерді таңдауға мүмкіндік береді. Қазіргі уақытта
математикалық модельдер лазерлік сәулеленудің көз тіндеріне әсерін
болжауға және лазердің параметрлерін оңтайландыруға көмектеседі.
Модельдер тіндердің құрылымын, температураның таралуын және лазерлік
энергияның әсерін ескеріп, лазердің әсер ету шекараларын анықтауға
бағытталуы тиіс.

Бұл зерттеудің мақсаты -- лазерлік коагуляцияның тор қабығына әсерін
сипаттайтын математикалық модельді құру, сондай-ақ лазердің әртүрлі
параметрлерінің диабеттік ретинопатияны емдеудегі тиімділігіне әсерін
зерттеу.

Зерттеу міндеттері: лазерлік сәулеленудің тор қабығына әсерін сипаттау,
жылу процестерін ескеру, лазердің параметрлерін талдау және олардың
әсерін тексеру, сондай-ақ зақымданудың азаюы үшін бұл параметрлерді
оңтайландыру. Бұл мақсаттарға жету үшін математикалық модельдеу және
жылу процестерін есептеу үшін сандық әдістер қолданылады.

Зерттеу гипотезасы -- лазерлік сәулеленудің әсерін дәл модельдеу, барлық
параметрлер мен көз тіндерінің ерекшеліктерін ескере отырып, лазерлік
терапияның қауіпсіздігін арттырып, оның тиімділігін жақсартуға мүмкіндік
береді. Зерттеудің теориялық маңызы -- лазерлік терапияны математикалық
модельдеудің жаңа тәсілдерін дамыту, ал практикалық маңызы -- лазердің
параметрлерін оңтайландыру және офтальмологиялық процедуралардың
қауіпсіздігін арттыруға құралдарды дамыту.

Қазіргі уақытта лазерлік терапия офтальмология саласында маңызды әдіс
болып табылады, әсіресе диабеттік ретонопатия мен макулярды ісіну сияқты
көз ауруларын емдеуде. Лазерлік сәулеленудің әсерін дәл модельдеу
лазерлік фотокоагуляция мен басқа да лазерлік емдеу әдістерінің
тиімділігі мен қауіпсіздігін арттыруға көмектеседі. Бұл зерттеулер
лазерлік сәулеленудің тор қабығына термикалық әсерін зерттеуге және оның
нәтижелерін болжауға бағытталған. Мысалы, авторлар оптикалық когерентті
томография арқылы көз түбінің үш өлшемді құрылымын қалпына келтіріп, тор
қабығын сегментациялау мен лазерлік әсерді модельдеудің алгоритмдерін
ұсынған. Бұл модель түрлі лазерлер мен лазерлік сәулеленудің әсері
бойынша коагуляцияның өлшемдерін, сондай-ақ зақымдану шекараларын
болжауға мүмкіндік береді. Модельдің нәтижелері интервалдардың саны екі
еселенгенде орташа квадраттық ауытқу 1,7-5,9 есе төмендегенін көрсетті,
бұл лазерлік терапияның тиімділігі мен қауіпсіздігін арттырып, оның
негізгі параметрлерін бағалауға мүмкіндік береді {[}1{]}. Сонымен қатар,
лазерлік сәулеленудің биологиялық тіндерге әсерін модельдеу маңызды
мәселе болып табылады. MILI моделін қолдану арқылы лазер сәулеленуінің
оңтайлы параметрлері, энергияның сіңірілуі, жылудың таралуы және
тіндердің өзгерістері есептеледі. Бұл моделдеу тіндердің лазерлік
сәулеленуге реакциясын болжауға және тіндердің қызып кету қаупін
бағалауға мүмкіндік береді, яғни лазерлік терапияның қауіпсіз әрі тиімді
болуына ықпал етеді {[}2{]}. Лазер мен тіндердің өзара әрекеттесу
механизмдерін сандық талдау лазерлік операциялардың тиімділігін
арттыруда маңызды рөл атқарады. Осы мақсатта, көпқабатты біртекті орта
моделін қолдана отырып, лазерлік сәуленің таралуы мен энергияны сіңіруді
модельдеу үшін көпқабатты Монте-Карло әдісі қолданылған. Бұл модельдің
нәтижелері фототермальды әсерлер мен жылулық зақымдануды лазер
сәулесінің белгілі параметрлеріне байланысты анықтауға мүмкіндік береді.
Бұл зерттеу лазерлік терапияның тиімділігін арттыру үшін маңызды болып
табылады, әсіресе клиникалық нәтижелерді жақсартуға бағытталған {[}3{]}.
Жасушалық деңгейде лазерлік сәулеленудің тор қабығына термикалық зақым
келтіретін энергияны есептеу үшін арнайы компьютерлік модельдер жасалды.
Модельдің нәтижелері эксперименттік деректермен салыстырылғанда 31\%
ауытқумен жақсы сәйкес келді, бұл лазерлік өнімдердің қауіпсіздігін
қамтамасыз ету үшін қажетті ғылыми негіздер болып табылады. Осы
модельдер лазерлік өнімдердің тәуекелдерін бағалауға және тор қабығының
зақымдану шегін болжауға мүмкіндік береді {[}4{]}. Көз түбінің
температуралық өзгерістерін зерттеу барысында түрлі лазерлер мен олардың
әсер ету режимдерін қолдана отырып, жаңа модельдер ұсынылды. Бұл
модельдер лазерлік сәулеленудің биологиялық тіндерге әсерін алдын-ала
болжауға мүмкіндік береді және лазерлік абляция мен фотодисрупцияның
нәтижелерін бағалауға көмектеседі. Лазерлік сәулеленудің әсерінен
тіндердегі жылулық зақымдану шегін болжау үшін фототермальды және
фотохимиялық процестерді in vitro моделінде ажырату қажеттілігі
айтылады. Бұл үшін Аррениустың бірінші ретті жылдамдық тұрақтысы мен
интегралы қолданылады {[}5{]}. Лазерлік фотокоагуляция әдісінің
тиімділігін арттыру үшін лазер сәулесінің параметрлерін дәл есептеу өте
маңызды. Model Predictive Control (MPC) әдісі лазерлік сәулеленудің жылу
теңдеуін шешуді қажет етеді, бұл лазердің сәулелену ұзындығы, уақыты мен
энергияның таралу профилі сияқты параметрлердің температураға әсерін
бағалауға мүмкіндік береді. Бұл әдіс лазердің қауіпсіздігін арттырып,
лазерлік терапияның тиімділігін жоғарылатуға көмектеседі {[}6{]}.
Зерттеулер көрсеткендей, температураның көтерілуіне сезімталдық жоғары,
әсіресе пигментті эпителийдің сіңіру коэффициентіне қатысты {[}7{]}.
Лазерлік сәулеленудің көз түбіне әсерін түсіну, оның зақымдану шектерін
болжауға және лазерлік офтальмохирургиядағы жаңа әдістерді әзірлеуге
ықпал етеді. Пеннеспила био-жылу теңдеуі негізіндегі модельдер лазерлік
сәулелену мен тіндердің өзара әрекеттесуін зерттеуге мүмкіндік береді
және клиникалық нәтижелерді бағалауға ықпал етеді {[}8{]}. Бұл модельдер
лазерлік терапияның қауіпсіздігін арттыруға және жаңа лазерлік
технологияларды клиникалық тәжірибеге енгізуге көмектеседі. Қорытындылай
келе, лазерлік сәулеленудің көз түбіне термикалық әсерін модельдеу өте
маңызды қадам болып табылады, себебі бұл зерттеулер лазерлік терапияны
тиімдірек және қауіпсіз ету үшін қажетті ғылыми негіздерді қамтамасыз
етеді. Моделдеу нәтижелері лазерлік сәулеленудің көз тіндеріне әсерін
дәл болжауға мүмкіндік береді, бұл лазерлік терапия әдістерін
жетілдіруге жол ашады {[}9{]}. Лазерлік сәулеленудің көз түбіне әсерін
зерттеу бойынша соңғы зерттеу 10 нс-тан 1 с-қа дейінгі сәулелену
импульстарының әсерін талдайды. Бұл зерттеу лазерлік офтальмохирургия
үшін негізгі физикалық процестерді математикалық модельдеу мен зерттеу
нәтижелерін ұсынады. Модель түрлі лазерлер мен олардың әсер ету
режимдері үшін коагуляцияның салыстырмалы өлшемдері мен зақымдану
шекараларын генерациялайды. Сондай-ақ, көз ауруларын емдеу мен диагноз
қоюға арналған нақты қолданбаларды ұсынады {[}10{]}. Келесі жұмыста
авторлар лазерлік сәулеленудің адам көзінде жылулық әсерін модельдеуге
арналған бұрын жарияланған жұмысқа кеңейту болып табылады.2D модель
3D-ге кеңейтілді, бұл көздің лазермен сәулеленуін нақтырақ бейнелеуге
мүмкіндік береді. Модель Пеннес био-жылу өткізгіштігі теңдеуімен
жасалып, шектеулі элементтер әдісімен шешіледі, Галеркин-Бубнов
процедурасы қолданылған. Көз ішіндегі температуралық өріс үшін алынған
нәтижелер офтальмологиялық процедуралар кезінде көз ішіндегі тіндерге
болатын зақымдануларды минимизациялауға пайдалы болуы мүмкін, сондай-ақ
лазер мен тіндердің өзара әрекеттесу механизмдерін талдауда да пайдалы
болады {[}11{]}. Авторлар зерттеулерін лазерлік сәулеленудің көзге
әсерінен болатын жылулық зақымдануды болжау моделін тексеруге арнаған.
Модель төрт кезеңнен тұрады: лазерлік бейненің тор қабығында есептелуі,
көз тіндерінде пайда болатын жылудың есептелуі, температураның көтерілуі
және зақымдану шегін анықтау. Техас университеті модельді тексеру үшін
клеткалармен және жануарлармен тәжірибелер жүргізді. Тәжірибелерде түрлі
лазерлік әсерлерге байланысты интенсивтілік, температура, шекті қуат
және зақымдану дәрежесі өлшенді {[}12{]}.

{\bfseries Материалдар мен әдістер.} Көз түбіне лазерлік әсерді модельдеу
үшін қолданылатын негізгі математикалық модельдер, лазерлік сәулеленудің
көз тіндеріне әсерін болжауға және осы әсердің жылу процестерін дәл
сипаттауға мүмкіндік береді. Бұл модельдер әртүрлі факторларды, оның
ішінде лазердің параметрлерін, көз тіндерінің қасиеттерін және
биомеханикалық ерекшеліктерін ескереді. Төменде осы модельдердің негізгі
түрлері мен олардың құрылымы қарастырылған. Лазерлік сәулелену көздің
тор қабығына әсер етіп, оның жылулық өзгерістеріне әкеледі. Бұл
әсерлердің барлық аспектілерін сипаттау үшін әртүрлі физикалық
процестерді біріктіретін математикалық модельдер қажет. Модельдер
негізінен лазерлік сәулеленудің тіндерге өтуін және лазердің әсерінен
туындайтын жылу таралуын есептеу үшін қолданылады.

Математикалық модельдеу жылу өткізгіштік процесін сипаттауға
негізделген, бұл процесс электр энергиясының сипаттамасына негізделеді:

\[\begin{array}{r}
W(r,z) = \Delta S \cdot \left\lbrack I(r,z) - I(r,z + \Delta z) \right\rbrack,\#(1)
\end{array}\]

мұндағы:

\emph{(r,z)} -- цилиндрлік координаттар, мұнда орталық лазер сәулесінің
орталығы мен сәйкес келеді;

\emph{I(r,z)} -- лазер сәулеленуінің интенсивтілігі (r, z) нүктесінде;

\emph{ΔS} -- қарапайым аймақтың ауданы;

\emph{Δz} -- қарапайым аймақтың қалыңдығы.

Лазерлік сәулеленудің интенсивтілігі қашықтықтың артуымен төмендейді.

Лазерлік сәулеленудің тіндерге өтуі барысында оның интенсивтілігі
тереңдікке қарай төмендейді. Бұл процесс Бугер заңымен сипатталады:

\[I(r,z) = I_{0}(r)e^{- \int_{0}^{z}{\beta(r,\xi)d\xi}},\ \ \ \ \ \ \ \ \ \ \ \ \ \ \ \ \ \ \ \ \ \ \ \ \ \ \ \ \ \ \ \ \ \ \ \ \ \ \ \ \ \ \ \ \ \ \ \ \ \ \ \ \ \ \ \ \ \ \ \ \ \ (2)\]

мұндағы:

\emph{I(r,z)} -- \emph{z} тереңдігінде лазерлік сәулеленудің
интенсивтілігі,

\emph{I\textsubscript{0}(r)} -- бастапқы интенсивтілік,

\emph{β(r,z)} -- лазерлік сәулеленудің сіңіру коэффициенті,

\emph{r,z} -- цилиндрлік координаттар, мұндағы r лазердің ортасынан
қашықтық, z тереңдік.

Лазерлік сәулеленудің тіндерде сіңірілген энергиясы жылуға айналады. Бұл
процесті сипаттау үшін энергия балансы теңдеуі пайдаланылады:

\[\Delta V \cdot C_{o}(x,y,z) \cdot \Delta T(x,y,z) = W\left( \sqrt{\left( x - x_{0} \right)^{2} + \left( y - y_{0} \right)^{2}},z \right) \cdot \Delta t,\ \ \ \ \ \ \ \ \ \ \ \ \ \ \ \ \ \ \ (3)\]

мұндағы:

\emph{ΔV} -- тіндердің қарапайым көлемі,

\emph{Co(x,y,z)} -- тіндердің көлемдік жылу сыйымдылығы,

\emph{ΔT(x,y,z)} -- температураның өзгерісі,

\emph{W --} лазердің тіндерге әсер ететін қуаты,

\emph{Δt --} уақыт интервалы.

Тіндердегі температураның өзгерісі лазерлік сәулеленудің әсерінен
туындайтын жылу энергиясының таралуына байланысты:

\[\Delta T(x,y,z) = \frac{I_{0}\left( \sqrt{\left( x\mathbf{-}x_{0} \right)^{2}\mathbf{+}\left( y\mathbf{-}y_{0} \right)^{2}} \right)e^{- \int_{0}^{z}{\beta(x,y,\xi)d\xi}}\mathbf{\cdot}\beta(x,y,z)\Delta t}{C_{o}(x,y,z)\sigma(x,y,z)},\ \ \ \ \ \ \ \ \ \ \ \ \ \ \ \ \ (4)\]

мұндағы:

σ(x,y,z) - тіннің жылу өткізгіштігі.

Лазерлік сәулелену осі бойынша қозғалатын кезде, қабаттардағы сіңіру
коэффициенті баяу өзгеріп, ол жеткілікті түрде кіші мәнге ие, сондықтан
бұл жағдайда оңайлату жасауға болады.

\[\begin{array}{r}
1 - e^{- \int_{z}^{2 + a}\mspace{2mu}\mspace{2mu}\beta(r,\xi)ds} \approx \beta(r,z)\Delta z,\#(5)
\end{array}\]

Мұндай оңайлатудың арқасында формулада (1) қарапайым көлем бірлігі
\emph{ΔV=ΔS⋅Δz} түрінде көрінеді, бұл температураның өзгерісін жазуға
мүмкіндік береді. Температураның өзгерісі лазерлік сәулеленудің ағымдағы
импульсі іске асырылғанға дейінгі және сол сәтте температураның таралуы
арасындағы айырмашылықты білдіреді:

\[\Delta T(x,y,z) = T(x,y,z,0) - T_{c}(x,y,z),\ \ \ \ \ \ \ \ \ \ \ \ \ \ \ \ \ \ \ \ \ \ \ \ \ \ \ \ \ \ \ \ (6)\]

мұндағы:

T(x,y,z,t) -- температураның t≥0 уақытта таралуы,

T\textsubscript{c}(x,y,z) -- лазерлік сәулелену әсерінен бұрынғы
температураның таралуы.

Бастапқы лазерлік сәулелену интенсивтілігі Гаусс әдісімен анықталады,
(7) формулаға сәйкес. Лазерлік коагуляция кезінде дақтың радиусы
өзгермейді. Лазер қуаты офтальмолог дәрігері тарапынан лазерлік импульс
алдында анықталады.

\[I_{0}(r) = \frac{P}{\pi a^{2}}e^{- \left( \frac{r}{a} \right)^{2}},\ \ \ \ \ \ \ \ \ \ \ \ \ \ \ \ \ \ \ \ \ \ \ \ \ \ \ \ \ \ \ \ \ \ \ \ \ \ \ \ \ \ \ \ \ \ \ \ \ \ (7)\]

мұндағы:

\emph{а} -- дақтың радиусы,

\emph{Р} -- лазер қуаты.

Ортаның біртексіздігіне байланысты жылуөткізгіштік теңдеуін жалпы түрде
қарастыру қажет {[}1-5{]}. Лазерлік сәулелену әсерінен кейін сыртқы
ықпал жоқ, сондықтан есеп келесі түрде қойылады (8).

\[\begin{array}{r}
\left\{ \begin{matrix}
C_{o\sigma}(x,y,z,T)\frac{\partial T}{\partial t} = div\left( k(x,y,z,T) \cdot {grad}_{xyz}(T) \right), \\
\left. \ T \right|_{t = 0} = \Delta T(x,y,z) + T_{c}(x,y,z), \\
\left. \ T \right|_{\Gamma} = T_{z},
\end{matrix} \right.\ \#(8)
\end{array}\]

мұндағы:

div -- векторлық өрістің дивергенциясы,

grad\emph{\textsubscript{xyz}} -- координаттар x, y, z бойынша градиент,

Г -- шекара,

Т\textsubscript{2} -- шекаралық жағдайды анықтайтын функция,

\(C_{o\sigma}(x,y,z,T)\) -- температура Т кезінде (x, y, z) нүктесіндегі
көлемдік жылу сыйымдылығы,

\(k(x,y,z,T)\) -- температура Т кезінде (x, y, z) нүктесіндегі жылу
өткізгіштік коэффициенті.

Т функциясын осылайша түрлендіруге болады, бұл шекаралық жағдайларды
нөлге айналдырады. Әдетте мұндай түрлендірулер кезінде біртекті теңдеу
біртекті емес теңдеуге түрленеді. Шекараларды лазерлік сәулеленудің
осінен жеткілікті қашықтықта аламыз, осылайша жылу энергиясы оларға
жетпейді.

{\bfseries Нәтижелер мен талқылау.} Жоғарыда сипатталған математикалық
модельдер мен әдістерді қолдана отырып, көз тіндеріне лазерлік әсерді
модельдеу нәтижелері келтірілген. Бұл бөлімде ұсынылған модельдер мен
әдістердің тиімділігі және олардың диабеттік ретинопатияны емдеудегі
ықтимал қолданылуы туралы негізгі нәтижелер ұсынылады.

Жұмыс барысында жылуөткізгіштік және лазерлік сәулеленудің тіндерде
сіңірілуін ескеретін математикалық моделдер (1)- (8) Python
бағдарламалау ортасында шешілді. Сандық шешім алу үшін~айқын
әдіс~қолданылды, жылуөткізгіштік теңдеуі уақыт және кеңістік бойынша
торға бөлініп, шекті айырымдар әдісімен модельденді.


\begin{figure}[H]
	\centering
	\includegraphics[width=0.8\textwidth]{media/ict/image20}
	\caption*{}
\end{figure}


{\bfseries 1-сурет. Фундус кескініндегі жылу картасының қабаттасуы}

Математикалық формулалар негізінде Питон тілінде жылу картасы жасалды
1-сурет, ол лазердің көз тіндеріне әсерін модельдейді, бұл дәрігерлер
мен зерттеушілерге лазерлік терапияның тор қабығына қалай әсер ететінін
түсінуге көмектеседі, дені сау тіндерді зақымдаудың тәуекелдерін
азайтады. Лазерлік әсерлерді модельдеу лазердің параметрлерін (мықты,
импульс ұзақтығы) дәл баптауға мүмкіндік береді, мысалы, диабеттік
ретинопатияны емдеуде, дені сау тіндерге зақым келтірмеу үшін. Бұл әдіс
медициналық персоналды оқытуда, жаңа емдеу әдістері мен диагностика
әзірлеуде, сондай-ақ лазерлік технологияларды жақсартуда, әсердің
дәлдігі мен қауіпсіздігін арттырады.


\begin{figure}[H]
	\centering
	\includegraphics[width=0.8\textwidth]{media/ict/image21}
	\caption*{}
\end{figure}


{\bfseries 2-сурет. Лазердің әсерінен температураның таралуы}

График 2-сурет лазерлік әсерден кейін температураның уақыт пен тереңдік
бойынша таралуын көрсетеді. Х осі тін тереңдігін, Ү осі уақытты, ал
түстер шкаласы температураны көрсетеді. Бұл график лазерлік әсердің
тіндерде таралуын бейнелейді, бұл емдеудің тиімділігін бағалау және
тіндерді артық қыздыру тәуекелін азайту үшін маңызды.


\begin{figure}[H]
	\centering
	\includegraphics[width=0.8\textwidth]{media/ict/image22}
	\caption*{}
\end{figure}


{\bfseries 3-сурет. Көздің әр қабаты бойынша температураның өзгеруі}

График тіннің әр қабатындағы температураның тереңдік бойынша өзгеруін
көрсетеді.


- {\bfseries Х осі:} Тін тереңдігі (м), температураның әр қабаттағы
өзгерісін көрсетеді.

- {\bfseries Ү осі:} Температура (градус Цельсиямен), тереңдікке байланысты
температураның өзгеруін көрсетеді.

- {\bfseries Қызы сызық:} Қасаң қабық (роговица), бұл жерде температура
лазерлік энергияның көп сіңірілуіне байланысты тез көтеріледі.

- {\bfseries Жасыл сызық:} Хороид, бұл жерде температураның өзгеруі баяу
жүріп, тұрақты болады.

- {\bfseries Көк сызық:} Көз торы (cетчатка), бұл жерде температура
тереңдікке қарай салыстырмалы түрде тұрақты қалады.
Бұл график 3-сурет лазерлік әсердің тіннің әр қабатына әсерін көрсетеді,
бұл лазерлік параметрлерді оңтайландыру және тіндерді зақымдаудан сақтау
үшін маңызды.


\begin{figure}[H]
	\centering
	\includegraphics[width=0.8\textwidth]{media/ict/image23}
	\caption*{}
\end{figure}


{\bfseries 4-сурет. Лазердің әсерінен температураның таралуы}

График лазерлік әсер мен температураның таралуын уақыт бойынша
көрсетеді.


- {\bfseries Х осі:} Тін тереңдігі (м), температураның әр түрлі
тереңдіктерде қалай өзгеретінін көрсетеді.

- {\bfseries Ү осі:} Температура (градус Цельсиямен), температураның
өзгеруін көрсетеді.

- {\bfseries Түсті сызықтар:} әр сызық әр уақыт аралығында температураның
өзгеруін көрсетеді (0-ден 4,5 секундқа дейін). Сызықтар уақыт өткен
сайын температураның тіндерінде қалай таралатынын көрсетеді.
Графикте 4-сурет тіннің бетінде температура лазерлік әсерден кейін
бірден артып, кейінірек тереңдікке қарай таралып, уақыт өткен сайын
тұрақтана бастайды. Бұл график лазерлік әсерден кейін температураның
таралу жылдамдығы мен оның ұзақ сақталуын талдауға пайдалы.


\begin{figure}[H]
	\centering
	\includegraphics[width=0.8\textwidth]{media/ict/image24}
	\caption*{}
\end{figure}


{\bfseries 5-сурет. Лазердің әсерінен температураның таралуы}

Бұл 5-сурет лазердің әр қабаттағы әсерін (роговица, хороида және
сетчатка) тереңдік бойынша уақыт өте келе қалай таралатынын көрсетеді.


- {\bfseries Сол график (Роговица):} Қасаң қабық (роговица) қабатының
тереңдігі бойынша температураның өзгеруін көрсетеді. Уақыт өткен сайын
температура жоғарылайды, кейін тереңдік бойынша біртіндеп таралады.

- {\bfseries Орталық график (Хороид):} Хороид қабатының тереңдігінен
температураның өзгеруі көрсетілген. Бұл қабаттағы температура өзгерісі
баяу болады.

- {\bfseries Оң график (Сетчатка):} Көз торы (cетчатка) қабатындағы
температураның өзгеруі көрсетілген. Бұл қабаттағы температураның
өзгерісі айтарлықтай баяу жүріп, аздап көтеріледі.
Әр графикте уақыт аралығында температураның қалай өзгеретіні көрсетілген
(0 секундтан 4.5 секундқа дейін). Бұл графиктер лазерлік терапияның әр
қабаттағы әсерін визуализациялап, тіндердің әртүрлі қабаттарында
температураның таралуын зерттеуге мүмкіндік береді.

Бұл зерттеуде диабеттік ретинопатияны емдеу мақсатында көздің түбі
тіндеріне лазерлік әсерді модельдеу үшін математикалық модельдер
қолданылды. Нәтижелер көрсеткендей, лазерлік энергия ең көп беткі
қабаттарда, мысалы, қасаң қабықта сіңіріледі, ал терең қабаттарда,
мысалы, көз торында, бұл энергия едәуір төмендейді. Бұл лазердің
параметрлерін (қуаты, импульс ұзақтығы, дақтың диаметрі) дәл баптаудың
дені сау тіндерге зақым келтірмеу үшін қаншалықты маңызды екенің
растайды.

Математикалық модельдер мен жылу карталарын қолдану арқылы көздің түбі
тіндеріндегі температураның уақыт бойынша таралуын талдауға мүмкіндік
береді. Алынған деректер беткі қабаттарда температураның жылдам
көтерілетініне және біршама уақыттан кейін терең қабаттарда
температураның тұрақталатынын көрсетті. Бұл лазерлік әсерді модельдеудің
жоғары деңгейдегі басқаруға қол жеткізуге мүмкіндік беретінін және
тиімді емдеу үшін қажетті қауіпсіздік шараларын қамтамасыз ететінін
дәлелдейді.


\begin{figure}[H]
	\centering
	\includegraphics[width=0.8\textwidth]{media/ict/image25}
	\caption*{}
\end{figure}


{\bfseries 6-сурет. Математикалық модельдердің программалық іске асуының
көрінісі}

Ғылыми жұмыстағы математикалық модельді жүзеге асыру және сандық
эксперименттер жүргізу үдерісінде заманауи ақпараттық-коммуникациялық
технологиялар (АКТ) кеңінен қолданылды (6-сурет). Жоғарыда айтылғандай
модельдеу процесі Python бағдарламалау тілінің негізінде құрылып,
есептеу және сандық талдау мақсатында NumPy мен SciPy кітапханалары
пайдаланылды. Температуралық өрістің және лазерлік параметрлердің
кеңістікте таралуын көрнекі түрде бейнелеу үшін Matplotlib және Seaborn
құралдары қолданылды. Зерттеу Jupyter Notebook интерактивті ортасында
жүргізілді, бұл ғылыми есептеулердің құрылымын модульдеу, нәтижелердің
дәлдігін қамтамасыз ету және зерттеу үдерісін қайталамалы етуге
мүмкіндік берді. Осы АКТ құралдарының үйлесімді қолданылуы модельдеу
сапасын арттырып, есептеу нәтижелерін тиімді талдауға және
визуализациялауға жол ашты.

Алдыңғы зерттеулермен салыстырғанда, бұл зерттеудің нәтижелері лазерлік
терапияны қолданудың тиімділігін растайды, бірақ лазер параметрлердің
дәл бапталуының қажеттілігін де айқындайды. Мысалы, авторлардың {[}5{]}
жұмыстарында энергия дозасын дәл есептеу мәселесі көтерілген болатын,
алайда біздің тәсіл көздің түбі тіндеріндегі жылу алмасуды дәл
модельдеуге мүмкіндік береді, бұл клиникалық нәтижелерді жақсартуға
әкелуі мүмкін.

{\bfseries Қорытынды.} Зерттеу нәтижесінде лазерлік терапия үшін
математикалық модель әзірленіп, оның негізінде лазерлік параметрлердің
көздің түбі тіндеріндегі температураға әсерін дәл болжауға мүмкіндік
беретін жүйе жасалды. Жылу әсерлерін модельдеу, оның ішінде
температураның көтерілуі, тіндер энергиясының сіңірілуі және таралуы,
лазерлік емдеудің қауіпсіз әрі тиімді әдістерін жасауға мүмкіндік
береді.

Алынған нәтижелерге сүйене отырып, лазер параметрлерін оңтайландыру дені
сау тіндерге зақым келтірмей, лазерлік әсердің дәлдігі мен тиімділігін
арттыратынын айтуға болады. Бұл офтальмологиядағы лазерлік
технологияларды қолдану мүмкіндіктерін жаңа деңгейге көтеріп, лазерлік
құрылғыларды жасауға негіз болады. Бұл құрылғылар лазерлік энергияны дәл
бағыттай отырып, тек зақымданған аймақтарды емдейді, сау тіндерді
зақымдамайды.

Келешекте жүргізілетін зерттеулер лазерлік энергияның түрлі тіндерге
әсерін тереңірек зерттеп, лазердің көп мәрте қолданылуының ұзақ мерзімді
әсерлерін бағалауға бағытталады.

{\bfseries References}

1. Shirokanev A., Ilyasova N., Andriyanov N., Zamytskiy E. A., Zolotarev
A., Kirsh D. V. Modeling of Fundus Laser Exposure for Estimating Safe
Laser Coagulation Parameters in the Treatment of Diabetic Retinopathy //
MATH.- 2021. -Vol.9(9).- P.967. DOI 10.3390/MATH9090967.

2. Pashchenko H., Tereshchenko M. Modeling the Impact of Laser
Irradiation on Temperature Changes in Biological Tissues // Vísnik
Kiívskogo Polítéchnícho Instítutu.2024. -№ 67(1).-P.96-102. DOI
10.20535/1970.67(1).2024.306876.

3. Chen B., Zhao Y. B., Li D. Numerical Simulation of Ophthalmic Laser
Surgeries by a Local Thermal Non-Equilibrium Two-Temperature Model //
International Journal of Numerical Methods for Heat \& Fluid Flow.-
2019. - Vol.29(12).- P.4706-4723. DOI 10.1108/HFF-05-2019-0397.

4. Mathieu J., Schulmeister K. Validation of a Computer Model to Predict
Laser Induced Retinal Injury Thresholds //Journal of Laser
Applications.- 2017. Vol.29(3).- P.032004. DOI 10.2351/1.4997831.

5. Cvetković M., Poljak D., Peratta A. Thermal Modelling of the Human Eye
Exposed to Laser Radiation // 2008.
\href{https://ieeexplore.ieee.org/xpl/conhome/4662489/proceeding}{16th
International Conference on Software, Telecommunications and Computer
Networks}, P.16-20. DOI 10.1109/SOFTCOM.2008.4669444.

6. Denton M. H., Clark C., Noojin G. D., West H., Stadick A., Khan T.
Unified Modeling of Photothermal and Photochemical Damage // Frontiers
in Ophthalmology.-2024.-Vol.4.

DOI 10.3389/fopht.2024.1408869.

7. \href{https://www.researchgate.net/profile/Manuel-Schaller-2?_tp=eyJjb250ZXh0Ijp7ImZpcnN0UGFnZSI6InB1YmxpY2F0aW9uIiwicGFnZSI6InB1YmxpY2F0aW9uIn19}{Schaller}
M. et al. Parameter Estimation and Model Reduction for Retinal Laser
Treatment //// arXiv: Electrical Engineering and Systems Science
\textgreater{} Systems and Control.2022.

DOI 10.48550/arxiv.2202.13806.

8. Kotzur S., Wahl S., Frederiksen A. Simulation of Laser Induced Retinal
Thermal Injuries for Non-Uniform Irradiance Profiles and Their
Evaluation According to the Laser Safety Standard // arXiv: Tissues and
Organs.2020. DOI 10.1117/12.2555492.

9. Zheltov G. I., Glazkov V. N., Kirkovsky A. I.,
Podol' tsev A. S. Mathematical Models of Laser/Tissue
Interactions for Treatment and Diagnosis in Ophthalmology // Laser
Applications in Life Sciences.- 1991.-Vol.1403.- P.752-753. DOI
10.1117/12.57371.

10. \href{https://www.researchgate.net/profile/Mario-Cvetkovic?_tp=eyJjb250ZXh0Ijp7ImZpcnN0UGFnZSI6InB1YmxpY2F0aW9uIiwicGFnZSI6InB1YmxpY2F0aW9uIn19}{~Cvetković}
M., Cavka D., Poljak D., Peratta A.3D FEM Temperature Distribution
Analysis of the Human Eye Exposed to Laser Radiation // WIT Transactions
on Engineering Sciences. -2010.- Vol.68.- P.303-312. DOI
10.2495/HT100261.

11. Welch A. J., Priebe L. A., Forster L. D., Gilbert R., Lee C.
Experimental Validation of Thermal Retinal Models of Damage from Laser
Radiation // 1979. DOI 10.21236/ADA074156.

12. Filippov, V. M.~Peripheral retinal laser photocoagulation in the
proactive treatment of proliferative diabetic retinopathy // Russian
Ophthalmology Online.- 2024.-Vol.2(4).-P.221-222.

DOI 10.25276/2312-4911-2024-4-221-222.

\emph{{\bfseries Авторлар туралы мәліметтер}}

Сапакова С.З.- ф.-м.ғ.к., қауымдастырылған профессор, «Компьютерлік
инженерия» кафедрасы, Халықаралық ақпараттық технологиялар
университеті,Алматы, Қазақстан, e-mail:~sapakovasz@gmail.com;
ORCID:~https://orcid.org/0000-0001-6541-6806

Есмұхамедов Н. С.- PhD докторанты, «Компьютерлік инженерия» кафедрасы,
Халықаралық ақпараттық технологиялар университеті, Қазақстан, Алматы,
e-mail:~yesmukhamedov.yeskendyr@gmail.com;\\
ORCID: https://orcid.org/0009-0006-0652-3082

Даниярова Д.Р. - т.ғ.к., қауымдастырылған профессор, Халықаралық білім
беру корпорациясы, Қазақстан, Алматы, e-mail:~dariia.daniyarova@mail.ru;
ORCID: https://orcid.org/0009-0000-5730-7407

Ембердиева А. Б.- ассистент G1, Халықаралық ақпараттық технологиялар
университеті, «Компьютерлік инженерия» кафедрасы, техника ғылымдарының
магистрі, Қазақстан, Алматы,

e-mail:~a.yemberdiyeva@iitu.edu.kz; ORCID:
https://orcid.org/0009-0005-5078-2412

Арманқызы Р.~-техника ғылымдарының магистрі, Халықаралық ақпараттық
технологиялар университеті, «Компьютерлік инженерия» кафедрасы, лектор,
Қазақстан, Алматы, e-mail:~armankyzyrenata@gmail.com;

ORCID: https://orcid.org/0009-0009-2236-559X

Қалдыбаева А. С.- аға оқытушы, «Ақпараттық технологиялар және кітапхана
ісі» кафедрасы, Қазақ ұлттық қыздар педагогикалық университеті,
Қазақста, Алматы, e-mail:~aizhan.seisebek@gmail.com.

ORCID: https://orcid.org/0000-0002-2062-182X

\emph{{\bfseries Information about authors}}

Sapakova S. Z.- PhD in Physics and Mathematics, Associate Professor,
Department of Computer Engineering, International University of
Information Technology, Kazakhstan, Almaty, e-mail:
\href{mailto:sapakovasz@gmail.com}{};

Yesmukhamedov N. S.- PhD student, Department of Computer Engineering,
International University of Information Technology, Kazakhstan, e-mail:
\href{mailto:yesmukhamedov.yeskendyr@gmail.com}{};

Daniyarova D.R.- PhD in Technical Sciences, Associate Professor at
International Educational Corporation, Kazakhstan, Almaty, e-mail:
\href{mailto:dariia.daniyarova@mail.ru}{};

Emberdieva A. B.- Assistant G1, International University of Information
Technology, Department of Computer Engineering, Master of Technical
Sciences, Kazakhstan, Almaty, e-mail:
\href{mailto:a.yemberdiyeva@iitu.edu.kz}{};

Armankyzy R.-- Master of Technical Sciences, International University of
Information Technology, Department of Computer Engineering, Lecturer,
Kazakhstan, Almaty, e-mail:
\href{mailto:armankyzyrenata@gmail.com}{};

Kaldybaeva A. S.- Senior Lecturer, Department of Information Technology
and Library Science, Kazakh National Women' s Pedagogical
University, Kazakhstan, Almaty, e-mail:
\href{mailto:aizhan.seisebek@gmail.com}{}.

ГРНТИ 28.23.15

{\bfseries ПРОГНОЗИРОВАНИЕ СОСТОЯНИЯ ПОЧВЫ В РАЗЛИЧНЫХ КЛИМАТИЧЕСКИХ ЗОНАХ
С ИСПОЛЬЗОВАНИЕМ МЕТОДОВ МАШИННОГО ОБУЧЕНИЯ}

{\bfseries \textsuperscript{1}М.Қ.
Болсынбек}
\begin{figure}[H]
	\centering
	\includegraphics[width=0.8\textwidth]{media/ict/image16}
	\caption*{}
\end{figure}
{\bfseries ,
\textsuperscript{1}Г.Б.
Абдикеримова}
\begin{figure}[H]
	\centering
	\includegraphics[width=0.8\textwidth]{media/ict/image16}
	\caption*{}
\end{figure}
{\bfseries ,
\textsuperscript{2}Ж.К.
Тасжурекова}
\begin{figure}[H]
	\centering
	\includegraphics[width=0.8\textwidth]{media/ict/image16}
	\caption*{}
\end{figure}
{\bfseries ,
\textsuperscript{1}А.А.Адамов}
\begin{figure}[H]
	\centering
	\includegraphics[width=0.8\textwidth]{media/ict/image16}
	\caption*{}
\end{figure}
,

{\bfseries \textsuperscript{1}С.К.
Серикбаева}
\begin{figure}[H]
	\centering
	\includegraphics[width=0.8\textwidth]{media/ict/image16}
	\caption*{}
\end{figure}
{\bfseries \textsuperscript{\envelope },
\textsuperscript{1}А.М.
Ануарбеков}
\begin{figure}[H]
	\centering
	\includegraphics[width=0.8\textwidth]{media/ict/image16}
	\caption*{}
\end{figure}


\textsuperscript{1}Евразийский национальный университет имени
Л.Н.Гумилева, Астана, Казахстан,

\textsuperscript{2}Таразский региональный университет им.М. Х. Дулати,
Тараз, Казахстан

{\bfseries \textsuperscript{\envelope }}Корреспондент-автор:
\href{mailto:inf_8585@mail.ru}{}

В статье рассматривается применение методов машинного обучения для
прогнозирования состояния почвы в различных климатических зонах.
Прогнозирование состояния почвы является ключевым элементом управления
сельскохозяйственными и экологическими системами, поскольку состояние
почвы влияет на урожайность, биоразнообразие и способность поглощать
углерод. Традиционные методы мониторинга почвы требуют значительных
временных и ресурсных затрат, в то время как методы машинного обучения
позволяют обрабатывать большие объемы данных и строить модели,
учитывающие множество факторов, таких как климат, гидрология и
агротехнические практики. В статье представлены современные подходы к
использованию машинного обучения для анализа данных дистанционного
зондирования, таких как вегетационные индексы (NDVI, SAVI), альбедо и
индексы влажности почвы (MSI, NDMI). Эти методы помогают улучшить
точность прогнозов, выявить участки с высоким риском эрозии и предложить
меры для предотвращения деградации земель. Особое внимание уделено
возможности адаптации моделей к различным климатическим условиям, что
способствует устойчивому развитию сельского хозяйства и эффективному
управлению земельными ресурсами.

{\bfseries Ключевые слова:} прогнозирование состояния почвы, машинное
обучение, климатические зоны, дистанционное зондирование, эрозия,
вегетационные индексы, альбедо, влажность почвы.

{\bfseries МАШИНАЛЫҚ ОҚЫТУ ӘДІСТЕРІН ҚОЛДАНЫП ӘРТҮРЛІ КЛИМАТТЫҚ
АЙМАҚТАРДАҒЫ ТОПЫРАҚ ЖАҒДАЙЫН БОЛЖАУ}

{\bfseries \textsuperscript{1}М.Қ. Болсынбек, \textsuperscript{1}Г.Б.
Абдикеримова, \textsuperscript{2}Ж.К. Тасжурекова,
\textsuperscript{1}А.А.Адамов,}

{\bfseries \textsuperscript{1}С.К. Серикбаева\textsuperscript{\envelope },
\textsuperscript{1}А.М. Ануарбеков}

\textsuperscript{1}Л.Н.Гумилев атындағы Еуразия ұлттық университеті,
Астана, Қазақстан,

\textsuperscript{2}М.Х.Дулати атындағы Тараз өңірлік университеті,
Тараз, Қазақстан,

e-mail: \href{mailto:inf_8585@mail.ru}{}

Мақалада әртүрлі климаттық аймақтардағы топырақ жағдайын болжау үшін
машиналық оқыту әдістерін қолдану қарастырылады. Топырақ жағдайын болжау
ауылшаруашылық және экологиялық жүйелерді басқарудың негізгі элементі
болып табылады, өйткені топырақ жағдайы өнімділікке, биоәртүрлілікке
және көміртекті сіңіру қабілетіне әсер етеді. Топырақты бақылаудың
дәстүрлі әдістері айтарлықтай уақыт пен ресурстарды қажет етеді, ал
Машиналық оқыту әдістері үлкен көлемдегі деректерді өңдеуге және климат,
гидрология және агротехникалық тәжірибелер сияқты көптеген факторларды
ескеретін модельдер жасауға мүмкіндік береді. Мақалада вегетациялық
индекстер (NDVI, SAVI), альбедо және топырақ ылғалдылығы индекстері
(MSI, NDMI) сияқты қашықтықтан зондтау деректерін талдау үшін машиналық
оқытуды қолданудың заманауи тәсілдері келтірілген. Бұл әдістер
болжамдардың дәлдігін жақсартуға, эрозия қаупі жоғары жерлерді анықтауға
және жердің деградациясының алдын алу шараларын ұсынуға көмектеседі.
Ауыл шаруашылығының тұрақты дамуына және жер ресурстарын тиімді
басқаруға ықпал ететін модельдерді әртүрлі климаттық жағдайларға
бейімдеу мүмкіндігіне ерекше назар аударылады.

{\bfseries Түйін сөздер:} топырақ жағдайын болжау, машиналық оқыту,
климаттық аймақтар, қашықтықтан зондтау, эрозия, вегетациялық индекстер,
альбедо, топырақ ылғалдылығы.

{\bfseries FORECASTING SOIL CONDITIONS IN DIFFERENT CLIMATIC ZONES USING
MACHINE LEARNING METHODS}

{\bfseries \textsuperscript{1}M.Bolsynbek}, \textsuperscript{1}{\bfseries G.
Abdikerimova, \textsuperscript{2}Zh. Taszhurekova, \textsuperscript{1}A.
Adamov,}

{\bfseries \textsuperscript{1}S.Serikbayeva\textsuperscript{\envelope },
\textsuperscript{1}A. Anuarbekov}

\textsuperscript{1}L.N. Gumilyov Eurasian National University, Astana,
Kazakhstan,

\textsuperscript{2}Taraz Regional University named after M.KH.Dulaty,
Taraz, Kazakhstan,

\begin{quote}
e-mail: \href{mailto:inf_8585@mail.ru}{}
\end{quote}

The article discusses the application of machine learning methods for
predicting soil conditions in various climatic zones. Predicting soil
conditions is a key element of agricultural and ecological systems
management, as soil conditions affect yields, biodiversity and the
ability to absorb carbon. Traditional soil monitoring methods require
significant time and resource costs, while machine learning methods
allow you to process large amounts of data and build models that take
into account many factors such as climate, hydrology and agrotechnical
practices. The article presents modern approaches to using machine
learning to analyze remote sensing data, such as vegetation indices
(NDVI, SAVI), albedo and soil moisture indices (MSI, NDMI). These
methods help to improve the accuracy of forecasts, identify areas with a
high risk of erosion and propose measures to prevent land degradation.
Special attention is paid to the possibility of adapting models to
different climatic conditions, which contributes to the sustainable
development of agriculture and effective land management.

{\bfseries Keywords:} soil condition forecasting, machine learning,
climatic zones, remote sensing, erosion, vegetation indices, albedo,
soil moisture.

{\bfseries Введение.} Прогнозирование состояния почвы - это ключевой
элемент в управлении аграрными системами и экосистемами, поскольку оно
помогает оптимизировать производство сельскохозяйственной продукции,
поддерживать здоровье экосистем и минимизировать воздействие изменения
климата на землю.

Состояние почвы напрямую влияет на урожайность, биоразнообразие и
способность поглощать углерод, что делает её важным объектом
исследований в контексте устойчивого развития. Однако почвенные
характеристики сильно варьируются в зависимости от региона и
климатических условий, что усложняет задачу точного прогнозирования.
Различные климатические зоны представляют собой уникальные комбинации
температурных режимов, уровней осадков и влажности, что влияет на
структуру и состав почвы. Например, в тропических зонах почва часто
подвергается интенсивной эрозии и потерям питательных веществ из-за
обильных осадков, тогда как в аридных регионах возникает проблема
деградации и засоления почв вследствие недостатка влаги {[}1{]}. Эти
факторы делают крайне важным адаптацию методов прогнозирования к
конкретным климатическим условиям. Традиционные методы исследования
состояния почвы, такие как полевые исследования и лабораторный анализ,
занимают много времени и ресурсов, а также требуют значительного
человеческого участия. Эти методы обеспечивают детальные данные, но
имеют ограниченные возможности для масштабирования и прогнозирования в
режиме реального времени. В условиях изменяющегося климата и нарастающих
экологических проблем возникает потребность в новых подходах, которые
позволяют быстро и точно оценивать состояние почвы.

Машинное обучение и современные методы обработки данных предлагают
инновационные решения для анализа почвенных условий. С их помощью можно
обрабатывать большие объемы данных, извлекать полезные закономерности и
строить модели, способные учитывать многовариантные зависимости между
факторами, такими как климат, гидрология и агротехнические практики. Это
позволяет значительно улучшить точность и скорость прогнозирования
состояния почвы в различных условиях.

В последнее время появилось множество исследований, посвященных
применению машинного обучения для задач сельского хозяйства и экологии.
Эти исследования показывают, что алгоритмы машинного обучения могут
эффективно использовать исторические данные о климате, почвенных
характеристиках и урожайности для предсказания будущих изменений
состояния почвы {[}2{]}. Такой подход значительно улучшает управление
земельными ресурсами, минимизирует риск неурожаев и повышает
устойчивость сельского хозяйства к изменениям климата.

Важным преимуществом машинного обучения является возможность
использования разнородных источников данных. В задачи прогнозирования
состояния почвы могут быть включены данные дистанционного зондирования,
метеорологические измерения, результаты лабораторных анализов почвы и
данные полевых наблюдений. Комплексный подход к анализу этих данных
позволяет получить более точные и полные прогнозы. Один из наиболее
перспективных подходов в области машинного обучения для прогнозирования
состояния почвы - это использование глубоких нейронных сетей и моделей,
основанных на временных рядах. Такие модели могут учитывать временную
динамику изменения почвенных характеристик и их зависимости от
климатических факторов. Это особенно важно для регионов с резко
меняющимися условиями, где почва подвержена значительным колебаниям в
короткие промежутки времени {[}3{]}.

Применение методов машинного обучения для прогнозирования состояния
почвы может значительно повысить эффективность сельскохозяйственного
производства. Сельхозпроизводители могут получать более точные
рекомендации по внесению удобрений, орошению и другим агротехническим
мероприятиям. Это, в свою очередь, позволяет уменьшить затраты на
ресурсы и снизить негативное воздействие на окружающую среду, например,
за счет уменьшения использования химических удобрений и пестицидов.
Машинное обучение также играет важную роль в мониторинге деградации
почв. В условиях изменения климата, особенно в засушливых и
полузасушливых регионах, почва подвержена деградации, что приводит к
снижению её плодородности. Прогностические модели на основе машинного
обучения могут помочь выявить ранние признаки деградации и предложить
превентивные меры, направленные на сохранение и восстановление
почвенного покрова.

Оценка влажности почвы - один из ключевых показателей для оценки её
состояния. Методы машинного обучения могут использовать спутниковые
данные для прогнозирования уровня влажности на больших территориях. Это
позволяет обеспечить более точные данные для систем орошения, тем самым
улучшая управление водными ресурсами в сельском хозяйстве.

Кроме того, использование машинного обучения для прогнозирования
состояния почвы помогает бороться с опустыниванием. В регионах с
повышенной подверженностью засухам и эрозии почв машинное обучение может
стать мощным инструментом для предсказания критических моментов и
принятия необходимых мер по восстановлению почвенного покрова. Также
важным аспектом является возможность применения машинного обучения для
прогнозирования химического состава почвы. Наличие или отсутствие
определенных элементов в почве, таких как азот, фосфор и калий, напрямую
влияет на рост растений и урожайность. Прогностические модели могут
помочь определить, в каких регионах требуется дополнительное внесение
удобрений, а где почва достаточно насыщена питательными веществами.

Проблемы опреснения и засоления почвы также могут решаться с помощью
методов машинного обучения. Эти процессы часто наблюдаются в засушливых
регионах и приводят к значительному снижению плодородности почвы.
Прогностические модели могут помочь идентифицировать области с высоким
риском засоления и предложить подходящие агротехнические меры для борьбы
с этой проблемой {[}4{]}. В условиях изменения климата важную роль
играет возможность прогнозирования воздействия экстремальных погодных
условий на почву. С помощью машинного обучения можно предсказать, как
засухи, наводнения и другие аномальные климатические явления повлияют на
состояние почвы в конкретных регионах. Это поможет заблаговременно
разработать стратегии адаптации для сельскохозяйственного сектора.
Особое внимание стоит уделить проблеме эрозии почвы, которая является
одной из главных причин деградации земель. Машинное обучение может
помочь определить, какие регионы наиболее подвержены эрозии, и
предложить эффективные методы её предотвращения. Это особенно важно для
горных и прибрежных регионов, где почва наиболее уязвима к разрушению.

Использование методов машинного обучения для прогнозирования состояния
почвы также способствует более эффективному управлению
сельскохозяйственными ресурсами на уровне страны. Национальные и
региональные правительства могут использовать прогнозные модели для
разработки долгосрочных стратегий по управлению земельными ресурсами,
что будет способствовать устойчивому развитию сельского хозяйства.

Прогнозирование состояния почвы является важным аспектом сельского
хозяйства и экологии, поскольку позволяет оптимизировать использование
земельных ресурсов и обеспечивать устойчивое развитие аграрных регионов
{[}5{]}. В разных климатических зонах почва подвержена различным внешним
воздействиям, таким как колебания температуры, осадков и уровня
влажности, что существенно влияет на её физические и химические
свойства. С развитием технологий машинного обучения появилась
возможность автоматизировать процессы анализа данных о почве и климате,
что значительно ускоряет и улучшает прогнозы её состояния. Методы
машинного обучения позволяют обрабатывать большие объемы информации,
учитывать многочисленные факторы и выявлять скрытые закономерности, что
повышает точность и надежность прогнозов. В данной статье
рассматриваются современные подходы к использованию машинного обучения
для прогнозирования состояния почвы в различных климатических зонах.

{\bfseries Методы и материалы.} Модели машинного обучения, применяемые для
прогнозирования состояния почвы, включают в себя современные алгоритмы
анализа больших данных, такие как XGBoost и глубокие нейронные сети
(DNN). XGBoost (Extreme Gradient Boosting) представляет собой метод
градиентного бустинга, который использует ансамбль решающих деревьев для
повышения точности предсказаний. Основное преимущество XGBoost
заключается в его способности эффективно обрабатывать разнородные данные
и справляться с пропусками в выборках. В процессе обучения алгоритм
итеративно создает деревья решений, каждое из которых исправляет ошибки
предыдущего, минимизируя отклонения в прогнозах. Для обучения модели
используются параметры состояния почвы, полученные из данных
дистанционного зондирования, такие как индексы вегетации (NDVI, SAVI),
влажность (MSI, NDMI) и альбедо. Этот подход позволяет модели точно
определять стадии эрозии и прогнозировать риск деградации земель.

Для успешного обучения моделей машинного обучения, таких как XGBoost и
глубокие нейронные сети, необходима тщательная подготовка данных. Этот
процесс начинается с сбора данных из различных источников: спутниковых
снимков (Sentinel-2), метеорологических станций и полевых наблюдений.
Данные включают спектральные индексы (NDVI, SAVI, MSI, NDMI),
температуру, уровень осадков и характеристики почвы. Полученные данные
часто имеют разную структуру и формат, поэтому первым этапом является их
преобразование в единую форму. Это включает в себя стандартизацию единиц
измерения, синхронизацию временных меток и удаление дубликатов.

После этого выполняется обработка аномалий и пропусков данных. Временные
ряды спутниковых наблюдений иногда содержат пропущенные значения из-за
облачности или технических сбоев в получении снимков. Для заполнения
таких пробелов применяются методы интерполяции, либо используются более
сложные техники, например, временные нейронные сети, которые могут
предсказывать недостающие значения на основе предыдущих данных. Также на
этом этапе выявляются выбросы --- значения, сильно отклоняющиеся от
нормы, например, резкие скачки индекса NDVI в засушливых регионах. Эти
аномалии могут быть вызваны ошибками измерений или временными природными
явлениями. Для их устранения применяется метод IQR (Interquartile Range)
или Z-score, которые помогают отфильтровать атипичные значения.

Завершающим этапом подготовки данных является нормализация и
масштабирование признаков. Поскольку разные параметры (например,
температура, альбедо и индексы вегетации) имеют различные диапазоны
значений, это может влиять на процесс обучения моделей. Для приведения
всех признаков к единому масштабу применяются методы Min-Max Scaling
(масштабирование значений в диапазон {[}0, 1{]}) или Standard Scaling
(преобразование к стандартному нормальному распределению). Это
необходимо, чтобы алгоритмы, такие как XGBoost и нейронные сети, могли
правильно учитывать влияние каждого признака на предсказание, не отдавая
приоритеты параметрам с большими числовыми значениями. После
нормализации данные становятся готовыми для обучения моделей, что
повышает их точность и стабильность при прогнозировании эрозии почвы и
деградации земель.

Помимо XGBoost, для анализа временных изменений почвенных характеристик
применяются глубокие нейронные сети. В частности, архитектуры на основе
LSTM (Long Short-Term Memory) и GRU (Gated Recurrent Unit) позволяют
моделям учитывать временную зависимость данных. Это особенно важно для
мониторинга почвы в условиях меняющегося климата, когда влажность и
состояние растительного покрова могут резко изменяться. Модели LSTM и
GRU способны запоминать долгосрочные зависимости и предсказывать будущие
изменения почвы, основываясь на исторических данных. В качестве входных
данных используются временные ряды спектральных индексов, температуры,
уровня осадков и показателей влажности. Обучение моделей происходит на
больших наборах данных, что позволяет повысить точность прогнозов даже в
условиях изменчивого климата.

Для верификации и оценки точности моделей применяются метрики, такие как
Accuracy, Precision, Recall и F1-Score. Эти показатели помогают
определить, насколько точно модели идентифицируют стадии эрозии и
предсказывают ухудшение состояния почвы. Для повышения надежности
результатов используется метод кросс-валидации, при котором данные
разбиваются на несколько частей, и обучение модели происходит на
различных подвыборках. Такой подход позволяет избежать переобучения и
повысить обобщающую способность моделей на новых данных. Кроме того,
результаты прогнозирования визуализируются в виде картографических
моделей, где состояние почвы отображается по уровням эрозии, что
позволяет легко идентифицировать критические зоны для последующих
восстановительных мероприятий.

Для анализа эрозии почв используются различные методики дистанционного
зондирования, основанные на спектральных характеристиках почвы и
растительности. Один из наиболее эффективных методов - это использование
вегетационных индексов. Индексы, такие как NDVI (Normalized Difference
Vegetation Index) и SAVI (Soil Adjusted Vegetation Index), позволяют
оценить состояние растительности, что является ключевым фактором для
анализа эрозии. NDVI вычисляется на основе разности отражения ближнего
инфракрасного (NIR) и красного (Red) спектров, и его значения
варьируются от -1 до 1. Низкие значения NDVI могут указывать на
отсутствие растительности, что часто связано с эрозией почвы. SAVI - это
модификация NDVI, которая учитывает влияние почвы на отражательную
способность и лучше подходит для регионов с разреженной растительностью,
где почва доминирует в спектре.

Кроме того, важную роль в анализе эрозии играет вычисление альбедо ---
отношения отражённой солнечной радиации к падающей. Альбедо,
представляющее собой долю отраженного солнечного излучения от
поверхности, играет важную роль в анализе эрозии почвы. Этот параметр
влияет на температурные режимы почвы и, соответственно, на процессы
испарения и влагонакопления. В условиях низкого альбедо, поверхность
почвы поглощает больше солнечного тепла, что может способствовать её
пересыханию, уменьшению связности грунта и, как следствие, усилению
эрозионных процессов. Для эрозии почвы, особенно в засушливых или
полузасушливых регионах, где почва более подвержена разрушению, важно
учитывать влияние альбедо на физические процессы, такие как испарение и
высыхание поверхностных слоев. Повышение температуры поверхности почвы
может усилить её подверженность ветровой эрозии, особенно если почвенный
покров нарушен или недостаточно увлажнен {[}6{]}.

С другой стороны, высокое альбедо может снижать испарение и уменьшать
скорость высыхания почвы, что способствует сохранению влаги в верхних
слоях почвы. Однако избыточное отражение солнечной энергии также может
приводить к охлаждению почвы, что иногда отрицательно сказывается на
биологических процессах, таких как рост растительности, которая играет
важную роль в укреплении почвенного покрова и противодействии эрозии.

Вычисление альбедо при прогнозировании эрозии почвы с использованием
методов машинного обучения позволяет учитывать динамику изменения
температуры поверхности и её взаимодействие с другими факторами, такими
как влажность и ветер. Это дает возможность строить более точные модели
предсказания эрозионных процессов, что в свою очередь помогает
заблаговременно разрабатывать меры по защите и сохранению почв в
уязвимых регионах.

Альбедо может быть вычислено на основе мультиспектральных данных с
использованием комбинации различных спектральных каналов, таких как
синий (B2), красный (B4), ближний инфракрасный (B8) и коротковолновой
инфракрасный (SWIR, B11 и B12). Высокие значения альбедо могут указывать
на деградированные или эродированные земли, что особенно актуально для
регионов, подверженных ветровой эрозии.

(1)

Другим важным аспектом анализа эрозии является оценка влажности почвы,
поскольку она играет ключевую роль в стабильности почвенного покрова и
его подверженности эрозионным процессам. Влажность почвы влияет на её
структуру, плотность и способность противостоять воздействию внешних
факторов, таких как ветер и вода. Влажная почва обычно обладает большей
связностью и устойчивостью к эрозии, в то время как сухая почва
становится более рыхлой, что делает её уязвимой для разрушения.

Низкая влажность почвы способствует увеличению риска ветровой эрозии.
Когда почва пересыхает, её частицы становятся менее сцепленными и легче
поднимаются ветром, особенно на открытых или незащищенных участках.
Ветровая эрозия часто наблюдается в засушливых регионах и может привести
к значительным потерям верхнего плодородного слоя почвы, что оказывает
негативное влияние на сельское хозяйство и экосистемы.

С другой стороны, чрезмерное увлажнение почвы, особенно во время
проливных дождей или наводнений, может способствовать водной эрозии.
Вода смывает поверхностные слои почвы, унося с собой питательные
вещества и органические вещества, что приводит к деградации земель. В
условиях склона или неравномерного рельефа этот процесс может
ускоряться, вызывая образование оврагов и оползней {[}7{]}.

В рамках использования методов машинного обучения для анализа эрозии
почвы оценка влажности является важным параметром, который помогает
предсказывать эрозионные риски {[}8{]}. С помощью моделей машинного
обучения можно анализировать данные о влажности почвы, поступающие из
различных источников, таких как датчики влажности, спутниковые снимки и
метеорологические данные {[}9{]}. Это позволяет учитывать, как
пространственную, так и временную динамику изменений влажности почвы,
что помогает прогнозировать периоды повышенной уязвимости почв к
эрозионным процессам.

Другим важным аспектом анализа эрозии является оценка влажности
почвы. Методы, основанные на индексах влажности, таких как MSI
(Moisture Stress Index) и NDMI (Normalized Difference Moisture Index),
позволяют оценить влагозапас почвы и растительности. MSI рассчитывается
как отношение коротковолнового инфракрасного спектра (SWIR) к ближнему
инфракрасному спектру (NIR) и показывает степень стресса от нехватки
влаги. NDMI позволяет оценить содержание влаги в растительности и почве
на основе разности NIR и SWIR спектров. Оба этих индекса играют ключевую
роль в анализе состояния почвы, так как эродированные земли часто теряют
свою способность удерживать влагу, что приводит к их высыханию и
деградации. В совокупности с вычислением альбедо и вегетационными
индексами, анализ влажности почвы предоставляет полное представление о
состоянии почвы и её подверженности эрозионным процессам {[}10{]}.

Важность влажности почвы для выявления эрозии:

Влажность почвы является важным показателем для определения состояния
земли. С помощью дистанционного зондирования можно оценить влажность
почвы и отличить сухие эродированные земли от здоровых участков.

Индексы влажности почвы:

{\bfseries MSI (Moisture Stress Index)}: Этот индекс измеряет степень
влажности почвы. Низкие значения MSI указывают на влажную почву, а
высокие значения --- на сухую, что может указывать на эрозию.

\(MSI = \frac{SWIR}{NIR}\) (2)

Где:

SWIR (B11 в Sentinel-2) --- коротковолновой инфракрасный диапазон;

NIR (B8 в Sentinel-2) --- ближний инфракрасный диапазон.

{\bfseries NDMI (Normalized Difference Moisture Index)}: Этот индекс
оценивает влажность растительности и почвы. Низкие значения NDMI могут
указывать на сухие участки.

\(NDMI = \frac{NIR - SWIR}{NIR + SWIR}\) (3)

Где:

SWIR --- коротковолновой инфракрасный диапазон (B11);

NIR --- ближний инфракрасный диапазон (B8).

Комбинированная математическая формула с учетом альбедо, \emph{MSI}, и
\emph{NDMI}:

Пусть:

\emph{A} --- альбедо (обозначающее отражательную способность
поверхности);

\emph{MSI} --- индекс стресса по влажности (Moisture Stress Index);

\emph{NDMI} --- индекс влажности почвы (Normalized Difference Moisture
Index);

\emph{A\textsubscript{min }}и A\textsubscript{max} --- пороговые
значения альбедо для эрозированных земель;

\emph{MSI\textsubscript{min }}и \emph{MSI\textsubscript{max}} ---
диапазон значений индекса стресса по влажности (чем больше \emph{MSI},
тем суше почва);

\emph{NDMI\textsubscript{min }}и \emph{NDMI\textsubscript{max}} ---
диапазон значений NDMI (чем меньше NDMINDMINDMI, тем суше почва).

(4)

где:

---
нормализованное значение альбедо, которое указывает на степень
оголённости поверхности (чем ближе значение к A\textsubscript{max}, тем
выше вероятность эрозии);

---
нормализованное значение \emph{MSI}, которое указывает на степень
сухости почвы (чем выше MSIMSIMSI, тем суше почва);

---
нормализованное обратное значение NDMI, которое используется для учета
влажности почвы (чем ниже NDMI, тем суше почва).

(4) формаула учитывает {\bfseries альбедо} (высокое альбедо указывает на
оголённые, возможно эрозированные земли), {\bfseries сухость почвы по MSI}
(чем выше значение MSI, тем суше почва), и {\bfseries влажность почвы по
NDMI} (чем меньше NDMI, тем суше почва).

Итоговая комбинация этих факторов позволяет оценить вероятность наличия
эрозии: чем выше результат по формуле, тем больше вероятность того, что
участок земли эродирован.

{\bfseries Результаты и обсуждение.} На основе сегментации, выполненной с
использованием спектральных индексов, альбедо и оценки влажности почвы,
был сформирован набор данных для обучения модели машинного обучения.
Этот набор содержит детализированную информацию о состоянии земельных
участков, разделённых на четыре категории: «Норма», «Первая стадия
эрозии», «Вторая стадия эрозии» и «Третья стадия эрозии». Для
сегментации использовались индексы вегетации, такие как NDVI, альбедо
для оценки отражательной способности поверхности, а также индексы
влажности почвы MSI и NDMI. Эти параметры являются ключевыми для
определения состояния земель и прогнозирования процессов эрозии.

Набор данных был построен на основе временных наблюдений, включающих
значения вегетационных индексов, альбедо и влажности почвы для каждого
участка. В каждом наблюдении фиксировались дата съемки, индекс NDVI,
отражающий уровень вегетации, альбедо - показатель отражения солнечного
излучения, MSI - индикатор стрессового состояния почвы по влажности, и
NDMI, оценивающий содержание влаги. Эти показатели обеспечивают
структуру данных для эффективного обучения моделей машинного обучения.

Каждая строка данных включает информацию о состоянии почвы и её
классификации в одну из четырёх категорий:


1. Норма: Участки с плотным растительным покровом, умеренным
альбедо и стабильной влажностью почвы. NDVI обычно варьируется от 0.3
до 0.6, что указывает на хорошую вегетацию. Альбедо низкое из-за
высокой абсорбции солнечной радиации. MSI и NDMI находятся в пределах
нормы, что говорит о достаточной влажности почвы.

1. Первая стадия эрозии: Земли с начальной деградацией.
Растительный покров истончается, NDVI находится в диапазоне 0.2--0.3.
Альбедо повышено, поскольку оголённая почва начинает отражать больше
солнечной энергии. MSI свидетельствует о дефиците влаги, что
способствует эрозионным процессам.

1. Вторая стадия эрозии: Значительное снижение вегетации. NDVI
снижается до 0.1--0.2, альбедо увеличивается, отражая потерю
растительности. MSI указывает на высокую сухость почвы, что затрудняет
её восстановление.

1. Третья стадия эрозии: Полная деградация земель. NDVI ниже
0.1, что свидетельствует о почти полном отсутствии растительности.
Альбедо превышает 0.3, а MSI и NDMI указывают на сильную сухость и
дефицит влаги. Эти земли требуют немедленного восстановления.
Набор данных был построен на основе спутниковых снимков с использованием
комбинации индексов NDVI, MSI, NDMI и альбедо, что позволило
классифицировать земельные участки по стадиям эрозии. Данные содержат
параметры, такие как дата наблюдения, значения NDVI, альбедо, MSI, NDMI
и класс эрозии. Это помогает прогнозировать дальнейшее развитие
эрозионных процессов и разрабатывать меры по восстановлению
деградированных земель.

Набор данных был создан на основе сегментации изображений, выполненной с
применением комбинации вегетационных индексов, таких как NDVI, SI, NDMI,
MSI, и значений альбедо. Это позволило классифицировать почвы по уровням
эрозии, разделив их на разные категории. Сегментация опиралась на
научные методы анализа растительности и почвы, что дало возможность
точно оценить степень деградации земель. В результате выделены четыре
класса, отражающие различные стадии эрозии.


1. Первый класс - "Норма". Этот класс включает земли, где эрозия
отсутствует или проявляется минимально. Земли находятся в
удовлетворительном состоянии, и на них наблюдаются нормальные процессы
вегетации. В данной категории содержится 1,775,535 экземпляров.

1. Второй класс - "Начальная стадия эрозии". В эту категорию входят
земли, на которых начинают проявляться первые признаки эрозии, такие
как потеря растительности или начальная деградация почвы. В этой
категории насчитывается 4,989 экземпляров.

1. Третий класс - "Средняя стадия эрозии". Данная категория включает
земли, на которых наблюдаются более выраженные признаки эрозии. Почвы
теряют способность удерживать влагу, а значения альбедо и MSI
повышаются. Эти земли требуют серьёзных мер по восстановлению. В этой
группе содержится 56,110 экземпляров.

1. Четвёртый класс - "Критическая стадия эрозии". Почвы в этой категории
почти полностью деградированы, они не способны удерживать влагу и
имеют высокие значения альбедо и MSI, что указывает на критическую
сухость и отсутствие растительности. Земли в этой категории
подвергаются сильным разрушительным процессам, таким как выветривание
и потеря плодородного слоя. В данной категории содержится 7,517
экземпляров.
Структура данных для каждого участка земли включает следующие параметры:

Дата получения спутникового снимка - важнейший параметр для анализа
временных изменений в динамике эрозии.

Индекс NDVI - показывает состояние растительности на участке: высокие
значения свидетельствуют о густой растительности, низкие -об её
отсутствии.

Индекс SI (Soil Index) - характеризует состояние почвы, вычисляется как
отношение красного канала к ближнему инфракрасному диапазону и позволяет
выявлять деградированные участки.

Альбедо - отражает способность поверхности отражать солнечное излучение:
высокие значения характерны для оголённых почв, низкие - для покрытых
растительностью.

Индекс MSI (Moisture Stress Index) - оценивает содержание влаги в почве;
высокие значения указывают на сильный водный стресс.

Индекс NDMI (Normalized Difference Moisture Index) - показывает уровень
влажности растительности и почвы, основанный на ближнем и среднем
инфракрасных диапазонах, и используется для оценки водного стресса.

Каждый участок классифицируется по уровню эрозии, обозначенному как
ErosionClass (0 - норма, 1 - начальная стадия эрозии, 2 - средняя стадия
эрозии, 3 - критическая стадия эрозии). Эта структура данных позволяет
проводить комплексный анализ земель, прогнозировать развитие эрозионных
процессов и разрабатывать меры по восстановлению деградированных
территорий.

После того как были применены различные методы анализа, все данные были
использованы для создания набора данных для машинного обучения.
Спектральные данные, включая вегетационные индексы, альбедо и индексы
влажности, были объединены для обучения моделей машинного обучения. Эти
данные включали множество различных спектральных характеристик, которые
указывали на наличие или отсутствие эрозии на различных участках земли.
Комбинированный анализ позволил выявить ключевые закономерности в
изменении почвы под воздействием эрозии и создать качественный набор
данных для обучения моделей.

На рисунке 1 представлено исходное изображение, полученное с
использованием спутника Sentinel-2. Оно демонстрирует регион, где
различимы участки с разными состояниями земель. На изображении можно
увидеть, как плодородные земли с плотным растительным покровом, так и
пустые участки, которые могут быть либо обработаны, либо находиться под
паром. Эти различия в типах земель позволяют использовать данные для
анализа состояния почвы и растительности, что является важным шагом для
сегментации и последующей классификации уровня эрозии.


\begin{figure}[H]
	\centering
	\includegraphics[width=0.8\textwidth]{media/ict/image31}
	\caption*{}
\end{figure}


\begin{figure}[H]
	\centering
	\includegraphics[width=0.8\textwidth]{media/ict/image32}
	\caption*{}
\end{figure}


{\bfseries Рис.1 - Оригинальное изображение}

Этот снимок является основой для дальнейшего анализа состояния почвы и
оценки эрозионных процессов с применением спектральных индексов. Он
предоставляет исходные данные, которые затем обрабатываются для
выявления признаков эрозии.

На рисунке 2 представлено изображение, которое демонстрирует результат
вычисления альбедо - метода, оценивающего способность поверхности
отражать солнечное излучение. На этом изображении участки земли с
разными уровнями альбедо выделены различными цветами, где более светлые
оттенки указывают на высокие значения альбедо. Такие участки, как
правило, соответствуют оголённым или эродированным землям, которые
теряют способность поглощать солнечную энергию и отражают её в большей
степени. Этот метод позволяет идентифицировать участки, подверженные
эрозионным процессам, и оценивать степень их деградации. На изображении
видно, что большая часть земель окрашена в жёлтый цвет, что указывает на
высокие значения альбедо. Это часто связано с эродированными или
оголёнными участками, где отсутствует растительный покров, что приводит
к повышенной отражательной способности поверхности. Такие участки
требуют особого внимания при прогнозировании эрозионных процессов.


\begin{figure}[H]
	\centering
	\includegraphics[width=0.8\textwidth]{media/ict/image33}
	\caption*{}
\end{figure}


\begin{figure}[H]
	\centering
	\includegraphics[width=0.8\textwidth]{media/ict/image34}
	\caption*{}
\end{figure}


{\bfseries Рис.2. - Использование метода альбедо}

Этот метод показывает, что почти вся поверхность земли обладает высоким
отражением, что может свидетельствовать о возможной эрозии. Однако это
не всегда верный признак. Некоторые участки, кажущиеся эродированными,
на самом деле могут быть паровыми землями, где уже был собран урожай.
Эти земли временно пусты, но могут сохранять достаточный уровень
влажности. Поэтому метод альбедо может приводить к ложноположительным
результатам, неверно интерпретируя плодородные или временно пустующие
земли как эродированные.

На рисунке 3 представлены три категории эрозии земель, отображённые с
использованием различных оттенков. Для анализа был применён
комбинированный метод, включающий несколько показателей: NDVI (индекс
вегетации), альбедо (отражательная способность), MSI (индекс стресса по
влажности) и NDMI (индекс влажности почвы). Эти индексы обеспечивают
комплексную оценку состояния земель и помогают распределить их по
категориям в зависимости от степени деградации.


\begin{figure}[H]
	\centering
	\includegraphics[width=0.8\textwidth]{media/ict/image35}
	\caption*{}
\end{figure}


\begin{figure}[H]
	\centering
	\includegraphics[width=0.8\textwidth]{media/ict/image36}
	\caption*{}
\end{figure}


{\bfseries Рис.3 - Использование комбинированного метода (альбедо + оценка
влажности})

Жёлтая область на рисунке 3 указывает на участки, находящиеся на первой
степени эрозии (начальная стадия). В этих зонах наблюдается снижение
индекса NDVI, что свидетельствует о начале деградации растительного
покрова, и умеренные значения альбедо, указывающие на повышение
отражательной способности поверхности. Это начальные признаки того, что
почва становится более уязвимой к эрозии. Дополнительные параметры,
такие как MSI (индекс сухости) и NDMI (индекс влажности), также
показывают, что почва начинает пересыхать, что может способствовать
дальнейшему ухудшению её состояния. Если не принять меры для
восстановления земель на этой стадии, эрозионные процессы могут
усилиться. Условия для первой степени эрозии включают NDVI в диапазоне
от 0.2 до 0.5, что указывает на среднее состояние растительности,
альбедо от 0.1 до 0.2, характеризующее умеренную отражательную
способность, и MSI от 0.8 до 1.5, что свидетельствует о умеренной
сухости почвы.

Оранжевые участки на изображении указывают на среднюю степень эрозии
почвы, что означает начавшийся процесс её деградации. Почва на этих
участках теряет способность поддерживать здоровый растительный покров, и
NDVI, характеризующий плотность растительности, здесь ниже, чем на
начальной стадии эрозии. Одновременно с этим увеличивается альбедо, что
указывает на оголённые или мало защищённые почвенные поверхности, более
подверженные воздействию солнечного излучения. Высокие значения MSI
(индекс сухости) и низкие NDMI (индекс влажности) свидетельствуют о том,
что почва испытывает повышенный дефицит влаги, что ускоряет эрозионные
процессы. Условия для второй степени эрозии включают NDVI в диапазоне от
0.1 до 0.3, что указывает на низкий уровень растительности, альбедо от
0.2 до 0.25, характеризующее повышенную отражательную способность
поверхности, и MSI выше 1.5, что свидетельствует о высокой сухости
почвы. На этих участках уже можно наблюдать серьёзные признаки
деградации, и, если не принять своевременные меры, ситуация может
ухудшиться, вплоть до перехода к более тяжёлым стадиям эрозии.

Красные участки на изображении указывают на высокую степень эрозии, где
почва сильно деградировала и растительный покров практически
отсутствует. Высокие значения альбедо указывают на оголённую,
незащищённую почву, а высокие MSI и низкие NDMI свидетельствуют о полной
потере влаги. Такие земли считаются непригодными для
сельскохозяйственного использования без серьёзных восстановительных
мероприятий. Условия для третьей степени эрозии включают NDVI менее 0.1,
что означает очень низкий или отсутствующий растительный покров, альбедо
выше 0.25, что характеризует очень высокую отражательную способность, и
MSI выше 1.5, свидетельствующий о крайне высокой сухости почвы.

На следующем, 4 рисунке, представлена полная сегментация земель на
четыре категории в зависимости от их состояния. Зелёные области
указывают на земли в нормальном состоянии, где эрозия отсутствует, а
почва остаётся пригодной для сельскохозяйственного использования. Эти
участки демонстрируют нормальные показатели вегетации (NDVI), умеренное
альбедо и стабильную влажность почвы, что свидетельствует об устойчивом
и здоровом состоянии почвенного покрова.


\begin{figure}[H]
	\centering
	\includegraphics[width=0.8\textwidth]{media/ict/image37}
	\caption*{}
\end{figure}


\begin{figure}[H]
	\centering
	\includegraphics[width=0.8\textwidth]{media/ict/image38}
	\caption*{}
\end{figure}


{\bfseries Рис.4 -Полная сегментация, зелёные участки это норма, жёлтые
это начальная стадия, оранжевая это прогресс эрозии, красная это уже
деградированные земля то есть непригодная земля}

Жёлтые, оранжевые и красные участки на изображении представляют собой
земли, подверженные различным стадиям эрозии: жёлтые области указывают
на начальную стадию, где почва уже демонстрирует первые признаки сухости
и деградации растительного покрова, но всё ещё может быть восстановлена
при правильном управлении; оранжевые области отражают среднюю стадию
эрозии, при которой почва потеряла значительную часть своей
плодородности, что сопровождается увеличением альбедо, снижением
влажности и ослаблением растительности; красные области свидетельствуют
о высокой стадии эрозии, где почва почти полностью утратила свою
пригодность для сельскохозяйственного использования, а её восстановление
требует значительных усилий, о чём говорят высокое альбедо и низкие
показатели индексов влажности.

Важно отметить, что такие элементы, как дороги, искусственные
сооружения, дома и другие постройки на изображении не учитываются в
анализе. Они автоматически классифицируются как "норма" (зелёный цвет) и
исключаются из расчётов при оценке состояния почвы, поскольку не
являются частью сельскохозяйственных или природных земель. Алгоритм
сегментации распознаёт эти объекты как неподверженные эрозии и не
включает их в анализ деградации почв. Благодаря сегментации удалось
получить более чёткое и наглядное представление о состоянии земель,
позволяя увидеть, какие территории требуют внимания и проведения
восстановительных мероприятий для предотвращения дальнейшей эрозии и
деградации. Особенно важно, что нормальные участки почвы (зелёные)
выделяются как потенциальные зоны, которые можно сохранить и защитить от
будущего ухудшения состояния.

Применение технологий машинного обучения оказалось крайне эффективным
для анализа и прогнозирования процессов эрозии почвы. Одним из наиболее
успешных методов, использованных в исследовании, стал алгоритм XGBoost,
основанный на градиентном бустинге. Данный метод позволяет выявлять
сложные взаимосвязи между различными входными параметрами, такими как
спектральные индексы, альбедо и показатели влажности, и целевыми
переменными --- например, наличие эрозии. XGBoost был выбран за его
способность справляться с большими объемами данных и за высокую
точность, что важно при анализе многофакторных природных явлений.
Преимуществом XGBoost является устойчивость к переобучению, что
позволяет создавать модели, которые сохраняют высокую точность на
различных типах данных.

Для обучения модели использовались данные дистанционного зондирования,
включающие показатели вегетации, альбедо и влажности почвы. Обученная
модель продемонстрировала высокую способность к обобщению, так как
данные охватывали различные типы почв и климатические зоны, что
позволило модели точно прогнозировать эродированные участки. Было
установлено, что наибольшую точность прогнозирования эрозии, особенно в
зонах, подверженных ветровой эрозии, давали комбинированные методы
анализа альбедо и влажности. Таким образом, использование методов
машинного обучения в сочетании с данными спутникового мониторинга
открывает новые возможности для точного и оперативного мониторинга
состояния почвы и предотвращения её деградации.


\begin{figure}[H]
	\centering
	\includegraphics[width=0.8\textwidth]{media/ict/image39}
	\caption*{}
\end{figure}

\begin{figure}[H]
	\centering
	\includegraphics[width=0.8\textwidth]{media/ict/image40}
	\caption*{}
\end{figure}


{\bfseries Рис.5. - Динамика изменения потерь и результат точности для
обучающей и валидационной выборки}

На рисунке 5 первый график изображает динамику изменения потерь (Log
Loss) на протяжении 100 итераций обучения модели. Синяя линия
представляет собой потери на обучающей выборке, а оранжевая -- на
валидационной. Можно заметить, что по мере увеличения количества
итераций потери на обеих выборках значительно уменьшаются и выходят на
плато, приближаясь к нулю. Это говорит о том, что модель успешно
обучается, минимизируя ошибки в прогнозировании как на тренировочных,
так и на валидационных данных, что указывает на хорошую способность
модели к обобщению. Стабилизация на низком уровне потерь свидетельствует
о высокой точности модели. Второй график на рисунке 5 демонстрирует
точность (Accuracy) для обучающей и валидационной выборок. Синяя линия
показывает, как растет точность модели на тренировочной выборке с
увеличением числа итераций, достигая практически 100\%. Однако для
валидационной выборки (оранжевая линия) наблюдается небольшое снижение
точности после 50 итераций, что может указывать на незначительное
переобучение модели. Тем не менее, обе линии остаются достаточно
высокими, что подтверждает эффективность модели. Процесс обучения модели
XGBoost проходит с хорошей конвергенцией: потери минимизируются, а
точность модели на тестовых данных остается на высоком уровне. Этот
результат указывает на то, что модель справляется с задачей
классификации эрозионных степеней, не только на тренировочных данных, но
и на новых данных, что делает её применимой для анализа и прогноза
деградации земельных участков в реальных условиях.

{\bfseries Выводы.} Данное исследование предложило эффективный подход к
выявлению и классификации эрозии почв с использованием данных
дистанционного зондирования и методов машинного обучения. Проблема
эрозии почв является одной из основных экологических угроз, снижая
плодородие земель и влияя на сельское хозяйство и устойчивость
экосистем. В ходе работы был разработан алгоритм, основанный на
комбинации спектральных индексов (NDVI, MSI, NDMI) и параметров альбедо,
что позволило разделить почву на четыре класса по степени эрозии:
«Норма», «Первая степень», «Вторая степень» и «Третья степень». Ключевым
аспектом методики является сочетание индексов растительности с оценкой
альбедо и влажности почвы, что позволяет избегать ошибочной
классификации участков, визуально похожих на эрозионные, но фактически
не подверженных эрозии.

Использование модели XGBoost для классификации эрозии почвы
продемонстрировало высокую эффективность, показывая точность как на
тренировочной, так и на тестовой выборках. Модель учитывает нелинейные
зависимости между входными признаками, что важно для анализа сложных
экологических процессов. Разработанная модель может быть полезна для
мониторинга больших территорий, подверженных эрозии, оперативно выявляя
участки, требующие восстановления. Методика имеет потенциал для
масштабных проектов по управлению земельными ресурсами и может быть
адаптирована для работы с другими регионами и типами данных. В будущем
важно рассмотреть возможность интеграции дополнительных спектральных
индексов и использования методов глубинного обучения для повышения
точности классификации и анализа сложных пространственно-временных
зависимостей в данных дистанционного зондирования.

{\bfseries Литература}


1. Жоголев А.В. Актуализация региональных почвенных карт на основе
спутниковых и геоинформационных технологий (на примере Московской
области): Автореф. дис. ... к. с.-х. н. - М., 2016. - 22 c.

1. Векшина В.Н. Построение цифровых моделей почвенного покрова западной
части Большеземельской тундры. Бюллетень Почвенного института имени
В.В. Докучаева.- 2019. Т.99.- С.21-46.
\href{https://doi.org/10.19047/0136-1694-2019-99-21-46}{DOI
10.19047/0136-1694-2019-99-21-46}.

1. Савин И.Ю., Прудникова Е.Ю. Об оптимальном сроке спутниковой съемки
для картографирования пахотных почв // Бюл. Почв. ин-та им. В.В.
Докучаева. -2014. -№ 74. -С.66-77.

1. Рожков В.А. Об информационном подходе в классификации почв//Бюллетень
Почвенного института имени В.В. Докучаева. -2012.-Т.(69).- С.4-24.
\href{https://doi.org/10.19047/0136-1694-2012-69-4-24}{DOI
/10.19047/0136-1694-2012-69-4-24}

1. Гребень А.С., Красовская И.Г. Анализ основных методик прогнозирования
урожайности с помощью данных космического мониторинга, применительно к
зерновым культурам степной зоны Украины // Радіоелектронні і
комп'ютерні системи. -2012. - № 2. - С.170-180.
\href{http://www.irbis-nbuv.gov.ua/cgi-bin/irbis_nbuv/cgiirbis_64.exe?I21DBN=LINK&P21DBN=UJRN&Z21ID=&S21REF=10&S21CNR=20&S21STN=1&S21FMT=ASP_meta&C21COM=S&2_S21P03=FILA=&2_S21STR=recs_2012_2_27}{http://nbuv.gov.ua/UJRN/recs\_2012\_2\_27}

1. Рожков В.А. Информатизация и теория классификации почв // Труды
Института государства и права РАН.- 2012.- № 6. -C.218-227

1. Hengl T., Mendes de Jesus J., Heuvelink G.B.M., Ruiperez Gonzalez M.,
Kilibarda M. SoilGrids250m: Global gridded soil information based on
machine learning // PLOS ONE. -2017. --Vol.12 (2):e0169748 DOI
10.1371/journal.pone.0169748

1. Чинилин А.В., Савин И.Ю. Крупномасштабное цифровое картографирование
содержания органического углерода почв с помощью методов машинного
обучения // Бюллетень Почвенного института им. В.В. Докучаева. -2018.
--Т.91. -- С.46-62. DOI 10.19047/0136-1694-2018-91-46-62

1. Masrur Ahmed, A.A., Deo, R.C., Raj, N., Ghahramani, A., Feng, Q., Yin,
Z., Yang, L. Deep learning forecasts of soil moisture: Convolutional
neural network and gated recurrent unit models coupled with
satellite-derived modis, observations and synoptic-scale climate index
data // Remote Sensing. -2021. --Vol.13 (554). -P.1-30. DOI
10.3390/rs13040554
10. Anton, C.A., Matei, O., Avram, A. Collaborative Data Mining in
Agriculture for Prediction of Soil Moisture and Temperature // Advances
in Intelligent Systems and Computing. -2019. -P.141-151. DOI


{\bfseries References}

1. Zhogolev A.V. Aktualizacija regional' nyh pochvennyh
kart na osnove sputnikovyh i geoinformacionnyh tehnologij (na primere
Moskovskoj oblasti): Avtoref. dis. ... k. s.-h. n. - M., 2016. - 22
s.{[}in Russian{]}

2. Vekshina V.N. Postroenie cifrovyh modelej pochvennogo pokrova zapadnoj
chasti Bol' shezemel' skoj tundry.
Bjulleten'{} Pochvennogo instituta imeni V.V.
Dokuchaeva.- 2019. T.99.- S.21-46. DOI 10.19047/0136-1694-2019-99-21-46.
{[}in Russian{]}

3. Savin I.Ju., Prudnikova E.Ju. Ob optimal' nom sroke
sputnikovoj semki dlja kartografirovanija pahotnyh pochv // Bjul. Pochv.
in-ta im. V.V. Dokuchaeva. -2014. -№ 74. -S.66-77. {[}in Russian{]}

4. Rozhkov V.A. Ob informacionnom podhode v klassifikacii
pochv//Bjulleten'{} Pochvennogo instituta imeni V.V.
Dokuchaeva. -2012.-Т.69.- S.4-24. DOI /10.19047/0136-1694-2012-69-4-24.
{[}in Russian{]}

5. Greben'{} A.S., Krasovskaja I.G. Analiz osnovnyh
metodik prognozirovanija urozhajnosti s pomoshh' ju
dannyh kosmicheskogo monitoringa, primenitel' no k
zernovym kul' turam stepnoj zony Ukrainy //
Radіoelektronnі і komp'juternі sistemi. -2012. - №2. - S.170-180.
http://nbuv.gov.ua/UJRN/recs\_2012\_2\_27. {[}in Russian{]}

6. Rozhkov V.A. Informatizacija i teorija klassifikacii pochv // Trudy
Instituta gosudarstva i prava RAN.- 2012.- № 6. -S.218-227. {[}in
Russian{]}

7. Hengl T., Mendes de Jesus J., Heuvelink G.B.M., Ruiperez Gonzalez M.,
Kilibarda M. SoilGrids250m: Global gridded soil information based on
machine learning // PLOS ONE. -2017. --Vol.12 (2):e0169748 DOI
10.1371/journal.pone.0169748

8. Chinilin A.V., Savin I.Ju. Krupnomasshtabnoe cifrovoe
kartografirovanie soderzhanija organicheskogo ugleroda pochv s
pomoshh' ju metodov mashinnogo obuchenija //
Bjulleten'{} Pochvennogo instituta im. V.V. Dokuchaeva.
-2018. --T.91. -- S.46-62. DOI 10.19047/0136-1694-2018-91-46-62. {[}in
Russian{]}

9. Masrur Ahmed, A.A., Deo, R.C., Raj, N., Ghahramani, A., Feng, Q., Yin,
Z., Yang, L. Deep learning forecasts of soil moisture: Convolutional
neural network and gated recurrent unit models coupled with
satellite-derived modis, observations and synoptic-scale climate index
data // Remote Sensing. -2021. --Vol.13 (554). -P.1-30. DOI
10.3390/rs13040554

10. Anton, C.A., Matei, O., Avram, A. Collaborative Data Mining in
Agriculture for Prediction of Soil Moisture and Temperature // Advances
in Intelligent Systems and Computing. -2019. -P.141-151. DOI


\emph{{\bfseries Сведения об авторах}}

Болсынбек М. Қ. - докторант кафедры информационных систем Евразийского
национального университета имени Л. Н. Гумилева, Астана, Казахстан,
е-mail:

\href{https://orcid.org/0009-0001-0233-1984}{}

Абдикеримова Г.Б. - PhD, и.о. доцент кафедры информационных систем
Евразийского национального университета им.Л. Н.Гумилева, Астана,
Казахстан, е-mail:
\href{mailto:gulzira1981@mail.ru}{}{\bfseries ;
\href{https://orcid.org/0000-0002-4953-0737}{}}

Тасжурекова Ж.К. - к.т.н., и.о. доцента кафедры «Прикладная
информатика и программирование» Таразского регионального университета
им. М. Х. Дулати, Тараз, Казахстан, е-mail:


Адамов А.А. - д.т.н профессор, кафедра «Математическое и
компьютерное моделирование» член-корр. НИА РК, Евразийский национальный
университет им. Л.Н. Гумилева, Астана, Казахстан, е-mail:

\href{https://orcid.org/0000-0001-9515-1263}{}

Серикбаева С.К. - PhD, старший преподаватель кафедры информационных
систем Евразийского национального университета им. Л. Н. Гумилева,
Астана, Казахстан, е-mail:


Ануарбеков А.М. - Евразийский национальный университет имени
Л.Н.Гумилева, Преподаватель кафедры криптологии, Астана, Казахстан,
е-mail:
\href{mailto:almasanuarbekov01@gmail.com}{}.
\href{https://orcid.org/0009-0008-5013-3553}{}

\emph{{\bfseries Information about the authors}}

Bolsynbek M. - doctoral student of the Department of Information Systems
of the L. N. Gumilyov Eurasian National University, Astana, Kazakhstan,
е-mail: \href{mailto:mbolsynbek@bk.ru}{};

Abdikerimova G.- PhD, acting associate professor of the Department of
Information Systems of L. N. Gumilyov Eurasian National University,
Astana, Kazakhstan, е-mail:
\href{mailto:gulzira1981@mail.ru}{};

Taszhurekova Zh. - acting associate professor of the Department «applied
informatics and programming» Taraz regional university named after M.
KH. Dulaty, Taraz, Kazakhstan. е-mail:


Adamov A. - Doctor of Technical Sciences, Professor, Department of
Mathematical and Computer Modeling, Corresponding Member, NIA RK, L.N.
Gumilyov Eurasian National University, Astana, Kazakhstan, е-mail:
adamov\_aa@enu.kz;

Serikbayeva S. - PhD, Senior Lecturer of the Department of Information
Systems, L.N. Gumilyov Eurasian National University, Astana, Kazakhstan.
E-mail:


Anuarbekov А. - L.N. Gumilyov Eurasian National University, Teacher,
Department of Cryptology, Astana, Kazakhstan, е-mail:
\href{mailto:almasanuarbekov01@gmail.com}{}
.\