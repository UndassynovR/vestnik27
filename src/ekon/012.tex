\id{МРНТИ 06.52.13}{}

\begin{articleheader}
\sectionwithauthors{А.Б. Мухамедханова, Н.Н. Жанакова, М.Р. Сергазиева, Е.К. Молдакенова, Г.К. Амренова}{АНАЛИЗ И ОЦЕНКА ВНЕШНЕГО ДОЛГА КАЗАХСТАНА}

{\bfseries
\textsuperscript{1}А.Б. Мухамедханова,
\textsuperscript{2}Н.Н. Жанакова\textsuperscript{\envelope },
\textsuperscript{3}М.Р. Сергазиева,
\textsuperscript{4}Е.К. Молдакенова,
\textsuperscript{4}Г.К. Амренова
}
\end{articleheader}

\begin{affiliation}
\textsuperscript{1}Южно-Казахстанский университет им. М. Ауэзова, Шымкент, Казахстан,

\textsuperscript{2}Институт экономических исследований, Астана, Казахстан,

\textsuperscript{3}Университет дружбы народов им. Академика А. Куатбекова, Шымкент, Казахстан,

\textsuperscript{4}Евразийский национальный университет им. Л.Н. Гумилева, Астана, Казахстан

\raggedright \textsuperscript{\envelope }Корреспондент-автор: nazikzhan291178@gmail.com
\end{affiliation}

В статье проведен комплексный анализ структуры, динамики и рисков
внешнего долга Казахстана, а также дана оценка ключевых факторов,
определяющих рост внешней задолженности. Исследование охватывает как
государственный, так и частный сектор, включая межфирменную
задолженность и обязательства квазигосударственного сектора.

Изучены секторальные особенности внешнего долга, включая обязательства
органов государственного управления, Национального банка, банковского
сектора и предприятий реального сектора экономики. ~Выявлено, что
наибольшую долю в структуре долга занимают «другие сектора», к которым
относятся нефинансовые организации и квазигосударственные компании.
Проанализированы также финансовые инструменты привлечения внешнего
долга, среди которых преобладают кредиты и займы.

Особое внимание уделено рискам, связанным с государственными гарантиями
по обязательствам частных и квазигосударственных компаний, а также с
колебаниями курса национальной валюты, которые могут увеличить стоимость
обслуживания долга. Отмечено, что рост гарантированного государством
внешнего долга создает дополнительные фискальные риски.

На основе проведенного анализа предложены рекомендации по усилению
контроля за заимствованиями, повышению прозрачности корпоративной
отчетности, диверсификации источников финансирования и разработке
механизмов хеджирования валютных и сырьевых рисков. Подчеркнута
необходимость укрепления фискальной дисциплины и развития внутреннего
рынка капитала для обеспечения долгосрочной финансовой устойчивости и
макроэкономической стабильности Казахстана.

{\bfseries Ключевые слова:} внешний долг, долговые обязательства, долговая
устойчивость, фискальные риски, внешние займы, долговая политика,
бюджетные правила.

\begin{articleheader}
{\bfseries ҚАЗАҚСТАННЫҢ СЫРТҚЫ БОРШЫНЫҢ ТАЛДАУЫ ЖӘНЕ БАҒАЛАУЫ}

{\bfseries
\textsuperscript{1}А.Б. Мухамедханова,
\textsuperscript{2}Н.Н. Жанакова\textsuperscript{\envelope },
\textsuperscript{3}М.Р. Сергазиева,
\textsuperscript{4}Е.К. Молдакенова,
\textsuperscript{4}Г.К. Амренова
}
\end{articleheader}

\begin{affiliation}
\textsuperscript{1}М.Әуезов атындағы Оңтүстік Қазахстан Университеті, Шымкент, Қазақстан,

\textsuperscript{2}Экономика зерттеулер институты, Астана, Қазақстан,

\textsuperscript{3}Академик Ә.Қуатбеков атындағы Халықтар достығы университеті, Шымкент, Қазақстан,

\textsuperscript{4}Л.Н.Гумилев атындағы Еуразия ұлттық университеті, Астана, Қазақстан,

e-mail: nazikzhan291178@gmail.com
\end{affiliation}

Мақалада Қазақстанның сыртқы қарыздың құрылымы, динамикасы және
тәуекелдері кешенді талданып, сыртқы қарыздың өсуін анықтайтын негізгі
факторлар бағаланған. Зерттеу мемлекеттік және жеке секторды, соның
ішінде фирмааралық қарыздар мен квазимемлекеттік компаниялардың
міндеттемелерін қамтиды.

Мемлекеттік басқару органдарының, Ұлттық банктің, банк секторының және
нақты сектор кәсіпорындарының міндеттемелерін қоса алғанда, сыртқы
қарыздың секторлық ерекшеліктері зерттелген. Қарыз құрылымында ең үлкен
үлесті қаржылық емес ұйымдар мен квазимемлекеттік компанияларды қамтитын
«басқа секторлар» алатыны анықталды. Сондай-ақ, сыртқы қарызды тартудың
қаржылық құралдары талданды, олардың ішінде несиелер мен қарыздар басым
болатыны атап өтілді.

Зерттеу барысында жеке және квазимемлекеттік компаниялардың
міндеттемелері бойынша мемлекеттің кепілдіктерімен, сондай-ақ ұлттық
валюта бағамының өзгеруімен байланысты тәуекелдерге ерекше назар
аударылды. Мемлекет кепілдік берген сыртқы қарыздың өсуі қосымша
бюджеттік тәуекелдер тудыратыны атап өтілді.

Жүргізілген талдау негізінде қарыз алуды бақылауды нығайту,
корпоративтік есептіліктің ашықтығын арттыру, қаржыландыру көздерін
әртараптандыру және валюталық және шикізаттық тәуекелдерді хеджирлеу
механизмдерін әзірлеу бойынша ұсыныстар көрсетілді. Қазақстанның ұзақ
мерзімді қаржылық тұрақтылығын және макроэкономикалық тұрақтылығын
қамтамасыз ету үшін бюджеттік тәртіпті нығайту мен ішкі капитал нарығын
дамыту қажеттілігі атап өтілді.

{\bfseries Түйін сөздер:}~сыртқы қарыз, қарыздық міндеттемелер, қарыздық
тұрақтылық, бюджеттік тәуекелдер, сыртқы несиелер, қарыздық саясат,
бюджеттік ережелер.

\begin{articleheader}
{\bfseries ANALYSIS AND ASSESSMENT OF KAZAKHSTAN' S EXTERNAL DEBT}

{\bfseries
\textsuperscript{1}A.B. Mukhamedkhanova,
\textsuperscript{2}N.N. Zhanakova\textsuperscript{\envelope },
\textsuperscript{3}M.R. Sergaziyeva,
\textsuperscript{4}Y.K. Moldakenova,
\textsuperscript{4}G.K. Amrenova
}
\end{articleheader}

\begin{affiliation}
\textsuperscript{1}M. Auezov South Kazakhstan University, Shymkent, Kazakhstan,

\textsuperscript{2}Economic Research Institute, Astana, Kazakhstan,

\textsuperscript{3}Peoples'{} Friendship University named after Academician A. Kuatbekov, Shymkent, Kazakhstan,

\textsuperscript{4}L.N. Gumilyov Eurasian National University, Astana, Kazakhstan,

e-mail: nazikzhan291178@gmail.com
\end{affiliation}

The article provides a comprehensive analysis of the structure,
dynamics, and risks of Kazakhstan' s external debt, as
well as an assessment of the key factors driving the growth of external
liabilities. The study covers both the public and private sectors,
including intercompany debt and obligations of the quasi-public sector.

The sectoral features of external debt are examined, encompassing the
liabilities of government bodies, the National Bank, the banking sector,
and enterprises in the real sector of the economy. It is revealed that
the largest share of the debt structure belongs to "other sectors,"
which include non-financial organizations and public companies. The
financial instruments used to attract external debt are also analyzed,
with loans and borrowings being the predominant forms.

Special attention is given to the risks associated with state guarantees
for the obligations of private and public companies, as well as to
fluctuations in the national currency exchange rate, which may increase
the cost of debt servicing. It is noted that the growth of
state-guaranteed external debt creates additional fiscal risks.

Based on the analysis, recommendations are proposed to strengthen
borrowing controls, enhance corporate reporting transparency, diversify
funding sources, and develop mechanisms for hedging currency and
commodity risks. The need to strengthen fiscal discipline and develop
the domestic capital market to ensure long-term financial stability and
macroeconomic stability in Kazakhstan is emphasized.

{\bfseries Keywords:} external debt, debt obligations, debt sustainability,
fiscal risks, external borrowing, debt policy, fiscal rules.

\begin{multicols}{2}
{\bfseries Введение.} Внешний долг является важным экономическим
индикатором, который отражает финансовую устойчивость страны и её
способность обслуживать обязательства перед внешними кредиторами. Он
играет значительную роль в экономике, обеспечивая приток внешних
ресурсов, необходимых для финансирования бюджетного дефицита, развития
инфраструктуры и реализации стратегических проектов. Однако высокий
уровень внешнего долга может создавать определенные риски, связанные с
его обслуживанием, зависимостью от внешних заимствований и изменениями в
глобальной финансовой среде.

Для Казахстана, как для развивающейся страны с экономикой,
ориентированной на экспорт природных ресурсов, эффективное управление
внешним долгом является ключом к поддержанию макроэкономической
стабильности и обеспечению роста в долгосрочной перспективе. В последние
годы структура внешнего долга Казахстана претерпела значительные
изменения, что требует детального анализа и оценки для понимания текущих
рисков и перспектив.

В этой связи, целью исследования является анализ структуры и динамики
внешнего долга Казахстана, а также оценка ключевых факторов,
определяющих рост задолженности. Исследование направлено на выявление
пределов роста внешнего долга согласно ковенантам внешнего долга,
утвержденным в Концепции управления государственными финансами.

Гипотеза исследования заключается в том, что внешний долг Казахстана
сохраняет управляемость благодаря диверсификации источников
финансирования, росту ненефтяного сектора и стратегическому накоплению
валютных резервов, несмотря на рост абсолютных показателей долга и
уязвимость к колебаниям сырьевых цен. Ключевым риском остается высокая
доля обязательств в иностранной валюте, которая при девальвации тенге
может увеличить нагрузку на бюджет и корпоративный сектор.

{\bfseries Материалы и методы.} Данное исследование основано на применении
широкого спектра научных методов, обеспечивающих аналитическое
сопровождение внешнего долга и структурных его элементов, позволивших
провести его оценку согласно утвержденным ковенантам внешнего долга.

Для комплексного анализа и оценки внешнего долга Казахстана использованы
следующие методы:

- системно-библиографический анализ, проведенного на основе изучения
существующих научных публикаций и методологических подходов, касающихся
анализа и оценки внешнего долга страны;

- институциональный анализ, позволивший изучить нормативно-правовые
документы, регулирующие государственные заимствования, в том числе займы
квазигосударственного сектора;

- методы сравнения и сопоставления на основе ключевых статистических
показателей, позволивших выявить структурные особенности внешнего долга
Казахстана, определить его позицию, а также оценить соответствие
долговой нагрузки установленным критериям долговой устойчивости;

- экспертная оценка позволила верифицировать количественные данные,
выявить ненаблюдаемые факторы риска и получить качественные инсайды о
долгосрочных последствиях текущей долговой политики Казахстана.

Источниками данных явились официальные данные Бюро национальной
статистики АСПР, Национального банка РК, Всемирного банка, а также
аналитические отчеты и информация из интернет-ресурсов.

Исследованию вопросов внешнего долга посвящено множество научных трудов,
каждый из которых подходит к изучению данного вопроса через призму
влияния внешнего долга на различные макроэкономические ситуации при
разных условиях.

Выявлено, что более высокий уровень внешнего долга связан с повышенной
экономической уязвимостью за счет увеличения масштабов и вероятности
внешних шоков, особенно в долгосрочном периоде, в то время, как
краткосрочный внешний шок, напротив, снижает экономическую уязвимость.
Определено, что государственный внешний долг оказывает более негативное
влияние на экономическую стабильность, тогда как частный внешний долг не
имеет явного эффекта {[}1{]}. При этом, экономический рост и инвестиции
способствуют снижению внешнего долга {[}2{]}, в том числе через
трансмиссионные макроэкономические каналы {[}3{]}, {[}4{]} тогда как
обменный курс, объем торговли, государственные расходы {[}5{]}, а также
международные резервы {[}6{]} приводят к его увеличению.

Исследования показывают, что в долгосрочной перспективе внешний долг
оказывает положительное влияние на безработицу и ожидаемую
продолжительность жизни, в то время как он оказывает негативное влияние
на чистый национальный доход {[}7{]}.

В целом, влияние внешнего долга на экономический рост подтверждает тезис
о важности взвешенного управления долгом особенно для развивающихся
экономик, уязвимых к внешним шокам {[}8{]} с качественной
институциональной средой {[}9{]}.

Несмотря на ряд проведенных исследований, возникает необходимость в
углубленном изучении структурных аспектов внешнего долга и его динамики,
а также влияния различных факторов и форм, влияющих на рост
заимствований с целью их оценки в рамках установленных пределов роста в
контексте эффективного управления государственными финансами.

{\bfseries Обсуждение и результаты.} Внешний долг представляет собой
совокупность финансовых обязательств государства и частного сектора
перед иностранными кредиторами, выраженных в иностранной валюте. Он
включает займы международных организаций, правительств других стран,
коммерческих банков и институциональных инвесторов.

Внешний долг играет двойственную роль в экономике: с одной стороны, он
служит инструментом привлечения ресурсов для финансирования дефицита
бюджета, развития инфраструктуры и поддержки ключевых отраслей, с другой
- создаёт риски макроэкономической нестабильности, особенно при
неэффективном управлении долговой нагрузкой. В этой связи вопрос
долговой устойчивости страны приобретает особое значение на фоне
глобальной нестабильности, внешних шоков и ужесточения монетарной
политики ведущих центральных банков, поскольку внешний долг страны, его
объемы, структура и динамика оказывают влияние на платежный баланс, курс
национальной валюты, уровень инфляции и способность выполнять свои
обязательства перед кредиторами.

Внешний долг РК, включая межфирменную задолженность, за период 2014-2023
годы в абсолютном выражении вырос со 157,1 млрд тенге в 2014 году до
163,6 млрд тенге в 2023 году (рост 4,1\%). В то время как его доля в ВВП
уменьшилась с 71,0\% в 2014 году до 62,7\% в 2023 году (рисунок 1). Хотя
это не критичный уровень по международным стандартам, рост внешнего
долга в номинальном выражении вызывает опасения для структуры экономики
с сырьевой направленностью, зависящей от волатильности цен на сырье.
\end{multicols}

{\bfseries Рис.1 - Внешний долг РК, включая межфирменную задолженность, в 2014-2023 годы, млрд долл. США / \% от ВВП}

\emph{Примечание - Составлено по источнику {[}10{]}}

\begin{multicols}{2}
Внешний долг без межфирменной задолженности в абсолютном выражении
сократился с 78,0 млрд долл. в 2014 году до 70,5 млрд долл. в 2023 году
(снижение на 9,6\%). Доля в ВВП так же сократилась с 35,2\% в 2014 году
до 27,0\% в 2023 году (на 8,2 п.п.) (рисунок 2).

При этом, важно отметить рост внешнего долга в кризисные 2016 и 2020
годы (рисунок 1), (рисунок 2).
\end{multicols}

{\bfseries Рис.2 - Внешний долг РК, исключая межфирменную задолженность, в 2014-2023 годы, млрд долл. США / \% от ВВП}

\emph{Примечание - Составлено по источнику {[}10{]}}

\begin{multicols}{2}
Как видно из рисунков 1 и 2, большую часть внешнего долга занимает
межфирменная задолженность казахстанских резидентов перед
аффилированными нерезидентами (например, материнскими компаниями,
дочерними предприятиями или партнерами за границей). То есть, это
накопленные прямые иностранные инвестиции без учета реинвестированной
прибыли. Речь идет о тех инвестициях, которые были привлечены в
Казахстан в виде долговых инструментов (например, займов, кредитов
внутри группы компаний), а не о прибыли, которую иностранные инвесторы
заработали и оставили в стране. Это важно, так как межфирменная
задолженность в рамках транснациональных корпораций, хотя может не
всегда нести такие же риски, как обычные внешние долги, но все же
остается значимой частью финансовых обязательств страны. Риски здесь
могут возникнуть в случае внезапных требований к выплатам при кризисах у
иностранных партнеров, формируя скрытые обязательства.

В секторальном разрезе к резидентам, имеющим внешний долг, относят:

а) органы государственного управления;

б) Национальный банк РК;

в) банки;

г) а также другие сектора.

Органы государственного управления во внешнем долге страны занимают
определенную долю. Их доля составляет в среднем 19\% за последние десять
лет от всего внешнего долга РК, что указывает на активное привлечение
внешних заимствований для финансирования бюджетных расходов. В ВВП их
доля составила 4,5\% в 2023 году. Хотя данный показатель не критичен,
важно учитывать возможные риски увеличения долговой нагрузки в случае
ухудшения экономической конъюнктуры.

Внешний долг Национального банка РК в структуре внешнего долга составил
2\% или 0,9\% от ВВП в 2023 году, оставаясь в целом незначительным, что
указывает на относительно консервативную политику привлечения внешних
средств и наличие достаточных резервов для выполнения обязательств.

Банковский сектор за последние десять лет занимает лишь 11\% в структуре
внешнего долга. В ВВП их доля составила 4,3\% в 2023 году. Это
свидетельствует о снижении притока внешнего финансирования в банковскую
систему, что вызвано как ужесточением регуляторных требований, так и
ограниченным доступом казахстанских банков к международным кредитным
ресурсам.

Значительную часть внешнего долга в общей структуре занимают «Другие
сектора». Их доля составила 67\% за последние десять лет или 17,4\% от
ВВП в 2023 году. К их числу относятся нефинансовые организации, то есть
предприятия реального сектора экономики, привлекающие внешние кредиты и
займы, а также квазигосударственные компании, такие как «Самрук Казына»,
«КазМунайгаз», и др., и частные корпорации, к числу которых относятся
крупные предприятия, использующие внешнее финансирование. Эти сектора
вносят значительный вклад в общий объем внешнего долга страны, особенно
в части корпоративного заимствования.

По срокам до погашения внешний долг подразделяется на: а) краткосрочный;
б) долгосрочный. Преобладающую долю занимает долгосрочный внешний долг
(94\%).6\% принадлежит краткосрочному внешнему долгу.

Структура внешнего долга с точки зрения участия государства
подразделяется на: а) государственный внешний долг; б) внешний долг
банков и организаций, контролируемых государством, в том числе: в)
гарантированный государством внешний долг. Структура внешнего долга с
точки зрения участия государства позволяет оценить степень
ответственности правительства за внешний долг и возможные бюджетные
риски.

За десятилетний период государственный внешний долг вырос в абсолютном
выражении с 8,3 млрд долл. в 2014 году до 14,0 млрд долл. в 2023 году
(рост в 1,7 раза). Доля государственного внешнего долга в ВВП также
выросла с 3,7\% в 2014 году до 5,4\% в 2023 году (рост на 1,6 п.п.)
(рисунок 3). То есть это означает, что обязательства правительства перед
международными кредиторами выросли.

Внешний долг банков и организаций, контролируемых государством, имея
значительно большую долю в общей структуре, тем не менее, согласно
рисунку 3, в абсолютном выражении снизился с 28,0 млрд долл. в 2014 году
до 15,6 млрд долл. в 2023 году (снижение в 1,8 раза). Их доля ВВП так же
снизилась (в 2 раза) с 12,6\% в 2014 году до 6,0\% в 2023 году (рисунок
3). Внешний долг банков и организаций, контролируемых государством,
охватывает долговые обязательства государственных предприятий и
финансовых институтов, находящихся под контролем государства. Такие
организации могут привлекать внешнее финансирование на льготных
условиях, однако высокий уровень их задолженности несет потенциальные
бюджетные риски в случае необходимости поддержки со стороны государства.
К числу таких организаций относятся квазигосударственные компании,
национальные управляющие холдинги, такие как АО «Банк Развития
Казахстана, АО «НУХ «Байтерек», «АО Самрук-Казына» и др.
\end{multicols}

{\bfseries Рис.3 - Государственный внешний долг и внешний долг банков и организаций, контролируемых государством, млрд долл. / \% от ВВП}

\emph{Примечание - Составлено по источнику {[}10{]}}

\begin{multicols}{2}
Зачастую за обязательства частных или государственных компаний
поручителем выступает правительство, и в случае дефолта заемщика эти
обязательства ложатся на бюджет, что создает потенциальные фискальные
риски. И обеспокоенность вызывает то, что гарантированный государством
внешний долг за десятилетний период вырос в 6,7 раза -- с 0,4 млрд долл.
в 2014 году до 2,8 млрд долл. в 2023 году.

Государственный внешний долг сформировался за счет привлечения кредитов
у международных финансовых организаций и двусторонних займов. За
десятилетний период основная часть долга приходится на Еврооблигации
(52,4\% от всего внешнего долга), что указывает на зависимость от
глобальных рынков капитала.

Международный банк реконструкции и развития, на которого приходится
26,1\% от всего внешнего долга, является стабильным многосторонним
кредитором с льготными условиями.

На Азиатский банк развития приходится 14,6\% от всего внешнего долга,
который финансирует инфраструктурные проекты в Центральной Азии.

К числу кредиторов входят также и Японский банк международного
сотрудничества, на долю которого приходится за последние десять лет
около 3,1\% от всего внешнего долга, Европейский банк реконструкции и
развития (1,3\%), Исламский банк развития (0,9\%), и другие.

В целом, внешний долг государственного сектора в расширенном
определении, который включает в себя государственный внешний долг и
внешний долг банков и организаций, контролируемых государством,
снижается в абсолютном выражении с 36,2 млрд долл. в 2014 году до 29,7
млрд долл. в 2023 году (снижение на 18,1\%). В отношении к ВВП данный
показатель так же снижается с 16,4\% в 2014 году до 11,4\% в 2023 году,
комплексно показывая долговую устойчивость на удовлетворительном уровне.

Для эффективного управления бюджетными ресурсами, государственным
долгом, доходами и расходами страны на долгосрочную перспективу в 2022
году принята Концепция управления государственными финансами до 2030
года {[}11{]}.

Согласно данной Концепции политика управления долгом стала
осуществляться в рамках следующих основных ограничений:

1. Внешний долг Правительства (с учетом внешнего гарантированного
государством долга) и внешний долг субъектов КГС не должен превышать
75\% валютных активов Национального фонда РК.

2. Совокупный государственный долг и долг квазигосударственного сектора
не должен превышать лимит 53,2\% к ВВП.

3. Долг квазигосударственного сектора не должен превышать 21,2\% к ВВП.

Из представленных ограничений оценку внешнего долга, в нашем случае,
возможно провести согласно первому пункту ограничений касательно
внешнего долга. Второй и третий пункты ограничений охватывают весь
государственный долг и долг квазигосударственного сектора, включающий
внутренний и внешний долг.

Согласно Концепции управления государственными финансами до 2030 года,
лимит внешнего долга Правительства (с учетом внешнего гарантированного
государством долга) и внешнего долга субъектов квазигосударственного
сектора не должен превышать 75\% валютных активов Национального фонда РК
{[}11{]}. По расчетам авторов, за анализируемый период внешний долг
государственного сектора в расширенном определении, который охватывает
государственный внешний долг и внешний долг банков и организаций,
контролируемых государством, не превысил установленный лимит в 75\% к
валютным активам Национального Фонда РК (рисунок 4).
\end{multicols}

{\bfseries Рис.4 - Государственный внешний долг и внешний долг банков и организаций, контролируемых государством к валютным активам Национального Фонда РК, млрд долл. / \%}

\emph{Примечание - Составлено по источнику {[}10{]}.}

\begin{multicols}{2}
Согласно данным рисунка 4, следует отметить рост данного показателя к
установленному лимиту в 75\% в 2017 году (69\%) и 2021 году (72\%),
объясняемый посткризисным восстановлением экономики, когда государство
брало на себя как прямые, так и условные обязательства после кризисных
годов.

Отдельное внимание необходимо уделить внешнему долгу частного сектора,
который включает обязательства независимых от государства компаний и
организаций перед иностранными кредиторами. Важно отметить их
преобладающую долю в общей структуре долга и рост с 120,9 млрд долл. в
2014 году до 133,9 млрд долл. в 2023 году (рост на 10,8\%). В процентном
отношении к ВВП внешний долг частного сектора сократился с 54,6\% в 2014
году до 51,3\% в 2023 году. Хотя данный вид долга напрямую не связан с
государственными финансами, кризисные ситуации могут потребовать
косвенной поддержки со стороны правительства, например, в случае
банкротства системообразующих предприятий. Поэтому необходимо тщательно
отслеживать динамику задолженности, уровень валютных рисков и
способность заемщиков обслуживать свои обязательства, чтобы
минимизировать потенциальные фискальные и макроэкономические риски.

Внешний долг Казахстана классифицируется по финансовым инструментам, что
позволяет детально анализировать структуру обязательств перед
иностранными кредиторами. По финансовым инструментам внешний долг
осуществляется через:

а) специальные права заимствования;

б) наличные деньги и депозиты;

в) долговые ценные бумаги;

г) кредиты и займы;

д) торговые кредиты и авансы;

е) прочие обязательства.

Среди всех финансовых инструментов основную долю занимают кредиты и
займы (70,2\%), динамика которых выросла за последние десять лет на
3,6\%, что свидетельствует о значительной зависимости страны от
международных заимствований (рисунок 5).

Долговые ценные бумаги в общей классификации финансовых инструментов
внешнего долга занимают 11\%, динамика которых за десять лет сократилась
в 1,6 раза, что указывает на изменение предпочтений в способах
привлечения капитала через кредиты и займы, а не путем выпуска
облигаций.

Торговые кредиты и авансы, хотя и занимают небольшую долю в структуре
применяемых финансовых инструментов (6,3\%), динамика их роста
увеличивается на 21,9\% за десять лет, что свидетельствует об усилении
внешнеэкономической деятельности.

Наличные деньги и депозиты занимают лишь 1,3\%, но динамика их также
значительно выросла и составила 3,5 раза за 2014-2023 годы, что
свидетельствует о росте ликвидности внешних активов Казахстана.

Специальные права заимствования занимают 0,5\%, но их рост составил 3,9
раза, что связано с перераспределением резервов МВФ в рамках глобальных
антикризисных мер.
\end{multicols}

{\bfseries Рис.5 - Внешний долг РК в классификации по финансовым инструментам за 2014-2023 годы, млрд долл.}

\emph{Примечание - Составлено по источнику {[}10{]}}

\begin{multicols}{2}
Проведенный анализ показывает, что внешний долг Казахстана сохраняет
устойчивость благодаря снижению доли в ВВП, а также соблюдению лимитов
согласно Концепции управления государственными финансами до 2030 года.
Однако рост номинального долга с преобладающей долей межфирменной
задолженности, зависимость от сырьевых доходов и скрытые фискальные
риски (гарантии государства) требуют усиления системы управления внешним
долгом.

Прежде всего, основные вызовы связаны с доминированием межфирменных и
частных обязательств, номинированные в иностранной валюте (доллар,
евро). Эти обязательства зачастую не отражаются в публичной отчетности
компаний, что затрудняет оценку реальной долговой нагрузки и создает
дополнительные риски для экономики. Проблема усугубляется отсутствием
жёстких лимитов на соотношение заёмных и собственных средств в рамках
внутригруппового финансирования, что может приводить к накоплению
долгов. В случае ослабления курса национальной валюты стоимость
обслуживания долга для резидентов существенно возрастает, что негативно
сказывается на их финансовой устойчивости. Ухудшение финансового
состояния компаний-заемщиков может привести к сокращению инвестиционной
активности, снижению темпов экономического роста и уменьшению занятости
населения. В долгосрочной перспективе это создает угрозу для
стабильности всей экономической системы.

Для минимизации данных рисков необходимы меры по повышению прозрачности
корпоративной отчетности, введению более жестких регуляторных требований
к заемной деятельности компаний, а также разработка механизмов,
способствующих снижению зависимости от иностранной валюты.

Кроме того, вызывают особую тревогу скрытые фискальные обязательства
государства, которое выступает поручителем по займам тех
системообразующих предприятий (например, в нефтегазовом секторе),
неспособных покрыть долговые обязательства, трансформируемые в бюджетные
расходы. В результате создается давление на платежный баланс, поскольку
резкий отток валюты для погашения долгов ухудшает текущий счет и снижает
золотовалютные резервы.

Такая ситуация не только подрывает финансовую стабильность, но и
ограничивает возможности государства для реализации стратегических
программ и поддержки экономики в периоды кризисов. Для предотвращения
подобных сценариев необходимы меры по повышению прозрачности финансовой
деятельности системообразующих компаний, а также разработка механизмов,
минимизирующих риски переложения долговых обязательств на
государственный бюджет.

Такие вызовы в комплексе формируют систему взаимоусиливающихся рисков,
образуя «эффект домино», когда проблемы у одного крупного заёмщика
(например, в сырьевом секторе) могут спровоцировать цепную реакцию в
связанных отраслях.

Без глубоких реформ ужесточения контроля за заимствованиями частного и
квазигосударственного сектора, стимулирования несырьевого экспорта,
создания механизмов хеджирования валютных и сырьевых рисков, страна
рискует столкнуться с долговым кризисом, аналогичным кризисам таких
развивающихся стран, как Аргентина, Турция в 1990-х гг. Текущие
показатели, хотя и находятся в «зелёной зоне» по международным
стандартам, маскируют накопленные структурные дисбалансы, которые могут
проявиться при следующем глобальном шоке.

{\bfseries Выводы.} Анализ структуры государственного долга Казахстана
показывает, что страна демонстрирует относительно успешное управление
государственным долгом, однако для обеспечения долгосрочной финансовой
устойчивости и макроэкономической стабильности требуется дальнейшее
совершенствование подходов к заимствованиям и снижение внешних рисков.
Ключевыми вызовами остаются зависимость от колебаний курса тенге,
глобальные изменения в финансовых условиях, а также необходимость
финансирования дефицита бюджета, что создает дополнительную нагрузку на
государственные ресурсы.

Для минимизации рисков важно продолжить диверсификацию источников
финансирования, укрепить внутренний рынок капитала и усилить контроль за
заимствованиями квазигосударственных и частных компаний. Особое внимание
следует уделить повышению прозрачности корпоративной отчетности и
разработке механизмов хеджирования валютных и сырьевых рисков, а также
созданию резервов для покрытия потенциальных обязательств.

Кроме того, необходимо внедрять более строгие нормативы для
внутригрупповых займов, стимулировать реинвестирование прибыли и
повышать прозрачность корпоративной отчетности. Укрепление фискальной
дисциплины, оптимизация структуры долга и развитие системы эффективного
управления государственными финансами будут способствовать снижению
долговой нагрузки, повышению доверия инвесторов и обеспечению
устойчивого экономического роста в долгосрочной перспективе.
\end{multicols}

\begin{center}
{\bfseries Литература}
\end{center}

\begin{references}
1. Dau N.H., Pham T., Luu H.N. \& Nguyen D.T. External Debt and Economic
Vulnerability: An International Evidence//Journal of Economic
Integration.-2024.- Vol.39(4). - P.969-990.
\href{https://doi.org/10.11130/jei.2024023}{DOI 10.11130/jei.2024023}.

2. Abdelaziz H., Rim B. \& Majdi K. External Debt, Investment, and
Economic Growth // Journal of Economic Integration. - 2019. --
Vol.34(4). - P.725-745.
\href{https://doi.org/10.11130/jei.2019.34.4.725}{DOI
10.11130/jei.2019.34.4.725}.

3. Dawood M., Feng Z.R., Ilyas M. \& Abbas G. External Debt,
Transmission Channels, and Economic Growth: Evidence of Debt Overhang
and Crowding-Out Effect // SAGE Open. - 2024. - 14(3). - P.1-20.
\href{https://doi.org/10.1177/21582440241263626}{DOI
10.1177/21582440241263626}.

4. World Bank Group. Human Development Global Practice. External Debt,
Fiscal Consolidation, and Government Expenditure on Education // Policy
Research Working Paper. -2023. -10475. P.1-13. - URL:
\href{https://documents1.worldbank.org/curated/en/099748506072325934/pdf/IDU09d7e7fa50fbff046e00a8a80e07ac5341e5b.pdf}{https://worldbank.org} .-
Date of address: 21.02.2025.

5. Dawood M., Baidoo S.T. \& Raza Shah S.M. An empirical investigation
into the determinants of external debt in Asian developing and
transitioning economies //
\href{https://www.tandfonline.com/journals/rdsr20}{Development Studies
Research}. - 2021.- Vol.8(1). - P.253-263.
\href{https://doi.org/10.1080/21665095.2021.1976658}{DOI
10.1080/21665095.2021.1976658}.

6. Mijiyawa A.G., Oloufade D.K. Effect of Remittance Inflows on External
Debt in Developing Countries // Open Economic Review.- 2023. -- Vol.34.
- P.437-470. \href{https://doi.org/10.1007/s11079-022-09675-5}{DOI
10.1007/s11079-022-09675-5}

7. Ayoub A., Wani T.A. \& Sultan A. External debt crisis \&
socio-economic fallout: Evidence from the BRICS nations //
\href{https://www.sciencedirect.com/journal/regional-science-policy-and-practice}{Regional
Science Policy \& Practice}. - 2024. -- Vol.16(6). - P.1-10.
\href{https://doi.org/10.1016/j.rspp.2024.100029}{DOI
10.1016/j.rspp.2024.100029}.

8. Elkhalfi O., Chaabita R., Benboubker M., Ghoujdam M., Zahraoui K., El
Alaoui H., Laalam S., Belhaj I. \& Hammouch H. The impact of external
debt on economic growth: The case of emerging countries//Research in
Globalization.-2024.- Vol.9.- P.1-10.
\href{https://doi.org/10.1016/j.resglo.2024.100248}{DOI
10.1016/j.resglo.2024.100248}.

9. Harsono E., Kusumawati A., Nirwana N. External Debt Determinants: Do
Macroeconomic and Institutional Ones Matter for Selected ASEAN
Developing Countries? // Economies. -2024. -- Vol.12(1). - P.1-16.
\href{https://doi.org/10.3390/economies12010007}{DOI
10.3390/economies12010007}

10. Национальный банк РК. Статистика. Внешний долг. -- URL:
\href{https://www.nationalbank.kz/ru/news/vneshniy-dolg}{https://www.nationalbank.kz}
.- Дата обращения 21.02.2025.11.Об утверждении Концепции управления
государственными финансами Республики Казахстан до 2030 года. Указ
Президента Республики Казахстан от 10 сентября 2022 года № 1005. //
\href{https://adilet.zan.kz/rus/docs/U2200001005}{https://adilet.zan.kz} .- Дата обращения
21.02.2025. 
\end{references}

\begin{center}
{\bfseries References}
\end{center}

\begin{references}
1. Dau N.H., Pham T., Luu H.N. \& Nguyen D.T. External Debt and Economic
Vulnerability: An International Evidence//Journal of Economic
Integration.-2024.- Vol.39(4). - P.969-990.
\href{https://doi.org/10.11130/jei.2024023}{DOI 10.11130/jei.2024023}.

2. Abdelaziz H., Rim B. \& Majdi K. External Debt, Investment, and
Economic Growth // Journal of Economic Integration. - 2019. --
Vol.34(4). - P.725-745.
\href{https://doi.org/10.11130/jei.2019.34.4.725}{DOI
10.11130/jei.2019.34.4.725}.

3. Dawood M., Feng Z.R., Ilyas M. \& Abbas G. External Debt,
Transmission Channels, and Economic Growth: Evidence of Debt Overhang
and Crowding-Out Effect // SAGE Open. - 2024. - 14(3). - P.1-20.
\href{https://doi.org/10.1177/21582440241263626}{DOI
10.1177/21582440241263626}.

4. World Bank Group. Human Development Global Practice. External Debt,
Fiscal Consolidation, and Government Expenditure on Education // Policy
Research Working Paper. -2023. -10475. P.1-13. - URL:
\href{https://documents1.https://worldbank.org/curated/en/099748506072325934/pdf/IDU09d7e7fa50fbff046e00a8a80e07ac5341e5b.pdf}{https://worldbank.org} .-
Date of address: 21.02.2025.

5. Dawood M., Baidoo S.T. \& Raza Shah S.M. An empirical investigation
into the determinants of external debt in Asian developing and
transitioning economies //
\href{https://www.tandfonline.com/journals/rdsr20}{Development Studies
Research}. - 2021.- Vol.8(1). - P.253-263.
\href{https://doi.org/10.1080/21665095.2021.1976658}{DOI
10.1080/21665095.2021.1976658}.

6. Mijiyawa A.G., Oloufade D.K. Effect of Remittance Inflows on External
Debt in Developing Countries // Open Economic Review.- 2023. -- Vol.34.
- P.437-470. \href{https://doi.org/10.1007/s11079-022-09675-5}{DOI
10.1007/s11079-022-09675-5}

7. Ayoub A., Wani T.A. \& Sultan A. External debt crisis \&
socio-economic fallout: Evidence from the BRICS nations //
\href{https://www.sciencedirect.com/journal/regional-science-policy-and-practice}{Regional
Science Policy \& Practice}. - 2024. -- Vol.16(6). - P.1-10.
\href{https://doi.org/10.1016/j.rspp.2024.100029}{DOI
10.1016/j.rspp.2024.100029}.

8. Elkhalfi O., Chaabita R., Benboubker M., Ghoujdam M., Zahraoui K., El
Alaoui H., Laalam S., Belhaj I. \& Hammouch H. The impact of external
debt on economic growth: The case of emerging countries//Research in
Globalization.-2024.- Vol.9.- P.1-10.
\href{https://doi.org/10.1016/j.resglo.2024.100248}{DOI
10.1016/j.resglo.2024.100248}.

9. Harsono E., Kusumawati A., Nirwana N. External Debt Determinants: Do
Macroeconomic and Institutional Ones Matter for Selected ASEAN
Developing Countries? // Economies. -2024. -- Vol.12(1). - P.1-16.
\href{https://doi.org/10.3390/economies12010007}{DOI
10.3390/economies12010007}

10. Nacional' nyj bank RK. Statistika. Vneshnij dolg. -
URL: \href{https://www.nationalbank.kz/ru/news/vneshniy-dolg}{https://www.nationalbank.kz}
.- Data obrashhenija 21.02.2025.{[}in Russian{]}

11. Ob utverzhdenii Koncepcii upravlenija gosudarstvennymi finansami
Respubliki Kazahstan do 2030 goda. Ukaz Prezidenta Respubliki Kazahstan
ot 10 sentjabrja 2022 goda № 1005. //
\href{https://adilet.zan.kz/rus/docs/U2200001005}{https://adilet.zan.kz} .- Data obrashhenija
21.02.2025. {[}in Russian{]}
\end{references}

\begin{authorinfo}
\emph{{\bfseries Сведения об авторах}}

Мухамедханова А.Б. - PhD, Южно-Казахстанский университет им. М. Ауэзова,
Шымкент, Казахстан, e-mail: Dia-2808@mail.ru;

Жанакова Н.Н. - кандидат экономических наук, ассоциированный профессор,
Институт экономических исследований, Астана, Казахстан, e-mail:
nazikzhan291178@gmail.com;

Сергазиева М.Р. -- кандидат экономических наук, Университет дружбы
народов имени академика А.Куатбекова, Шымкент, Казахстан, e-mail:
vipforever@mail.ru;

Молдакенова Е.К. - PhD, Евразийский национальный университет им. Л.Н.
Гумилева, Астана, Казахстан, e-mail:
erke\_totai\_77@mail.ru;

Амренова Г.К. - старший преподаватель кафедры «Менеджмент», Евразийский
национальный университет им. Л.Н. Гумилева, Астана, Казахстан, e-mail:
amrenova1969@mail.ru

\emph{{\bfseries Infotmation about the authors}}

Mukhamedkhanova A. - PhD, South-Kazakhstan University named after
M.Auezov, Shymkent, Kazakhstan, e-mail: Dia-2808@mail.ru;

Zhanakova N. - Candidate of Economic Sciences, Associate Professor,
Economic Research Institute, Astana, Kazakhstan, e-mail:
nazikzhan291178@gmail.com;

Sergaziyeva M. -- Candidate of Economic Sciences,
Peoples'{} Friendship University named after Academician
A. Kuatbekov, Shymkent, Kazakhstan, e-mail: vipforever@mail.ru;

Moldakenova Y.K. - Ph.D, L.N. Gumilyov Eurasian National University,
Astana, Kazakhstan, e-mail: erke\_totai\_77@mail.ru;

Amrenova G.K. - Senior Lecturer, Department of Management; L.N. Gumilyov
Eurasian National University, Astana, Kazakhstan, e-mail:
amrenova1969@mail.ru
\end{authorinfo}
