\newpage
\let\cleardoublepage\clearpage
\part{Экономика, бизнес и услуги}
\let\cleardoublepage\clearpage
\chapter{Экономика, бизнес и услуги}
\ID{IRSTI 06.81.85}{https://doi.org/10.58805/kazutb.v.2.27-811}

\begin{articleheader}
\sectionwithauthors{D.N. Kelesbayev, A.A. Kuralbayev, B.Zh. Keneshbayev, G.R. Mombekova, A.A. Mutaliyeva}{ASSESSMENT AND ANALYSIS OF MANAGEMENT EFFICIENCY OF HEALTHCARE ORGANIZATIONS}

{\bfseries
\textsuperscript{1}D.N. Kelesbayev\alink{https://orcid.org/0000-0002-4193-8121}\textsuperscript{\envelope },
\textsuperscript{1}A.A. Kuralbayev\alink{https://orcid.org/0000-0002-6564-9711},
\textsuperscript{1}B.Zh. Keneshbayev\alink{https://orcid.org/0000-0002-4504-1418},
\textsuperscript{1}G.R. Mombekova\alink{https://orcid.org/0009-0000-2374-7316},
\textsuperscript{2}A.A. Mutaliyeva\alink{http://orcid.org/0000-0002-4268-9382}
}
\end{articleheader}

\begin{affiliation}
\textsuperscript{1}Akhmet Yassawi University, Turkestan, Kazakhstan,

\textsuperscript{2}Regional Innovation University, Shymkent, Kazakhstan

\raggedright \textsuperscript{\envelope }Correspondent-author: dinmukhamed.kelesbayev@ayu.edu.kz
\end{affiliation}

This research paper, prepared in this direction, considers the problems
of efficient use of resources in hospitals in order to increase the
efficiency of health care organizations. Increasing competition and
rising costs in the healthcare industry are forcing hospitals to
optimize their operations. DEA (Data Envelopment Analysis) method is
used to measure the efficiency level of hospitals, this method helps
hospital managers to identify inefficient departments and learn how to
use resources more efficiently. The main goal of the research work is to
conduct a comparative efficiency analysis in 11 departments with
inpatient services based on the data of the Turkestan City Central
Hospital in 2023. As a result of the study, effective and ineffective
departments were identified, and recommendations for resource
optimization were developed. The information obtained during the
research allows the hospital management to effectively use the resour\-ces
allocated by the state and provide quality medical services. The results
of the study recommend that hospital management take concrete steps to
improve the efficiency of their departments, which will contrib\-ute to
increasing the productivity of the overall health care system.

{\bfseries Keywords:} healthcare organizations, management, efficiency, use
of resources, competition, costs, DEA method, comparative efficiency
analysis, medical services.

\begin{articleheader}
{\bfseries ДЕНСАУЛЫҚ САҚТАУ ҰЙЫМДАРЫНЫҢ БАСҚАРУ ТИІМДІЛІГІН БАҒАЛАУ ЖӘНЕ ТАЛДАУ}

{\bfseries
\textsuperscript{1}Д.Н. Келесбаев\textsuperscript{\envelope },
\textsuperscript{1}А.А. Құралбаев,
\textsuperscript{1}Б.Ж. Кенешбаев,
\textsuperscript{1}Г.Р. Момбекова,
\textsuperscript{2}А.А. Муталиева
}
\end{articleheader}

\begin{affiliation}
\textsuperscript{1}Ахмет Ясауи университеті, Түркістан, Қазақстан,

\textsuperscript{2}Аймақтық инновациялық университеті, Шымкент, Қазақстан,

e-mail: dinmukhamed.kelesbayev@ayu.edu.kz
\end{affiliation}

Бұл зерттеу жұмысы денсаулық сақтау ұйымдарының тиімділігін арттыру
мақсатында ауруханалардағы ресурстарды тиімді пайдалану мәселелерін
қарастырады. Денсаулық сақтау саласындағы бәсекелестіктің артуы мен
шығындардың өсуі ауруханаларды өз қызметін оңтайландыруға мәжбүрлейді.
Ауруханалардың тиімділік деңгейін өлшеу үшін DEA (Data Envelopment
Analysis) әдісі қолданылады, бұл әдіс аурухана басшыларына тиімсіз
бөлімдерді анықтап, ресурстарды қалай тиімді пайдалануға болатынын
білуге көмектеседі. Зерттеу жұмысының негізі мақсаты Түркістан қалалық
орталық ауруханасының 2023 жылғы мәліметтеріне сүйеніп, стационарлық
қызметі бар 11 бөлімде салыстырмалы тиімділік анализін жүргізу болып
табылады. Зерттеу нәтижесінде тиімді және тиімсіз бөлімдер анықталып,
ресурстарды оңтайландыру бойынша ұсыныстар әзірленді. Зерттеу барысында
алынған ақпараттар аурухана басшылығына мемлекет бөлетін ресурстарды
тиімді пайдалану мен медициналық қызметтерді сапалы көрсетуге мүмкіндік
береді. Зерттеу нәтижелері аурухана басшылығына өз бөлімдерінің
тиімділігін арттыру үшін нақты қадамдар жасауды ұсынады, бұл жалпы
денсаулық сақтау жүйесінің өнімділігін жоғарылатуға септігін тигізеді.

{\bfseries Түйін сөздер:} денсаулық сақтау ұйымдары, менеджмент, тиімділік,
ресурстарды пайдалану, бәсекелестік, шығындар, DEA әдісі, салыстырмалы
тиімділік анализі, медициналық қызметтер.

\begin{articleheader}
{\bfseries ОЦЕНКА И АНАЛИЗ ЭФФЕКТИВНОСТИ УПРАВЛЕНИЯ ОРГАНИЗАЦИЯМИ ЗДРАВООХРАНЕНИЯ}

{\bfseries
\textsuperscript{1}Д.Н. Келесбаев\textsuperscript{\envelope },
\textsuperscript{1}А.А. Құралбаев,
\textsuperscript{1}Б.Ж. Кенешбаев,
\textsuperscript{1}Г.Р. Момбекова,
\textsuperscript{2}А.А. Муталиева
}
\end{articleheader}

\begin{affiliation}
\textsuperscript{1}Университет Ахмеда Ясави, Туркестан, Казахстан,

\textsuperscript{2}Региональный инновационный университет, Шымкент, Казахстан,

e-mail: dinmukhamed.kelesbayev@ayu.edu.kz
\end{affiliation}

В данной научной работе, подготовленной в этом направлении, рассмотрены
проблемы эффективного использования ресурсов больниц с целью повышения
эффективности деятельности организаций здравоохранения. Растущая
конкуренция и рост затрат в сфере здравоохранения вынуждают больницы
оптимизировать свою деятельность. Метод DEA (анализ оболочки данных)
используется для измерения уровня эффективности больниц. Этот метод
помогает руководителям больниц выявлять неэффективные отделения и
учиться более эффективно использовать ресурсы. Основная цель
научно-исследовательской работы -- провести сравнительный анализ
эффективности в 11 отделениях стационарной помощи на основе данных
Туркестанской городской центральной больницы за 2023 год. В результате
исследования были выявлены эффективные и неэффективные подразделения, а
также разработаны рекомендации по оптимизации ресурсов. Информация,
полученная в ходе исследования, позволяет руководству больницы
эффективно использовать ресурсы, выделяемые государством, и
предоставлять качественные медицинские услуги. Результаты исследования
рекомендуют руководству больниц предпринять конкретные шаги по повышению
эффективности работы своих отделений, что будет способствовать повышению
продуктивности всей системы здравоохранения.

{\bfseries Ключевые слова:} организации здравоохранения, менеджмент,
эффективность, использование ресурсов, конкуренция, затраты, метод DEA,
сравнительный анализ эффективности, медицинские услуги.

\begin{multicols}{2}
{\bfseries Introduction.} Today, hospitals, which are considered healthcare
organizations, are the most important providers of healthcare services.
The continuous continuation of private hospitals depends, first of all,
on the provision of treatment and other necessary services at low costs
and with maximum efficiency, while public hospitals depend on the
efficient use of resources, without wasting public resources. Therefore,
the efficient use of resources directly affects the productivity and
success of hospitals, as in any enterprise. In this context, concepts
such as efficiency and productivity play a very important role in the
work of healthcare organizations. Measuring the concept of efficiency
provides important information for the organizations in question in
terms of their situation in their sectors, as well as for preparing
reports on the current state, status and what issues need to be worked
on more. For all these reasons, conducting an analysis of efficiency and
productivity in healthcare organizations over a certain period of time
will undoubtedly bring various advantages and important information.

The effectiveness of healthcare management is largely determined by
ensuring consistency in management {[}1{]}. The new conditions of the
healthcare system impose different requirements on the management
capacity, its functions and responsibilities should be significantly
expanded. In addition, it is important to note that the activity of the
management mechanism should be achieved by providing a set of measures
that contribute to the formation, use and development of human resources
{[}2{]}. Since competition in the provision of medical services is
increasing day by day, and the costs of the industry are rapidly
increasing, healthcare organizations, which are the core of this
industry, are directing their resources to more efficient and effective
use. For these reasons, all hospitals should measure their efficiency
levels, identify and determine what needs to be done to increase or
decrease resources to the most effective level.

However, efficiency and productivity measurement methods provide
important information that determines how well inputs are used to
achieve target outcomes. Data Envelopment Analysis (DEA) offers many
possibilities when methods such as parametric methods and relational
analysis are not sufficient, especially when measuring multiple inputs
and outputs {[}3{]}. DEA allows for the analysis of the relationship
between different inputs and outputs and is very useful in cases where
this relationship cannot be explained functionally {[}4{]}.

The purpose of the work prepared in this context is to determine the
efficiency of management of healthcare organizations with inpatient
services in the city of Turkestan using the DEA method and conduct a
comparative efficiency analysis. The results of the analysis obtained
are intended to identify effective and ineffective healthcare
organizations, indicate which resources should be increased or decreased
for ineffective healthcare organizations, and share recommendations with
the organization' s management to improve their
efficiency.

{\bfseries Materials and methods.} When considering the effectiveness of
healthcare organizations, it is important to distinguish between public
and private organizations in terms of the consistency of the results
obtained. Since public healthcare organizations are non-profit
organizations, they should be evaluated in terms of the quality and
efficiency of the services they provide, while private healthcare
organizations should be evaluated in the context of the efficient and
productive use of profits and financial resources to ensure the
continuity of their work {[}5{]}. When assessing at the organizational
level, the management of healthcare organizations, that is, management,
is the main department responsible for the results. Since the management
defines the policies, goals, strategies of the institution, implements
them and evaluates the results, the success of the organization is
directly proportional to the quality of their performance {[}6{]}.
Performance management is the collection of timely and forward-looking
information about the organization in order to guide the organization
towards its goals and the implementation of necessary measures aimed at
improving performance in the light of the information collected {[}7{]}.

Nowadays, the increasing competition in the provision of healthcare
services and the rapid increase in costs are pushing hospitals, which
are considered the main organizations in the healthcare sector, to use
their resources more efficiently and productively {[}8{]}. For these
reasons, hospitals need to measure their efficiency levels, identify
inefficient departments, determine the amount of income/expenditure that
needs to be increased or decreased, and know what to do to achieve the
most efficient level.

Most of the health services in our country are provided by hospitals.
Therefore, various factors affect the efficiency or inefficiency of
hospitals, which are the locomotives of health services. Analyzing data
using the DEA method and thereby determining the level of efficiency of
hospitals provides important indicators for hospital managers {[}9{]}.
Hospital managers can focus on low-efficiency departments based on the
indicators of the analysis results. The reason for the failure to take
various measures aimed at maximizing the output from existing revenues
is that organizations providing health services are not managed by
modern management methods {[}10{]}. From this point of view, it can be
assumed that the failure of hospitals to achieve the intended level of
efficiency is due to the insufficient functioning of the control
mechanism. Therefore, it is necessary to identify the factors that
reduce the qualitative and quantitative values
\hspace{0pt}\hspace{0pt}of health services through various methodologies
and analyzes, and develop various measures to increase efficiency. Thus,
it is possible to provide high-quality medical services to the public by
ensuring efficiency and productivity.

In this regard, the purpose of this research work is to conduct a
comparative efficiency analysis with the DEA method in 11 departments of
the Turkestan City Central Hospital, which, according to 2023 data, have
inpatient services and can provide the necessary data for the DEA
method, which was determined as a research method. The DEAP 2.1. program
was used during the analysis. The research work prepared in this
direction is a written study of the retrospective research type.
Therefore, during this study, an approbation was conducted in the
Turkestan City Central Hospital and 11 departments affiliated to it and
the data for 2023 were taken as a basis. The capacity of the hospital,
which was the basis for this study, is 403 beds; 159 of the beds are
used for surgery, 202 for therapy, and 42 for life support services. The
hospital employs 178 doctors, including 14 general practitioners, 36
assistant physicians, 128 specialist physicians, and a total of 397
healthcare workers, including 64 midwives, 309 nurses, and 24 laboratory
technicians.

In this regard, the set and sample of the research work are determined
as follows:

The Turkestan City Central Hospital is the set (base mass) of the
research work. As a sample (sample group) of the research work, 11
departments (DMD-Decision-making departments) of this hospital that have
inpatient services and produce similar results using similar inputs were
taken. They are:

- Neurosurgery

- Pediatric Surgery

- General Surgery

- Ophthalmology

- Obstetrics and Gynecology

- Thoracic Surgery

- Cardiac Surgery

- Otorhinolaryngology

- Orthopedics and Traumatology

- Plastic Surgery

- Urology
\end{multicols}

\begin{table}[H]
\caption*{Table 1 - Technical efficiency results based on a constant income assumption according to the measurement scale}
\centering
\begin{tabular}{|l|l|l|} 
\hline
N  & Name of the department           & Technical efficiency result  \\ 
\hline
1  & Neurosurgery                     & 0,773                        \\ 
\hline
2  & Pediatric Surgery                & 0,653                        \\ 
\hline
3  & General Surgery                  & 0,723                        \\ 
\hline
4  & Ophthalmology                    & 1,000                        \\ 
\hline
5  & Obstetrics and Gynecology        & 1,000                        \\ 
\hline
6  & Thoracic Surgery                 & 0,355                        \\ 
\hline
7  & Cardiac Surgery                  & 0,602                        \\ 
\hline
8  & Otorhinolaryngology              & 1,000                        \\ 
\hline
9  & Orthopedics and Traumatology     & 0643                         \\ 
\hline
10 & Plastic Surgery                  & 1,000                        \\ 
\hline
11 & Urology                          & 0,895                        \\ 
\hline
\multicolumn{2}{|l|}{Average value}   & 0,786                        \\
\hline
\end{tabular}
\caption*{\normalfont\emph{Note -- Compiled by the authors}}
\end{table}

\begin{multicols}{2}
According to the DEA method, which is defined as a research method, if
the number of inputs is determined as m and the number of outputs is
determined as p, the number of DMDs must be at least m + p + 1 for the
analysis to yield results. The study of the work and services provided
by healthcare organizations is mainly focused on the use of hospital
resources. This research work is also aimed at the efficient use of
resources. Therefore, in this study, 2 inputs and 3 outputs were
selected according to the nature of the HSEs and the available data.
They are:

- 2 inputs:

- Number of doctors.

- Number of beds.

- 3 outputs:

- Number of treatments.

- Number of surgeries.

- Number of discharges.

The annual average of the analyzed inputs was taken and the total of the
outputs was estimated. In short, managers (leaders) in hospitals and
similar healthcare organizations have the competence and authority to
manage and control not only the organization' s outputs
(products), but also its revenues. Within this logic, the model created
during the research work is income-oriented (minimum income-maximum
expenditure) and is designed in such a way that it provides a fixed and
variable income result according to the measurement scale.

{\bfseries Results and discussion.} Managers in hospitals and similar
healthcare organizations have the authority to manage, direct, and
control not only the organization' s output but also its
revenues. Within this logic, the model in this study is designed to be
revenue-oriented (minimum revenue, maximum expense), with fixed and
variable revenue outcomes according to the measurement scale. First, the
tables and their explanations are presented based on the fixed revenue
forecast according to the measurement scale, and then on the variable
revenue forecast according to the measurement scale.

As shown in Table 1, out of 11 departments identified as DMD, only 4
(ophthalmology, obstetrics and gynecology, otorhinolaryngology, plastic
surgery) were found to be fully efficient. Based on this result, it can
be said that this hospital is not operating with high efficiency. On the
other hand, it can be seen that the efficiency level of the remaining
departments, except for urology and the 4 fully efficient departments,
is lower than the average efficiency value.

When examining the departments one by one, the differences between the
input and output components and the target values
and the departments that are exemplary for them
are shown in Table 2.
\end{multicols}

\tcap{Table 2 -- Consistent income according to the measurement scale: exemplary departments}
\begin{longtblr}[
  label = none,
  entry = none,
]{
  colspec = {X X X X X X},
  cell{1}{1} = {r=2}{},
  cell{3}{1} = {r=2}{},
  cell{5}{1} = {r=2}{},
  cell{7}{1} = {r=2}{},
  cell{9}{1} = {r=2}{},
  cell{11}{1} = {r=2}{},
  cell{13}{1} = {r=2}{},
  cell{15}{1} = {r=2}{},
  cell{17}{1} = {r=2}{},
  cell{19}{1} = {r=2}{},
  cell{21}{1} = {r=2}{},
  vlines,
  hline{1,3,5,7,9,11,13,15,17,19,21,23} = {-}{},
  hline{2,4,6,8,10,12,14,16,18,20,22} = {2-6}{},
}
Neurosurgery & Exemplary section & Plastic Surgery & Ophthalmology & Obstetrics and gynecology & \\
 & Lambda value & 1,439 & 0,139 & 0,087 & \\
Pediatric Surgery & Exemplary section & Ophthalmology & Obstetrics and gynecology &  & \\
 & Lambda value & 0,060 & 0,044 &  & \\
General Surgery & Exemplary section & Plastic Surgery & Obstetrics and gynecology &  & \\
 & Lambda value & 1,377 & 0,466 &  & \\
Ophthalmology & Exemplary section & Ophthalmology &  &  & \\
 & Lambda value & 1,000 &  &  & \\
Obstetrics and gynecology & Exemplary section & Obstetrics and gynecology &  &  & \\
 & Lambda value & 1,000 &  &  & \\
Breast surgery & Exemplary section & Obstetrics and gynecology &  &  & \\
 & Lambda value & 0,036 &  &  & \\
Otorhinolaryn\-gology & Exemplary section & Obstetrics and gynecology &  &  & \\
 & Lambda value & 0,219 &  &  & \\
Orthopedics and Traumatology & Exemplary section & Plastic Surgery & Otorhinolaryn\-gology & Obstetrics and gynecology & Ophthalmology\\
 & Lambda value & 0,388 & 0,345 & 0,288 & 0,270\\
Plastic Surgery & Exemplary section & Plastic Surgery &  &  & \\
 & Lambda value & 1,000 &  &  & \\
Urology & Exemplary section & Obstetrics and gynecology & Plastic Surgery & Otorhinolaryn\-gology & \\
 & Lambda value & 0,171 & 0,381 & 0,400 & \\
Cardiac surgery & Exemplary section & Obstetrics and gynecology &  &  & \\
 & Lambda value & 0,219 &  &  & 
\end{longtblr}
\begin{center}
\vspace{-1em}
\emph{Note -- Compiled by the authors}
\end{center}

\begin{table}[H]
\caption*{Table 3 - Comparative results of fixed, variable income and overall efficiency according to the measurement scale}
\centering
\begin{tabular}{|l|l|p{0.16\textwidth}|p{0.16\textwidth}|p{0.16\textwidth}|} 
\hline
N & Name of the department & CRS: Constant Income According to the Measurement Scale & VRS: Variable Remuneration Scale & Overall measurement efficiency \\ 
\hline
1 & Neurosurgery & 0,773 & 0,786 & 0,983(drs) \\ 
\hline
2 & Pediatric Surgery & 0,653 & 1,000 & 0,653(irs) \\ 
\hline
3 & General Surgery & 0,723 & 0,874 & 0,827(drs) \\ 
\hline
4 & Ophthalmology & 1,000 & 1,000 & 1,000 \\ 
\hline
5 & Obstetrics and Gynecology & 1,000 & 1,000 & 1,000 \\ 
\hline
6 & Thoracic Surgery & 0,355 & 1,000 & 0,355(irs) \\ 
\hline
7 & Cardiac Surgery & 0,602 & 0,632 & 0,952(irs) \\ 
\hline
8 & Otorhinolaryngology & 1,000 & 1,000 & 1,000 \\ 
\hline
9 & Orthopedics and Traumatology & 0643 & 0,664 & 0,968(drs) \\ 
\hline
10 & Plastic Surgery & 1,000 & 1,000 & 1,000 \\ 
\hline
11 & Urology & 0,895 & 0,899 & 0,996(irs) \\ 
\hline
\multicolumn{2}{|l|}{\uline{Average value}} & \uline{0,786} & \uline{0,896} & \uline{0,885} \\
\hline
\end{tabular}
\caption*{\normalfont\emph{Note -- Compiled by the authors}}
\end{table}

\begin{multicols}{2}
As shown in Table 2, the departments of ophthalmology, obstetrics and
gynecology, otorhinolaryngology, and plastic surgery are exemplary
departments for inefficient departments. Since the 4 departments with
the highest overall efficiency have a lambda value of 1, no exemplary
departments have been identified for them. To give an example here,
according to the data in the table, the departments that are exemplary
for the neurosurgery department are: the plastic surgery department with
the first lambda value of 1.439, ophthalmology with the second lambda
value of 0.139, and obstetrics and gynecology with the third lambda
value of 0.087. Based on these exemplary departments and their lambda
values, a formula indicating what revenues the neurosurgery department
should have to be efficient was determined using the number of revenues
in the departments and lambda values, and is shown below.

- Target income = {[}(5; 1) x 1.439{]} + {[}(20; 11) x 0.087{]} + {[}(8;
5) x 0.139{]} = (10.046; 3.091).

As mentioned in the ``Materials and methods'' section of the study,
overall efficiency is the sum of technical efficiency and measurement
efficiency. For this reason, it is necessary to consider technical and
measurement efficiency separately in the analysis. Table 3 shows the CRS
score for overall efficiency, the VRS score for technical efficiency and
the measurement efficiency scores for the departments together.

When Table 3 above is examined in detail, the CRS values
\hspace{0pt}\hspace{0pt}of the 11 departments whose efficiency was
analyzed are the same as the CRS values
\hspace{0pt}\hspace{0pt}presented in Table 1. As mentioned earlier, CRS
values \hspace{0pt}\hspace{0pt}are a measure of overall efficiency for
departments. However, the VRS values \hspace{0pt}\hspace{0pt}presented
in Table 3 provide technical efficiency results based on variable
success predictions according to the measurement scale. According to
these results, the departments of ophthalmology, obstetrics and
gynecology, otorhinolaryngology, and plastic surgery were also
determined to be technically efficient as determined by CRS values. In
addition, the departments of thoracic surgery and pediatric surgery,
although having the lowest overall efficiency, had the highest score in
technical efficiency. When these departments were evaluated in terms of
measurement efficiency, while the departments of neurosurgery, general
surgery, orthopedics, and traumatology had decreasing success according
to the measurement scale, the departments of pediatric surgery, thoracic
surgery, cardiac surgery, and urology had increasing success according
to the measurement scale.
\end{multicols}

\tcap{Table 4 - Variable income according to the measurement scale: sample sections}
\begin{longtblr}[
  label = none,
  entry = none,
]{
  colspec = {X X X X X},
  cell{1}{1} = {r=2}{},
  cell{3}{1} = {r=2}{},
  cell{5}{1} = {r=2}{},
  cell{7}{1} = {r=2}{},
  cell{9}{1} = {r=2}{},
  cell{11}{1} = {r=2}{},
  cell{13}{1} = {r=2}{},
  cell{15}{1} = {r=2}{},
  cell{17}{1} = {r=2}{},
  cell{19}{1} = {r=2}{},
  cell{21}{1} = {r=2}{},
  vlines,
  hline{1,3,5,7,9,11,13,15,17,19,21,23} = {-}{},
  hline{2,4,6,8,10,12,14,16,18,20,22} = {2-6}{},
}
Neurosurgery & Exemplary section & Plastic Surgery & Ophthalmology & Obstetrics and gynecology & \\
 & Lambda value & 0,661 & 0,208 & 0,131 & \\
Pediatric Surgery & Exemplary section & Pediatric Surgery &  &  & \\
 & Lambda value & 1,000 &  &  & \\
General Surgery & Exemplary section & Plastic Surgery & Obstetrics and gynecology & Otorhinolaryn\-gology & \\
 & Lambda value & 0,290 & 0,473 & 0,237 & \\
Ophthalmology & Exemplary section & Ophthalmology &  &  & \\
 & Lambda value & 1,000 &  &  & \\
Obstetrics and gynecology & Exemplary section & Obstetrics and gynecology &  &  & \\
 & Lambda value & 1,000 &  &  & \\
Breast surgery & Exemplary section & Obstetrics and gynecology &  &  & \\
 & Lambda value & 1,000 &  &  & \\
Otorhinolaryn\-gology & Exemplary section & Obstetrics and gynecology &  &  & \\
 & Lambda value & 1,000 &  &  & \\
Orthopedics and Traumatology & Exemplary section & Plastic Surgery & Otorhinolaryn\-gology & Obstetrics and gynecology & Ophthalmology\\
 & Lambda value & 0,066 & 0,416 & 0,303 & 0,215\\
Plastic Surgery & Exemplary section & Plastic Surgery &  &  & \\
 & Lambda value & 1,000 &  &  & \\
Urology & Exemplary section & Obstetrics and gynecology & Plastic Surgery & Otorhinolaryn\-gology & Pediatric Surgery\\
 & Lambda value & 0,169 & 0,391 & 0,395 & 0,045\\
Cardiac surgery & Exemplary section & Obstetrics and gynecology & Plastic Surgery &  & \\
 & Lambda value & 0,153 & 0,847 &  & 
\end{longtblr}
\begin{center}
\vspace{-1em}
\emph{Note -- Compiled by the authors}
\end{center}

\begin{multicols}{2}
As shown in Table 4, the departments of plastic surgery, obstetrics and
gynecology, ophthalmology, otorhinolaryngology, pediatrics, and thoracic
surgery are exemplary departments for inefficient departments. Since the
6 departments with the highest overall efficiency have a lambda value of
1, no exemplary departments have been identified for them. To give an
example here, according to the data in the table, the departments that
are exemplary for the neurosurgery department are: plastic surgery with
a lambda value of 0.661 in the first place, ophthalmology with a lambda
value of 0.208 in the second place, and obstetrics and gynecology with a
lambda value of 0.131 in the third place. It is important to note here
that the study is input-oriented and cannot participate in output.

{\bfseries Conclusions.} In general, this study conducted an efficiency
analysis of 11 departments of the Turkestan City Central Hospital and
evaluated their conditions, with varying results. While there are many
studies in the literature that use DEA to analyze groups of private and
public hospitals and their health centers, there are very few studies
that analyze and evaluate departments and clinics within a single
hospital. In addition, there are few studies that use similar inputs and
outputs to those used in this study.

According to the results of the analysis of the 11 departments
considered in this study, the departments of ophthalmology, obstetrics
and gynecology, otorhinolaryngology, and plastic surgery were found to
be fully inefficient based on the CRS forecast. And according to the
fixed income forecast according to the measurement scale, only about
36\% of these departments were recognized as fully efficient. Based on
the VRS forecast, the departments of pediatric surgery, ophthalmology,
obstetrics and gynecology, otorhinolaryngology, and plastic surgery were
technically efficient. And according to the variable income forecast
according to the measurement scale, about 55\% of these departments were
considered technically efficient. The greater efficiency of departments
according to the VRS forecast compared to the results obtained under the
CRS forecast is due to the greater flexibility of the VRS forecast and
the positive effect on efficiency through the use of variable income. An
important point to note in the analysis results is the need to reduce
the number of doctors and beds in all departments recognized as
inefficient. This situation indicates that inefficient departments use
human resources and physical capabilities inefficiently and
unproductively.

In this regard, based on the data obtained from the study, the following
recommendations are made on improving efficiency in relation to low
efficiency and highlighting the importance of efficiency:

- Since the results of the DEA method show the relative efficiency
between departments, inefficient departments should be taken as a basis
for model departments.

- The excess and empty number of doctors and beds in all inefficient
departments should be reorganized, human resources and physical
capabilities should be reorganized.

- In order to effectively use human resources and physical capabilities,
the relevant department needs excellent managers who can measure and
evaluate efficiency and effectiveness.

- Inefficient departments should work directly with departments that
need excess and empty resources and share these resources.

- The principles of efficiency and productivity should be given special
importance in the management of healthcare organizations.

- Hospitals that use the bulk of the allocated resources in the
healthcare sector should conduct more efficiency and productivity
analyses.

- Hospital management should use the results of the efficiency and
productivity analysis as a roadmap and allocate resources in accordance
with these analyses.

- In the course of efficiency analyses, it is necessary to achieve a
wide range of results by using various inputs and outputs.

In conclusion, all these recommendations indicate ways to effectively
use scarce resources. Therefore, we believe that this research work
conducted at the Turkestan City Central Hospital will be useful in terms
of effective use of resources, consistent quality provision of health
services, and most importantly, "doing the right thing well."
\end{multicols}

\begin{center}
{\bfseries References}
\end{center}

\begin{references}
1. Kostyrin, E.V. Economic and mathematical models of financial
incentives for the personnel at medical organization departments //
International Journal of Pharmaceutical Research. -2020. Vol.12(4). -P.
1769--1780. DOI
\href{https://doi.org/10.28991/ESJ-2023-07-03-017}{10.28991/ESJ-2023-07-03-017}

2. Nundoochan, A. Improving public hospital efficiency and fiscal space
implications: The case of Mauritius // International Journal for Equity
in Health.- 2020.-Vol.19(1).- P.152-160. DOI 10.1186/s12939-020-01262-9

3. Ahmadvand, S., Pishvaee, M.S. An efficient method for kidney
allocation problem: a credibility-based fuzzy common weights data
envelopment analysis approach // Health Care Management Science. -2018.
Vol.21(4). - P.587--603. DOI 10.1007/s10729-017-9414-6

4. Kim, C., Kim, H.J. A study on healthcare supply chain management
efficiency: using bootstrap data envelopment analysis // Health Care
Management Science. -2019.- Vol.22(3). - P.534-548. DOI\\
10.1007/s10729-019-09471-7

5. Duffourc, M.N. Filling voice promotion gaps in healthcare through a
comparative analysis of error reporting and learning systems and open
communication and disclosure policies in the United States and Germany
// American Journal of Law and Medicine.- 2018.- Vol.44(4). -P.
579--605. DOI \\10.1177/0098858818821137

6. Björkelund, C., Svenningsson, I., Westman, J., Petersson, E.-L.,
Hange, D., Holst, A., Wallin, L., Udo, C. Effects of a care manager
organization for care of people with mild-moderate depression in Swedish
primary care // Lakartidningen.-2019.-Vol.116(1).- P.29-40. PMID:
31688945

7. Yamin, M., Alharthi, S. Measuring impact of healthcare information
systems in administration and operational management // International
Journal of Information Technology (Singapore).- 2020. -Vol.- 12(3).- P.
767-774. DOI 10.1007/s41870-019-00329-3

8. Shi, Z., Wu, F., Huang, H., Sun, X., Zhang, L. Comparing economics,
environmental pollution and health efficiency in China // International
Journal of Environmental Research and Public Health.- 2019.
-Vol.16(23).- P.258-264. DOI 10.3390/ijerph16234827

9. Chai, P., Zhang, Y., Zhou, M., Liu, S., Kinfu, Y. Technical and scale
efficiency of provincial health systems in China: A bootstrapping data
envelopment analysis // BMJ Open. -- 2019. -Vol.9(8). -P.77--92. DOI
10.1136/bmjopen-2018-027539

10. Peixoto, M.G.M., Musetti, M.A., Mendonça, M.C.A. Multivariate
analysis techniques applied for the performance measurement of Federal
University Hospitals of Brazil // Computers and Industrial \\Engineering .-
2018. -Vol.126.- P.16-29. DOI 10.1016/j.cie.2018.09.020
\end{references}

\begin{authorinfo}
\hspace{1em}\emph{{\bfseries Information about the authors}}

Kelesbayev D.N. - PhD, Professor, Akhmet Yassawi University, Turkestan,
Kazakhstan, е-mail: \\dinmukhamed.kelesbayev@ayu.edu.kz;

Kuralbayev A.A. - PhD, Senior Lecturer, Akhmet Yassawi University,
Turkestan, Kazakhstan, е-mail: \\almas.kuralbayev@ayu.edu.kz;

Keneshbayev B.Zh. - PhD, Senior Lecturer, Akhmet Yassawi University,
Turkestan, Kazakhstan, е-mail: \\keneshbayev\_bektur@ayu.edu.kz;

Mombekova G.R. - PhD, Associate Professor, Akhmet Yassawi University,
Turkestan, Kazakhstan, е-mail: \\gulmira.mombekova@ayu.edu.kz;

Mutaliyeva A.A. - PhD, Senior Lecturer, Regional Innovation University,
Shymkent, Kazakhstan, е-mail: Alua012@mail.ru.

\hspace{1em}\emph{{\bfseries Сведения об авторах}}

Келесбаев Д.Н. - PhD, профессор, Университет Ахмеда Ясави, Туркестан,
Казахстан, е-mail: \\dinmukhamed.kelesbayev@ayu.edu.kz;

Куралбаев А.А. - PhD, Старший преподаватель, Университет Ахмеда Ясави,
Туркестан, Казахстан, е-mail: \\almas.kuralbayev@ayu.edu.kz;

Кенешбаев Б.Ж. - PhD, Старший преподаватель, Университет Ахмеда Ясави,
Туркестан, Казахстан, е-mail: \\keneshbayev\_bektur@ayu.edu.k;

Момбекова Г.Р. - PhD, доцент, Университет Ахмеда Ясави, Туркестан,
Казахстан, е-mail: gulmira.mombekova@ayu.edu.kz;

Муталиева А.А. - PhD, Старший преподаватель, Региональный инновационный
университет, Шымкент, Казахстан, е-mail: Alua012@mail.ru.
\end{authorinfo}
