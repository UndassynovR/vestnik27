\id{МРНТИ 06.03.15}{}

\begin{articleheader}
\sectionwithauthors{А.А.Жалгасбаева, С.К.Буканова, А.Н.Абдигалиева}{ПРИОРИТЕТНЫЕ ОТРАСЛИ АТЫРАУСКОГО РЕГИОНА: КЛЮЧЕВЫЕ КРИТЕРИИ И ФАКТОРЫ}

{\bfseries
А.А. Жалгасбаева\alink{https://orcid.org/0000-0003-4040-6855}\textsuperscript{\envelope },
С.К. Буканова\alink{https://orcid.org/0000-0001-8508-1863},
А.Н. Абдигалиева\alink{https://orcid.org/0009-0003-7907-6875}
}
\end{articleheader}

\begin{affiliation}
НАО «Атырауский университет нефти и газа имени Сафи Утебаева», Атырау, Казахстан

\textsuperscript{\envelope }Корреспондент-автор: a.zhalgasbayeva@aogu.edu.kz
\end{affiliation}

В статье рассматриваются ключевые критерии и факторы, определяющие
приоритетные отрасли экономики Атырауского региона. Особое внимание
уделяется анализу промышленного сектора, включая нефтегазовую отрасль,
которая занимает доминирующее положение, а также перерабатывающую
промышленность, строительство и сельское хозяйство.

Проведен анализ социально-экономического развития региона за последние
годы, включая динамику валового регионального продукта (ВРП), объемов
промышленного производства и уровня занятости населения. Рассмотрены
структурные изменения в экономике региона, влияющие на развитие
приоритетных отраслей, а также влияние государственных программ и
стратегий на экономическую стабильность.

Отдельное внимание уделено вопросам инновационного развития,
технологического прогресса и факторов, способствующих росту ключевых
отраслей. В статье представлены статистические данные, характеризующие
динамику производства, инвестиционную активность и тенденции занятости в
различных секторах экономики.

Также рассмотрены вызовы, с которыми сталкивается регион, включая
зависимость от нефтегазового сектора, необходимость развития несырьевых
отраслей, улучшения инфраструктуры и повышения уровня квалификации
рабочей силы. Подчеркивается важность диверсификации экономики,
увеличения объемов инвестиций в перерабатывающую промышленность, а также
внедрения современных технологий в ключевые отрасли.

Научная новизна исследования заключается в разработке
научно-методического подхода, позволяющего формировать приоритетные
отрасли, которые позволят обеспечить рационализацию перераспределения
инвестиционных потоков в развитие различных отраслей региональной
экономики нефтегазодобывающего региона, с целью диверсификации и
модернизации промышленности на инновационной основе, снижения уровня ее
монопрофильности, реализацию социальных и экологических программ в целях
повышения уровня устойчивого развития и социальной ориентации
региональной экономики в условиях рынка.

Научное обоснование приоритетных направлений социально-экономического
развития нефтегазодобывающих регионов является важнейшей составляющей
управленческих процессов, определяющих состояние их экономики. Имеющиеся
средства должны быть сконцентрированы на наиболее значимых направлениях
социально-экономического развития. Выделение приоритетных направлений
развития региона позволяет создать кумулятивный эффект и активизировать
деятельность в смежных отраслях.

Практическая значимость определения приоритетных отраслей для региона
заключается в следующем:

1. Эффективное распределение ресурсов -- позволяет направить инвестиции
и усилия в наиболее перспективные направления, обеспечивая рост и
устойчивое развитие.

2. Диверсификация экономики -- снижает зависимость от одной отрасли
(например, нефтегазовой), способствуя развитию других секторов.

3. Создание рабочих мест -- развитие приоритетных отраслей стимулирует
занятость, особенно в высокотехнологичных и инновационных сферах.

4. Повышение конкурентоспособности региона -- укрепление отраслей с
высоким экспортным и инновационным потенциалом усиливает позиции региона
на внутреннем и внешнем рынках.

5. Социальное развитие -- способствует реализации социальных программ,
улучшению качества жизни населения и повышению уровня образования.

Результаты исследования включают следующие ключевые выводы:

1. Выделены приоритетные отрасли -- на основе анализа
социально-экономических показателей региона определены отрасли с
наибольшим потенциалом для роста (нефтегазовая промышленность,
строительства и сельское хозяйство).

2. Оценено влияние приоритетных отраслей на регион -- выявлено, что
развитие определённых отраслей способствует снижению монозависимости,
созданию рабочих мест, росту ВРП и устойчивому развитию.

3. Сформированы сведения о влиянии внешних факторов на отдельные отрасли
региона (например, мировые цены на нефть, экологические аспекты), а
также определено влияние государственных программ и проектов на развитие
приоритетных отраслей.

В заключении делаются выводы о перспективах развития Атырауской области,
возможностях диверсификации экономики и необходимости поддержки
стратегически значимых отраслей. Предложены рекомендации по повышению
конкурентоспособности региона за счет развития инновационного
потенциала, улучшения делового климата и привлечения инвестиций в
высокотехнологичные производства.

{\bfseries Ключевые слова:} Атырауская область, региональное развитие,
приоритетные отрасли экономики, промышленность, человеческий капитал,
жилищное строительство.

\begin{articleheader}
{\bfseries АТЫРАУ ӨҢІРІНІҢ БАСЫМ САЛАЛАРЫ: ТҮЙІНДІ КРИТЕРИЙЛЕР МЕН ФАКТОРЛАР}

{\bfseries
А.А. Жалғасбаева\textsuperscript{\envelope },
С.К. Бұқанова,
А.Н. Әбдіғалиева
}
\end{articleheader}

\begin{affiliation}
"Сафи Өтебаев атындағы Атырау мұнай және газ университеті" КЕАҚ, Атырау, Қазақстан,

e-mail: a.zhalgasbayeva@aogu.edu.kz
\end{affiliation}

Мақалада Атырау өңірі экономикасының басым салаларын айқындайтын негізгі
критерийлер мен факторлар қарастырылады. Өнеркәсіп секторын, оның ішінде
үстем жағдайға ие мұнай-газ саласын, сондай-ақ қайта өңдеу өнеркәсібін,
құрылыс пен ауыл шаруашылығын талдауға ерекше назар аударылады.

Жалпы өңірлік өнімнің (ЖӨӨ) серпінін, өнеркәсіптік өндіріс көлемдерін
және халықты жұмыспен қамту деңгейін қоса алғанда, өңірдің соңғы
жылдардағы әлеуметтік-экономикалық дамуына талдау жүргізілді. Басым
салалардың дамуына әсер ететін өңір экономикасындағы құрылымдық
өзгерістер, сондай-ақ Мемлекеттік бағдарламалар мен стратегиялардың
экономикалық тұрақтылыққа әсері қаралды.

Инновациялық даму, технологиялық прогресс және негізгі салалардың өсуіне
ықпал ететін факторлар мәселелеріне ерекше назар аударылды. Мақалада
өндіріс динамикасын, инвестициялық белсенділікті және экономиканың
әртүрлі секторларындағы жұмыспен қамту тенденцияларын сипаттайтын
статистикалық мәліметтер келтірілген.

Сондай-ақ, мұнай-газ секторына тәуелділікті, шикізаттық емес салаларды
дамыту, инфрақұрылымды жақсарту және жұмыс күшінің біліктілік деңгейін
арттыру қажеттілігін қоса алғанда, өңірдің алдында тұрған сын-қатерлер
қаралды. Экономиканы әртараптандырудың, қайта өңдеу өнеркәсібіне
инвестициялар көлемін ұлғайтудың, сондай-ақ негізгі салаларға заманауи
технологияларды енгізудің маңыздылығы атап өтіледі.

Зерттеудің ғылыми жаңалығы -- мұнай-газ өндіруші өңірлердің экономикасын
әртараптандыру және жаңғырту мақсатында инвестициялық ағындарды әртүрлі
салаларға ұтымды қайта бөлуді қамтамасыз етуге мүмкіндік беретін басым
салаларды қалыптастыруға бағытталған ғылыми-әдістемелік тәсілді
әзірлеуінде көрініс табады. Бұл тәсіл өнеркәсіптің инновациялық негізде
дамуын, оның монопрофильділігін төмендетуді, сондай-ақ әлеуметтік және
экологиялық бағдарламаларды іске асыру арқылы өңір экономикасының
орнықты дамуы мен әлеуметтік бағыттылығын арттыруды көздейді.

Мұнай-газ өндіруші өңірлердің әлеуметтік-экономикалық дамуының басым
бағыттарын ғылыми негіздеу -- өңірлік экономика жағдайын айқындайтын
басқарушылық процестердің маңызды құрамдас бөлігі болып табылады. Қолда
бар ресурстар әлеуметтік-экономикалық дамудың неғұрлым өзекті
бағыттарына шоғырландырылуы тиіс. Өңір дамуының басым бағыттарын
айқындау жинақталу (кумулятивті) әсерін туындатып, оған іргелес
салалардағы қызметтің белсенділігін арттыруға ықпал етеді.

Аймақ үшін басым салаларды анықтаудың практикалық маңыздылығы келесідей:

1. Ресурстарды тиімді бөлу -- инвестициялар мен күш-жігерді неғұрлым
перспективалы бағыттарға шоғырландыруға мүмкіндік береді, бұл өз
кезегінде өсуді және тұрақты дамуды қамтамасыз етеді.

2. Экономиканы әртараптандыру -- бір салаға (мысалы, мұнай-газ саласына)
тәуелділікті төмендетіп, басқа секторлардың дамуына ықпал етеді.

3. Жұмыс орындарын құру -- басым салалардың дамуы, әсіресе жоғары
технологиялық және инновациялық салаларда жұмыспен қамтуды арттырады.

4. Өңірдің бәсекеге қабілеттілігін арттыру -- жоғары экспорттық және
инновациялық әлеуеті бар салаларды нығайту арқылы өңірдің ішкі және
сыртқы нарықтардағы орнын күшейтеді.

5. Әлеуметтік даму -- әлеуметтік бағдарламаларды жүзеге асыруға, халықтың
өмір сүру сапасын жақсартуға және білім деңгейін көтеруге ықпал етеді.

Зерттеу нәтижелері келесі негізгі қорытындыларды қамтиды:

1. Басым салалар айқындалды -- өңірдің әлеуметтік-экономикалық
көрсеткіштеріне талдау жүргізу негізінде өсу әлеуеті жоғары салалар
(мұнай-газ өнеркәсібі, құрылыс және ауыл шаруашылығы) анықталды.

2. Басым салалардың өңірге әсері бағаланды -- жекелеген салалардың дамуы
монотәуелділікті азайтуға, жұмыс орындарын ашуға, жалпы өңірлік
өнімнің өсуіне және орнықты дамуға ықпал ететіні көрсетілді.

3. Сыртқы факторлардың өңірдің жекелеген салаларына әсері туралы
мәліметтер жинақталды (мысалы, мұнайдың әлемдік бағасы, экологиялық
аспектілер), сондай-ақ мемлекеттік бағдарламалар мен жобалардың басым
салалардың дамуына ықпалы анықталды.

Қорытындыда Атырау облысының даму перспективалары, экономиканы
әртараптандыру мүмкіндіктері және стратегиялық маңызы бар салаларды
қолдау қажеттілігі туралы қорытындылар жасалады. Инновациялық әлеуетті
дамыту, іскерлік ахуалды жақсарту және жоғары технологиялық өндірістерге
инвестициялар тарту есебінен өңірдің бәсекеге қабілеттілігін арттыру
бойынша ұсыныстар ұсынылды.

{\bfseries Түйін сөздер:} Атырау облысы, өңірлік даму, экономиканың басым
салалары, өнеркәсіп, адами капитал, тұрғын үй құрылысы
\newpage
\begin{articleheader}
{\bfseries PRIORITY SECTORS OF THE ATYRAU REGION: KEY CRITERIA AND FACTORS}

{\bfseries
A.A. Zhalgasbayeva\textsuperscript{\envelope },
S.K. Bukanova,
A.N. Abdigalieva
}
\end{articleheader}

\begin{affiliation}
Atyrau University of Oil and Gas named after Safi Utebayev, Atyrau, Kazakhstan,

e-mail: a.zhalgasbayeva@aogu.edu.kz
\end{affiliation}

The article examines the key criteria and factors determining the
priority sectors of the Atyrau region' s economy. Special
attention is paid to the analysis of the industrial sector, including
the oil and gas industry, which occupies a dominant position, as well as
the processing industry, construction and agriculture.

The analysis of the socio-economic development of the region in recent
years, including the dynamics of the gross regional product (GRP),
industrial production and employment levels. Structural changes in the
region' s economy affecting the development of priority
industries, as well as the impact of government programs and strategies
on economic stability, are considered.

Special attention is paid to the issues of innovative development,
technological progress and factors contributing to the growth of key
industries. The article presents statistical data characterizing the
dynamics of production, investment activity and employment trends in
various sectors of the economy.

The challenges faced by the region, including dependence on the oil and
gas sector, the need to develop non-resource industries, improve
infrastructure and improve the skills of the workforce, were also
consid\-ered. The importance of economic diversification, increased
investment in the processing industry, as well as the introduction of
modern technologies in key industries is emphasized.

The scientific novelty of the research lies in the development of a
scientific and methodological approach that enables the identification
of priority sectors. These sectors will help ensure the rational
redistribution of investment flows for the development of various
industries in the regional economy of an oil and gas producing area,
with the aim of diversifying and modernizing industry on an innovative
basis, reducing the region' s dependence on a single
industry, and implementing social and environmental programs to enhance
sustainable development and the social orientation of the regional
economy in a market environment.

The scientific justification of priority areas for the socio-economic
development of oil and gas producing regions is a crucial component of
the management processes that determine the state of their economies.
Available resources should be concentrated on the most significant areas
of socio-economic development. Identifying priority directions for
regional development allows for a cumulative effect and stimulates
activity in related sectors.

The practical significance of identifying priority industries for the
region is as follows:

1. Efficient allocation of resources - allows investments and efforts to
be directed to the most promising areas, ensuring growth and sustainable
development.

2. Diversification of the economy - reduces dependence on one industry
(for example, oil and gas), promoting the development of other sectors.

3. Job creation - the development of priority sectors stimulates
employment, especially in high-tech and innovative areas.

4. Improving the region' s competitiveness -
strengthening industries with high export and innovation potential
strengthens the region' s position in domestic and
foreign markets.

5. Social development - facilitates the implementation of social
programmes, improves the quality of life of the population and raises
the level of education.

The results of the study include the following key findings:

1. Priority industries are identified - based on the analysis of
socio-economic indicators of the region, the industries with the highest
potential for growth (oil and gas industry, construction and
agriculture) were identified.

2. the impact of priority industries on the region was assessed - it was
found that the development of certain industries contributes to the
reduction of mono-dependence, job creation, GRP growth and sustainable
development.

3. Information on the impact of external factors on individual
industries in the region (e.g., world oil prices, environmental aspects)
is compiled, and the impact of government programmes and projects on the
development of priority industries is determined.

In conclusion, conclusions are drawn about the prospects for the
development of Atyrau region, the possibilities of economic
diversification and the need to support strategically important
industries. Recom\-mendations are proposed to increase the competitiveness
of the region by developing innovative potential, improving the business
climate and attracting investment in high-tech industries.

{\bfseries Keywords:} Atyrau region, regional development, priority sectors
of the economy, industry, human capital, housing construction

\begin{multicols}{2}
{\bfseries Введение}. Атырауская область является одной из самых динамично
развивающихся областей Казахстана, имеющей выгодное географическое
положение области, где находятся уникальные запасы нефтегазового и
газоконденсатного сырья, богатые месторождения калиевых и натриевых
солей, строительных материалов.

Целью исследования является определение ключевых критериев и факторов,
влияющих на развитие приоритетных отраслей экономики Атырауского
региона, а также анализ их динамики, структуры и перспектив с учетом
текущих социально-экономических условий и государственной политики.
Исследование актуально в контексте реализации государственных программ,
направленных на модернизацию промышленности, поддержку малого и среднего
бизнеса, а также повышение инвестиционной привлекательности региона.

Реальная новизна данного исследования по сравнению с уже существующими
подходами заключается в том, что обоснованы результаты исследования
реальными показателями и определены положения о том, что
нефтегазодобывающие регионы имеют существенные особенности в своем
социально-экономическом развитии, так как общенациональное и мировое
значение нефти и газа, открытость экономики, рентный характер большей
части доходов таких территорий, высокая степень участия государства в
регулировании развития нефтегазового сектора, а также доминирующее
положение вертикально интегрированных компаний в экономике таких
регионов - все это объективно накладывает отпечаток на особенности
управления их социально-экономическим развитием.

{\bfseries Материалы и методы}. Исследование основано на анализе
статистических данных, нормативно-правовых документов и стратегических
программ социально-экономического развития Атырауской области. В
качестве источников информации использованы данные Бюро национальной
статистики Республики Казахстан, региональные и отраслевые отчеты, а
также материалы научных публикаций.

{\bfseries Результаты и обсуждение.} В объеме промышленной продукции
Атырауской области доля нефтегазового сектора составляет 88\%. Вместе с
тем, регион является важнейшим рыбопромысловым районом.

За последние пять лет чис­ленность экономически активного населения
региона увеличилась и на 1 января 2024 г. составила~704,1 тыс. человек,
в том числе 389,9 тыс. человек (55,4\%) -- городских, 314,2 тыс. человек
(44,6\%) -- сельских жителей. Лидером по ВРП стала нефтяная Атырауская
область -- более 9,5 млн тенге на душу населения.

Согласно Государственной программе развития регионов на 2020--2025 годы
основными отраслями экономики в перспективе Атырауской области являются
углубление нефтепереработки, нефтехимии, производство строительных
материалов агропромышленная и рыбные отрасли.

По мнению респондентов экономическая стабильность региона обеспечивается
за счет развития  сфер.

Важными отраслями экономики, которые имеют стратегическую значимость для
развития области отметили 102 (17 \%) чел. -- сельского хозяйства, 91
(15 \%) чел. -- перерабатывающую отрасль, 85 (14 \%) чел. -- добывающую
промышленность. В разрезе отраслей по итогам 2023 года большая часть
работников заняты в сельском хозяйстве (12,7\%) и промышленности
(12,4\%).

{\bfseries Промышленность}. Основные показатели промышленности приводятся в
таблице 1.
\end{multicols}

\tcap{Таблица 1 - Основные показатели промышленности Атырауской области на 2018--2023 годы {[}1{]}.}
\begin{longtblr}[
  label = none,
  entry = none,
]{
  width = \linewidth,
  colspec = {Q[20]Q[400]Q[60]Q[60]Q[60]Q[60]Q[65]Q[56]},
  cells = {c},
  hlines,
  vlines,
}
№ & Показатели & 2018 & 2019 & 2020 & 2021 & 2022 & 2023\\
1 & Число предприятий и производств (ед.) & 363 & 396 & 395 & 394 & 291 & 415 \\
2 & Индекс физического объема промышленного производства, в процентах к предыдущему году & 110,6 & 105,4 & 94,5 & 102,8 & 97,9 & 111,2 \\
3 & Объем производства промышленной продукции, млрд. тенге & 7 077,5 & 7 888,1 & 5 174,8 & 8 557,6 & 13 133,7 & 10 815
\end{longtblr}

\begin{multicols}{2}
С 2018 по 2022 года сохранялась положительная динамика показателей
социально-экономического развития области (таблица 2). По итогам 2022
года объем производства промышленной продукции составил 13 133,7 млрд.
тенге. Хотя количество предприятий сократилось на 72 ед.

В 2023 году объем производства промышленной продукции в регионе
снизился на 2318,7 млрд. тенге по сравнению с аналогичным периодом
прошлого года. В горнодобывающей промышленности произведено продукции на
10 047,3~млрд тенге,~индекс физического объема --~111,5\%. В
обрабатывающей промышленности объем продукции составил 696,1~млрд~тенге
или~106,5\%.

Энергетический шок 2022 года, экономические трудности, возникшие в
разных регионах, стремительный рост цен на энергоносители, который
невозможно было представить два года назад, и геополитические конфликты
-- все это в совокупности подтолкнуло многие компании к пересмотру
стратегий энергетического перехода.

В рамках такого пересмотра делается вывод, что энергетический переход
должен предполагать обеспечение энергетической безопасности -- то есть
наличия энергоресурсов в надлежащих объемах и по разумным ценам -- чтобы
заручиться поддержкой населения и избежать серьезных экономических
осложнений.

В 2023 году произведено промышленной продукции на сумму 10~815~028 млн.
тенге, из них в горнодобывающей отрасли на 9 851~177 млн. тенге (91 \%
от общего объема) и снизился на 2318720 млн. тенге по сравнению с
аналогичным периодом прошлого года и в обрабатывающей на 791136 млн.
тенге (9\%) и вырос на 118297 млн. тенге соответственно.

В горнодобывающей промышленности и разработке карьеров в 2023 году ИПП
составил 97,9 \%, что обусловлено спадом добычи сырой нефти (98,0 \%),
природного газа (92,7 \%), наблюдается рост предоставление услуг в
горнодобывающей промышленности на 16,0 \%.

В обрабатывающей промышленности производство увеличилось на 1,3 \%.
Увеличилось производство продуктов питания, переработка и
консервирование рыбы, ракообразных и моллюсков, производство молочных
продуктов, производство напитков, производство одежды, производство
деревянных и пробковых изделий; кроме мебели, производство изделий из
соломки и материалов для плетения, полиграфическая деятельность и
воспроизведение записанных носителей информации, производство продуктов
химической промышленности, производство прочей не металлической
минеральной продукции, металлургическое производство, машиностроение.

В снабжении электроэнергией, газом, паром, горячей водой и
кондиционированным воздухом ИПП составил 92,6\%, в основном за счет
снижение объемов производства, передачи и распределения электроэнергии
на 10,7 \%.

В водоснабжении, сборе, обработке и удалении отходов, деятельности по
ликвидации загрязнений ИПП в 2023 году составил 103,1\%. Снизились
объемы по сбору, обработке и распределению воды, сбору, обработке и
удалению отходов; утилизации (восстановлении) материалов.
\end{multicols}

\tcap{Таблица 2 - Основные показатели социально-экономического развития Атырауской области на 2018--2023 годы {[}2-5{]}.}
\begin{longtblr}[
  label = none,
  entry = none,
]{
  width = \linewidth,
  colspec = {Q[263]Q[108]Q[108]Q[108]Q[108]Q[117]Q[117]},
  cells = {c},
  cell{5}{1} = {c=7}{0.928\linewidth},
  cell{8}{1} = {c=7}{0.928\linewidth},
  cell{11}{1} = {c=7}{0.928\linewidth},
  cell{14}{1} = {c=7}{0.928\linewidth},
  hlines,
  vlines,
}
Показатели & 2018 & 2019 & 2020 & 2021 & 2022 & 2023\\
1. ВРП, млрд. тенге & 7~818 & 9~327 & 7~795 & {~\\
			
			10~627
		} & {~\\
			
			14~236
		} & {~\\
			
			9~682
		}\\
2. млн. долларов
			США & 22 682,3 & 24 369,1 & 18 877,2 & 24 945,6 & {~\\
			
			30949
		} & {~\\
			
			21094
		}\\
3. в \% к предыду-щему
			году & 113,3 & 107,4 & 93,7 & 105,8 & {~\\
			
			134
		} & {~\\
			
			68
		}\\
Валовой региональный
			продукт на душу населения &  &  &  &  &  & \\
4. тыс. тенге & 12465,5 & 14 584,4 & 11 970,8 & 16 037,4 & 19
			974,1 & 13~887,1\\
5. тыс. долларов
			США & 36,2 & 38,1 & 30,0 & 37,6 & {~\\
			
			44,4
		} & {~\\
			
			31,2
		}\\
Объем промышленного
			производства &  &  &  &  &  & \\
5. млн. тенге & 7 077 540 & 7 888 134 & 5 174 828 & 8 557 592 & 13 133 748 & 10~815
			028\\
6. в \% к предыду-щему
			году & 110,6 & 105,4 & 94,5 & 102,8 & {~\\
			
			153,4
		} & {~\\
			
			82
		}\\
Горнодобывающая
			промышленность &  &  &  &  &  & \\
7. млрд. тенге & 6~411 & 7~268 & 4~554 & 7 720~ & 12 320 & 9 851\\
8. в \% к предыду-щему
			году & 111,2 & 104,7 & 94,1 & 101,9 & {~\\
			
			159,1
		} & {~\\
			
			80
		}\\
Обрабатывающая
			промышленность &  &  &  &  &  & \\
9. млрд. тенге & 584 & 525 & 526 & 704 & 673 & 791\\
10. в \% к предыду-щему
			году & 109,6 & 109,7 & 99,6 & 106,8 & {~\\
			
			98,7
		} & {~\\
			
			118
		}
\end{longtblr}

\begin{multicols}{2}
Валовой региональный продукт области в 2023 году составил 9~682,3 млрд
тенге. Индекс физического объема -- 109,3\%.

Удельный вес области в объеме валового внутреннего продукта республики
-- 12,8\%. ВРП на душу населения -- 13~887,1 тыс. тенге.

Объем промышленной продукции составил 10 895,7 млрд тенге, индекс
физического объема -- 111,1\%. В горнодобывающей промышленности
произведено продукции на 10 047,3 млрд. тенге, индекс физического объема
-- 111,5\%. В обрабатывающей промышленности объем продукции составил
696,1 млрд тенге или 106,5\%.

Важность нефтегазовой отрасли для региона увеличивает уязвимость
региональной экономики перед внешними потрясениями. В частности, сильная
зависимость экономики региона от доходов, связанных с углеводородами,
усугубляет ее чувствительность к колебаниям мировых цен на нефть, что
находило неоднократное подтверждение в последние годы. В частности, как
сокращение ВВП Атырауской области в 2020 году, так и его последующее
сильное восстановление в 2021 году во многом повторяло траекторию
мировых цен на нефть, а замедление роста ВВП региона в 2022-23 гг.
совпало с замедлением роста цен на нефть в 2022 году и их снижением в
2023 году. Однако, несмотря на вышеуказанные факторы уязвимости,
ожидается, что энергетика в целом и углеводородная отрасль в частности
останется ключевым драйвером экономического роста Атырауского региона на
протяжении всего прогнозного периода до 2050 года.

Однако Казахстан будет сталкиваться с растущей конкуренцией за
ограниченные объемы иностранных инвестиций во всем мире (в том числе со
стороны других крупных стран-производителей углеводородов). Компании
инвесторы по-прежнему будут рассматривать новые возможности, но теперь
они придерживаются гораздо более строгой дисциплины управления
капиталом, и странам, располагающим ресурсами, будет все труднее
бороться за доступные инвестиции в новые проекты. В складывающейся
ситуации властям Казахстана важно принимать действенные меры -- за счет
продуманной фискальной и иной политики -- чтобы продемонстрировать, что
страна располагает «выгодными» ресурсами, разработка, добыча и
транспортировка которых возможна по относительно низкой цене и с
незначительным углеродным следом. Не менее важными факторами являются
предсказуемая нормативная среда и своевременное принятие решений. Именно
эти критерии будут играть главную роль при выборе объектов для
инвестирования международными компаниями.
\end{multicols}

\tcap{Таблица 3 - Основные производители сельскохозяйственной продукции Атырауской области на 2023 год {[}6{]}}
\begin{longtblr}[
  label = none,
  entry = none,
]{
  width = \linewidth,
  colspec = {Q[40]Q[178]Q[254]Q[137]Q[346]},
  cells = {c},
  hlines,
  vlines,
}
№ & Показатели & Количество
			юридических лиц & Количество ИП & Крестьянские или
			фермерские хозяйства\\
1 & Атырауская обл & 111 & 352 & 3 969\\
2 & Атырау г.а. & 61 & 75 & 847\\
3 & Жылыой & 8 & 17 & 464\\
4 & Индер & 9 & 6 & 464\\
5 & Исатай & 4 & 5 & 365\\
6 & Курмангазы & 4 & 209 & 642\\
7 & Кызылкога & 15 & 3 & 547\\
8 & Макат & - & 6 & 105\\
9 & Махамбет & 10 & 31 & 535
\end{longtblr}

\begin{multicols}{2}
{\bfseries Сельское хозяйство} является одним из важнейших источников роста
для долгосрочного устойчивого развития, экономической диверсификации и
повышения уровня жизни населения в Казахстане. Сельское хозяйство
продолжает выполнять важную роль в экономике страны, обеспечивая
продовольственную безопасность, способствуя устойчивому развитию
регионов и создавая рабочие места. Государство активно поддерживает этот
сектор, предоставляя льготное финансирование, субсидии и другие меры для
повышения конкурентоспособности агропромышленной продукции на внутреннем
и внешних рынках.~

Общая площадь земель Атырауской области составляет 11 863,1 тыс. га, из
них земли сельскохозяйственного назначения -- 2 982,6 тыс.га. Площадь
лесного фонда -- 56,0 тыс. га. Водные ресурсы области складываются из
ресурсов реки Урал, Кигач.

Сегодня в агропромышленном секторе Атырауской области отмечается
положительная динамика развития. Доля валовой добавленной стоимости
сельского, лесного и рыбного хозяйства региона с 2018 года менялась
незначительно (с 4,3 до 5,2\%).

Действующие производители сельскохозяйственной продукции приводим в
таблице 3.

В 2023 году по Атырауской области зарегистрированы 4432 организации в
области сельского хозяйства, из них 3969 ед. (90\%) крестьянские или
фермерские хозяйства, самое большое количество которых находится в
Махамбетском районе -- 535.

Динамика количества занятого населения по сельскому хозяйству по
Атырауской области за 2018--2023 годы приводим в таблице 4.
\end{multicols}

\tcap{Таблица 4 - Количество занятого населения по сельскому хозяйству по Атырауской области за 2018-2022 годы (тыс.чел) {[}2, 3, 6, 7{]}.}
\begin{longtblr}[
  label = none,
  entry = none,
]{
  width = \linewidth,
  colspec = {Q[454]Q[71]Q[83]Q[83]Q[83]Q[83]Q[75]},
  cells = {c},
  hlines,
  vlines,
}
Показатели & 2018 & 2019 & 2020 & 2021 & 2022 & 2023\\
Занято в экономике, всего & 304 & 316,3 & 314,5 & 317,7 & 326,7 & 335\\
Сельское, лесное и рыбное хозяйство & 8 & 8 & 8 & 8,2 & 8,2 & 9,1
\end{longtblr}

\tcap{Таблица 5 - Валовый выпуск продукции (услуг) сельского хозяйства Атырауской области на 2018--2023 годы (в фактически действовавших ценах, млрд. тенге).}
\begin{longtblr}[
  label = none,
  entry = none,
]{
  width = \linewidth,
  colspec = {Q[556]Q[62]Q[62]Q[62]Q[62]Q[62]Q[62]},
  row{1} = {c},
  row{2} = {c},
  cell{2}{1} = {c=7}{0.928\linewidth},
  cell{3}{2} = {c},
  cell{3}{3} = {c},
  cell{3}{4} = {c},
  cell{3}{5} = {c},
  cell{3}{6} = {c},
  cell{3}{7} = {c},
  cell{4}{2} = {c},
  cell{4}{3} = {c},
  cell{4}{4} = {c},
  cell{4}{5} = {c},
  cell{4}{6} = {c},
  cell{4}{7} = {c},
  cell{5}{2} = {c},
  cell{5}{3} = {c},
  cell{5}{4} = {c},
  cell{5}{5} = {c},
  cell{5}{6} = {c},
  cell{5}{7} = {c},
  cell{6}{2} = {c},
  cell{6}{3} = {c},
  cell{6}{4} = {c},
  cell{6}{5} = {c},
  cell{6}{6} = {c},
  cell{6}{7} = {c},
  hlines,
  vlines,
}
Показатели & 2018 & 2019 & 2020 & 2021 & 2022 & 2023\\
Все категории хозяйств &  &  &  &  &  & \\
1. Валовый выпуск продукции (услуг) сельского хозяйства, в том числе: & 67 & 80 & 86 & 115 & 133 & 118\\
1.1 Продукция растениеводства & 25 & 32 & 36 & 42 & 50 & 52\\
1.2 Продукция животноводства & 40 & 45 & 48 & 70 & 81 & 62\\
1.3 Вспомогательные виды деятельности в области выра-щивания сельскохозяйственных культур и разведения животных & 0,5 & 0,5 & 0,4 & 0,6 & 0,6 & 0,3
\end{longtblr}

\begin{multicols}{2}
В 2023 году в сельском хозяйстве по региону трудятся 9144 чел. (2,7 \%)
из 335399 чел. активного населения {[}8{]}.

В 2023 году объем производства в сельском хозяйстве составил 11,7 млрд.
тенге и снизился на один млрд. тенге по сравнению с аналогичным периодом
прошлого года. Индекс физического объема продукции 102,8 \%. Высокую
долю в структуре объема валовой продукции (услуг) сельского хозяйства
занимает - продукция животноводства (60\%) (таблица 5).

Объем валового выпуска продукции (услуг) сельского, лесного и рыбного
хозяйства в январе 2024 года составил~5503,9 млн. тенге, что меньше
на~5,2\%~чем в январе 2023 г.

В рамках Комплексного плана социально-экономического развития Атырауской
области на 2021 -- 2025 годы предусмотрено создание индустриальной зоны
площадью 400 га, строительство завода по производству терефталевой
кислоты и оптово-распределительного центра.

По развитию сельского хозяйства планируется строительство 6
теплиц,~5~откорм площадок,~3~птицефабрик,~7~прудовых хозяйств для
выращивания рыб, цехов по переработке молока и другие.

К концу 2025 года ожидается увеличение объема валовой продукции
сельского хозяйства региона на~11\%, до~96 млрд.~тенге.

Для улучшения показателей данной отрасли служили меры, направленные на
повышение эффективности аграрного сектора и улучшение условий для
агробизнеса.

Впервые общий объем кредитования весенне-полевых и уборочных работ
доведен до 580 млрд тенге. В дополнение к имеющимся на начало года 180
млрд тенге льготного финансирования, правительство обеспечило
привлечение дополнительных 400 млрд тенге для кредитования фермеров под
5\% годовых. Это решает ключевую проблему отрасли --
недофинансированность. Достаточное количество льготных заемных средств
позволит аграриям соблюдать технологии для получения высоких урожаев.

Для упрощения получения займов был привлечен еще один канал
финансирования -- Социально-предпринимательские корпорации регионов.
Благодаря более гибкой залоговой политике и удобству подачи заявок, этот
канал стал востребованным среди заемщиков. Атырауская область заложила
15 млрд тенге на льготное кредитование весенне-полевых и уборочных работ
через СПК.Для решения проблемы дефицита залогов, особенно из-за
пролонгации кредитов прошлого года, введен механизм гарантирования
займов через фонд «Даму». Этот механизм позволяет покрыть гарантией до
85\% от суммы займа.

С марта 2024 года по поручению Главы государства внедрен механизм
авансового субсидирования отечественных удобрений. Это новый
стимулирующий механизм, при котором аграрии заранее обеспечиваются
удешевленными отечественными удобрениями.~

{\bfseries Строительство.} Объем строительных работ в Казахстане в 2023
году составил 7,6128 триллиона тенге и по сравнению с 2022 годом
увеличился на 15,1\% (см.таблица 6). Увеличение объема строительных
работ в 2023 году по сравнению с 2022 годом связано с работами по
строительству и ремонту сооружений (на 26,4\%) и жилых зданий (на 7\%).
Доля региона в строительной сфере республики составляет 16,3\%.
\end{multicols}

\tcap{Таблица 6 - Объем выполненных строительных работ и количество подрядных строительных организаций Атырауской области на 2018--2023 годы.}
\begin{longtblr}[
  label = none,
  entry = none,
]{
  width = \linewidth,
  colspec = {Q[519]Q[67]Q[67]Q[67]Q[67]Q[67]Q[67]},
  cells = {c},
  hlines,
  vlines,
}
Показатели & 2018 & 2019 & 2020 & 2021 & 2022 & 2023\\
Количество подрядных строительных организаций, единиц & 219 & 240 & 208 & 188 & 221 & 241\\
Объем выполненных строительных работ, млрд. тенге & 638 & 868 & 873 & 960 & 1 164 & 1 221\\
Индексы физического объема строительных работ, в процентах к предыдущему году & 107,4 & 133,8 & 101,1 & 106,6 & 118,4 & 104,8
\end{longtblr}

\begin{multicols}{2}
В 2023 году в сфере строительства по региону трудятся 62715 чел. (19\%)
из 335 тыс. чел. активного населения (таблица 7).

Объем выполненных строительных работ Атырауской области в 2023 году
1~220~669 358 тыс. тенге. Количество подрядных строительных организаций
в регионе в 2023 году 241 ед. Объем выполненных строительных работ
увеличился на 2 раза по сравнению с 2018 годом согласно таблице 8.
\end{multicols}

\tcap{Таблица 7 - Количество занятого населения в строительстве по атырауской области за 2018-2023 годы (тыс.чел).}
\begin{longtblr}[
  label = none,
  entry = none,
]{
  width = \linewidth,
  colspec = {Q[400]Q[83]Q[90]Q[90]Q[90]Q[90]Q[83]},
  cells = {c},
  hlines,
  vlines,
}
Показатели & 2018 & 2019 & 2020 & 2021 & 2022 & 2023\\
Занято в экономике, всего & 304 & 316,3 & 314,5 & 317,7 & 326,7 & 335\\
Строительство & 53,8 & 55,1 & 54,4 & 55,5 & 61 & 62,7
\end{longtblr}

\tcap{Таблица 8 - Объем выполненных строительных работ по формам собственности Атырауской области на 2018--2023 годы {[}9{]}.}
\begin{longtblr}[
  label = none,
  entry = none,
]{
  width = \linewidth,
  colspec = {Q[548]Q[62]Q[62]Q[62]Q[62]Q[67]Q[67]},
  row{1} = {c},
  cell{2}{2} = {c},
  cell{2}{3} = {c},
  cell{2}{4} = {c},
  cell{2}{5} = {c},
  cell{2}{6} = {c},
  cell{2}{7} = {c},
  cell{3}{2} = {c},
  cell{3}{3} = {c},
  cell{3}{4} = {c},
  cell{3}{5} = {c},
  cell{3}{6} = {c},
  cell{3}{7} = {c},
  cell{4}{2} = {c},
  cell{4}{3} = {c},
  cell{4}{4} = {c},
  cell{4}{5} = {c},
  cell{4}{6} = {c},
  cell{4}{7} = {c},
  cell{5}{2} = {c},
  cell{5}{3} = {c},
  cell{5}{4} = {c},
  cell{5}{5} = {c},
  cell{5}{6} = {c},
  cell{5}{7} = {c},
  hlines,
  vlines,
}
Показатели & 2018 & 2019 & 2020 & 2021 & 2022 & 2023\\
1. Объем выполненных строительных работ, млрд. тенге & 638 & 868 & 873 & 960 & 1 164 & 1 221\\
в том числе: государственная & 27 & - & - & - & - & 421\\
частная & 362 & 514 & 324 & 331 & 548 & 687\\
иностранная & 276 & 354 & 548 & 629 & 616 & 533
\end{longtblr}

\begin{multicols}{2}
Судя по структуре распределения заказов между участниками, строительный
рынок в регионе принадлежит частным казахстанским (56 \%) и иностранным
компаниям (44 \%).

Стоит отметить, что с 2018 года до 2021 года иностранные строительные
фирмы увеличили свое влияние на казахстанском рынке. В 2022--2023 годы
идет спад и в 2023 году они занимали долю лишь 44 \%.

Фактическая стоимость строительства введенных в эксплуатацию объектов по
Атырауской области в 2023 году составляет 3~278~713 417 тыс.тенге, в том
числе жилые здания -- 106049375 (3,2 \%), нежилые здания -- 2948078136
(90 \%).

Большая часть денег в строительном бизнесе формируется за счет
выполнения заказов на возведение нежилых зданий (90 \% от общего
объема).

В 2023 году средние фактические затраты на строительство 1 кв. метра
общей площади жилых домов в Казахстане в городской местности (без учета
жилых домов, построенных населением) составили 493 тыс.тенге,
увеличившись на 13\% за год. Самая высокая средняя стоимость за один
квадратный метр сложилась Атырауской области -- 387 тыс. тенге {[}10{]},
так как кроме воды, в Атырау ничего нет, ни ПГС, ни щебня, ни цемента.
Большая часть сырья, необходимого для строительства, поступает из
России. Газоблок доставляется из Ирана, Актюбинской области. Многие
товары поставляются из России -- железо, камень, электрические кабели,
стройматериалы, и нестабильность рубля тоже оказывает свое негативное
влияние.

{\bfseries Выводы.} Рассмотрев Планы развития региона и статистические
отчеты нами были определены главенствующие отрасли, выступающие некими
драйверами развития региональной экономики. Основными факторами,
влияющими на развитие приоритетных отраслей региона являются: инновации
и технологический прогресс, развитие человеческого капитала, поддержка
государства, рыночные потребности и социальные и экологические
требования.

Результаты проведенного исследования могут быть применены при
формировании эффективной системы устойчивого развития предприятий
инфраструктуры Атырауского региона, а также в прогнозировании видов
производственной инфраструктуры Республики Казахстан и её субъектов.
Вместе с тем выводы и рекомендации могут найти применение и в разработке
учебных программ, научно-методических пособий, а также в курсах
повышения квалификации персонала и руководителей государственного
управления региона.

\emph{{\bfseries Финансирование}. Данное исследование финансировалось
Комитетом науки Министерства науки и высшего образования Республики
Казахстан (грант № BR21882382).}
\end{multicols}

\begin{center}
{\bfseries Литература}
\end{center}

\begin{references}
1. Атырауская область - Статистика регионов РК {[}Электронный ресурс{]}.
URL:  Дата обращения: 20.01.2025.

2. Статистический ежегодник. Регионы Казахстана в 2022 г. Астана, 2023.
URL:
\href{https://stat.gov.kz/ru/publication/collections/?year=2022&name=17195&period=year-}{}
Дата обращения: 20.01.2025.

3. Статистический ежегодник Атырауской области за 2020 год. Атырау,
2021. URL:
\href{https://stat.gov.kz/ru/region/atyrau/collections/?year=2020&period=year&name=120885-}{}
Дата обращения: 20.01.2025.

4. Комплексный план социально-экономического развития Атырауской области
на 2021--2025 годы. {[}Электронный ресурс{]}. URL:
\href{https://adilet.zan.kz/rus/docs/P2100000337-}{}
Дата обращения: 20.01.2025.

5. Концепция развития МСП Республики Казахстан до 2030 года.
{[}Электронный ресурс{]}. URL:
\href{https://adilet.zan.kz/rus/docs/P2300001050/history}{}
-Дата обращения: 20.01.2025.

6. Краткий статистический ежегодник Департамента Бюро национальной
статистики Агентства по стратегическому планированию и реформам
Республики Казахстан по Атырауской области.2023. URL:
\href{https://stat.gov.kz/ru/region/atyrau/collections/?year=2022&period=year&name=52710-}{}
Дата обращения: 20.01.2025.

7. План развития Атырауской области на 2021--2025 годы. {[}Электронный
ресурс{]}. URL:
\href{https://adilet.zan.kz/rus/docs/P2100000337}{}
- Дата обращения: 20.01.2025

8. Концепция развития рынка труда Республики Казахстан на 2024--2029
годы. {[}Электронный ресурс{]}. URL:
\href{https://adilet.zan.kz/rus/docs/P2300001050/history-}{}
Дата обращения: 20.01.2025.

9. Программа реновации жилищного фонда города Атырау на 2022-2026 года.
URL:
\href{https://legalacts.egov.kz/npa/view?id=14132167-}{}
Дата обращения: 20.01.2025

10. Пояснительная записка к проекту «Об областном бюджете на 2024-2026
годы»~ Дата обращения: 20.01.2025

11. Национальный энергетический доклад \textsc{KAZENERGY 2023.}
\href{https://www.kazenergy.com/ru/operation/ned/2117/.-}{\textsc{}}
Дата обращения: 20.01.2025
\end{references}

\begin{center}
{\bfseries References}
\end{center}

\begin{references}
1. Atyrauskaya oblast'{} - Statistika regionov RK
{[}Elektronnyi resurs{]}. URL: \href{https://stat.gov.kz/ru/region/atyrau}{https://stat.gov.kz}.
-Data obrash\-cheniya: 20.01.2025.

2. Statisticheskii ezhegodnik. Regiony Kazakhstana v 2022 g. Astana,
2023. URL:
\href{https://stat.gov.kz/ru/publication/collections/?year=2022&name=17195&period=year}{https://stat.gov.kz}
- Data obrashcheniya: 20.01.2025.

3. Statisticheskii ezhegodnik Atyrauskoi oblasti za 2020 god. Atyrau,
2021. URL:
\href{https://stat.gov.kz/ru/region/atyrau/collections/?year=2020&period=year&name=120885}{https://stat.gov.kz}
- Data obrashcheniya: 20.01.2025.

4. Kompleksnyi plan sotsial' no-ekonomicheskogo razvitiya
Atyrauskoi oblasti na 2021--2025 gody. {[}Elek\-tronnyi resurs{]}. URL:
\href{https://adilet.zan.kz/rus/docs/P2100000337}{https://adilet.zan.kz}
-Data obrashcheniya: 20.01.2025.

5. Kontseptsiya razvitiya MSP Respubliki Kazakhstan do 2030 goda.
{[}Elektronnyi resurs{]}. URL:\\
\href{https://adilet.zan.kz/rus/docs/P2300001050/history}{https://adilet.zan.kz} - Data obrashcheniya:
20.01.2025.6. Kratkii statisticheskii ezhegodnik Departamenta Byuro
natsional' noi statistiki Agentstva po strategicheskomu
planirovaniyu i reformam Respubliki Kazakh\-stan po Atyrauskoi oblasti.
2023. URL:
\href{https://stat.gov.kz/ru/region/atyrau/collections/?year=2022&period=year&name=52710}{https://stat.gov.kz}
-Data obrashcheniya: 20.01.2025.

7. Plan razvitiya Atyrauskoi oblasti na 2021--2025 gody. {[}Elektronnyi
resurs{]}. URL: \href{https://adilet.zan.kz/rus/docs/P2100000337}{https://adilet.zan.kz} - Data
obrashcheniya: 20.01.2025.

8. Kontseptsiya razvitiya rynka truda Respubliki Kazakhstan na
2024--2029 gody. {[}Elektronnyi resurs{]}. URL:
\href{https://adilet.zan.kz/rus/docs/P2300001050/history}{https://adilet.zan.kz}
- Data obrashcheniya: 20.01.2025.

9. Programma renovatsii zhilishchnogo fonda goroda Atyrau na 2022-2026
goda. URL:\\
\href{https://legalacts.egov.kz/npa/view?id=14132167}{https://legalacts.egov.kz}
- data obrashcheniya: 20.01.2025.

10. Poyasnitel' naya zapiska k proektu «Ob oblastnom
byudzhete na 2024-2026 gody»
\href{https://www.gov.kz/memleket/entities/atyrau-economy/documents/details/597525?lang=ru}{https://www.gov.kz}
-. Data obrashcheniya:
20.01.2025.11. Natsional' nyi energeticheskii doklad KAZENERGY 2023.
\textsc{\href{https://www.kazenergy.com/ru/operation/ned/2117/.-}{}}
Data obrashcheniya: 20.01.2025.
\end{references}

\begin{authorinfo}
\emph{{\bfseries Сведения об авторах}}

Жалгасбаева А. - кандидат экономических наук, доцент, директор Института
повышения квалификации и дополнительного образования Атырауского
университета нефти и газа имени Сафи Утебаева, г. Атырау, Республика
Казахстан, e-mail: a.zhalgasbayeva@aogu.edu.kz; 

Буканова С. - магистр нефтехимии, PhD докторант, Атырауского
университета нефти и газа имени Сафи Утебаева, г. Атырау, Казахстан,
e-mail: s.bukanova@aogu.edu.kz;

Абдигалиева А. - магистр техники и технологии, старший преподаватель,
Атырауский университет нефти и газа им. Сафи Утебаева, г. Атырау,
Казахстан,
e-mail: a.abdigalieva@aogu.edu.kz

\emph{{\bfseries Information about the authors}}

Zhalgasbayeva А. - candidate of economic sciences, associate professor
of the KKSON, director of the Institute of advanced training and
additional education of the Atyrau oil and gas university named after
Safi Utebayev, Atyrau, Republic of Kazakhstan,
e-mail: a.zhalgasbayeva@aogu.edu.kz; 

Bukanova S. - master of petrochemistry, PhD student, senior lecturer at
the Institute of chemical engineering and ecology of Atyrau oil and gas
university named after Safi Utebayev, Atyrau, Republic of Kazakhstan,
e-mail: s.bukanova@aogu.edu.kz;

Abdigaliyeva A. - master of engineering and technology, senior lecturer
at faculty of Information Technology, Atyrau oil and gas university
named after Safi Utebayev, Atyrau, Republic of Kazakhstan,
e-mail: a.abdigalieva@aogu.edu.kz
\end{authorinfo}
