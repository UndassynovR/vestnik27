\id{ҒТАМР 06.35.31}{}

\begin{articleheader}
\sectionwithauthors{М.С. Искакова, А.К. Ибраева, Д.М. Акишева}{СТРАТЕГИЯЛЫҚ БАСҚАРУ ЕСЕБІН ҰЙЫМДАСТЫРУ}

{\bfseries
М.С. Искакова\textsuperscript{\envelope },
А.К. Ибраева,
Д.М. Акишева
}
\end{articleheader}

\begin{affiliation}
«Семей қаласының Шәкәрім атындағы университеті» КеАҚ, Семей, Қазақстан

\raggedright \textsuperscript{\envelope }Корреспондент-автор: mis0508@mail.ru
\end{affiliation}

Стратегиялық басқару есебінің маңызды міндеті ұйымның көрсеткіштерінің
орындалуын тұрақты бақылау және басшылықты шығындар тәртібі туралы
ақпаратпен қамтамасыз ету мақсатында стратегиялық шығындарды есепке алу
болып табылады.

Кәсіпорын қызметінің қаржылық нәтижелері стратегиялық есеп пен
бақылаудың негізгі объектісі болып табылады, оның оңтайлы қалыптасуы
кәсіпорынның миссиясын тиімді жүзеге асыруға, бәсекелестік жағдайын және
қаржылық тұрақтылығын нығайтуға ықпал етеді.

Қаржылық нәтижелерді тиімді бақылау және реттеу мақсатында стратегиялық
басқарушылық есеп жүйесінде стратегиялық бюджеттеудің, стратегиялық есеп
пен талдаудың, стратегиялық контроллингтің заманауи технологиялары
енгізілуі тиіс.

Ұйымның стратегиялық позицияларын нығайту теріс факторлар мен
тәуекелдердің әсерін жою үшін шаруашылықтың сыртқы және ішкі жағдайларын
кешенді бақылауды және реттеуді талап етеді. Стратегиялық менеджмент
жүйесіндегі бұл тәсіл компанияның стратегиясын іске асыру тиімділігінің
деңгейін, оның бәсекеге қабілеттілік деңгейін және қаржылық тұрақтылығын
арттыруға мүмкіндік береді.

Шаруашылық субъектісінің стратегиялық позицияларын нығайту теріс
факторлар мен тәуекелдердің әсерін жою үшін шаруашылықтың сыртқы және
ішкі жағдайларын кешенді бақылауды және реттеуді талап етеді.
Стратегиялық менеджмент жүйесіндегі бұл тәсіл компанияның стратегиясын
іске асыру тиімділігінің деңгейін, оның бәсекеге қабілеттілік деңгейін
және қаржылық тұрақтылықты арттыруға мүмкіндік береді.

Кәсіпорын қызметінің ең маңызды көрсеткіші оның қаржылық нәтижелері,
кірістер мен шығыстар арасындағы айырмашылық болып табылады.

Стратегиялық менеджмент жүйесіндегі қаржылық нәтижелер ұйымның қаржылық
жағдайының қалыптасуын және компанияның маңызды стратегиялық
сипаттамасының көлемін - бизнестің нарықтық құнын анықтайтын негізгі
факторды білдіреді.

{\bfseries Түйін сөздер:} стратегиялық шығындар, бюджеттеу, басқарушылық
шешімдер, талдау.

\begin{articleheader}
{\bfseries ОРГАНИЗАЦИЯ СТРАТЕГИЧЕСКОГО УПРАВЛЕНЧЕСКОГО УЧЕТА}

{\bfseries
М.С. Искакова\textsuperscript{\envelope },
А.К. Ибраева,
Д.М. Акишева
}
\end{articleheader}

\begin{articleheader}
НАО «Университет имени Шакарима города Семей», Семей, Казахстан,

e-mail: mis0508@mail.ru
\end{articleheader}

\vspace{-1em}
Важной задачей стратегического управленческого учета является ведение
стратегических затрат с целью постоянного контроля за выполнением
показателей деятельности организации и предоставления руководству
информации о динамике затрат.

Финансовые результаты деятельности предприятия являются основным
объектом стратегического учета и контроля, оптимальное формирование
которого способствует эффективной реализации миссии предприятия,
укреплению его конкурентных позиций и финансовой устойчивости.

Для эффективного контроля и регулирования финансовых результатов в
систему стратегического управленческого учета необходимо внедрять
современные технологии стратегического 

бюджетирования, стратегического
учета и анализа, стратегического контроллинга.

Укрепление стратегических позиций организации требует комплексного
мониторинга и регулирования внешних и внутренних условий экономики для
устранения влияния негативных факторов и рисков. Такой подход к системе
стратегического управления позволяет повысить уровень эффективности
реализации стратегии компании, уровень ее конкурентоспособности и
финансовую устойчивость.

Укрепление стратегических позиций хозяйствующего субъекта требует
комплексного контроля и регулирования внешних и внутренних условий
экономики для устранения влияния негативных факторов и рисков. Такой
подход к системе стратегического управления позволяет повысить уровень
эффективности реализации стратегии компании, уровень ее
конкурентоспособности и финансовой устойчивости.

Важнейшим показателем деятельности предприятия является его финансовый
результат --- разница между доходами и расходами.

Финансовые результаты в системе стратегического управления представляют
собой основной фактор, определяющий формирование финансового положения
организации и величину важной стратегической характеристики компании --
рыночной стоимости бизнеса.

{\bfseries Ключевые слова:} стратегические затраты, бюджетирование,
управленческие решения, анализ
\vspace{2em}
\begin{articleheader}
{\bfseries ORGANIZING STRATEGIC MANAGEMENT ACCOUNTING}

{\bfseries
M.S. Iskakova\textsuperscript{\envelope },
A.K. Ibraeva,
D.M. Akisheva
}
\end{articleheader}

\begin{affiliation}
NPJSC «Shakarim University of Semey», Semey, Kazakhstan,

e-mail: \href{mailto:mis0508@mail.ru}{}
\end{affiliation}

An important task of strategic management accounting is the accounting
of strategic costs for the purpose of continuous monitoring of the
organization' s performance indicators and providing
management with information on cost dynamics.

The financial results of an enterprise' s activities are
the main object of strategic accounting and control, the optimal
formation of which contributes to the effective implementation of the
enterprise' s mission, strengthening its competitive
positions and financial stability.

For effective control and regulation of financial results, it is
necessary to introduce modern technologies of strategic budgeting,
strategic accounting and analysis, and strategic controlling into the
strategic manage\-ment accounting system.

Strengthening the strategic positions of an organization requires
comprehensive monitoring and regulation of external and internal
economic conditions to eliminate the influence of negative factors and
risks. This approach to the strategic management system allows
increasing the level of efficiency of the company' s
strategy implementation, its level of competitiveness and financial
stability.

Strengthening the strategic positions of an economic entity requires
comprehensive control and regulation of external and internal economic
conditions to eliminate the influence of negative factors and risks.
This approach to the strategic management system allows increasing the
level of efficiency of the company' s strategy
implementation, its level of competitiveness and financial stability.

Financial results in the strategic management system are the main factor
determining the formation of the financial position of the organization
and the value of an important strategic characteristic of the company -
the market value of the business.

{\bfseries Keywords:} strategic costs, budgeting, management decisions,
analysis

\begin{multicols}{2}
{\bfseries Кіріспе.} Басқару есебі өз мәнінде есеп процесі мен ұйымды
басқару арасындағы байланыстырушы буын болып табылады. Ал ұйымды басқару
жүйесіне қатысты стратегия - ұйымның миссиясы мен мақсатына жетуге
арналған егжей-тегжейлі, кешенді жоспар.

Зерттеудің мақсаты - ұйымдарда стратегиялық басқару есебінің
маңыздылығын анықтап, басшылықты тұрақты түрде ақпараттармен қамтамасыз
ету мен оны өңдеу бойынша ұсыныстарды қарастыру.

Осы мақсатқа сәйкес бірнеше міндеттер қойылды:

- ұйымдарда әрекет етуші басқару есебінің теориялық негіздерін зерделеу;

- стратегиялық басқару есебінің басты функцияларын анықтау;

- ұйым стратегиясын іске асырғанда пайда болатын шығындарды бақылау;

- ұйымның бәсекеге қабілеттілігін арттыру мақсатында стратегиялық басқару
есебін ұйымдастырудың оңтайлы әдістемесін ұсыну.

{\bfseries Материалдар мен әдістер.} Стратегия - ұйымның миссиясын жүзеге
асыруға және оны жаңа қалыпқа айналдыруға бағытталған басқару
шешімдерінің жүйесі. Ұйымдастырушылық әлеуетті қалыптастыру және дамыту
бағыттары; бәсекелестік артықшылықтарға қол жеткізу үшін нарықтағы
қызметті жүзеге асыру әдістері осы стратегиялық есеп жүргізудің
басымдықтары болып табылады.

Ұйымда стратегиялық басқару есебін жетілдіру мәселелерін қарастыру
барысында зерттеудің мынадай әдістері қолданылды:

- стратегиялық басқару есебінің қажеттілігін негіздеуге бағытталған
зерттеу мақсаты мен міндеттерін анықтау кезінде абстрактты-логикалық;

- стратегиялық басқару есебін жетілдіру мәселелерін қарастыру және
оларды шешуге бағытталған ұсыныстарды өңдеу барысында монографиялық
әдіс қолданылды.

Зерттеу барысында стратегиялық басқару есебіне сипаттама беріп, ұйымдағы
қалыптасу жағдайына талдау жасау негізінде, оны жетілдіруге бағытталған
ұсыныстар жұмыстың ғылыми жаңалығы болып табылады.

{\bfseries Нәтижелер және талқылау.} Стратегиялық бюджеттеу, есепке алу
және бақылау үлгілерінің әртүрлілігін ескере отырып, стратегиялық есеп
объектісі ретінде қаржылық нәтижелерді көрсету әдістемесін әзірлеуге
жүйелі көзқарас жеткілікті түрде әзірленбеген.

Басқарудың стратегиялық есебінің қажеттілігін сипаттайтын мына
жағдайлардың болуы:

- бюджеттік жүйенің болмауы;

- ішкі есеп беру жүйесінің жетілмегендігі;

- қызметтің және өнімнің жекелеген түрлерінің табыстылығын талдау
жүйесінің болмауы;

- баға белгілеуге, өндірістік бағдарламаларға, инвестициялық жобаларды
бағалауға байланысты басқару шешімдерін талдау мен қабылдаудың
жетілмегендігі;

- шығындарды азайту және операциялық тиімділікті арттыру үшін
қызметкерлердің жауапкершілігі мен мотивациясының жеткіліксіз деңгейі;

- бақылау көрсеткіштері жүйесінің, оларды жоспарлаудың нормативтік
құқықтық актілерінің болмауы.

Сонымен қатар, стратегиялық басқару есебінің қажеттілігі келесі
себептермен түсіндіріледі:

- басқарушылық есептің жаңа әдістерінің пайда болуы;

- даму стратегиясын өзгерту қажеттілігін анықтау;

- операциялық және стратегиялық қызмет арасындағы байланыстың
қажеттілігі;

- сыртқы орта өзгерген кезде басқару шешімдерін қабылдау жылдамдығын
арттыру;

- талдау фокусын ішкі ортадан ұйым дамып жатқан немесе дамытуды көздеп
отырған сыртқы іскерлік ортаға ауыстыру {[}1{]}.

Стратегиялық шығындарды есепке алу жүйесі маңызды салаларды басқару
шешімдерін қабылдау үшін ақпаратпен қамтамасыз етуге бағытталған.
Стратегиялық басқару шешімдерінің түрлері тиісті стратегиялық
басқарушылық есеп ақпаратын талап етеді.

Қазіргі заманғы кәсіпорынның басшылығы үнемі проблемаға тап болады

стратегиялық басқару шешімдерін қабылдау процесіне әсер ететін оңтайлы
және нақты ақпаратты таңдау қажеттілігі, сондай-ақ бизнес жағдайларының
өзгеруі кезінде шығындардың ерекшеліктерін ескеру қажеттілігі. Осы
мәселелерге тиісті көңіл бөлу арқылы ғана белгілі бір ресурстары бар
және оларды жеткілікті түрде ұтымды пайдаланатын кәсіпорынның пайдалы
қызметіне қол жеткізуге болады {[}2{]}.

Стратегиялық басқару шешімдерін қабылдау процесіне әсер ететін оңтайлы
және нақты ақпаратты таңдау қажеттілігін, сондай-ақ бизнес жағдайларының
өзгеруі кезінде шығындардың ерекшеліктерін ескеру қажеттілігін ескере
отырып кәсіпорынның пайдалы қызметіне қол жеткізуге болады.
\end{multicols}

{\bfseries 1 - сурет. Ұйымдағы бухгалтерлік есептің құрамдастары}

\emph{Ескерту -- {[}3{]} әдебиет негізінде құрастырылған}

\begin{multicols}{2}
Сурет 1 -де көрсетілгендей ұйымдағы қаржылық, басқару, стратегиялық
басқару есептері бір-бірімен тығыз байланысты. Стратегиялық есеп --
қаржылық, салықтық және басқарушылық есеп деректеріне негізделген, ұзақ
мерзімді перспективаға бағытталған және сыртқы факторлардың әсерін
есепке алатын есеп жүйесі.

Стратегиялық басқару есебі ұйымды басқару жүйесі элементтерінің бірі
ретінде стратегиялық мақсаттар үшін пайдаланылатын белгілі бір уақыт
кезеңінде алынған бухгалтерлік есепті, жоспарлау мен болжауды,
нормативтік және басқа ақпаратты жинау, өңдеу және ұсыну тәртібін
реттейді.

Стратегиялық басқару есебі мен стратегиялық талдау арасындағы байланыс
алынған бухгалтерлік ақпаратты стратегиялық бағалау үшін талдамалық
ақпаратқа айналдырып, трансформациялауға мүмкіндік береді.

Кәсіпорынның стратегиялық мақсаттарын жүзеге асыратын бюджеттеу,
жоспарлау, талдау, бақылау және реттеуге басқару есебінің ақпараттық
қамтамасыз етілуі тікелей әсер етеді. Ірі компанияларда жүргізілетін
қаржылық талдаудың негізін құрайтын бухгалтерлік есеп деректерінсіз
стратегияларды қалыптастыру мүмкін емес. Ұзақ мерзімді стратегиялық
шешімдер ұйымның болашағын және оның одан әрі даму болашағын
анықтайды.Тиімді басқару шешімдерін қабылдау, талдау үшін жедел және
басқарушылық сенімді ақпаратты пайдалану керек {[}4{]}.

Цифрлық технологиялардың қарқынды дамуы мен ақпараттық жүйелерді кеңінен
қолданудың арқасында стратегиялық басқару есебінде қажетті ақпаратты
өңдеу процесі жеделдетіледі және құжат айналымының еңбек сыйымдылығы
төмендейді.

Стратегиялық басқару есебінде ұйым жұмыс істейтін іскерлік ортаның
талдауы жүргізіліп, компанияның нарықтағы стратегиялық позициясы
ескерілуі,өндірілген өнімді саралау мүмкіндігі қарастырылуы және
функциялардың құн тізбегі құрылуы қажет. Құн тізбегі кәсіпорынның
стратегиялық қызметі туралы түсінік береді және шығындар қозғалысын
бақылауға мүмкіндік беріп, шығындарды азайтудың әлеуетті көздерін
анықтауға және жеке операцияларды оңтайландыруға мүмкіндік береді,
осылайша оның нарықтағы бәсекелестерінен артықшылықтарға ие болады.
Мұның бәрі компанияның нарықта дұрыс орналасуына, оның тұрақты
бәсекелестік артықшылығын сақтауға және дамытуға, ұзақ уақыт бойы жақсы
қаржылық-экономикалық нәтижелерді сақтауға әсері бар.

Әлемдік экономиканың жаһандануы компаниялар үшін олардың
институционалдық ортамен өзара әрекеттесуінің әртүрлі салаларына әсер
ететін жаңа бәсекелестік артықшылықтар тудырады. Егер бұрын бәсекенің
мақсаты компаниялар арасындағы пайданы көбейту үшін күрес болса, қазір
әр тұтынушы үшін күрес, нарықтағы позицияны жоғарылату {[}5{]}.

Басқарудың стратегиялық есебі кәсіпорынның ұйым шеңберінен шығып,
нарықтық қатынастардың басқа субъектілерімен өзара әрекеттесу
қажеттілігімен байланысты. Шығындарды азайту --компанияның өнімділікті
арттыру, ысырапты жою және шығындарды қатаң бақылау арқылы
бәсекелестеріне қарағанда төмен шығындарға ие болу мүмкіндігі. Өз
саласында шығындарды азайту, бәсекелестік артықшылық беретін, бірегей
өнімдер мен қызметтерге қарағанда төмен сату бағасы ұсыну - нарықтағы
көптеген ұйымдар ұмтылатын әрекеттер. Кәсіпорынды басқарудың стратегиясы
мен нәтижелері туралы әртүрлі көзқарастар, компанияның стратегиялық
мәселелерді шешуі туралы әртүрлі идеялар бар. Тиімді нарыққа
бағдарланған корпоративтік менеджментті құру түбегейлі жаңа құралдарды,
демек, стратегиялық басқару есебінде қолданылатын заманауи әдістерді
үнемі жетілдіруді және енгізуді талап етеді {[}6{]}.

Жаһандық бәсекелестіктің, тұтынушылардың өнім мен қызметтерге деген
талаптарының өсуі стратегиялық шығындарды басқарудың жаңа тәсілдерін
пайдалану компанияға тұрақты бәсекелестік артықшылыққа және жоғары
қаржылық және нарықтық көрсеткіштерге қол жеткізуге мүмкіндік береді.
\end{multicols}

{\bfseries 2 - сурет. Стратегиялық басқару үшін ақпарат көздері}

\emph{Ескерту -- {[}7{]} әдебиет негізінде құрастырылған}

\begin{multicols}{2}
2 -Суретте көрсетілгендей басқарудың стратегиялық есебін ұйымдастыру
және оның әр бөліммен байланыс жүйелілігі мен күрделілігі барлық
деңгейлерде стратегиялық мақсаттарды анықтайды.

Стратегиялық басқару шешімдерін қабылдауға арналған ақпарат
кәсіпорындағы өндіріс процесі мен оның нәтижелері туралы мәліметтерді
қамтиды. Ұзақ мерзімді перспективада қажетті ақпаратпен қамтамасыз ете
отырып, ұйымның жалпы бәсекелестік стратегиясын қолдауға бағытталуы
керек.

Стратегиялық талдаудың негізгі әдістері мен жүйелеріне мыналар жатады:

Стратегиялық шығындарды басқару --компанияның ресурстық әлеуетін сыртқы
ортаның мүмкіндіктері мен қауіптерімен салыстыру негізінде стратегиялық
шығындар туралы шешімдерді қабылдау және жүзеге асыру процесі.

Стратегиялық бизнес бөлімдердің тұжырымы - қызметтің белгілі бір түріне
жауапты немесе нақты функцияларды орындайтын компания бөлімшелерінің
анықтамасы.

STEP талдау кәсіпорынның қызметіне әсер ететін әлеуметтік (S),
технологиялық (Т), экономикалық (Е) және саяси (P) факторларды талдау
құралы болып табылады. Талдаудың бұл түрі сыртқы орта деректері
негізінде нарықта не болып жатқанын бағалауға мүмкіндік береді.

SWOT талдау --компанияның күшті және әлсіз жақтарын анықтайтын талдау
әдісі, ол артықшылықтарды (Strength), кемшіліктерді (Weakness), сыртқы
ортаның қолайлы факторларын (Opportunities) және сыртқы ортаның
қауіптерін (Troubles) талдаудан тұрады.

Портердің құн тізбегі моделі -- бұл құн өндірілетін және шығындар пайда
болатын стратегиялық өзара байланысты тоғыз қызметтің тізбегі. Олардың
ішінде бастапқы (жеткізу, инвентарлық қорлар, материалдар, өндіріс,
сақтау және жылжыту, сату және маркетинг, дилерлік қолдау және сатудан
кейінгі қызмет көрсету) және қосалқы (инфрақұрылым, жоспарлау, қаржы,
ақпараттық және заң қызметтері; технология, ғылыми-зерттеу, өнімді
әзірлеу, дизайн, адам ресурстарын дамыту) қызмет түрлері бар.

Қазіргі жоғары бәсекелестік жағдайында стратегиялық шешімдерді
ақпараттық-талдамалық қамтамасыз ету үшін басқарушылық есептің жаңа
әдістері қажет. Бұл әдістердің комбинациясы әдетте стратегиялық басқару
есебі деп аталады. Оның басты ерекшелігі -- ұйым қызметінің сыртқы
факторларын талдауға және оның стратегиялық жоспарына бағытталған
{[}8{]}.

Біздің ойымызша, стратегиялық басқару есебінің қажеттілігін келесі
себептермен түсіндіруге болады:

- талдау фокусын ішкі ортадан ұйымды жан-жақты дамытуды көздеп отырған
сыртқы іскерлік ортаға ауыстыру;

- сыртқы іскерлік ортаның өзгеруіне байланысты басқару шешімдерін
қабылдау жылдамдығының артуы;

- кәсіпорынның ағымдағы және стратегиялық қызметі арасындағы мұқият
ойластырылған қарым-қатынастың қажеттілігі, өйткені стратегиялық басқару
есебі негізгі мақсаттарға қол жеткізуді нақты көрсеткіштер жүйесінде
білдіреді;

- кәсіпорынның даму стратегиясын өзгерту қажеттілігін анықтау;

- басқарушылық есептің жаңа әдістерінің пайда болуы.

Стратегиялық басқарушылық есептің нарықтық бағыттылығы сыртқы ортаның
бәсекелестік позициясы мен мүмкіндіктері туралы ақпаратты жүйелеуге,
жинауға және өңдеуге қайта бағдарлануына ықпал етеді. Табысты
стратегияның негізі - тұрақты бәсекелестік артықшылықтарды жасау және
күтпеген жағдайларға, күшті бәсекелестікке, одан туындайтын ішкі
мәселелерге қарамастан табысты жұмыс істей алатын кәсіпорын құру.

Стратегия бірнеше қажетті элементтерді қамтиды. Қазіргі уақытта
стратегиялық басқарушылық есеп концепциясын бухгалтерлік және талдамалық
тәсіл ретінде ғана емес, бизнесті одан әрі кеңейту және дамыту үшін
жоғары сапалы және икемді басқарудың ажырамас элементі ретінде қарастыру
керек {[}9{]}.

{\bfseries Қорытынды.} Стратегиялық басқару есебі күрделі сала болып
табылады, өйткені

компания басшылығы көптеген дәстүрлі және жаңа есеп әдістерін түсінуі
керек. Стратегиялық есепті жүзеге асыру кезінде компания басшылығы тап
болуы мүмкін қиындықтар:

- басқарушылық есеп жүйесін автоматтандырудың төмен деңгейі;

- бухгалтерлік есеп пен жедел басқару шешімдерінің өзара байланысының
нысандары мен көрсеткіштерін әзірлеудің күрделілігі;

- стратегиялық мақсаттар мен міндеттерді жүзеге асыру үшін ақпаратты
таңдаудың еңбек сыйымдылығы.

Жүргізілген зерттеулер негізінде, стратегиялық басқару есебі ұйымның
толыққанды қызметі үшін маңыздылығы анықталды:

- стратегиялық есеп пен талдау ұйымның бәсекеге қабілеттілігін
арттырудың құралы болып табылады;

- бизнес жүргізу барысында есеп жүйесінің тиімділігі артады;

- стратегиялық есеп пен талдаудың әдістері мен тәсілдерін қолдану
тиімділігі ұйымның іскерлік қызметте мақсаттарына қол жеткізуге
мүмкіндік береді;

- ұйымның стратегиялық бағыттарының ақпараттық негізі болып табылады.

Стратегиялық есеп басқару есебінің маңызды бөлігі, ол кәсіпорын
басшылығына компанияның дамуына әсер ететін ұзақ мерзімді шешімдер
қабылдауға мүмкіндік береді. Кәсіпкерге өз компаниясының жағдайын және
дамуын генерациялау үшін ресурстарды қалай дұрыс бөлуге болатынын түсіну
үшін де осы стратегиялық есептің маңызы ерекше. Стратегиялық есеп әрбір
бизнес сегментінің қаржылық деректерін анықтайды және әр саланы дамытуға
қанша қаражат жұмсалатынын, болжамды кіріс мөлшерін көрсетеді {[}10{]}.

Осы деректерге сүйене отырып, басшылық рентабельді емес аймақтарды жабу,
жарнамалық бюджетті қайта бөлу және кадрлық өзгерістер туралы шешім
қабылдай алады. Стратегиялық басқару есебі күрделі сала, өйткені
кәсіпорын басшылығы көптеген дәстүрлі және жаңа есеп әдістерін түсінуі
керек.
\end{multicols}

\begin{center}
{\bfseries Әдебиеттер}
\end{center}

\begin{references}
1. Глущенко А.В., Яркова И.В., Стратегический учет: учебник и практикум
для бакалавриата, специалитета и магистратуры. -М.: Издательство
Юрайт, 2017. - 240 с. ISBN 978-5-534-05061-5.

2. Dixon R., \& Smith D., Strategic management accounting // Omega.
-1993. -Vol.21(6). -P.605-618. DOI
\href{https://doi.org/10.1016/0305-0483(93)90003-4}{10.1016/0305-0483(93)90003-4}.

3. Попова Л.А.,Каренова Г.С.Управленческий учет -1 Учебник -Караганда:
2016. - 272 с. ISBN 978-601-80533-9-9.

4. Тайгашинова К.Т. Углубленный управленческий учет: учебник.- Алматы:
Экономика, 2014. - 184 с. ISBN 978-601-225-625-3.

5. Нургазина Ж.К. Управленческий учет: Учебник -- Астана, 2014. -411с.
ISBN: 978-601-289-126-3.

6. Алданиязов К.Н. Управленческий учет и анализ: Учебное пособие. -
Алматы: Нур-пресс, 2008. -- 368 с. - ISBN 9965-813-32-9.

7. Кузьмина М.С., Акимова Б.Ж. Управление затратами предприятия
(организации): учебное пособие. -- М.: КНОРУС, 2015. - 310 с. ISBN
978-5-406-10524-5.

8. Тайгашинова К.Т. Управленческий учет - Алматы: ТОО «Издательство LEM»,
2010. - 350 с. - ISBN 978-601-239-117-6.

9. Виханский, О.С. Стратегическое управление: Учебник. -- 2-е изд.,
перераб. и доп. - М.: Гардарика, 1999. -- 296 с. ISBN 5-8297-0021-2.

10. Казакова Н.А. Современный стратегический анализ: учебник и практикум -
М. Издательство Юрайт, 2016. - 502 с. ISBN 978-5-9916-6785-2.
\end{references}

\begin{center}
{\bfseries References}
\end{center}

\begin{references}
1. Glushhenko A.V., Jarkova I.V., Strategicheskij uchet: uchebnik i
praktikum dlja bakalavriata, specialit\-eta i magistratury. -M.:
Izdatel' stvo Jurajt, 2017. - 240 s. ISBN
978-5-534-05061-5 {[}in Russian{]}

2. Dixon R., \& Smith D., Strategic management accounting // Omega.
-1993. -Vol.21(6). -P.605-618. DOI 10.1016/0305-0483(93)90003-4 {[}in
Russian{]}

3. Popova L.A.,Karenova G.S.Upravlencheskiy uchet -1 Uchebnik -
Karaganda: 2016. -- 272 s. ISBN 978-601-80533-9-9

4. Tajgashinova K.T. Uglublennyj upravlencheskij uchet: uchebnik.-
Almaty: Jekonomika, 2014. -- 184 s. ISBN 978-601-225-625-3

5. Nurgazina Zh.K. Upravlencheskij uchet: Uchebnik - Astana, 2014.
-411s. ISBN: 978-601-289-126-3. {[}in Russian{]}

6. Aldanijazov K.N. Upravlencheskij uchet i analiz: Uchebnoe posobie. -
Almaty: Nur-press, 2008. -- 368 s. ISBN 9965-813-32-9

7. Kuz' mina M.S., Akimova B.Zh. Upravlenie zatratami
predprijatija (organizacii): uchebnoe posobie. - M.: KNORUS, 2015. --
310 s. ISBN 978-5-406-10524-5

8. Taygashinova K.T. Upravlencheskiy uchet -- Almaty: TOO
«Izdatel' stvo LEM», 2010. - 350 s. - ISBN
978-601-239-117-6

9. Vihanskij O.S. Strategicheskoe upravlenie: Uchebnik. -- 2-e izd.,
pererab. i dop. - M.: Gardarika, 1999. -- 296 s. ISBN 5-8297-0021-2
{[}in Russian{]}

10. Kazakova N.A. Sovremennyy strategicheskiy analiz: uchebnik i
praktikum -- M. Izdatel' stvo Yurayt, 2016. - 502 s. ISBN
978-5-9916-6785-2
\end{references}

\begin{authorinfo}
\emph{{\bfseries Авторлар туралы мәліметтер}}

Искакова М.С.- PhD, «Семей қаласының Шәкәрім атындағы университеті»
КеАҚ, Семей, Қазақстан, email: \\mis0508@mail.ru;

Ибраева А.К. -- экономика ғылымдарының кандидаты, «Семей қаласының
Шәкәрім атындағы университеті» КеАҚ, Семей, Қазақстан, email: Zeretai@mail.ru;

Акишева Д.М. -PhD, «Семей қаласының Шәкәрім атындағы университеті» КеАҚ,
Семей, Қазақстан, email: \\Dana\_m@mail.ru

\emph{{\bfseries Information about the authors}}

Iskakova M.S. - PhD, Shakarim University of Semey, Kazakhstan, email:
mis0508@mail.ru;

Ibraeva A.K. - candidate of economic sciences Shakarim University of
Semey, Kazakhstan, email: Zeretai@mail.ru;

Akisheva D.M.- - PhD, Shakarim University of Semey, Kazakhstan, email:
Dana\_m@mail.ru
\end{authorinfo}
