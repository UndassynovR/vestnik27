\id{МРНТИ \href{https://grnti.ru/?p1=06&p2=71&p3=02}{06.71.02}, \href{https://grnti.ru/?p1=06&p2=61&p3=23}{06.61.23}}{}

\begin{articleheader}
\sectionwithauthors{Б.У. Сыздыкбаева}{ПРОЕКТИРОВАНИЕ И ПОВЫШЕНИЕ ЭФФЕКТИВНОСТИ РАЗМЕЩЕНИЯ
ЛОГИСТИЧЕСКОЙ ИНФРАСТРУКТУРЫ ДЛЯ ХРАНЕНИЯ, РАСПРЕДЕЛЕНИЯ И ТОРГОВЛИ
АГРОПРОДОВОЛЬСТВЕННОЙ ПРОДУКЦИЕЙ}

{\bfseries Б.У. Сыздыкбаева}
\end{articleheader}

\begin{affiliation}
Евразийский национальный университет им. Л.Н. Гумилева, Астана, Казахстан

\raggedright \textsuperscript{\envelope } Корреспондент-автор: \href{mailto:bakyt_syzdykbaeva@mail.ru}{\nolinkurl{bakyt\_syzdykbaeva@mail.ru}}
\end{affiliation}

В статье рассматриваются результаты исследования проектирования
размещения логистической инфраструктуры агропродовольственной продукции
с применением собственного методологического подхода к формированию и
определению потенциальных мест ее размещения: складских хранилищ,
оптово-распределительных центров по хранению, сбыту и торговле
агропродовольственной продукцией на территории страны, а также
предполагаемых зон их обслуживания, способствующих повышению
эффективности их функционирования.

Исследования проведены с использованием методов стандартизации
показателей, кластерного, корреляционно-регрессионного, факторного
анализа, методов центра тяжести, ранжирования. В статье показана
реализация предложенного подхода на примере использования статистических
данных 204 объектов - административных районов и городов Казахстана.

На первом этапе исследования определены потенциальные места размещения
распределительных центров сельскохозяйственной продукции на основе
предложенных критериев. На основе факторного анализа сформированы новые
факторы с целью определения критериев выбора мест размещения
логистической инфраструктуры. Кластерный анализ позволил
дифференцировать территории и определить характеристики логистической
инфраструктуры для их размещения, в зависимости от их специализации.

На втором этапе уточнены потенциальные места размещения логистической
инфраструктуры и географическая привязка объектов на территории
административно-территориальных единиц.

На третьем этапе, с помощью весовых коэффициентов и интегрального
показателя привлекательности расположения логистических инфраструктур,
определена зона обслуживания логистической инфраструктуры хранения,
сбыта и торговли в соответствии с потенциальными местами размещения на
территории регионов.

Предлагаемый подход, в отличие от существующих ранее, учитывает
особенности функционирования инфраструктуры хранения, сбыта и торговли,
взаимоувязывает эти объекты с транспортной и складской доступностью на
территории регионов.

{\bfseries Ключевые слова:} логистическая инфраструктура, критерий выбора и
размещения, оптово - распределительный центр, транспортная и складская
доступность, транспортная и складская сеть, факторный и кластерный
анализ, хранение, распределение и торговля агропродовольственной
продукции, скоропортящаяся продукция

\begin{articleheader}
{\bfseries АУЫЛ ШАРУАШЫЛЫҒЫ ӨНІМДЕРІН САҚТАУ, ТАРАТУ ЖӘНЕ САТУ ҮШІН
ЛОГИСТИКАЛЫҚ ИНФРАҚҰРЫЛЫМДЫ ОРНАЛАСТЫРУДЫ ЖОБАЛАУ ЖӘНЕ ТИІМДІЛІГІН
АРТТЫРУ}

{\bfseries Б.Ұ. Сыздықбаева}
\end{articleheader}

\begin{affiliation}
Л.Н. Гумилев атындағы Еуразия ұлттық университеті, Астана, Қазақстан,

e-mail: \href{mailto:bakyt_syzdykbaeva@mail.ru}{}
\end{affiliation}

Мақалада өзіндік әдіснамалық тәсілді қолдана отырып ел аумағында
агроазық-түлік өнімдерінің логистикалық инфрақұрылымын: сақтау
қоймаларын, агроазық-түлік өнімдерін сақтау, өткізу және сату жөніндегі
көтерме-тарату орталықтарын орналастыруды жобалауды зерттеу және оны
орналастырудың әлеуетті орындарын

анықтау нәтижелері, сондай-ақ олардың жұмыс істеу тиімділігін арттыруға
ықпал ететін қызмет көрсету аймақтары қарастырылады.

Зерттеулер индикаторларды стандарттау әдістерін, кластерлік,
корреляциялық-регрессиялық, факторлық талдауды, ауырлық центрінің
әдістерін, рейтингтерді қолдана отырып жүргізілді. Мақалада, ұсынылған
тәсілдің жүзеге асырылуы Қазақстанның 204 әкімшілік аудандары мен
қалаларының статистикалық деректерін пайдалану мысалында көрсетілген.

Зерттеудің бірінші кезеңінде ұсынылған критерийлер негізінде ауыл
шаруашылығы өнімдерін тарату орталықтарын орналастырудың әлеуетті
орындары анықталды. Логистикалық инфрақұрылымды орналастыру орындарын
таңдау критерийлерін анықтау мақсатында факторлық талдау жасалынып жаңа
факторлар қалыптастырылды. Кластерлік талдау аумақтарды саралауға және
олардың мамандануына байланысты оларды орналастыру үшін логистикалық
инфрақұрылымның сипаттамаларын анықтауға мүмкіндік берді.

Екінші кезеңде әкімшілік-аумақтық бірліктер аумағындағы объектілердің
логистикалық \\инфрақұрылымын орналастырудың және географиялық
байланыстырудың әлеуетті орындары нақтыланды.

Үшінші кезеңде салмақ коэффициенттері мен логистикалық
инфрақұрылымдардың орналасу тартымдылығының интегралды көрсеткіші
көмегімен өңірлердің аумағында әлеуетті орналастыру орындарына сәйкес
сақтаудың, өткізудің және сауданың логистикалық инфрақұрылымына қызмет
көрсету аймағы айқындалды.

Ұсынылған тәсіл, бұрынғылардан айырмашылығы, сақтау, сату және сауда
инфрақұрылымының жұмыс істеу ерекшеліктерін ескереді, бұл объектілерді
аймақтардағы көліктік және қоймалық қолжетімділікпен байланыстырады.

{\bfseries Түйін сөздер:} логистикалық инфрақұрылым, таңдау және
орналастыру критерийі, көтерме-тарату орталығы, көліктік және қоймалық
қолжетімділік, көлік және қойма желісі, факторлық және кластерлік
талдау, агроазық-түлік өнімдерін сақтау, тарату және сату, тез бұзылатын
өнім

\begin{articleheader}
{\bfseries DESIGNING AND IMPROVING THE EFFICIENCY OF LOGISTICS
INFRASTRUCTURE PLACEMENT FOR STORAGE, DISTRIBUTION AND TRADE OF
AGRI-FOOD PRODUCTS}

{\bfseries B.U. Syzdykbayeva}
\end{articleheader}

\begin{affiliation}
L.N. Gumilyov Eurasian National University, Astana, Kazakhstan,

e-mail: \href{mailto:bakyt_syzdykbaeva@mail.ru}{\nolinkurl{bakyt\_syzdykbaeva@mail.ru}}
\end{affiliation}

The article considers the results of a study of the design of the
logistics infrastructure of agri-food products placement using our own
methodological approach to the formation and determination of potential
locations of its placement: warehouses, wholesale distribution centers
for storage, marketing and trade of agri-food products in the country,
as well as their intended service areas that contribute to improving the
efficiency of their functioning.

The research was carried out using methods for standardizing indicators,
cluster analysis, correlation and regression analysis, factor analysis,
center of gravity methods and ranking. The article shows the
implementation of the proposed approach on the example of using
statistical data from 204 objects - administrative districts and cities
of Kazakhstan.

At the first stage of the study, potential locations of agricultural
product distribution centers were determined based on the proposed
criteria. Based on the factor analysis, new factors are formed to
determine the criteria for selecting locations for logistics
infrastructure. Cluster analysis made it possible to \\differentiate
territories and determine the characteristics of the logistics
infrastructure for their placement, depending on their specialization.

At the second stage, the potential locations of logistics infrastructure
and geographical reference of objects on the territory of administrative
divisions are clarified.

At the third stage, using weighting coefficients and an integral
indicator of the attractiveness of the location of logistics
infrastructures, the service area of the logistics infrastructure of
storage, sales and trade was determined in accordance with potential
locations on the territory of the regions.

The proposed approach, as opposed to previous ones, takes into account
the specifics of the functioning of the storage, sales and trade
infrastructure, and connects these objects with transport and warehouse
availability in the regions.

{\bfseries Keywords:} logistics infrastructure, selection and placement
criteria, wholesale distribution center, \\transport and warehouse
accessibility, transport and warehouse network, factor and cluster
analysis, storage, distribution and trade of agri-food products,
perishable products

\begin{multicols}{2}
{\bfseries Введение.} В современных условиях актуальной для аграрного
сектора является не только задача увеличения объемов производства
продукции, но и проблема ее хранения, реализации и доведения до
конечного потребителя.

Проблема сбыта сельскохозяйственной продукции остро стоит перед
аграриями, так как продукция зачастую реализуется без учета региональной
и мировой рыночной конъюнктуры, что позволяет многочисленным
посредническим структурам скупать сельскохозяйственную продукцию по
заниженным ценам, получать значительную прибыль на ее перепродаже и
приводит к снижению конкурентоспособности отечественной продукции по
сравнению с зарубежными аналогами.

Для решения данной проблемы, в большинстве стран Европы оптовая торговля
всей скоропортящейся продукции организована через оптовые
продовольственные рынки. Примерами успешных товаропроводящих сетей на
базе сетей оптово-распределительных центров (ОРЦ) являются оптовый
продовольственный рынок «Rungis» (Франция)
(\url{http://www.rungisinternational.com/}), сеть
оптовых продовольственных рынков «Mercasa» (Испания)
(\url{http://www.mercasa.es/}) и оптовый продовольственный
рынок «Bronisze» (Польша)
(\href{https://www.bronisze.com.pl/ru}{https://www.bronisze.com.pl}).

Актуальность развития сети ОРЦ для сбыта сельскохозяйственной продукции
в Казахстане обусловлена целым рядом проблем, решение которых
стратегически важно для продовольственной безопасности Казахстана.

Во-первых, большая территория страны и низкая плотность населения,
неравномерное размещение основных производителей сельскохозяйственной
продукции и потребителей по регионам являются причиной высоких затрат на
доставку и высоких потерь продукции в Казахстане. Степень концентрации
(размещения) производителей продукции агропромышленного комплекса (АПК)
и объектов торговли в регионах неравномерная - варьируется от низкой
плотности до высокой плотности размещения.

Более 75\% продукции сельского хозяйства перевозится автомобильным
транспортом. Транспортировка на дальние расстояния также является
причиной высоких затрат и потерь продукции.

Несмотря на различные меры государственной поддержки, имеется большой
дефицит логистической инфраструктуры по хранению, упаковке и
транспортировке, современных форматов торговли, что приводит к большим
потерям продукции, которые составляют в плодоовощном хозяйстве до 40\%,
в животноводстве - до 20-25\%. Особенно это проявляется в деятельности
мелких сельскохозяйственных производителей.

Во-вторых, быстрый рост и расширение торговых сетей по стране ведет к
росту импорта продукции, по причине того, что большая часть продукции
мелких сельхозпроизводителей и их товарный вид не удовлетворяют
требованиям торговых сетей. Отсутствуют устойчивые рынки сбыта.
Отсутствие ОРЦ, основной задачей которых является консолидация,
обработка и переработка, упаковка, оптовая и мелкооптовая продажа
продукции, ограничивает возможности местных оптовых и розничных торговых
рынков работать с крупными торговыми сетями.

Отсутствует система современных оптовых рынков и ОРЦ, способных
оптимально аккумулировать и распределять продукцию отечественных
товаропроизводителей, прежде всего мелких производителей продукций АПК.
Структура сельскохозяйственных производителей весьма неоднородная.
Основную их массу (около 85\%) составляют мелкие домашние хозяйства,
которые производят большую часть всей сельскохозяйственной продукции и
продукции растениеводства в стране.

В настоящее время распределение товаров и услуг в целом по
товаропроводящей цепи, по нашему мнению, нерационально и носит случайный
характер.

Для решения данной проблемы в 2020 году в Казахстане принято решение о
строительстве сети ОРЦ {[}1{]}.

Очевидно, что решение проблемы оптимального размещения и зоны
обслуживания логистических инфраструктур по хранению, сбыту и торговле
агропромышленной продукции по территории страны и их эффективной работы
является актуальной для Казахстана.

Несмотря на многочисленные научные разработки по выбору места размещения
распределительных центров (РЦ) для сбыта агропродовольственной продукции
и их проектирования, а также оценки эффективности их деятельности, можно
констатировать, что имеющиеся решения научной проблемы в области
организационно-методического обеспечения формирования логистической
инфраструктуры требуют более обстоятельного изучения, в особенности,
применительно к сельскохозяйственной продукции.

Это обусловлено, во-первых, отраслевой спецификой и существенным
отличием сельскохозяйственной продукции от продукции других видов
промышленного производства. Во-вторых, существенными недостатками в
методическом инструментарии -- дефицитом современных методик,
базирующихся на научно обоснованных критериях оценки рационального
размещения логистических объектов, что и определило выбор цели и задачи
исследования. При большой территории и низкой плотности населения и
размещения инфраструктуры это задача требует своего научно-обоснованного
решения с учетом различных факторов, которые влияют на выбор и
эффективность функционирования логистической инфраструктуры.

Цель исследования - разработка организационных решений по
формированию и эффективному размещению логистической инфраструктуры по
хранению, сбыту и торговле агропродовольственной продукции, в частности,
складских хранилищ и ОРЦ.

В настоящем исследовании основное внимание сосредоточено на объектах
логистической инфраструктуры, относящихся к группе посреднических,
распределительных и торговых организаций: складские хранилища
сельскохозяйственной продукции (зерно-, овоще-, фрукто-, ягодо-,
картофелехранилища) и ОРЦ. Поскольку эти организации оказывают
непосредственное влияние на процесс товарообращения продовольственной
продукции, то есть хранения, сбыта и торговли.

Решение проблемы рационального размещения ОРЦ дает возможность
реализовать проекты по развитию АПК РК на 2021-2025гг. {[}2{]} и
торговли на 2021-2025гг. {[}1{]}, определить долгосрочные инвестиции в
развитие ОРЦ по регионам Казахстана.

{\bfseries Материалы и методы.} Важным структурным компонентом
распределительной системы является логистическая инфраструктура,
обеспечивающая хранение, распределение и торговлю продукции. Роль
логистической инфраструктуры заключается в оптимальном размещении и
обеспечении эффективной доставки грузов при скоординированном
взаимодействии множества структур, элементов и звеньев логистической
инфраструктуры: сервисной, торговой, складской и транспортной сети
{[}3{]}.

Проблема выбора оптимального места размещения складов существует
достаточно давно, и для ее решения предложено множество моделей
{[}4-7{]}. Проектирование и размещение распределительных мощностей и
потребителей изучено во многих зарубежных трудах. Данный вопрос решается
путем разбиения территории на участки, на которых располагаются
распределительные пункты {[}8{]}, путем моделирования спроса на
продукцию, плотности распределения населения {[}9{]}, учета
географических {[}10{]}, социально-экономических и инфраструктурных
факторов {[}11, 12{]}.

Многие компании сталкиваются со стратегическим решением относительно
количества РЦ, их местоположения и клиентов, которых они обслуживают
{[}13{]}. Одной из целей компании, принимающей это решение, является
поддержание приемлемого уровня обслуживания при минимизации постоянных
затрат на эксплуатацию РЦ, затрат на хранение запасов в РЦ и
транспортных расходов между заводами и РЦ, а также РЦ и клиентами.

Модель поставки сельскохозяйственной продукции состоит из функции
распределения, хранения, переработки, сертификации и мониторинга
объектов инфраструктуры, информационного и транспортного обеспечения
{[}14{]}. Актуальным является определение оптимального,
научно-обоснованного, стандартизированного и практичного местоположения
логистической инфраструктуры. В предыдущих исследованиях установлены
требования к оценке местоположения логистической инфраструктуры с точки
зрения условий товародвижения {[}15{]}, законов и политики {[}16{]},
ресурсов {[}17{]}, бизнес-среды {[}18{]}, окружающей природной среды
{[}19{]}, затрат и качества информации {[}20{]}.

Обзор литературы цепочки поставок продовольствия за 2010-2021гг. показал
{[}21{]}: в 56\% статей используются математические и вычислительные
методы оптимизации процесса определения местоположения и мощностей
логистических объектов, 40\% - передовые технологии, модели планирования
и оптимизации поставок. Данные исследования подтверждают важность
изучения проблемы потерь и порчи продукции из-за неэффективной работы
логистической инфраструктуры.

В настоящее время остро ощущается нехватка качественной инфраструктуры
хранения производимой сельхозпродукции в Казахстане, что из года в год
увеличивает издержки аграрного сектора. Основной объем валового сбора
скоропортящейся продукции и картофеля осуществляется в южных регионах
(Алматинской, Туркестанской, Жамбылской областях) и на востоке страны
(Восточно-Казахстанской области), картофеля -- северных областях страны
(Павлодарской, Карагандинской, Северо-Казахстанской, Акмолинской
областях).

В начале 2023 года мощности скоропортящейся плодоовощной продукции
(овоще-, фруктохранилища) составили 1,95 млн тонн. Мощности хранения и
распределения 5 единиц ОРЦ составили 65 тыс. тонн, мощности 8 единиц ТЛЦ
- 72 тыс. тонн. Дефицит мощностей хранилищ плодоовощной продукции
составил 35,5\%.

Нехватка инфраструктуры хранения по регионам также неравномерна. В
отдельных регионах недостаточно развита складская инфраструктура по
хранению плодоовощной продукции и пищевых продуктов. Наибольший дефицит
(свыше 500 тыс. тонн в каждом регионе) наблюдается в Алматинской,
Жамбылской, Туркестанской, Северо-Казахстанской, Туркестанской областях
и в г. Шымкент. Дефицит объемом в пределах 50-100 тыс. тонн наблюдается
в Атырауской, Актюбинской, Восточно-Казахстанской,
Западно-Казахстанской, Мангистауской, Карагандинской, Костанайской,
Северо-Казахстанской областях, а также в г. Астана.

Сельскохозяйственные товаропроизводители испытывают серьезные
затруднения с продажей произведенной продукции. Так, через
оптово-розничные продовольственные рынки реализуется в порядке 45\%
продукции сельского хозяйства; через прямые поставки фермерами на рынок
- 5-10\%; через торгово-логистические центры и ОРЦ 5-10\%; 35-40\%
мелких фермеров продают свою продукцию прямо со своего фермерского
хозяйства торговцам; через платформы электронной площадки - около 1\%.

Наибольший удельный вес в общем объеме розничной торговли
агропродовольственной продукции республики в 2023г. приходится на города
Алматы (32,2\%) и Астана (12,9\%), а также Карагандинскую (8\%) и
Восточно-Казахстанскую (6\%) области.

Для решения проблем сбыта продукции, прямого выхода на торговые объекты,
отсутствия овощехранилищ возникает актуальный вопрос строительства ОРЦ
по хранению, распределению и торговле агропродовольственной продукции.~

Построение модели ОРЦ агропродовольственной продукции проводилось в
последовательности, включающей следующие этапы.

\emph{Этап 1.} Определение потенциальных мест размещения ОРЦ: анализ и
отбор показателей, определяющих места выбора и размещения;
стандартизация показателей; определение весовых коэффициентов
показателей; факторный анализ показателей и выявление наиболее важных
факторов; кластерный анализ для выявления территориального распределения
производств продукций АПК и формирование кластеров; расчет рейтингов и
дифференциация районов по уровню привлекательности размещения
логистических инфраструктур.

\emph{Этап 2.} Уточнение потенциальных мест размещения: определение
радиуса действия логистической инфраструктуры; определение потенциальных
мест размещения

\emph{Этап 3.} Определение зоны обслуживания: определение зоны
обслуживания ОРЦ; экономическая интерпретация полученных результатов.

{\bfseries Результаты и обсуждение.}

\emph{Этап 1. Определение потенциальных мест размещения ОРЦ}

1.1. Анализ, и отбор показателей, определяющих места выбора и размещения
по статистическим данным за 2021-2023гг.

Для характеристики ОРЦ нами из различных литературных источников были
выбраны 40 показателей, прямо или косвенно влияющих на выбор места
размещения логистических объектов по статистическим данным Бюро
национальной статистики Республики Казахстан (БНС РК). Вычисление
корреляционной матрицы для переменных, участвующих в анализе, позволило
исключить зависимые (коррелирующие) параметры, отобрать наиболее
значимые 19 факторов. Из них 15 показателей имеются на сайте БНС РК
(\href{https://stat.gov.kz/en/}{}), а
остальные 3 показателя (X9, X10 и X11) выбраны на основе изучения
онлайн-карты (https://yandex.kz/maps/ru/) транспортных маршрутов по всем
регионам Казахстана.

Были оценены значимость показателей и исключены незначимые факторы
(оценка значимых показателей корреляции по p-value: р\textless0,05;
р\textless0,01; р\textless0,001).

В таблице 1 представлены результаты оценки значимых показателей.
\end{multicols}

{\bfseries Таблица 1 - Переменные после исключения коррелирующих параметров}

%% \begin{longtable}[]{@{}
%%   >{\centering\arraybackslash}p{(\linewidth - 10\tabcolsep) * \real{0.0758}}
%%   >{\raggedright\arraybackslash}p{(\linewidth - 10\tabcolsep) * \real{0.1818}}
%%   >{\raggedright\arraybackslash}p{(\linewidth - 10\tabcolsep) * \real{0.0910}}
%%   >{\raggedright\arraybackslash}p{(\linewidth - 10\tabcolsep) * \real{0.2878}}
%%   >{\raggedright\arraybackslash}p{(\linewidth - 10\tabcolsep) * \real{0.1212}}
%%   >{\raggedright\arraybackslash}p{(\linewidth - 10\tabcolsep) * \real{0.2424}}@{}}
%% \toprule\noalign{}
%% \begin{minipage}[b]{\linewidth}\centering
%% №№
%% \end{minipage} & \begin{minipage}[b]{\linewidth}\raggedright
%% Факторы
%% \end{minipage} & \begin{minipage}[b]{\linewidth}\raggedright
%% Пока-зате-ли
%% \end{minipage} & \begin{minipage}[b]{\linewidth}\raggedright
%% Название переменной
%% \end{minipage} & \begin{minipage}[b]{\linewidth}\raggedright
%% Единицаизмере-ния
%% \end{minipage} & \begin{minipage}[b]{\linewidth}\raggedright
%% Описание показателя
%% \end{minipage} \\
%% \midrule\noalign{}
%% \endhead
%% \bottomrule\noalign{}
%% \endlastfoot
%% \multirow{2}{=}{1} & \multirow{2}{=}{Социальные факторы} & Х1 &
%% Численность населения & чел. & Среднегодовая численность населения \\
%% & & Х2 & Доход на душу населения & тыс. тг. & Отношение годового объема
%% денежных доходов на количество месяцев и на среднегодовую численность
%% населения \\
%% \multirow{5}{=}{2} & \multirow{5}{=}{Экономичес-кие факторы} & Х3 &
%% Количество производителей сельскохозяйственной продукции & ед. &
%% Показатель уровня концентрации производителей продукции \\
%% & & Х4 & Объем грузоперевозки агропродовольственной продукции & т &
%% Характеристика объемов сельскохозяйствен-ной перевозки продовольственной
%% продукции в регионах \\
%% & & Х5 & Объем производства сельскохозяйственной продукции & млн тг. &
%% Объемы производства продукции \\
%% & & Х6 & Розничный и оптовый товарооборот продовольственных товаров &
%% млн тг. & Объемы розничной и оптовой реализации продовольственных
%% товаров \\
%% & & Х7 & Инвестиции в основной капитал логистической инфраструктуры
%% (торговля, транспорт и складирование, связь) & млн тг. & Инвестиции в
%% инфраструктуру в регионах \\
%% \multirow{4}{=}{3} & \multirow{4}{=}{Региональные факторы} & Х8 &
%% Доступность (наличие) логистических мощностей хранения, распределения и
%% торговли сельхозпродукции, от 0 до 1 & да - 1, нет - 0 & Физическая
%% доступность (наличие) мощностей. Определяется по статистическим данным
%% по каждому региону \\
%% & & Х9 & Доступность железной дороги, от 0 до 1 & нет - 0, есть - 1,
%% частично -- 0,5 & Показатель уровня обеспеченности железнодорожной
%% сетью. Определяется на основе карты транспортных дорог регионов
%% https://yandex.kz/maps/ru/ \\
%% & & Х10 & Среднее время транспортировки на автотранспорте & час &
%% Показатель, определяющий быстроту доставки продукции до места
%% назначения.
%% 
%% Время перевозки от районного центра до областного центра. Определяется
%% на основе онлайн карты https://yandex.kz/maps/ru/ \\
%% & & Х11 & Расстояние от районного центра до областного центра на
%% автотранспорте & км & Расстояние перевозки -определяется по карте
%% маршрута https://yandex.kz/maps/ru/ \\
%% \multirow{6}{=}{4} & \multirow{6}{=}{Производст-венные факторы} & Х12 &
%% Производство и реализация мяса и молока & т & \multirow{6}{=}{Объем
%% реализации продукции в каждом регионе} \\
%% & & Х13 & Производство и реализация овощей и бахчевых & т \\
%% & & Х14 & Производство и реализация фруктов и ягод & т \\
%% & & Х15 & Производство зерновых & тыс. т \\
%% & & Х16 & Объем производства и реализация яиц & тыс. шт. \\
%% & & Х17 & Объем производства и реализация картофеля & т \\
%% \multirow{2}{=}{5} & \multirow{2}{=}{Экологи-ческие факторы} & Х18 &
%% Объем твердых отходов при реализации продукции & т & Количество отходов
%% при реализации продукции на основе статистики https://stat.gov.kz \\
%% & & Х19 & Выбросы в атмосферу загрязняющих веществ & т & Выбросы в
%% атмосферу определены на основе статистики https://stat.gov.kz \\
%% \multicolumn{6}{@{}>{\centering\arraybackslash}p{(\linewidth - 10\tabcolsep) * \real{1.0000} + 10\tabcolsep}@{}}{%
%% \emph{Примечание: Данные получены в результате определения значимости
%% факторов}} \\
%% \end{longtable}

Описательная статистика выбранных показателей представлена в таблице 2.

{\bfseries Таблица 2 - Описательная статистика выбранных показателей
(\emph{n}=204, \emph{m}=19)}

%% \begin{longtable}[]{@{}
%%   >{\centering\arraybackslash}p{(\linewidth - 14\tabcolsep) * \real{0.1490}}
%%   >{\centering\arraybackslash}p{(\linewidth - 14\tabcolsep) * \real{0.1242}}
%%   >{\centering\arraybackslash}p{(\linewidth - 14\tabcolsep) * \real{0.1336}}
%%   >{\centering\arraybackslash}p{(\linewidth - 14\tabcolsep) * \real{0.1351}}
%%   >{\centering\arraybackslash}p{(\linewidth - 14\tabcolsep) * \real{0.1212}}
%%   >{\centering\arraybackslash}p{(\linewidth - 14\tabcolsep) * \real{0.1602}}
%%   >{\centering\arraybackslash}p{(\linewidth - 14\tabcolsep) * \real{0.1009}}
%%   >{\centering\arraybackslash}p{(\linewidth - 14\tabcolsep) * \real{0.0758}}@{}}
%% \toprule\noalign{}
%% \begin{minipage}[b]{\linewidth}\centering
%% Показатели
%% \end{minipage} & \begin{minipage}[b]{\linewidth}\centering
%% Значение
%% \end{minipage} & \begin{minipage}[b]{\linewidth}\centering
%% Стандарт-ное отклоне-
%% 
%% ние
%% \end{minipage} & \begin{minipage}[b]{\linewidth}\centering
%% Медиана
%% \end{minipage} & \begin{minipage}[b]{\linewidth}\centering
%% Эксцесс
%% \end{minipage} & \begin{minipage}[b]{\linewidth}\centering
%% Асимметрия
%% \end{minipage} & \begin{minipage}[b]{\linewidth}\centering
%% Тест Харке-Бера
%% \end{minipage} & \begin{minipage}[b]{\linewidth}\centering
%% p-зна-че-ние*
%% \end{minipage} \\
%% \midrule\noalign{}
%% \endhead
%% \bottomrule\noalign{}
%% \endlastfoot
%% Х1 & 93665,9 & 195561,2 & 39478,5 & 55,31 & 6,67 & 0,95 & 0,41 \\
%% Х2 & 125394,7 & 38483,4 & 119790,0 & 2,32 & 1,29 & 1,12 & 0,24 \\
%% Х3 & 9267,5 & 8627,2 & 6743,0 & 6,67 & 2,26 & 1,35 & 0,31 \\
%% Х4 & 233,1 & 224,4 & 172,9 & 5,34 & 1,82 & 2,17 & 0,51 \\
%% Х5 & 35791,7 & 25323,2 & 32653,5 & 1,87 & 1,09 & 1,85 & 0,05 \\
%% Х6 & 43085,7 & 291685,9 & 1759,4 & 167,57 & 12,52 & 0,87 & 0,14 \\
%% Х7 & 3797,8 & 4533,7 & 2011,1 & 8,55 & 2,36 & 0,97 & 0,41 \\
%% Х8 & 33280,7 & 60580,9 & 5825,0 & 7,95 & 2,68 & 1,23 & 0,15 \\
%% Х9 & 17,6 & 26,6 & 6,0 & 6,18 & 2,31 & 2,47 & 0,52 \\
%% Х10 & 3,3 & 2,0 & 2,9 & 0,22 & 0,75 & 1,84 & 0,50 \\
%% Х11 & 216,0 & 147,0 & 195,0 & 0,03 & 0,64 & 0,78 & 0,34 \\
%% Х12 & 215,8 & 150,3 & 194,8 & 0,00 & 0,63 & 2,68 & 0,45 \\
%% Х13 & 36945,1 & 92134,7 & 6184,0 & 27,85 & 4,82 & 2,74 & 0,38 \\
%% Х14 & 2994,5 & 10905,2 & 150,4 & 49,13 & 6,50 & 0,81 & 0,27 \\
%% Х15 & 80521,7 & 107472,3 & 30717,3 & 2,26 & 1,64 & 1,21 & 0,31 \\
%% Х16 & 24655,9 & 59291,9 & 6209,3 & 13,69 & 3,69 & 0,87 & 0,12 \\
%% Х17 & 19762,7 & 26773,7 & 8121,2 & 5,61 & 2,24 & 2,37 & 0,49 \\
%% Х18 & 3369,9 & 3094,9 & 2435,0 & 49,46 & 6,10 & 1,41 & 0,34 \\
%% Х19 & 12983,7 & 44868,1 & 2133,2 & 31,38 & 5,47 & 0,97 & 0,24 \\
%% \multicolumn{8}{@{}>{\centering\arraybackslash}p{(\linewidth - 14\tabcolsep) * \real{1.0000} + 14\tabcolsep}@{}}{%
%% \emph{Примечание: *достоверно при p\textless0.05}
%% 
%% \emph{Источник: Статистические данные по РК за 2021-2023гг.}} \\
%% \end{longtable}

\begin{multicols}{2}
Коэффициент асимметрии и эксцесс близки к нулю, что делает возможным
приближение к нормальному распределению. Гипотезу о нормальности можно
принять на основе статистики Харке-Бера на уровне 5\%. Таким образом,
все параметры, изучаемые в таблице 3, могут быть использованы в качестве
методов параметрической статистики для дальнейшего анализа.

1.2. Стандартизация показателей осуществлена переходом от матрицы
исходных данных к матрице стандартизированных показателей матрицы
размерностью 204 х 19 (204 - количество территорий, 19 -- переменные,
взятые из таблицы 1).

1.3. Определение весовых коэффициентов показателей позволяет определить
их значимость для формирования рейтинга. Данные действия осуществлялись
с использованием программы SPSS 21.0. Расчеты показали, что выбранные
показатели являются существенными, так как коэффициент дисперсии равен
93,2\% (табл.3).
\end{multicols}

{\bfseries Таблица 3 - Весовые коэффициенты показателей, влияющих на выбор
места размещения}

%% \begin{longtable}[]{@{}
%%   >{\raggedright\arraybackslash}p{(\linewidth - 2\tabcolsep) * \real{0.8093}}
%%   >{\raggedleft\arraybackslash}p{(\linewidth - 2\tabcolsep) * \real{0.1907}}@{}}
%% \toprule\noalign{}
%% \begin{minipage}[b]{\linewidth}\raggedright
%% Показатели\emph{~}
%% \end{minipage} & \begin{minipage}[b]{\linewidth}\centering
%% Весовые коэффициенты
%% \end{minipage} \\
%% \midrule\noalign{}
%% \endhead
%% \bottomrule\noalign{}
%% \endlastfoot
%% Численность населения (Х1) & -0,9784 \\
%% Доходы на душу населения (Х2) & 0,5314 \\
%% Количество производителей сельскохозяйственной продукции (Х3) &
%% -0,9529 \\
%% Объем перевозки грузов сельскохозяйственной продукции автомобильным
%% транспортом (Х4) & -0,8194 \\
%% Объем реализации сельскохозяйственной продукции (Х5) & 0,9930 \\
%% Объем оптового и розничного товарооборота продовольственных товаров (Х6)
%% & 0,4546 \\
%% Инвестиции в основной капитал логистической инфраструктуры (Х7) &
%% 0,4387 \\
%% Доступность (наличие) мощностей хранения сельхозпродукции (Х8) &
%% 0,8754 \\
%% Доступность железной дороги* (Х9) & 0,6654 \\
%% Время транспортировки продукции на автотранспорте** (Х10) & -0,8427 \\
%% Расстояние от центра района до РЦ в областном центре** (Х11) &
%% -0,7123 \\
%% Производство сельскохозяйственной продукции в натуральном выражении
%% (Х12-Х17) & 0,9508 \\
%% Объем твердых отходов при реализации продукции (Х18) & -0,6412 \\
%% Выбросы в атмосферу загрязняющих веществ & -0,4517 \\
%% Общая дисперсия & 2,2305 \\
%% Доля общей дисперсии & 0,9327 \\
%% \multicolumn{2}{@{}>{\raggedright\arraybackslash}p{(\linewidth - 2\tabcolsep) * \real{1.0000} + 2\tabcolsep}@{}}{%
%% \emph{Примечания: * определены на основе онлайн-карты железных дорог РК.
%% https://nkregion.kz/info/maps/63-railways.html}
%% 
%% \emph{** определены на основе карты https://www.google.kz/maps}
%% 
%% \emph{Источник: Статистические данные по Казахстану}} \\
%% \end{longtable}

1.4. Проведенный факторный анализ позволил сформировать, сократить
количество переменных и сгруппировать их. На основе анализа построены
матрицы значений факторов \emph{F\textsubscript{ji}} для всех 204
территорий (районов, городских агломераций), которые в дальнейшем
использованы для расчета рейтинга места размещения логистических
объектов, полученного с использованием программы SPSS 21.0.

В первом блоке показателей главная компонента объяснила 27,38\% вариации
\emph{xj,} во втором блоке - 20,17\% вариации и 15,95; 11,39; 6,76\%
вариации соответственно по компонентам (таблица 4).

{\bfseries Таблица 4 - Объясненная совокупная дисперсия (метод главных
компонент)}

%% \begin{longtable}[]{@{}
%%   >{\centering\arraybackslash}p{(\linewidth - 10\tabcolsep) * \real{0.3476}}
%%   >{\centering\arraybackslash}p{(\linewidth - 10\tabcolsep) * \real{0.1372}}
%%   >{\centering\arraybackslash}p{(\linewidth - 10\tabcolsep) * \real{0.1220}}
%%   >{\centering\arraybackslash}p{(\linewidth - 10\tabcolsep) * \real{0.1220}}
%%   >{\centering\arraybackslash}p{(\linewidth - 10\tabcolsep) * \real{0.1373}}
%%   >{\centering\arraybackslash}p{(\linewidth - 10\tabcolsep) * \real{0.1338}}@{}}
%% \toprule\noalign{}
%% \multirow{2}{=}{\begin{minipage}[b]{\linewidth}\centering
%% Начальные собственные значения
%% \end{minipage}} &
%% \multicolumn{5}{>{\centering\arraybackslash}p{(\linewidth - 10\tabcolsep) * \real{0.6524} + 8\tabcolsep}@{}}{%
%% \begin{minipage}[b]{\linewidth}\centering
%% Компонента
%% \end{minipage}} \\
%% & \begin{minipage}[b]{\linewidth}\centering
%% 1
%% \end{minipage} & \begin{minipage}[b]{\linewidth}\centering
%% 2
%% \end{minipage} & \begin{minipage}[b]{\linewidth}\centering
%% 3
%% \end{minipage} & \begin{minipage}[b]{\linewidth}\centering
%% 4
%% \end{minipage} & \begin{minipage}[b]{\linewidth}\centering
%% 5
%% \end{minipage} \\
%% \midrule\noalign{}
%% \endhead
%% \bottomrule\noalign{}
%% \endlastfoot
%% Всего & 4,108 & 3,026 & 2,393 & 1,709 & 1,015 \\
%% \% дисперсии & 27,387 & 20,174 & 15,954 & 11,393 & 6,767 \\
%% Суммарный \% & 27,387 & 47,561 & 63,516 & 74,908 & 81,675 \\
%% \multicolumn{6}{@{}>{\centering\arraybackslash}p{(\linewidth - 10\tabcolsep) * \real{1.0000} + 10\tabcolsep}@{}}{%
%% \emph{Примечание: Результаты вычисления с использованием программы SPSS
%% 21.0}} \\
%% \end{longtable}

Использование данного метода позволило в каждом из блоков показателей,
характеризующих эффективность функционирования логистической
инфраструктуры, выделить главные компоненты путем «сжатия» переменных
(таблица 5).

{\bfseries Таблица 5 -Факторные нагрузки по главным компонентам для выбора
места размещения логистической инфраструктуры хранения, сбыта и
торговли}

%% \begin{longtable}[]{@{}
%%   >{\centering\arraybackslash}p{(\linewidth - 10\tabcolsep) * \real{0.2203}}
%%   >{\centering\arraybackslash}p{(\linewidth - 10\tabcolsep) * \real{0.1399}}
%%   >{\centering\arraybackslash}p{(\linewidth - 10\tabcolsep) * \real{0.1556}}
%%   >{\centering\arraybackslash}p{(\linewidth - 10\tabcolsep) * \real{0.1554}}
%%   >{\centering\arraybackslash}p{(\linewidth - 10\tabcolsep) * \real{0.1746}}
%%   >{\centering\arraybackslash}p{(\linewidth - 10\tabcolsep) * \real{0.1542}}@{}}
%% \toprule\noalign{}
%% \multirow{2}{=}{\begin{minipage}[b]{\linewidth}\centering
%% Показатели
%% \end{minipage}} &
%% \multicolumn{5}{>{\centering\arraybackslash}p{(\linewidth - 10\tabcolsep) * \real{0.7797} + 8\tabcolsep}@{}}{%
%% \begin{minipage}[b]{\linewidth}\centering
%% Средние значения компонент
%% 
%% за 2021-2023гг.
%% \end{minipage}} \\
%% & \begin{minipage}[b]{\linewidth}\centering
%% F1
%% \end{minipage} & \begin{minipage}[b]{\linewidth}\centering
%% F2
%% \end{minipage} & \begin{minipage}[b]{\linewidth}\centering
%% F3
%% \end{minipage} & \begin{minipage}[b]{\linewidth}\centering
%% F4
%% \end{minipage} & \begin{minipage}[b]{\linewidth}\centering
%% F5
%% \end{minipage} \\
%% \midrule\noalign{}
%% \endhead
%% \bottomrule\noalign{}
%% \endlastfoot
%% Х1 & 0,058 & {\bfseries 0,826} & -0,087 & -0,193 & 0,014 \\
%% Х2 & -0,346 & {\bfseries 0,753} & 0,371 & 0,052 & 0,035 \\
%% Х3 & {\bfseries 0,856} & -0,027 & -0,048 & -0,074 & 0,034 \\
%% Х4 & {\bfseries 0,854} & -0,057 & 0,315 & 0,003 & 0,196 \\
%% Х5 & {\bfseries 0,835} & -0,085 & 0,280 & 0,043 & 0,378 \\
%% Х6 & -0,076 & {\bfseries 0,972} & -0,047 & -0,073 & -0,013 \\
%% Х7 & -0,455 & {\bfseries 0,954} & -0,001 & -0,008 & 0,014 \\
%% Х8 & -0,083 & -0,014 & {\bfseries 0,924} & -0,043 & -0,095 \\
%% Х9 & 0,002 & -0,026 & {\bfseries 0,922} & -0,077 & -0,081 \\
%% Х10 & -0,022 & -0,088 & -0,049 & {\bfseries 0,979} & -0,088 \\
%% Х11 & -0,007 & -0,149 & -0,036 & {\bfseries 0,976} & -0,077 \\
%% Х12 & {\bfseries 0,739} & -0,081 & 0,010 & 0,164 & 0,316 \\
%% Х13 & {\bfseries 0,728} & 0,036 & -0,158 & 0,011 & -0,166 \\
%% Х14 & {\bfseries 0,700} & 0,050 & -0,044 & -0,139 & -0,247 \\
%% Х15 & {\bfseries 0,726} & -0,106 & 0,204 & 0,044 & 0,320 \\
%% Х16 & {\bfseries 0,828} & 0,040 & -0,012 & -0,163 & 0,062 \\
%% Х17 & {\bfseries 0,698} & -0,013 & 0,010 & -0,114 & 0,372 \\
%% Х18 & 0,024 & 0,217 & -0,171 & 0,008 & {\bfseries 0,854} \\
%% Х19 & 0,173 & 0,018 & {\bfseries 0,719} & 0,131 & 0,042 \\
%% \multicolumn{6}{@{}>{\centering\arraybackslash}p{(\linewidth - 10\tabcolsep) * \real{1.0000} + 10\tabcolsep}@{}}{%
%% \emph{Примечание: Метод выделения факторов: метод главных компонент.
%% Метод вращения: Варимакс с нормализацией Кайзера. Вращение сошлось за 5
%% итераций}} \\
%% \end{longtable}

\begin{multicols}{2}
По факторному анализу выведены следующие результаты: альфа-Кронбаха
равен 0,81 - выбранные переменные приемлемые. Кайзер-Мейер-Олкин (КМО),
равный 0,73, является удовлетворительным.

Получена матрица значений для пяти групп факторов: F1, F2, F3, F4 и F5
по каждому из 204 территорий.

В матрице главных компонент среднее значение весовых коэффициентов
определилось следующими переменными: F1 (Х3, Х4, Х5, Х12, Х13, Х14, Х15,
Х16, Х17), F2 (Х1, Х2, Х6, Х7), F3 (Х8, Х9, Х19), F4 (Х10, Х11), F5
(Х18).

Таким образом, выделены следующие факторы: фактор F1 --
производственные, факторы F2 - торговые, фактор F3 -- транспортная и
складская доступность, фактор F4 - географические (время и расстояния),
фактор F5 -- экологические.

Содержания факторов F1, F3 и F5 определяют сводную характеристику
хранения продукции в регионе: количество производителей продукции, объем
производства и грузоперевозки, объемы производства в натуральном
выражении, твердые отходы и выбросы загрязняющих веществ в атмосферу.

Содержания факторов F2 и F4 характеризуют возможности распределения и
торговли в регионе: численность и доход на душу населения региона, объем
торговли и инвестиции в инфраструктуру, время и расстояния перевозки
грузов в регионе.

Содержание фактора F2 характеризует возможности торговли в регионе:
численность и доход на душу населения региона, объем торговли и
инвестиции в инфраструктуру.

Связь между переменными и главной компонентой представлены следующей
зависимостью:

F1=0,856*Х3+0,854*Х4+ 0,835*Х5 + 0,739*Х12+0,728*Х13 +0,7*Х14 +0,726*Х15
+0,828*Х16 + 0,698*Х17

F2=0,826*Х1+0,753*Х2+0,972*Х6+0,954*Х7

F3=0,924*Х8+0,922*Х9+0,719*Х19

F4=0,979*Х10 +0,976*Х11

F5=0,854*Х18

Решением уравнения главных компонент явилось построение матрицы значений
факторов F1, F2, F3, F4 и F5 для 204 объектов исследования, которая
стала основанием для проведения кластерного анализа с целью
дифференциации регионов по уровню готовности к размещению логистической
инфраструктуры.

1.5. Для изучения территориального распределения сельскохозяйственного
производства районы Казахстана были разбиты на группы с помощью метода
кластерного анализа. Были использованы статистические данные за
2019-2023гг., обработка произведена с использованием программы SPSS
21.0. С целью определения значимости главных компонент, обуславливающих
кластеризацию, использовался \emph{F}-критерий. Чем больше его значение,
тем больший вклад вносит главная компонента в кластеризацию (таблица 6).
\end{multicols}

{\bfseries Таблица 6 -Значения \emph{F}-критерия главных компонент за 2021-2023 годы}

%% \begin{longtable}[]{@{}
%%   >{\raggedright\arraybackslash}p{(\linewidth - 10\tabcolsep) * \real{0.2199}}
%%   >{\raggedright\arraybackslash}p{(\linewidth - 10\tabcolsep) * \real{0.1529}}
%%   >{\raggedright\arraybackslash}p{(\linewidth - 10\tabcolsep) * \real{0.1426}}
%%   >{\raggedright\arraybackslash}p{(\linewidth - 10\tabcolsep) * \real{0.1529}}
%%   >{\raggedright\arraybackslash}p{(\linewidth - 10\tabcolsep) * \real{0.1684}}
%%   >{\raggedright\arraybackslash}p{(\linewidth - 10\tabcolsep) * \real{0.1633}}@{}}
%% \toprule\noalign{}
%% \multirow{3}{=}{\begin{minipage}[b]{\linewidth}\raggedright
%% Значение \emph{F}-критерия
%% \end{minipage}} &
%% \multicolumn{5}{>{\centering\arraybackslash}p{(\linewidth - 10\tabcolsep) * \real{0.7801} + 8\tabcolsep}@{}}{%
%% \begin{minipage}[b]{\linewidth}\centering
%% Главная компонента
%% \end{minipage}} \\
%% & \begin{minipage}[b]{\linewidth}\raggedright
%% F1
%% \end{minipage} & \begin{minipage}[b]{\linewidth}\raggedright
%% F2
%% \end{minipage} & \begin{minipage}[b]{\linewidth}\raggedright
%% F3
%% \end{minipage} & \begin{minipage}[b]{\linewidth}\raggedright
%% F4
%% \end{minipage} & \begin{minipage}[b]{\linewidth}\raggedright
%% F5
%% \end{minipage} \\
%% & \begin{minipage}[b]{\linewidth}\raggedright
%% 51.39
%% \end{minipage} & \begin{minipage}[b]{\linewidth}\raggedright
%% 58.15
%% \end{minipage} & \begin{minipage}[b]{\linewidth}\raggedright
%% 37.69
%% \end{minipage} & \begin{minipage}[b]{\linewidth}\raggedright
%% 24.08
%% \end{minipage} & \begin{minipage}[b]{\linewidth}\raggedright
%% 8.77
%% \end{minipage} \\
%% \midrule\noalign{}
%% \endhead
%% \bottomrule\noalign{}
%% \endlastfoot
%% \multicolumn{6}{@{}>{\raggedright\arraybackslash}p{(\linewidth - 10\tabcolsep) * \real{1.0000} + 10\tabcolsep}@{}}{%
%% \emph{Примечание: Получено на основе расчета моделей F1- F5 по данным
%% таблицы 5}} \\
%% \end{longtable}

\begin{multicols}{2}
Основными переменными, определяющими группировку, явились те, что
относятся к торговым факторам, производственным факторам, транспортной и
складской доступности. Также кластеризацию определяют результаты
переменных социального, производственного и, меньше всего, экологических
факторов главной компоненты. Это отражает государственную политику
развития торговой инфраструктуры агропродовольственного рынка на
сельских территориях и районах.

Из всех имеющихся методов проведения кластерного анализа был выбран
самый популярный метод \emph{k}-средних. Из проведенного
разведывательного анализа оптимальным явилось разбиение регионов на
шесть кластеров, которые включают в себя административно-территориальные
районы, качественно различающиеся между собой.

Принадлежность территорий к кластерам имеет следующие особенности.

В первый кластер вошли 7 районов, которые имеют развитое многоотраслевое
сельское хозяйство с производством растениеводческой продукции. В нем
производится 5\% объема сельскохозяйственной продукции (в стоимостном
выражении), 11\% - картофеля, 36\% - яиц, 5,1\% - молока, 4,1\% мяса (в
натуральном выражении).

Во второй кластер вошли районы, которые имеют в большей степени
животноводческую направленность, а также развитое растениеводство,
картофелеводство и овощеводство. В данный кластер вошли 159 районов всех
областей. В нём производится 67,0\% сельскохозяйственной продукции
Казахстана (в стоимостном выражении), из них мясо -79,7\%, молоко
-71,8\%, картофель - 66,2\%, зерновые -- 55,5\%, овощи - 49,9\%, фрукты
и ягоды - 54,9\% (в натуральном выражении).

Третий, пятый и шестой кластеры образуют районы и городские округа с
низким уровнем развития сельского хозяйства с долей объема производства
продукции - 2,0; 4,7 и 2,3\% соответственно.

В третьем кластере находится одна территория (Енбекшиказахский район
Алматинской области), где преобладают фрукты и ягоды -16,9\%, овощи и
бахчевые - 4,1\%.

В четвертый кластер входят районы с выраженным развитием отрасли
растениеводства, особенно производство сои и зерновых культур (37,8\%) и
картофелеводство (16,3\%), а также животноводство мясного и молочного
направления: мясо - 10,4\%, молоко - 15,8\%, яйца - 7,3\%, фрукты и
ягоды - 6,4\%. В нем производится 18,9\% сельскохозяйственной продукции
Казахстана (в стоимостном выражении).

В пятом кластере развито растениеводство: фрукты и ягоды -17,9\% всех
кластеров, овощи и бахчевые - 22,0\%, картофель - 2,9\%, молоко - 3,5\%,
мясо - 2,6\%, яйца - 3,8\%. Доля объема производства
сельскохозяйственной продукции в стоимостном выражении составляет 4,7\%.
В данный кластер вошли районы южных областей Казахстана, производящие в
основном овощи и фрукты.

В шестом кластере развиты овощеводство и бахчевые - 18,4\% всех
кластеров. Доля объема выпуска продукции в стоимостном выражении -
2,3\%. В данный кластер вошли южные регионы, производящие бахчевые.

Полученные результаты кластеров обобщены по производственным и
региональным потенциальным характеристикам инфраструктуры в итоговой
таблице 7.
\end{multicols}

{\bfseries Таблица 7 - Характеристика логистической инфраструктуры в зависимости от выделенных кластеров РК}

%% \begin{longtable}[]{@{}
%%   >{\centering\arraybackslash}p{(\linewidth - 6\tabcolsep) * \real{0.1061}}
%%   >{\centering\arraybackslash}p{(\linewidth - 6\tabcolsep) * \real{0.1212}}
%%   >{\raggedright\arraybackslash}p{(\linewidth - 6\tabcolsep) * \real{0.2893}}
%%   >{\raggedright\arraybackslash}p{(\linewidth - 6\tabcolsep) * \real{0.4833}}@{}}
%% \toprule\noalign{}
%% \begin{minipage}[b]{\linewidth}\centering
%% Номер клас-тера
%% \end{minipage} & \begin{minipage}[b]{\linewidth}\centering
%% Коли-
%% 
%% чество районов
%% \end{minipage} & \begin{minipage}[b]{\linewidth}\centering
%% Характеристика производственного потенциала
%% \end{minipage} & \begin{minipage}[b]{\linewidth}\centering
%% Потенциальные характеристики инфраструктуры
%% \end{minipage} \\
%% \midrule\noalign{}
%% \endhead
%% \bottomrule\noalign{}
%% \endlastfoot
%% 1 & 8 & Растениеводство (картофель) и зерновые, мясо и молоко, яйца &
%% зерно- и картофелехранилища, холодильное оборудование для молока и мяса
%% животных и птиц \\
%% 2 & 159 & Животноводство молочного направления, пчеловодство,
%% растениеводство (соя, овес, кукуруза, гречиха), картофелеводство,
%% овощеводство & зерно- и картофелехранилища, холодильное оборудование для
%% молока и мяса животных и птиц, фрукто- и ягодохранилища,
%% овощехранилища \\
%% 3 & 2 & Фрукты и ягоды, овощи и бахчевые, картофель, мясо, молоко &
%% картофелехранилища, фрукто- и ягодохранилища, овощехранилища,
%% холодильное оборудование для молока и мяса животных \\
%% 4 & 27 & Животноводство мясо-молочного направления, птицеводство,
%% растениеводство (соя, пшеница, овес, кукуруза, гречиха),
%% картофелеводство, овощеводство & зерно- и картофелехранилища,
%% холодильное оборудование для молока и мяса животных и птиц, фрукто- и
%% ягодохранилища \\
%% 5 & 5 & Растениеводство (соя, овес, кукуруза, гречиха),
%% картофелеводство, овощеводство, фрукты и ягоды, птицеводство & крупные
%% фрукто- и ягодохранилища, овощехранилища, картофелехранилища, а также
%% холодильное оборудование для молока и мяса животных и птиц \\
%% 6 & 3 & Овощеводство и бахчевые & крупные овощехранилища, а также
%% фрукто- и ягодохранилища, холодильное оборудование для молока и мяса
%% животных \\
%% \end{longtable}

\begin{multicols}{2}
Результаты классификации кластеров: 1 кластер - зерно- и
картофелехранилища, холодильное оборудование для молока и мяса животных
и птиц. Для развития кластера потребуются элеваторы, картофелехранилища,
холодильные оборудования для хранения молочной и мясной продукции.

2 кластер - зерно- и картофелехранилища, холодильное оборудование для
молока и мяса животных и птиц, фрукто- и ягодохранилища, овощехранилища.

3 кластер -- картофелехранилища, фрукто- и ягодохранилища,
овощехранилища, холодильное оборудование для молока и мяса животных.
Данному кластеру, в первую очередь, необходима инфраструктура
растениеводства.

4 кластер - зерно- и картофелехранилища, холодильнее оборудование для
молока и мяса животных и птиц, фрукто- и ягодохранилища.

5 кластер -- крупные фрукто- и ягодохранилища, овощехранилища,
картофелехранилища, а также холодильное оборудование для молока и мяса
животных и птиц. Районам нужны универсальные ОРЦ для хранения.

6 кластер -- крупные овощехранилища, а также фрукто- и ягодохранилища,
холодильное оборудование для молока и мяса животных. Данные районы
требует развития инфраструктуры хранения бахчевых и овощей.

Результаты классификации позволяют сделать вывод о том, что размещать
универсальные хранилища для сельскохозяйственных продукции необходимо на
территории районов, вошедших в 1-2, 4-5 кластеры.

2 и 3 кластеры -- зернохранилища; 1, 2 и 4 кластеры --
картофелехранилища; 1, 2 и 4 кластеры -- холодильное оборудование для
хранения мяса животных и птицы; 2 и 4 кластеры - холодильное
оборудование для молока; 2, 3, 4 и 5 кластеры - фрукто- и
ягодохранилища; 2, 3, 5 и 6 кластеры - овощехранилища. Данные районы
обладают значительным производственным потенциалом и производят 95,7\%
сельскохозяйственной продукции РК.

3 и 6 кластеры, где преобладают фрукты и ягоды (16,9\%), овощи и
бахчевые (18,4\%) с низкой долей населения. Эти районы следует
обеспечить специализированными овоще-фруктовыми хранилищами.

1.6. Расчет интегрального показателя, дифференциация районов по уровню
рейтингового значения

Интегральный рейтинг районов/городов определялся по средним значениям
сгруппированных факторов F1-F5, скорректированным на их веса.

Указанные группы факторов F1, F2, F3, F4 и F5 явились основанием для
выбора видов ОРЦ. Средние значение критериев F1, F2, и F3 - для выбора
ОРЦ хранения, средние значения F2 и F4 -- для выбора ОРЦ распределения и
F3 -- для выбора ОРЦ торговли.

\emph{Этап 2. Уточнение потенциальных мест размещения
оптово-распределительных центров}

На данном этапе уточнены потенциальные места размещения логистических
интегрированных РЦ с учетом оптимальности расположения, минимизации
транспортных расходов и потерь продукции.

2.1. Определение радиуса действия ОРЦ

Для уточнения радиуса действия ОРЦ использовали метод «центра тяжести»,
суть которого заключается в поиске места размещения ОРЦ таким образом,
чтобы расстояние от сельскохозяйственного товаропроизводителя,
привозящего свою продукцию в ОРЦ, и от ОРЦ до потребителя было
минимальным. Расчеты центра тяжести для выбора радиуса действия ОРЦ
показали, для каждого региона они находится в пределах от 142,2 км
(Атырауская область) до 424,2 км (Акмолинская область), что говорит о
большом разбросе (разница в 3 раза) потенциальных ОРЦ от потенциальных
потребителей в областных центрах. Данное обстоятельство приводит к
увеличению затрат на перевозки и увеличению потерь продукций при
транспортировке, что также является не рациональным подходом. В этой
связи мы использовали дифференцированный подход при выборе места
размещения ОРЦ. К примеру, обслуживание регионов, которые находятся на
большом радиусе обслуживания (например 424,2 км, Акмолинская), можно
перевести в г.Астану, расположенную ближе (в радиусе до 150 км), а часть
районов перевести в г.Павлодар (радиус около 250 км). Таким образом,
можно оптимизировать зоны обслуживания районов с помощью рационального
расположения ОРЦ и снижения затрат.

2.2. Определение потенциальных мест размещения ОРЦ

Проведенные расчеты позволили уточнить степень влияния выбранных точек
формирования ОРЦ на близлежащие территории и создать круг потребителей
услуг ОРЦ.

Расчет рейтингов субъектов региона по F1-F5 позволил выделить основные
территории, где размещение ОРЦ наиболее благоприятно. Для формирования
крупного ОРЦ с последующим включением в национальную сеть предлагается
использовать территории близлежащих к городам республиканского значения:
Алматы, Астана (ОРЦ торговли и распределения), Шымкент (ОРЦ торговли), а
также крупные РЦ - в городах Актобе и Караганда (ОРЦ торговли).

ОРЦ распределения (Алаколь, Жезказган, Бейнеу, Аягуз, Курчум) следует
разместить ближе к крупным областным центрам, имеющим высокие рейтинги
(по F2 и F4); ОРЦ хранения - в местах сосредоточения
сельскохозяйственного производства и переработки продукции -- в сельских
районах, имеющих высокие рейтинги или в близлежащих городских
агломерациях по F1, F3 и F5.

В целом сложилось 25 ОРЦ, из них: ОРЦ хранения - 13 единиц, ОРЦ
распределения - 7 единиц, ОРЦ торговли - 5 единиц.

\emph{Этап 3. Определение зоны обслуживания ОРЦ}

3.1. Зоны обслуживания ОРЦ определяются (группы сельскохозяйственных
районов) в соответствии с потенциальными местами размещения на
территории районов и городских агломераций.

Логистическая сеть сельскохозяйственных РЦ в РК будет представлена
следующим образом.

- Сельскохозяйственные РЦ должны обслуживать от 142 километровой зоны
(Атырауская) до 424 километровой зоны (Акмолинская), исходя из
расстояния районов до областного центра, где предполагается создание ОРЦ
или оптового рынка. При этом, при определении зоны обслуживания каждого
ОРЦ, нужно учитывать полученные результаты по центрам тяжести, которые
определены строго по каждому региону (области) раздельно. Для этого
можно использовать онлайн карты местности и на основе этого районам,
находящимся далеко от центра потребления, необходимо выбрать ОРЦ,
находящийся поблизости, но относящийся к другой области.

- ОРЦ хранения (13 ОРЦ), кроме выполнения основных функций, должны стать
перевалочной базой для крупных районов областей. Потоки
сельскохозяйственной продукции и сырья могут быть направлены на
перерабатывающие предприятия в городских агломерациях.

3.2. Экономическая интерпретация полученных результатов

В результате формирования и дальнейшего функционирования
сельскохозяйственных РЦ в Казахстане будет создана более эффективная
система распределения сельскохозяйственного сырья и продовольствия на
основе организованного рынка, произойдет постепенное упорядочение
бессистемной деятельности различных посредников. Реализация системы
развития сельскохозяйственных РЦ позволит на основе экономического
стимулирования сельскохозяйственных товаропроизводителей увеличить
объемы производства и повысить качество продукции, осуществить
наполнение продовольственного рынка страны товарами отечественного
производства и улучшить социально-экономическую ситуацию в целом по АПК
региона. Решение вышеперечисленных задач будет способствовать реализации
мультипликативного (умножающего) эффекта: снижению потерь продукции
производителей, увеличению притока налоговых поступлений, созданию новых
рабочих мест, сокращению расходов населения на приобретение
социально-значимых продовольственных товаров.

Предварительные расчеты показывают, что создание системы оптового
продовольственного рынка в Казахстане целесообразно, как по
экономическим показателям эффективности, так и по социальному значению.

Проведение типизации основано на расчете интегрального показателя,
позволяющего выявить территории, относящиеся к неблагоприятным для
ведения сельскохозяйственного производства.

К районам, определенным как наиболее благоприятные для развития
растениеводства (картофель) и зерновых, мяса и молока, яиц (кластеры 1,
3, 5 и 6), также должны быть предъявлены особые условия, заключающиеся
не только в повышенной государственной помощи в построении
специализированных складских хранилищ, таких как зерно- и
картофелехранилища, фрукто- и ягодохранилища, овощехранилища,
холодильное оборудование, но и в разработке для данных районов
программных документов, развивающих предпринимательскую инициативу с
учетом повышенных рисков ведения сельскохозяйственного производства, а
также особые условия взаимодействия данных районов с торговыми сетями.
Это может касаться быстрой и бесперебойной доставки в течение года,
финансирования торговых сетей хозяйственных субъектов по выращиванию
сельскохозяйственной продукции.

К районам, определенным как наиболее благоприятным для развития
животноводства молочного направления, пчеловодства, растениеводства
(кластеры 2 и 4), также должны быть представлены условия построения и
развития универсальных сельскохозяйственных РЦ. В районах, входящих в
данные кластеры, должны быть созданы условия для совместного
использования торгово-сбытовой инфраструктуры ввиду их многочисленности
(159 территорий из 204).

Проведение анализа с использованием собственной методики позволило
разделить исследуемые районы на 6 кластеров, в зависимости от
специализации производства и инфраструктуры сбыта сельскохозяйственной
продукции.

Рейтинговая оценка, определенная на основе интегрального показателя,
выделила регионы-лидеры.

Реализация концептуальных положений механизма функционирования рынка
сбыта агропродовольственной продукции требует изменения традиционных
подходов к оценке его выбора и размещения. На региональном уровне
критерии оптимальности и эффективности размещения целесообразно измерять
с использованием интегрального мультипликативного показателя.

Результаты классификации позволяют делать вывод о том, что размещать
универсальные сельскохозяйственные РЦ необходимо на территории районов,
вошедших в 1, 2, 4, 5 кластеры.

Районы, входящие в 3 и 6 кластеры, где преобладает производство фруктов
и ягод (16,9\%), овощей и бахчевых (18,4\%), наблюдается низкая доля
населения, следует обеспечить специализированными овощно-фруктовыми ОРЦ.

Расчет по представленной методике с использованием статистических данных
и данных онлайн карты транспортных сетей регионов позволил определить
потенциальные регионы, наиболее подходящие для строительства ОРЦ
хранения, ОРЦ распределения и ОРЦ торговли в различных регионах
Казахстана. К примеру, по фактору F3 нами выбраны 5 ОРЦ торговли в
крупных городах Казахстана: Алматы, Астана, Шымкент, Актобе и Караганда.

На основе полученных результатов расчета определены количество,
потенциальные места размещения, а также зоны обслуживания ОРЦ хранения,
ОРЦ распределения, ОРЦ торговли. Полученные результаты согласуются с
деятельностью Правительства РК по созданию товаропроводящей системы, в
рамках которой планируется создание овощехранилищ и сети ОРЦ, где
предусмотрено строительство 24 ОРЦ общей стоимостью 273 млрд тг.
{[}1{]}.

{\bfseries Выводы.} В статье исследована проблема создания логистической
инфраструктуры по распределению, хранению и торговле сельхозпродукцией.
Использование автором собственного методологического подхода позволило
определить месторасположение ключевых объектов логистической
инфраструктуры на территории Казахстана.

Предлагаемый методологический подход состоит из трех этапов. На первом
этапе определяются территории (районы и городские агломерации), где
целесообразно размещение ключевых объектов логистической инфраструктуры.
Для решения данной задачи авторы предлагают подход, основанный на
использовании совокупного потенциала двухэтапного кластерного,
факторного анализа и метод рейтинга. Согласованность результатов
совместного использования указанных методов продемонстрирована автором
на примере Казахстана. На втором этапе осуществляется географическая
привязка объектов логистической инфраструктуры на местности. Для этого
используется метод центра тяжести для определения оптимальных мест
размещения инфраструктуры с учетом времени и расстояния перевозки
автотранспортом. Соответствующие задачи оптимизации решаются с целью
минимизации затрат на продвижение материального потока от поставщиков к
потребителям. На третьем этапе определяются зоны обслуживания ОРЦ,
исходя из минимизации затрат на обслуживание каждого района.

Отличительная особенность предлагаемой методики состоит в том, что при
формировании логистической инфраструктуры возникает возможность
одновременного учета целевого назначения и вида товароносителя:
хранение, сбыт, торговля. Исходя из этого возможен выбор необходимых
мощностей, типов холодильного и вентиляционного оборудования,
специализированных транспортных средств (рефрижераторы, фургоны
изотермические) и т.д.

Автор полагает, что следующие направления исследования могут быть
связаны с определением мощности логистических инфраструктур по мере
накопления информации по объемам потребления, производства, импорта и
экспорта в каждом регионе. Будущие исследования могут быть улучшены за
счет включения более специфических факторов, таких как климатические
факторы, которые, на наш взгляд, также могут влиять на выбор места
размещения и их мощности.

\emph{{\bfseries Финансирование.} Данное исследование выполнялось в рамках
грантового проекта, финансируемого Комитетом по науке Министерства науки
и высшего образования Республики Казахстан (АР19677634 «Развитие
логистической инфраструктуры и устойчивых цепей поставок скоропортящейся
продовольственной продукции на территории Казахстана», 2023 -- 2025
гг.).}
\end{multicols}

\begin{center}
{\bfseries Литература}
\end{center}

\begin{references}
1. Концепция государственной программы развития торговли Республики
Казахстан на 2021-2025 годы: Постановление Правительства Республики
Казахстан от 2 марта 2020 года.
\href{https://www.gov.kz/memleket/entities/mti/documents/details/61426?lang=ru}{}
-Дата обращения: 21.11.2024.

1. Государственная программа развития агропромышленного комплекса
Республики Казахстан на 2017 -- 2021 годы: Постановление Правительства
Республики Казахстан от 12 июля 2018 года № 423.
\href{https://adilet.zan.kz/rus/docs/P1800000423-}{}
Дата обращения: 25.11.2024.

1. \href{https://www.webofscience.com/wos/author/record/34864370}{Feng
Y.Q}.,
\href{https://www.webofscience.com/wos/author/record/33997583}{Liu
Y.K}.,
\href{https://www.webofscience.com/wos/author/record/34807908}{Chen
Y.J}.
\href{https://www.webofscience.com/wos/woscc/full-record/WOS:000859686100002}{Distributionally
robust location-allocation models of~distribution centers for fresh
products with uncertain demands} //
\href{https://www.sciencedirect.com/journal/expert-systems-with-applications}{Expert
Systems with Applications}. - 2022. - Vol.209. - P.518-531.
\href{https://doi.org/10.1016/j.eswa.2022.118180}{DOI
10.1016/j.eswa.2022.118180}

1. Дыбская В.В. Управление складированием в цепях поставок. -
М.:~\href{https://publications.hse.ru/books/?pb=57130529}{Альфа-Пресс},
2009. -- 720 с. ISBN 978-5-94280-355-1

1. Baker P., Canessa M. Warehouse design A structured approach //
European Journal of Operation Research. -- 2009. -- Vol.193(2). -
P.425 -- 436. DOI
\href{http://dx.doi.org/10.1016/j.ejor.2007.11.045}{10.1016/j.ejor.2007.11.045}

1. Anil Varghese Mangalan, Sandeep Kuriakose, H. Mohamed, A. Ray. Optimal
location of warehouse using weighted MOORA approach // International
Conference on Electrical, Electronics and Optimization Techniques
(ICEEOT).-2016. DOI 10.1109/ICEEOT.2016.7754764

1. Волхин Е.Г. Модели размещения распределительных центров // Управленец.
- 2018. --Т.9. - №2. -- С.54 -- 60. DOI 10.29141/2218-5003-2018-9-2-9

1. Daskin M. Network and Discrete Location: Models, Algorithms and
Applications (2nd Edition). - New York: John Wiley \& Sons Inc, 2013.
-- 544 p. ISBN 978-0-470-90536-4

1. Daskin M. Network and Discrete Location, Models, Algorithms, and
Applications. -Hoboken: John Willey \& Sons Ltd, 1995.
\href{http://dx.doi.org/10.1002/9781118032343}{DOI
10.1002/9781118032343}

1. Geoffrion A.M., Morris J., Webster S. Distribution System Design.
Facility Location: A Survey of Applications and Methods. -- New York:
Springer, 1995. -P.181--198. ISBN 0387945458.

1. Romeijn H.E., Shu J. \& Teo C.-P. Designing Two-Echelon Supply Network
// European Journal of Operational Research. -2007. -Vol.178(2).
-P.449--462.
DOI 10.1016/j.ejor.2006.02.016

1. Попов П. В. Методология построения логистической инфраструктуры на
территории региона / П. В. Попов, И. Ю. Мирецкий // Экономика региона.
- 2019. - Том 15, вып.2. -С.483-492. DOI
\href{http://dx.doi.org/10.17059/2019-2-13}{10.17059/2019-2-13}

1. Erlebacher S.J., Meller R.D. The interaction of location and inventory
in designing distribution systems // IIE Transactions. -2000. -Vol.
32. -P.155-166. DOI 10.1023/A:1007614431718

1. Tuti Sarma Sinaga, Yosi Agustina Hidayat, Rachmawati Wangsaputra,
Senator Nur Bahagia. The development of a conceptual rural logistics
system model to improve products distribution in Indonesia // Journal
of Industrial Engineering and Management. -- 2022. -- Vol.15(4). -- P.
670-687. DOI 10.3926/jiem.4011

1. Lu L. and Qin J. Multi-regional logistics center location algorithm
based on improved K-means clustering // Computer system application --
2019. -- Vol.28(8). -- P.251--255.

1. Musolino G., Rindone C., Polimeni A. \& Vitetta A. Planning urban
distribution center location with variable restocking demand
scenarios: general methodology and testing in a medium-size town //
Transport Policy. -2019. -Vol.80. -P.157-166. DOI
10.1016/j.tranpol.2018.04.006

1. Ozmen M.Ё. and Aydo˘gan E.K. Robust multi-criteria decision making
methodology for real life logistics center location problem //
Artificial Intelligence Review. --2020. --Vol.53(12). -P.725-751.
DOI 10.1007/s10462-019-09763-y

1. Sharma R. \& Sungheetha A. Analysis of influencing factors of
logistics center location based on comprehensive quality training //
Indian Journal of Public Health Research \& Development. -- 2018. --
Vol.4(12). -- P.289-293. DOI 10.25236/ISMEEM.2019.006
19. Rao C., Goh M., Zhao Y. and Zheng J. Location selection of city
logistics centers under sustainability //
\href{https://www.sciencedirect.com/journal/transportation-research-part-d-transport-and-environment}{Transportation
Research. -2015.
-}Vol.\href{file:///C:/Users/admin/Desktop/загрузки/36}{36}. -P.29-44.
DOI 10.1016/j.trd.2015.02.008

20. Mieczyґnska M. and Czarnowski I. K-means clustering for SAT-AIS data
analysis //WMU Journal of Maritime Affairs. -2021. -Vol.20(3). -
P.377-400. DOI 10.1007/s13437-021-00241-3

21. Gurrala, K., Hariga, M.
\href{https://journal.oscm-forum.org/publication/article/key-food-supply-chain-challenges-a-review-of-the-literature-and-research-gaps}{Key
Food Supply Chain Challenges: A Review of the Literature and Research
Gaps} // Operations and Supply Chain Management. -- 2022. -- Vol.15(2).
-- Pp.441-460. DOI
\href{http://doi.org/10.31387/oscm0510358}{10.31387/oscm0510358}
\end{references}

\begin{center}
{\bfseries References}
\end{center}

\begin{references}
1. Koncepcija gosudarstvennoj programmy razvitija torgovli Respubliki
Kazahstan na 2021
2025 gody: Postanovlenie Pravitel' stva Respubliki
Kazahstan ot 2 marta 2020 goda.
\href{https://www.gov.kz/memleket/entities/mti/documents/details/61426?lang=ru-}{}
Data obrashhenija: 21.11.2024. {[}in Russian{]}

2. Gosudarstvennaja programma razvitija agropromyshlennogo kompleksa
Respubliki Kazahstan
na 2017 -- 2021 gody: Postanovlenie Pravitel' stva
Respubliki Kazahstan ot 12 ijulja 2018 goda № 423.
https://adilet.zan.kz/rus/docs/P1800000423 -Data obrashhenija
25.11.2024. {[}in Russian{]}.

3. \href{https://www.webofscience.com/wos/author/record/34864370}{Feng
Y.Q}.,
\href{https://www.webofscience.com/wos/author/record/33997583}{Liu
Y.K}.,
\href{https://www.webofscience.com/wos/author/record/34807908}{Chen
Y.J}.
\href{https://www.webofscience.com/wos/woscc/full-record/WOS:000859686100002}{Distributionally
robust location-allocation models of~distribution centers for fresh
products with uncertain demands} //
\href{https://www.sciencedirect.com/journal/expert-systems-with-applications}{Expert
Systems with Applications}. - 2022. - Vol.209. - P.518-531.
\href{https://doi.org/10.1016/j.eswa.2022.118180}{DOI
10.1016/j.eswa.2022.118180}

4. Dybskaja V.V. Upravlenie skladirovaniem v cepjah postavok. - M.:
Al' fa-Press, 2009. -- 720 s. ISBN 978-5-94280-355-1.
{[}in Russian{]}

5. Baker P., Canessa M. Warehouse design A structured approach //
European Journal of

Operation Research. -- 2009. -- Vol.193(2). - P.425 -- 436. DOI
\href{http://dx.doi.org/10.1016/j.ejor.2007.11.045}{10.1016/j.ejor.2007.11.045}

6. Anil Varghese Mangalan, Sandeep Kuriakose, H. Mohamed, A. Ray. Optimal
location of warehouse using weighted MOORA approach // International
Conference on Electrical, Electronics and Optimization Techniques
(ICEEOT).-2016. DOI 10.1109/ICEEOT.2016.7754764

7. Volhin E.G. Modeli razmeshhenija raspredelitel' nyh
centrov // Upravlenec. - 2018. -Tom 9. - №2. -- S.54 -- 60. DOI
10.29141/2218-5003-2018-9-2-9. {[}in Russian{]}

8. Daskin M. Network and Discrete Location: Models, Algorithms and
Applications (2nd Edition). - New York: John Wiley \& Sons Inc, 2013. --
544 p. ISBN 978-0-470-90536-4

9. Daskin M. Network and Discrete Location, Models, Algorithms, and
Applications. -Hoboken: John Willey \& Sons Ltd, 1995.
\href{http://dx.doi.org/10.1002/9781118032343}{DOI
10.1002/9781118032343}

10. Geoffrion A.M., Morris J., Webster S. Distribution System Design.
Facility Location: A Survey of Applications and Methods. -- New York:
Springer, 1995. -P.181--198. ISBN 0387945458.

11. Romeijn H.E., Shu J. \& Teo C.-P. Designing Two-Echelon Supply
Network // European Journal of Operational Research. -2007. -Vol.178(2).
-P.449--462.

DOI 10.1016/j.ejor.2006.02.016

12. Popov P. V. Metodologija postroenija logisticheskoj infrastruktury na
territorii regiona / P. V. Popov, I. Ju. Mireckij // Jekonomika regiona.
- 2019. -- T.15, vyp.2. -S.483-492.

DOI \href{http://dx.doi.org/10.17059/2019-2-13}{10.17059/2019-2-13}.
{[}in Russian{]}

13. Erlebacher S.J., Meller R.D. The interaction of location and
inventory in designing distribution systems // IIE Transactions. -2000.
-Vol.32. -P.155-166. DOI 10.1023/A:1007614431718

14. Tuti Sarma Sinaga, Yosi Agustina Hidayat, Rachmawati Wangsaputra,
Senator Nur Bahagia. The development of a conceptual rural logistics
system model to improve products distribution in Indonesia // Journal of
Industrial Engineering and Management. -- 2022. -- Vol.15(4). -- P.
670-687. DOI 10.3926/jiem.4011

15. Lu L. and Qin J. Multi-regional logistics center location algorithm
based on improved K-means clustering // Computer system application --
2019. -- Vol.28(8). -- P.251--255.

16. Musolino G., Rindone C., Polimeni A. \& Vitetta A. Planning urban
distribution center location with variable restocking demand scenarios:
general methodology and testing in a medium-size town // Transport
Policy. -2019. -Vol.80. -P.157-166. DOI 10.1016/j.tranpol.2018.04.006

17. Ozmen M.Ё. and Aydo˘gan E.K. Robust multi-criteria decision making
methodology for real life logistics center location problem //
Artificial Intelligence Review. --2020. --Vol.53(12). -P.725-751. DOI
10.1007/s10462-019-09763-y

18. Sharma R. \& Sungheetha A. Analysis of influencing factors of
logistics center location based on comprehensive quality training //
Indian Journal of Public Health Research \& Development. -- 2018. --
Vol.4(12). -- P.289-293. DOI 10.25236/ISMEEM.2019.006

19. Rao C., Goh M., Zhao Y. and Zheng J. Location selection of city
logistics centers under sustainability //
\href{https://www.sciencedirect.com/journal/transportation-research-part-d-transport-and-environment}{Transportation
Research. -2015.
-}Vol.\href{file:///C:/Users/admin/Desktop/загрузки/36}{36}. -P.29-44.
DOI 10.1016/j.trd.2015.02.008

20. Mieczyґnska M. and Czarnowski I. K-means clustering for SAT-AIS data
analysis //WMU Journal of Maritime Affairs. -2021. -Vol.20(3). -
P.377-400. DOI 10.1007/s13437-021-00241-3

21. Gurrala, K., Hariga, M.
\href{https://journal.oscm-forum.org/publication/article/key-food-supply-chain-challenges-a-review-of-the-literature-and-research-gaps}{Key
Food Supply Chain Challenges: A Review of the Literature and Research
Gaps} // Operations and Supply Chain Management. -- 2022. -- Vol.15(2).
-- Pp.441-460. DOI
\href{http://doi.org/10.31387/oscm0510358}{10.31387/oscm0510358}
\end{references}

\begin{authorinfo}
\emph{{\bfseries Сведения об авторе}}

Сыздыкбаева Б.У. -- д.э.н., профессор Евразийского национального
университета имени Л.Н. Гумилева, Астана, Казахстан, e-mail:
\href{mailto:bakyt_syzdykbaeva@mail.ru}{}

\emph{{\bfseries Information about the author}}

Syzdykbayeva B.U. - doctor of economic sciences, professor of
L.N. Gumilyov Eurasian National University, Astana, Kazakhstan, e-mail:
\href{mailto:bakyt_syzdykbaeva@mail.ru}{}
\end{authorinfo}
