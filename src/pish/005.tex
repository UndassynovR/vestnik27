\id{IRSTI 65.59.31}{}

\begin{articleheader}
\sectionwithauthors{Sh.B. Baitukenova, А.D. Kalitova, S.B. Baitukenova, U.A. Ryspaeva, S.A. Kardenov}{A DEVELOPMENT OF BOILED SAUSAGE TECHNOLOGY BASED ON HORSE MEAT}

{\bfseries
\textsuperscript{1}Sh.B. Baitukenova\alink{https://orcid.org/0000-0003-0200-8455}\textsuperscript{\envelope },
\textsuperscript{1}А.D. Kalitova\alink{https://orcid.org/0009-0001-3160-2528},
\textsuperscript{2}S.B. Baitukenova\alink{https://orcid.org/0000-0001-8200-4280},
\textsuperscript{2}U.A. Ryspaeva\alink{https://orcid.org/0000-0002-5862-9085},
\textsuperscript{2}S.A. Kardenov\alink{https://orcid.org/0000-0001-6198-1189}
}
\end{articleheader}

\begin{affiliation}
\textsuperscript{1}Kazakh Agrotechnical Research University named after S. Seifullin, Astana, Kazakhstan,

\textsuperscript{2} Kazakh University of Technology and Business named after K. Kulazhanov, Astana, Kazakhstan

\raggedright \textsuperscript{\envelope }Корреспондент-автор: baytukenova75@mail.ru
\end{affiliation}

The article presents experimental materials on the development and
optimization of a recipe for boiled sausage based on horse meat with the
addition of a balanced mixture of powders. During the development of the
new product, the formulation of boiled sausage "Donskoy" was taken as a
control sample, powders of dried beetroot and orange peel in various
concentrations were added to the formulation. For the research, a
control sample and three samples of boiled sausages were produced with
the addition of a mixture of powders in a ratio of 1:1 in the amount of
0.5, 1.0 and 1.5\%. In the course of the research, optimal doses of
additives in the amount of 1.0\% were identified. Organoleptic,
physicochemical and microbiological parameters were studied in samples
of boiled sausages.

The formulation and production technology of a new product has been
developed -- horse boiled sausage with the addition of a mixture of
dried beetroot and orange peel powders. A comparative analysis showed
that the use of beetroot and orange peel powders of 1.0\% each in a 1:1
ratio improves the organoleptic characteristics of boiled sausage
without worsening microbiological parameters and without reducing its
safety. The use of powders led to a slight improvement in the functional
and technological properties of the sausage, improved its quality and
increased the yield of finished products.

{\bfseries Keywords:} horsemeat; boiled sausage made from horsemeat, food
additives; mixtures of dried beetroot powder and orange peel,
nutritional value, organoleptic and microbiological parameters,
physicochemical properties.

\begin{articleheader}
{\bfseries РАЗРАБОТКА ТЕХНОЛОГИИ ВАРЕНОЙ КОЛБАСЫ НА ОСНОВЕ МЯСА КОНИНЫ}

{\bfseries
\textsuperscript{1}Ш.Б.Байтукенова\textsuperscript{\envelope },
\textsuperscript{1}А.Д.Калитова,
\textsuperscript{2}С.Б.Байтукенова,
\textsuperscript{2}У.А.Рыспаева,
\textsuperscript{1}С.А. Карденов
}
\end{articleheader}

\begin{affiliation}
\textsuperscript{1}НАО «Казахский агротехнический исследовательский
университет имени С. Сейфуллина», Астана, Казахстан,

\textsuperscript{2}АО «Казахский университет технологии и бизнеса имени
К. Кулажанова», Астана, Казахстан

e-mail: baytukenova75@mail.ru
\end{affiliation}

В статье представлены экспериментальные материалы по разработке и
оптимизации рецептуры вареной колбасы на основе мяса конины с
добавлением сбалансированной смеси порошков. При разработке нового
продукта была взята за контрольный образец рецептура вареной колбасы
первого сорта «Донская», в рецептуру были добавлены порошки сушеных
свеклы и апельсиновой цедры в различных концентрациях. Для исследований
были изготовлены контрольный образец и три образца вареных колбас с
добавлением смеси порошков в соотношении 1:1 в количестве 0,5, 1,0 и
1,5\%. В ходе исследований были выявлены оптимальные дозы внесения
добавок в количестве 1,0\%. В образцах вареных колбас изучались
органолептические, физико-химические и микробиологические показатели.

Разработана рецептура и технология производства нового продукта --
конская вареная колбаса с добавлением смеси порошков из сушеных свеклы и
апельсиновой цедры. Сравнительный анализ показал, что использование
порошков свеклы и апельсиновой цедры по 1,0\% в соотношении 1:1 улучшает
органолептические показатели вареной колбасы без ухудшения
микробиологических показателей и без снижения ее безопасности.
Использование порошков привело к незначительному улучшению
функционально-технологических свойств колбасы, позволило улучшение ее
качества и увеличение выхода готовой продукции.

{\bfseries Ключевые слова:} конина, вареная колбаса из конины, пищевые
добавки, смеси порошка из сушеных свеклы и апельсиновой цедры, пищевая
ценность, органолептические и микробиологические показатели,
физико-химические свойства.

\begin{articleheader}
{\bfseries ЖЫЛҚЫ ЕТІНЕН ПІСІРІЛГЕН ШҰЖЫҚ ТЕХНОЛОГИЯСЫН ЖАСАУ}

{\bfseries
\textsuperscript{1}Ш.Б.Байтукенова\textsuperscript{\envelope },
\textsuperscript{1}А.Д.Калитова,
\textsuperscript{2}С.Б.Байтукенова,
\textsuperscript{2}У.А.Рыспаева,
\textsuperscript{1}С.А. Карденов
}
\end{articleheader}

\begin{affiliation}
\textsuperscript{1}КеАҚ «С. Сейфуллин атындағы Қазақ агротехникалық зерттеу университеті», Астана, Қазақстан,

\textsuperscript{2}АҚ «Қ. Құлажанов атындағы Қазақ технология және бизнес университеті», Астана, Қазақстан,

e-mail: baytukenova75@mail.ru
\end{affiliation}

Мақалада теңдестірілген ұнтақ қоспасын қосу арқылы жылқы етіне
негізделген пісірілген шұжық өнімінің рецептурасын әзірлеу және
оңтайландыру бойынша эксперименттік материалдар берілген. Жаңа өнімді
әзірлеу кезінде «Донская» 1-ші сұрыпты пісірілген шұжықтың рецептурасы
бақылау үлгісі ретінде алынды, рецептура құрамына әртүрлі
концентрациядағы кептірілген қызылша мен апельсин қабығының ұнтақтары
қосылды. Зерттеулер жүргізу барысында 0,5, 1,0 және 1,5 \% мөлшерінде
ұнтақ қоспасы қосылып, бақылау үлгісі мен пісірілген шұжықтардың үш
үлгісі жасалды. Зерттеулер нәтижесінде 1,0 \% мөлшерінде қоспаларды
енгізудің оңтайлы мөлшері анықталды. Пісірілген шұжықтар үлгілерінде
органолептикалық, физика-химиялық және микробиологиялық көрсеткіштері
зерттелді.

Жаңа өнімнің рецептурасы мен технологиясы, яғни кептірілген қызылша мен
апельсин қабығынан дайындалған ұнтақ қоспасы қосылған жылқы етінен
пісірілген шұжық әзірленді. Салыстырмалы талдау көрсеткендей, қызылша
мен апельсин қабығының ұнтақтарын 1:1 қатынасында пісірілген шұжықтың
құрамына 1,0 \% қолдану органолептикалық және микробиологиялық
көрсеткіштерді төмендетпейді және оның қауіпсіздігін жақсартады.
Пісірілген шұжық өндірісінде ұнтақ қоспасын қолдану дайын өнімнің
функционалдық-технологиялық қасиеттерінің біршама жақсаруына әкелді,
оның сапасын жақсартуға және шығымдылығын арттыруға мүмкіндік берді.

{\bfseries Түйінді сөздер:} жылқы еті, жылқы етінен пісірілген шұжық,
тағамдық қоспалар,кептірілген қызылша мен апельсин қабығынан алынған
ұнтақ қоспалары, тағамдық құндылығы, органолептикалық және
микробиологиялық көрсеткіштер, физика-химиялық қасиеттері.

\begin{multicols}{2}
{\bfseries Introduction.} The development of food security and a
sustainable agricultural sector are key priorities for Kazakhstan, as
the President of the Republic of Kazakhstan has repeatedly emphasized in
his annual Messages to the people. In his last address, President
Kassym-Jomart Tokayev stressed the importance of food independence,
stating: "We must strengthen food security by increasing domestic
agricultural production and developing the processing of agricultural
raw materials." These words confirm the strategic importance of local
resources and traditional methods in increasing the
country' s food potential {[}1{]}.

Horse meat, traditionally eaten in Kazakhstan, fits perfectly into this
concept. With rising global meat prices and a growing demand for
healthier foods, horsemeat provides a unique opportunity to increase
both nutritional value and economic efficiency.

Horse meat is rich in protein, amino acids and other essential
nutrients, making it an ideal product for innovative processing. In
addition, compared to other types of meat, it contains less fat and
cholesterol, which meets modern nutritional requirements.

The use of beetroot powder and orange peel in the production of boiled
horse meat sausages opens up new opportunities for improving their
characteristics. Beetroot powder is a natural source of antioxidants, B
vitamins, iron and dietary fiber, and also gives sausages a rich color,
improving their organoleptic properties.

Orange peel powder has been used in the formulation of boiled horse meat
sausage due to its natural antioxidant and aromatic properties. It
contains essential oils and vitamin C. The addition of zest powder
improves the organoleptic characteristics of the product, giving it a
light citrus aroma and enhancing the overall taste appeal. The use of
this component reduces the need for synthetic additives and flavor
enhancers, which makes the product more natural and safer for the
consumer.

The purpose of this work was to develop boiled sausages using a mixture
of dried beetroot and orange peel powders.

To achieve the goal, the following tasks were set:

- to justify the choice of the proposed vegetable ingredient -- a
mixture of dried beetroot powder and orange peel -- in the production of
boiled sausages;

- to determine the optimal dose of the herbal component in the
formulations of enriched boiled and smoked sausages;

- to study the organoleptic, microbiological and physicochemical
parameters of the finished product;

- select the optimal ratio of vegetable components, develop a recipe and
present a technological scheme for the production of enriched boiled and
smoked sausages.

The use of natural additives in the production of boiled horse meat
sausages not only improves their organoleptic and nutritional
properties, but also helps to expand the range of functional meat
products that meet modern requirements of a healthy diet.

Scientists pay special attention to the influence of both animal and
vegetable raw materials on the properties of boiled sausages,
emphasizing the importance of choosing ingredients to achieve the
necessary characteristics. In the works devoted to the preservative
ability of extracts of moringa leaves and orange peel in chicken
sausages, the authors evaluated their antioxidant effect and the ability
to prolong shelf life. The antioxidant and preservative properties of
moringa and orange peel extracts in chilled chicken sausages were
studied, revealing that they slow down fat oxidation and increase shelf
life {[}2, 3, 4{]}.

The proposed article examines the effect of dried beetroot and orange
peel powders on the physicochemical, microbiological and organoleptic
properties of boiled sausage, resulting in improved color, juiciness and
texture. Unlike extracts, the use of powders simplifies the
technological process, making them convenient for production.

The scientists also used beetroot as a source of natural dyes for ham.
The effect of beetroot extracts (Beta vulgaris L.) on the staining of
meat products (ham) and their possible cytotoxicity against the AGS cell
line was studied. The extracts were used encapsulated in nanosystems
based on soy lecithin and maltodextrin, and then added to the ham
formulation during pilot production. The color of the finished product
was visually assessed using colorimetry {[}5,6{]}.

In contrast to this study, we studied the effect of dried beetroot and
orange peel powders on the quality of boiled sausage, including
physicochemical, organoleptic, technological and microbiological
parameters. The main attention was paid to changing the moisture-binding
(MBC), moisture-retaining (MRC) and fat-retaining (FRC) properties of
minced meat and the finished product. Instead of extraction and
encapsulation, whole vegetable powders were used, which not only
affected the color of the sausage, but also improved its juiciness and
texture.

{\bfseries Materials and methods.} The following materials were used for
the study: grade 1 cored horse meat, raw lamb fat, poultry meat,
beetroot powder and powdered orange peel. The «Donskoy» sausage,
developed in accordance with GOST 31780-2012, served as a prototype for
the development of formulations. Experimental samples of boiled sausage
were prepared with the addition of beetroot and orange peel powders in
various concentrations (0,5\%, 1,0\%, 1,5\%) {[}7{]}.

The mass fraction of proteins, fats, and carbohydrates in the finished
product was determined. Functional and technological properties were
evaluated: moisture-binding capacity (MBC), moisture-retaining capacity
(MRC), fat-retaining capacity (FRC). The organoleptic evaluation of
boiled sausages was carried out in accordance with GOST 9959-2015
{[}8{]}. Microbiological parameters were studied in accordance with GOST
54354-2011 {[}9{]}.

The acidic medium (pH) in the finished product using beetroot powder and
orange peel was determined in a pH meter. The shelf life was determined
by the dynamics of microbiological parameters (TAMC, CFU/g) during 14
days of storage at a temperature of 4°C {[}10{]}. The organoleptic
analysis was performed using a 5-point scale, which included an
assessment of color, aroma, taste and texture.15 tasters participated
in the study.

{\bfseries Results and discussion.} Horse meat is a valuable source of
high-quality protein with a low fat content, which makes it a dietary
raw material for sausage products. It is rich in unsaturated fatty
acids, B vitamins (B\textsubscript{12}, B\textsubscript{6}, niacin) and
minerals (iron, zinc, selenium) that help improve metabolism and
immunity. The caloric content is 120-150 kcal / 100 g, which is
significantly less than in other types of red meat.

Beetroot powder improves the color of the product due to betalains. It
also gives it a light sweetish taste. During storage, the color remained
stable, which confirms the resistance of the added ingredients to
oxidative processes.

Orange peel powder enriches the product with vitamin C and fiber, giving
it a light citrus flavor.

The use of these ingredients increases the nutritional and functional
value of sausages in line with healthy eating trends.

As a result of the conducted research, it was found that minced meat
from horse meat is a favorable environment for the uniform distribution
of dried beetroot and orange peel powder. The optimal dosage of
additives ranged from 0.5\% to 1.5\% of the total weight of the raw
material in the ratio of beet powder and orange peel 1:1. This ratio
makes it possible to obtain boiled sausage with improved organoleptic
properties, increased nutritional value and additional health benefits
due to the content of natural antioxidants and dietary fiber.

The addition of beetroot powder and orange peel had no significant
effect on the pH of minced meat (6.1±0.2). Microbiological analysis
showed that the content of TAMC in the control sample on the 10th day of
storage was 4.2×10⁴ CFU/g, while in the samples with additives it was
3.8×10⁴ CFU/g. This indicates a possible antimicrobial effect of the
powders. At the same time, the growth of pathogenic microflora of E.
Coli and S. aureus was not recorded during the entire shelf life.

During the development of the new product, the formulation of boiled
sausage «Donskoy» was used, powders of dried beetroot and orange peel in
various concentrations were added to the formulation (Table 1). The
addition of these ingredients made it possible to improve the color
characteristics, taste and aroma of the finished product, as well as
enhance its functional and technological properties.

Boiled sausage made from horse meat with the addition of dried beetroot
powder and dried orange peel powder turned out to be fragrant and
unusual. Beetroot gave it a beautiful pinkish hue and a light sweetness,
which harmoniously combined with the natural taste of horse meat. The
orange peel added a delicate citrus flavor and a subtle bitterness,
refreshing the overall taste. The consistency turned out to be dense,
but juicy due to good moisture retention. When sliced, the sausage had a
pleasant spicy aroma with hints of coriander and nutmeg. The taste was
balanced, with a slight piquancy and natural sweetness. This product
turned out to be original and could interest fans of unusual meat
products. The addition of these powders slightly changed the energy
value of the sausage. The carbohydrate content in beetroot against the
background of the overall composition of the product did not
significantly affect the calorie content. The orange peel added some
dietary fiber and essential oils, but its mass in the formulation was
small.

Organoleptic evaluation of boiled horse meat sausage. The samples were
evaluated on a 5-point scale, where 5 is the highest score that meets
the quality criteria. Parameters such as appearance, color in section,
aroma, taste, texture, juiciness, and overall score were studied.

The 2nd table shows the average values of the estimates for the control
sample and experimental samples with the addition of 0.5\%, 1\% and
1.5\% beetroot powder and orange peel.
\end{multicols}

\tcap{Table 1 -- Formulation of control and experimental samples of boiled sausages}
\begin{longtblr}[
  label = none,
  entry = none,
]{
  width = \linewidth,
  colspec = {Q[300]Q[200]Q[100]Q[100]Q[100]},
  row{1} = {c},
  row{2} = {c},
  cell{1}{1} = {r=2}{},
  cell{1}{2} = {r=2}{},
  cell{1}{3} = {c=3}{},
  cell{3}{2} = {c},
  cell{3}{3} = {c},
  cell{3}{4} = {c},
  cell{3}{5} = {c},
  cell{4}{2} = {c},
  cell{4}{3} = {c},
  cell{4}{4} = {c},
  cell{4}{5} = {c},
  cell{5}{2} = {c},
  cell{5}{3} = {c},
  cell{5}{4} = {c},
  cell{5}{5} = {c},
  cell{6}{2} = {c},
  cell{6}{3} = {c},
  cell{6}{4} = {c},
  cell{6}{5} = {c},
  cell{7}{2} = {c},
  cell{7}{3} = {c},
  cell{7}{4} = {c},
  cell{7}{5} = {c},
  cell{8}{2} = {c},
  cell{8}{3} = {c},
  cell{8}{4} = {c},
  cell{8}{5} = {c},
  cell{9}{2} = {c},
  cell{9}{3} = {c},
  cell{9}{4} = {c},
  cell{9}{5} = {c},
  cell{10}{2} = {c},
  cell{10}{3} = {c},
  cell{10}{4} = {c},
  cell{10}{5} = {c},
  cell{11}{2} = {c},
  cell{11}{3} = {c},
  cell{11}{4} = {c},
  cell{11}{5} = {c},
  vlines,
  hline{1,3-12} = {-}{},
  hline{2} = {3-5}{},
}
Ingredients, \%                                                       & {The control sample\\Boiled sausage «Donskoy»\\(GOST 31780-2012)} & {Experimental samples with the \\addition of a mixture of dried beetroot \\powder and orange peel} &              &              \\
                                                                      &                                                                   & №1                                                                                           & №2           & №3           \\
High-grade veneered horse meat                                        & 75,0                                                              & 60,0                                                                                         & 60,0         & 60,0         \\
Raw mutton fat                                                        & 25,0                                                              & 15,0                                                                                         & 15,0         & 15,0         \\
Poultry meat                                                          & -                                                                 & 25,0                                                                                         & 25,0         & 25,0         \\
\textbf{Total:}                                                       & \textbf{100}                                                      & \textbf{100}                                                                                 & \textbf{100} & \textbf{100} \\
A mixture of dried beetroot and dried orange peel powders             & -                                                                 & 0,5                                                                                          & 1,0          & 1,5          \\
\textbf{Spices and materials kg per 100 kg of unsalted raw materials} &                                                                   &                                                                                              &              &              \\
Table salt                                                            & 2,5                                                               & 2,5                                                                                          & 2,5          & 2,5          \\
Food additives and spices                                             & 0,21                                                              & 0,21                                                                                         & 0,21         & 0,21         \\
Added water                                                           & 0,2                                                               & 0,2                                                                                          & 0,2          & 0,2          
\end{longtblr}

\tcap{Table 2 -- Organoleptic evaluation of boiled horse meat sausage with the addition of dried beet powder and orange zest}
\begin{longtblr}[
  label = none,
  entry = none,
]{
  width = \linewidth,
  colspec = {Q[150]Q[96]Q[254]Q[244]Q[254]},
  cells = {c},
  hlines,
  vlines,
}
Parameter     & Control sample & Prototype with addition of 0.5\% powder mixture & Prototype with addition of 1\% powder mixture & Prototype with addition of 1.5\% powder mixture \\
Appearance    & 5              & 4.6                                             & 4.7                                           & 4.6                                             \\
Color on cut  & 5              & 4.7                                             & 4.9                                           & 4.6                                             \\
Aroma         & 5              & 4.8                                             & 4.9                                           & 4.6                                             \\
Taste         & 5              & 4.7                                             & 4.7                                           & 4.6                                             \\
Texture       & 5              & 4.5                                             & 4.5                                           & 4.4                                             \\
Jusiness      & 5              & 4.4                                             & 4.4                                           & 4.4                                             \\
Overall score & 5              & 4.6                                             & 4.7                                           & 4.5                                             \\
Average score & 5              & 4,6                                             & 4,7                                           & 4,5                                             
\end{longtblr}

\begin{multicols}{2}
Organoleptic evaluation boiled horse meat sausage with the addition of
1\% beetroot powder and orange peel powder showed high performance
according to several criteria. The structure of the product turned out
to be smooth and uniform, with a natural color and small inclusions of
beetroot powder and orange peel. Thanks to the beetroot, the color in
the section acquired a slightly pinkish tinge, which made the appearance
more attractive.

The aroma of the product was rich, with pronounced meat notes and light
fruity and citrus notes, giving it originality. The taste turned out to
be harmonious and rich, with moderate saltiness and light sweetness from
beetroot, as well as fresh citrus aromas that balanced the overall
taste.

The texture of the product is dense, juicy and elastic, making it easy
to chew. The juiciness was good; the meat didn' t feel
dry. The visual and taste characteristics of this sample were rated
higher than those of the variants with 0.5\% to 1.5\% powder mixture.
The color, aroma, and overall flavor balance were particularly good.
Compared to the control sample, the product was slightly inferior in
terms of traditional meat saturation, but it was distinguished by an
interesting combination of meat flavor with light fruity notes.

In general, the addition of 1\% powder of beetroot and orange peel led
to an improvement in the appearance, aroma and taste of the sausage,
preserving its juiciness and dense texture. This sample can be
considered the most successful among the experimental ones, as it
demonstrated high scores on key parameters and a well-balanced taste
profile.

Table 3 shows the results of measuring acidity using a pH meter. During
the study, the pH value in the finished product was measured using a pH
meter. The average pH value in the control sample was 6,2±0,1, while in
the sample was 6,2±0,1. These results indicate that the additives did
not significantly affect the acidity of the product, keeping it within
the standard vales.
\end{multicols}

\tcap{Table 3 -- The acidity values of the sample with 1\% of the additive and the control sample}
\begin{longtblr}[
  label = none,
  entry = none,
]{
  width = \linewidth,
  colspec = {Q[685]Q[252]},
  hlines,
  vlines,
}
Sample                                            & Average pH value \\
Control sample (without additives)                & 6,2±0,1          \\
With addition of beetroot and orange peel powders & 6,1±0,1          
\end{longtblr}

Table 4 shows the organoleptic and microbiological parameters of the
control and experimental samples.

\tcap{Table 4 - Organoleptic and microbiological indicators of boiled sausages}
\begin{longtblr}[
  label = none,
  entry = none,
]{
  width = \linewidth,
  colspec = {Q[131]Q[188]Q[208]Q[202]Q[210]},
  row{odd} = {c},
  row{2} = {c},
  row{8} = {c},
  row{10} = {c},
  cell{1}{1} = {r=2}{},
  cell{1}{2} = {c=4}{0.808\linewidth},
  cell{3}{1} = {c=5}{0.939\linewidth},
  cell{4}{1} = {c},
  cell{4}{2} = {c=4}{0.808\linewidth},
  cell{5}{2} = {c=4}{0.808\linewidth},
  cell{6}{1} = {c},
  cell{7}{1} = {c=5}{0.939\linewidth},
  vlines,
  hline{1,3-11} = {-}{},
  hline{2} = {2-5}{},
}
Name of the indicator                                    & Value of the indicator for sausages                                                                    &                                                                                                             &                                                                                                       &                                                                                                            \\
                                                         & Control sample                                                                                         & Sample with addition of 0,5\% powder mixture                                                                & Sample with addition of 1\% powder mixture                                                            & Sample
with addition of  1,5\% powder mixture                                                              \\
Organoleptic indicators:                                 &                                                                                                        &                                                                                                             &                                                                                                       &                                                                                                            \\
Appearance                                               & Loaves with a clean, dry surface, without damaged shells, minced meat drips, slips and fatty swellings &                                                                                                             &                                                                                                       &                                                                                                            \\
Consistency                                              & Elastic                                                                                                &                                                                                                             &                                                                                                       &                                                                                                            \\
Color and appearance of minced meat on the cut           & The mince is pink in color, evenly mixed, without grey spots and contains the pieces of raw mutton fat & The mince is light red in color, evenly mixed, without grey spots and contains the pieces of raw mutton fat & The mince is red in color, evenly mixed, without grey spots and contains the pieces of raw mutton fat & The mince is dark red in color, evenly mixed, without grey spots and contains the pieces of raw mutton fat \\
Microbiological indicators:                              &                                                                                                        &                                                                                                             &                                                                                                       &                                                                                                            \\
TBC (Total Bacterial Count), CFU/g                       & ≤ 1×10⁵                                                                                                & ≤ 1×10⁵                                                                                                     & ≤ 1×10⁵                                                                                               & ≤ 1×10⁵                                                                                                    \\
Coliforms (BGKP), CFU/0.01 g                             & Not allowed                                                                                            & Not detected                                                                                                & Not detected                                                                                          & Not detected                                                                                               \\
{Pathogenic micro\-organisms, including Salmonella, in 25 g} & Not allowed                                                                                            & Not detected                                                                                                & Not detected                                                                                          & Not detected                                                                                               
\end{longtblr}

\begin{multicols}{2}
The addition of powders did not lead to a deterioration in organoleptic
characteristics, but, on the contrary, contributed to an improvement in
color and taste properties. The color of the sausage has become more
saturated due to the natural beetroot pigment, and the aroma has
acquired light citrus notes, which makes the product more attractive to
consumers. The consistency remained homogeneous and dense, which
indicates the preservation of the technological properties of the minced
meat.

The microbiological parameters of the samples comply with safety
requirements. No pathogenic microorganisms (Salmonella, Listeria
monocytogenes, S. aureus) were found in all variants, and the total
number of bacteria was within acceptable values. This indicates that the
addition of powders did not adversely affect the microbiological purity
of the product.

A comparative analysis showed that the use of beetroot and orange peel
powders of 1\% each in a 1:1 ratio improves the organoleptic
characteristics of boiled sausage without deterioration of
microbiological parameters and without reducing its safety. In this
regard, the results of the analyses of this sample are presented in the
following study.

The main caloric content of the finished product was provided by horse
meat proteins and fats, and the total energy value remained
approximately the same as in the control sample. In general, the changes
in the composition had a greater effect on taste than on nutritional
value. Table 5 below shows comparative indicators of the chemical
composition of boiled sausages (control and experimental samples with
the addition of 1\% powder mixture in a ratio of 1:1).
\end{multicols}

\tcap{Table 5 -- Chemical composition of boiled and smoked sausages (control and experimental samples)}
\begin{longtblr}[
  label = none,
  entry = none,
]{
  width = \linewidth,
  colspec = {Q[304]Q[183]Q[454]},
  cells = {c},
  hlines,
  vlines,
}
Indicator                          & The control sample & A prototype with the addition of 1 \% powder mixture \\
Mass fraction of moisture, \%      & 62,5±0,3           & 63,0±0,2                                             \\
Mass fraction of protein, \%       & 12,0±0,3           & 11,0±0,3                                             \\
Mass fraction of fat, \%           & 23,0±0,2           & 21,7±0,2                                             \\
Mass fraction of carbohydrates, \% & 1,0±0,3            & 2,3±0,2                                              \\
Mass fraction of ash, \%           & 1,5±0,4            & 2,0±0,3                                              \\
Energy value, kcal/100 g           & 259,0±0,2          & 248,5±0,3                                            
\end{longtblr}

\begin{multicols}{2}
The consistency of the finished product remained dense, but became a
little juicier due to the ability of beetroot to retain moisture. The
nutritional value of boiled horsemeat sausage with the addition of a
powdered mixture has changed slightly, mainly due to a slight increase
in the proportion of carbohydrates. The mass fraction of moisture
decreased slightly, while the carbohydrate content increased to about
1.5\%. The protein and fat content remained virtually unchanged, while
the ash content increased slightly. As a result, the energy value
decreased from 259.0 kcal/100 g to about 248.5 kcal/100 g. These changes
did not significantly affect the calorie content, but made the product
juicier in taste and attractive in appearance.

To analyze the effect of the addition of dried beetroot and orange peel
powders on the properties of minced meat and boiled sausage, their main
physicochemical parameters were studied (Table 6). For comparison, four
options were considered:

- raw minced meat without added powders -- the initial composition of a
mixture of horse meat, fat and chicken meat;

- boiled sausage «Donskoy» is the final product after heat treatment
without additives;

- raw minced meat with the addition of 1\% powders -- the composition of
minced meat after the introduction of beetroot and orange peel;

- boiled sausage with the addition of 1\% powders is a finished product
after heat treatment with additives.
\end{multicols}


\tcap{Table 6 -- Results of moisture binding capacity (MBC), moisture retention capacity (MRC), fat retention capacity (FRC)}
\begin{longtblr}[
  label = none,
  entry = none,
]{
  width = \linewidth,
  colspec = {Q[100]Q[212]Q[144]Q[263]Q[252]},
  cells = {c},
  cell{1}{1} = {r=2}{},
  cell{1}{2} = {c=4}{},
  vlines,
  hline{1,3-6} = {-}{},
  hline{2} = {2-5}{},
}
Indicators & The studied samples                       &                          &                                                         &                                                        \\
           & Raw minced meat without added ingredients & Boiled sausage «Donskoy» & Raw minced meat with the addition of 1\% powder mixture & Boiled sausage with the addition of 1\% powder mixture \\
MBC, \%    & 59,5±0,3                                  & 53,46±0,4                & 59,71±0,3                                               & 53,7±0,4                                               \\
MRC, \%    & 50,4±0,2                                  & 42,84±0,3                & 50,65±0,2                                               & 43,05±0,3                                              \\
FRC, \%    & 70,0±0,4                                  & 66,5±0,3                 & 69,7±0,3                                                & 66,22±0,5                                              
\end{longtblr}

\begin{multicols}{2}
In the sausage production process, moisture-binding (MBC),
moisture-retaining (MRC) and fat-retaining (FRC) properties are
important, since they determine the consistency, juiciness and stability
of the fat phase. The MBC shows how much moisture minced meat can retain
before heat treatment, the MBC characterizes the amount of moisture
remaining after cooking, and the MBC reflects the ability of a meat
product to retain fat when heated.

Without the addition of powders, the initial MBC was 59.5\%, but after
heat treatment it decreased to 53.46\% due to moisture loss during
heating. Similarly, the percentage of fat decreased from 50.4\% to
42.84\%, as some of the water evaporated. The fat content has also
decreased from 70.0\% to 66.5\%, as some of the fat melts during
cooking.

The addition of 1\% dried beetroot powder and orange peel led to a
slight increase in MBC to 59.71\%, since the plant components are highly
hygroscopic. After cooking, the FRC also remained higher than in the
control sample, amounting to 53.7\%. The concentration of additives also
increased (50.65\% in the raw and 43.05\% in the cooked product),
indicating better moisture retention after heat treatment.

On the contrary, the fat content decreased slightly -- 69.7\% in raw and
66.22\% in cooked foods, since powders do not contribute to fat
retention. However, these changes are minor and do not impair the
quality of the sausage. In general, the addition of powders made it
possible to reduce moisture loss during cooking, which had a positive
effect on the juiciness of the finished product.

The use of powders has led to a slight improvement in the
moisture-retaining properties of sausage, which can be useful for
improving its quality and increasing the yield of finished products.

Based on the conducted research, a technology for the production of
boiled sausages using a mixture of powders from dried beetroot and dried
orange peel has been developed (Scheme 1).
\end{multicols}

{\bfseries Scheme 1 -- The technological scheme of production of boiled sausages with the addition of dried beetroot powder and orange peel}

\begin{multicols}{2}
This scheme is based on the traditional technology of production of
boiled sausage, but adapted for horse meat with natural additives. The
deposition stage ensures an even distribution of moisture and
stabilization of the structure, and heat treatment guarantees safety by
destroying pathogenic microorganisms. Optimal storage conditions help to
preserve freshness and extend shelf life without compromising quality.

{\bfseries Conclusion.} Based on the conducted studies of organoleptic,
physicochemical and microbiological parameters, it was found that the
combination of horse meat with the addition of 1\% dried beet powder and
orange peel in a ratio of 1:1 ensures the production of boiled sausage
with high quality characteristics. The final product has a balanced
taste with light sweet citrus notes, pleasant aroma and attractive
appearance. The sausage has a juicy and delicate consistency, uniform
pink color due to the addition of beetroot, as well as the absence of
foreign odors and flavors.

The developed prototypes of boiled sausage with the addition of dried
beetroot and orange peel powders can be recommended as a useful and safe
product for wide use, which has not only high taste qualities, but also
additional nutritional value due to the inclusion of natural plant
components.

The results obtained demonstrate that the proposed recipe for boiled
sausage with the addition of beetroot powder and orange peel has
advantages over known analogues. Additives improve the color, texture
and moisture-binding properties of the product, reduce the level of
lipid oxidation and reduce the amount of synthetic components, while
maintaining regulatory pH values and microbiological parameters.
\end{multicols}

\begin{center}
{\bfseries References}
\end{center}

\begin{references}
1. Tokaev K.Zh. (2023). Poslanie Glavy gosudarstva Kasym-Zhomarta
Tokaeva narodu Kazahstana \\«Jekonomicheskij kurs Spravedlivogo
Kazahstana». Oficial' nyj sajt Prezidenta Respubliki
Kazahstan.
\href{https://www.akorda.kz/ru/poslanie-glavy-gosudarstva-kasym-zhomarta-tokaeva-narodu-kazahstana-ekonomicheskiy-kurs-spravedlivogo-kazahstana-18588}{https://www.akorda.kz}
- Data obrashhenija: 28.10.2024.{[}in Russian{]}

2. Bishnoi, S., Yadav, S., Jairath, G., Mohamed Ahmed, I. A., Rani, M.,
\& Singh, Y. (2025). Quality and microbial assessment of chicken
sausages treated with moringa leaf and orange peel green extracts//CyTA
- Journal of Food.-2024.-Vol.23(1).- P.1-12.\\
DOI 10.1080/19476337.2024.2446835

3. Aykın-Dinçer, E., Güngör, K., Çağlar, E. \& Erbaş, M. The use of
beetroot extract and extract powder in sausages as natural food
colorant//International Journal of Food Engineering, 2021.-17(1).-
P.75-82. \url{https://doi.org/10.1515/ijfe-2019-0052}{}

4. V' jun M.A. Proizvodstvo sardelek s
ispol' zovaniem rastitel' noj dobavki
«svekol' nyj poroshok» // Materialy VI Mezhdunarodnoj
studencheskoj nauchnoj konferencii «Studencheskij nauchnyj forum» URL:
https://scienceforum.ru/2014/article/2014006458. {[}in Russian{]}

5. Dias S., Pereira D. M., Castanheira E. M. S., Fortes A. G., Pereira
R., \& Gonçalves M. S. T. (2019, November). Beetroot as a source of
natural dyes for ham//Proceedings.-2019.-Vol.41(1) 41(1):82\\
10.3390/ecsoc-23-06626

6. Valentik M., Stepanjanc V., Sokolova Ju.D. (2021) Issledovanie
jekstrakcii pigmentov iz svekly stolovoj razlichnyh sortov v zavisimosti
ot temperatury sushki korneplodov// Tendencii razvitija

nauki i obrazovanija.-2021.-№70(2).- S.6-10.{[}in Russian{]}

7. GOST 31780-2012. Kolbasy varenye iz koniny. Mezhgosudarstvennyj
standart. Moskva, Standartinform, 2013 g. {[}in Russian{]}

8. GOST 9959-91. Produkty mjasnye. Obshhie uslovija provedenija
organolepticheskoj ocenki. {[}in \\Russian{]}

9. GOST R 54354-2011. Mjaso i mjasnye produkty. Obshhie trebovanija i
metody mikrobiologicheskogo analiza. {[}in Russian{]}

10. ST RK 1730-2007. Mjaso i mjasnye produkty. Obshhie tehnicheskie
uslovija. {[}in Russian{]}
\end{references}

\begin{authorinfo}
\emph{{\bfseries Information about the authors}}

Baitukenova Sh.B. - Candidate of Technical Sciences, Associate Professor
of the Department ``Technology of food and process\-ing industries'',
Kazakh Agrotechnical Research University named after S.Seyfullin,
Astana, Kazakhstan, e-mail: \\baytukenova75@mail.ru:

Kalitova A.D. - Master' s student of 1 course of the
department ``Technology of food and processing industries'', Kazakh
Agrotechnical Research University named after S.Seyfullin, Astana,
Kazakhstan, e-mail: kalitova03@mail.ru;

Baitukenova S.B. -- Head of Department ``Technology and
Standardization'', Candidate of Technical Sciences, Associate Professor,
Kazakh University of Technology and Business named after K. Kulazhanov,
Astana, Kazakhstan, e-mail: saule7272@mail.ru;

Ryspaeva U.A. - Master' s degree, Lecturer, Department of
Technology and Standardization, Kazakh University of Technology and
Business named after K. Kulazhanov, Astana, Kazakhstan, e-mail:
ulzhan.ryspaeva@bk.ru

\emph{{\bfseries Сведения об авторах}}

Байтукенова Ш.Б. -- к.т.н., ассоциированный профессор кафедры
«Технология пищевых и перерабатывающих производств», НАО «Казахский
агротехнический исследовательский университет им. С.Сейфуллина», Астана,
Казахстан, e-mail: baytukenova75@mail.ru

Калитова А.Д. -- магистрант 1 курса кафедры «Технология пищевых и
перерабатывающих производств», НАО «Казахский агротехнический
исследовательский университет им. С.Сейфуллина», Астана, Казахстан,
e-mail: kalitova03@mail.ru;

Байтукенова С.Б. -- к.т.н., ассоциированный профессор кафедры
технологиии стандартизации, Казахский университет технологии и бизнеса
им. К. Кулажанова, Астана, Казахстан, e-mail: saule7272@mail.ru;

Рыспаева У.А. - магистр, преподаватель кафедры технологии и
стандартизации, Казахский университет технологии и бизнеса им. К.
Кулажанова, Астана, Казахстан, e-mail: ulzhan.ryspaeva@bk.ru
\end{authorinfo}
