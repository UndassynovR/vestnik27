\id{МРНТИ 65.55.33}{}

\begin{articleheader}
\sectionwithauthors{Г.Е. Жумалиева, У.Ч. Чоманов, А.Н. Асан}{КОМПЛЕКСНОЕ ИССЛЕДОВАНИЕ ПРЯНЫХ ТРАВ КАК ФУНКЦИОНАЛЬНОГО СЫРЬЯ ДЛЯ ПИЩЕВОЙ ПРОМЫШЛЕННОСТИ}

{\bfseries
Г.Е. Жумалиева\textsuperscript{\envelope },
У.Ч. Чоманов,
А.Н. Асан\textsuperscript{\envelope }
}
\end{articleheader}

\begin{affiliation}
ТОО «Казахский научно-исследовательский институт перерабатывающей и
пищевой промышленности», Алматы, Казахстан

\raggedright \textsuperscript{\envelope }Корреспондент-автор:
\href{mailto:guljan\_7171@mail.ru}{\nolinkurl{guljan\_7171@mail.ru}},
\href{mailto:arailym\_178@mail.ru}{\nolinkurl{arailym\_178@mail.ru}}
\end{affiliation}

Статья посвящена актуальной теме - исследованию пряных трав,
выращиваемых и перерабатываемых на территории Казахстана, с учётом
растущего спроса на органические продукты, развитие гастрономической
культуры, фармацевтической промышленности и экспортного потенциала
страны. Целью исследования является разработка технологии производства
порошков и экстрактов из пряных трав для использования в кулинарных
изделиях и пищевых продуктах. Впервые в комплексе изучены особенности
отечественного сырья: чеснока, укропа, базилика, петрушки, красного
перца и сельдерея. Представлен анализ их вкусовых, ароматических и
питательных свойств, физико-химических показателей и аминокислотного
состава. Научная новизна заключается в системном подходе к изучению
пряных трав казахстанского происхождения и предложении новых
технологических решений для их переработки. В ходе анализа источников за
1994-2024 гг. показано, что при благоприятных климатических условиях
Казахстан обладает высоким потенциалом для развития собственного
производства, однако пока уступает России по уровню технологических
разработок в данной области. Практическая значимость заключается в
снижении зависимости от импорта, формировании экспортного потенциала,
создании новых рабочих мест и стимулировании роста местного агробизнеса.
Результаты исследования могут быть использованы для разработки
функциональных и диетических продуктов, продвижения органического
сельского хозяйства и поддержки устойчивого развития регионов
Казахстана. В результате анализа физико-химических и биохимических
свойств различных пряных трав установлено, что они обладают высокой
пищевой и биологической ценностью. Наибольшее содержание белка выявлено
у петрушки (28,31\%), жиров --- у базилика (3,96\%) и петрушки (3,90\%).
Красный перец лидирует по содержанию витаминов A (665,20 мг/100 г) и C
(250 мг/100 г), а базилик - по кальцию (2279,56 мг/100 г) и железу
(99,36 мг/100 г). Укроп отличается высоким уровнем магния (530 мг/100
г), чеснок - селеном (0,096\%). Также чеснок и петрушка богаты
незаменимыми аминокислотами, включая валин, лейцин и треонин. Полученные
данные подтверждают потенциал пряных трав как функциональных
ингредиентов для здорового питания.

{\bfseries Ключевые слова:} укроп, петрушка, сельдерей, красный перец,
базилик, чеснок, пряные травы, экстракты, порошки.

\begin{articleheader}
{\bfseries АСПАЗДЫҚ ӨНЕРКӘСІПКЕ АРНАЛҒАН ФУНКЦИОНАЛДЫҚ ШИКІЗАТ РЕТІНДЕ
ДӘМДЕУІШ ШӨПТЕРДІ КЕШЕНДІ ЗЕРТТЕУ}

{\bfseries
Г.Е. Жумалиева\textsuperscript{\envelope },
У.Ч. Чоманов,
А.Н. Асан\textsuperscript{\envelope }
}
\end{articleheader}

\begin{affiliation}
«Қазақ қайта өңдеу және тамақ өнеркәсібі ғылыми-зерттеу институты» ЖШС,
Алматы, Қазақстан,

e-mail:
\href{mailto:guljan\_7171@mail.ru}{\nolinkurl{guljan\_7171@mail.ru}},
\href{mailto:arailym\_178@mail.ru}{\nolinkurl{arailym\_178@mail.ru}}
\end{affiliation}

Мақала Қазақстан аумағында өсірілетін және өңделетін дәмдеуіш шөптерді
зерттеуге арналған өзекті тақырыпқа арналған. Зерттеу органикалық
өнімдерге сұраныстың артуы, гастрономиялық мәдениеттің дамуы,
фармацевтикалық өнеркәсіп пен елдің экспорттық әлеуеті аясында
жүргізіледі. Зерттеудің мақсаты - дәмдеуіш шөптерден тағамдық өнімдер
мен кулинарлық бұйымдарға арналған ұнтақтар мен экстрактілер өндіру
технологиясын әзірлеу. Алғаш рет қазақстандық шикізаттың - сарымсақ,
аскөк, райхан, ақжелкен, қызыл бұрыш және балдыркөк - ерекшеліктері
кешенді түрде зерттелді. Олардың дәмдік, хош иістік және тағамдық
қасиеттері, физика-химиялық көрсеткіштері мен аминқышқылдық құрамы
талданды. Ғылыми жаңалық ретінде дәмдеуіш шөптерді өңдеудің жаңа
технологиялық шешімдері ұсынылып, оларды жүйелі зерттеу жүргізілді.
1994-2024 жылдар аралығындағы дереккөздерді талдау нәтижесінде
Қазақстанның қолайлы климаттық жағдайы бұл саланың дамуына үлкен
мүмкіндік беретінін көрсетті, алайда технологиялық даму жағынан Ресейден
артта қалып отыр. Практикалық маңыздылығы - импортқа тәуелділікті
азайту, экспорттық әлеует қалыптастыру, жаңа жұмыс орындарын ашу және
жергілікті агробизнесті дамыту. Зерттеу нәтижелері функционалдық және
диеталық өнімдер жасауда, органикалық ауыл шаруашылығын ілгерілетуде
және Қазақстан аймақтарының тұрақты дамуын қолдауда пайдаланылуы мүмкін.
Физикалық-химиялық және биохимиялық қасиеттерін талдау нәтижесінде түрлі
дәмдеуіш өсімдіктердің жоғары тағамдық және биологиялық құндылыққа ие
екендігі анықталды. Ең жоғары ақуыз мөлшері ақжелкеде тіркелді
(28,31\%), ал май мөлшері базилик (3,96\%) пен ақжелкеде (3,90\%) көп.
Қызыл бұрыш А дәрумені (665,20 мг/100 г) мен С дәрумені (250 мг/100 г)
бойынша көшбасшы болып табылады. Базилик құрамында кальций (2279,56
мг/100 г) мен темір (99,36 мг/100 г) ең көп. Аскөк магнийге (530 мг/100
г), ал сарымсақ селенге (0,096\%) бай. Сонымен қатар, сарымсақ пен
ақжелке валин, лейцин және треонин секілді алмастырылмайтын
аминқышқылдарына бай. Бұл деректер дәмдеуіш шөптердің сау тамақтануда
қолдануға болатын функционалдық құрамдас бөлік екенін дәлелдейді.

{\bfseries Түйін сөздер}: аскөк, ақжелкен, балдыркөк, қызыл бұрыш,
насыбайгүл, сарымсақ, шөптер.

\begin{articleheader}
{\bfseries COMPREHENSIVE STUDY OF SPICE HERBS AS FUNCTIONAL RAW MATERIALS
FOR THE FOOD INDUSTRY}

{\bfseries
G.E Zhumalieva\textsuperscript{\envelope },
U.Ch.Chomanov,
A.N.Asan\textsuperscript{\envelope }
}
\end{articleheader}

\begin{affiliation}
LTD "Kazakh Research Institute of Processing and Food Industry" Almaty, Kazakhstan,

e-mail:
\href{mailto:guljan\_7171@mail.ru}{\nolinkurl{guljan\_7171@mail.ru}},
\href{mailto:arailym\_178@mail.ru}{\nolinkurl{arailym\_178@mail.ru}}
\end{affiliation}

The article is devoted to a relevant topic - the study of spice herbs
cultivated and processed in Kazakhstan, in the context of growing demand
for organic products, the development of gastronomic culture, the
pharmaceutical industry, and the country's export potential. The purpose
of the research is to develop a technology for producing powders and
extracts from spice herbs for culinary and food applications. For the
first time, the properties of local raw materials - garlic, dill, basil,
parsley, red pepper, and celery - have been comprehensively studied. The
analysis includes their taste, aromatic and nutritional properties,
physicochemical indicators, and amino acid composition. The scientific
novelty lies in the systematic study of Kazakhstan-grown herbs and the
proposal of new technological solutions for their processing. A review
of sources from 1994 to 2024 demonstrates that, despite favorable
climatic conditions and high potential, Kazakhstan still lags behind
Russia in terms of technological development in this field. The
practical significance includes reducing import dependency, generating
export opportunities, creating new jobs, and supporting the growth of
local agribusiness. The results can be used in the development of
functional and dietary products, the promotion of organic agriculture,
and the sustainable development of Kazakhstan's regions. The analysis of
physicochemical and biochemical properties revealed that various herbs
and spices possess high nutritional and biological value. Parsley showed
the highest protein content (28.31\%), while basil (3.96\%) and parsley
(3.90\%) had the most fat. Red pepper led in vitamin A (665.20 mg/100 g)
and vitamin C (250 mg/100 g) content. Basil was richest in calcium
(2279.56 mg/100 g) and iron (99.36 mg/100 g). Dill had the highest
magnesium level (530 mg/100 g), and garlic contained the most selenium
(0.096\%). Additionally, garlic and parsley were rich in essential amino
acids such as valine, leucine, and threonine. These findings confirm the
potential of herbs as functional components in healthy nutrition.

{\bfseries Keywords:} dill, parsley, celery, red pepper, basil, garlic,
herbs, extracts, powders.

\begin{multicols}{2}
{\bfseries Введение.} Производство пряных трав в Казахстане актуально
благодаря растущему спросу на органические продукты, развитию
гастрономической культуры и фармацевтической промышленности, а также
экономическому потенциалу страны на мировом рынке. Пряные травы
востребованы как в национальной кухне, так и в производстве биологически
активных добавок и эфирных масел, что создает возможности для местного
агробизнеса и экспорта. Благоприятные климатические условия в разных
регионах Казахстана позволяют выращивать широкий спектр трав, что делает
это направление перспективным для сельского хозяйства.

Научной новизной данного исследования является изучение в комплексе
особенности пряных трав отечественного производства.

Пряные травы представляют собой широкую категорию растений с не менее
широким спектром применения. Они делают пищу и напитки более приятными,
а не обеспечивают значительную калорийность или другие питательные
преимущества. Относительно небольшие количества придают изюминку
продуктам, обычно не опасаясь негативных последствий {[}1{]}.

С целью раскрыть потенциал этих пряных трав используются такие виды, как
чеснок, укроп, базилик, петрушка, сельдерей, красный перец для создания
порошков и экстрактов. Каждая из них добавляет в рецепты свою особую
нотку, обогащая вкус и аромат блюд. Экстракты помогают лучше раскрыть
эти качества, делая использование трав более удобным и разнообразным. Их
гармоничное сочетание открывает новые возможности в кулинарии, позволяя
создавать более насыщенные и разнообразные вкусовые композиции.

Основные причины использования данных пряных травах:

- разнообразие вкуса: каждая трава добавляет свой неповторимый вкус, что
позволяет создавать сложные и многогранные блюда; - польза для здоровья:
все выбранные травы богаты витаминами, минералами и антиоксидантами, что
помогает поддерживать общее здоровье и укрепляет иммунитет; -
универсальность: эти травы популярны в кухнях разных стран, что делает
их подходящими для множества рецептов - от супов до мясных и овощных
блюд; - сохранение свежести: использование этих трав помогает сохранить
свежесть и аромат готовых изделий, что особенно важно для тех, кто ищет
натуральные и полезные ингредиенты; -доступность: эти травы легко найти
и они широко используются, что делает их практичным выбором для создания
кулинарных продуктов {[}2{]}. Выбор этих пряных трав обеспечивает
идеальное сочетание вкуса, аромата и здоровья.

Рассмотрены нормы потребления рассматриваемых пряных трав по
рекомендуемым суточным нормам (таблица 1).

Чеснок (\emph{Allium sativum}) - это широко используемая пряная
культура, известная своими кулинарными и медицинскими свойствами. С
древних времён чеснок применялся и не только в кулинарии, но и в
традиционной медицине, что обусловлено его богатым химическим составом.
Суточная потребность в чесноке составляет примерно 1-2 зубчика в день.
Это количество достаточно для получения его полезных свойств, таких как
поддержка иммунной системы и улучшение сердечно-сосудистого здоровья.
Если использовать чеснок в порошке, то это будет около 1/4 - 1/2 ч.
ложки {[}3{]}.
\end{multicols}

{\bfseries Таблица 1 - Рекомендуемые суточные нормы Рекомендуемая суточная норма потребления {[}4{]}}

%% \begin{longtable}[]{@{}
%%   >{\raggedright\arraybackslash}p{(\linewidth - 2\tabcolsep) * \real{0.4294}}
%%   >{\centering\arraybackslash}p{(\linewidth - 2\tabcolsep) * \real{0.5706}}@{}}
%% \toprule\noalign{}
%% \begin{minipage}[b]{\linewidth}\centering
%% Наименование показателей
%% \end{minipage} & \begin{minipage}[b]{\linewidth}\centering
%% Рекомендуемая суточная норма (г)
%% \end{minipage} \\
%% \midrule\noalign{}
%% \endhead
%% \bottomrule\noalign{}
%% \endlastfoot
%% Белки & 50-100 \\
%% Жиры & 70-100 (менее 30\% насыщенных) \\
%% Углеводы & 250-400 \\
%% Клетчатка & 25-30 (женщины), 30-38 (мужчины) \\
%% Витамин C & 90 (мужчины), 75 (женщины) \\
%% Витамин B\textsubscript{6} & 1,3-2,0 \\
%% Витамин B\textsubscript{1} & 1,1 (женщины), 1,2 (мужчины) \\
%% Витамин B\textsubscript{2} & 1,1 (женщины), 1,3 (мужчины) \\
%% Калий & 1000-4700 (взрослые) \\
%% 1 & 2 \\
%% Фосфор & 700 (взрослые) \\
%% Кальций & 1000 (взрослые), 1200 (женщины старше 50) \\
%% Магний & 400-420 (мужчины), 310-320 (женщины) \\
%% Железо & 8 (мужчины), 18 (женщины) \\
%% \end{longtable}

В таблице 2 показана пищевая ценность чеснока, укропа, базилика,
петрушки, сельдерея, красного перца.

{\bfseries Таблица 2-Пищевая ценность и калорийность пряных трав (чеснок,
укроп, базилик, петрушка, сельдерей, красный перец) {[}5{]}}

%% \begin{longtable}[]{@{}
%%   >{\raggedright\arraybackslash}p{(\linewidth - 12\tabcolsep) * \real{0.1979}}
%%   >{\centering\arraybackslash}p{(\linewidth - 12\tabcolsep) * \real{0.1068}}
%%   >{\centering\arraybackslash}p{(\linewidth - 12\tabcolsep) * \real{0.1008}}
%%   >{\centering\arraybackslash}p{(\linewidth - 12\tabcolsep) * \real{0.1210}}
%%   >{\centering\arraybackslash}p{(\linewidth - 12\tabcolsep) * \real{0.1365}}
%%   >{\centering\arraybackslash}p{(\linewidth - 12\tabcolsep) * \real{0.1371}}
%%   >{\centering\arraybackslash}p{(\linewidth - 12\tabcolsep) * \real{0.2000}}@{}}
%% \toprule\noalign{}
%% \multirow{3}{=}{\begin{minipage}[b]{\linewidth}\centering
%% Наименование показателей
%% \end{minipage}} &
%% \multicolumn{6}{>{\centering\arraybackslash}p{(\linewidth - 12\tabcolsep) * \real{0.8021} + 10\tabcolsep}@{}}{%
%% \begin{minipage}[b]{\linewidth}\centering
%% Виды пряных трав
%% \end{minipage}} \\
%% & \begin{minipage}[b]{\linewidth}\centering
%% чеснок
%% \end{minipage} & \begin{minipage}[b]{\linewidth}\centering
%% укроп
%% \end{minipage} & \begin{minipage}[b]{\linewidth}\centering
%% базилик
%% \end{minipage} & \begin{minipage}[b]{\linewidth}\centering
%% петрушка
%% \end{minipage} & \begin{minipage}[b]{\linewidth}\centering
%% сельдерей
%% \end{minipage} & \begin{minipage}[b]{\linewidth}\centering
%% красный перец
%% \end{minipage} \\
%% &
%% \multicolumn{6}{>{\centering\arraybackslash}p{(\linewidth - 12\tabcolsep) * \real{0.8021} + 10\tabcolsep}@{}}{%
%% \begin{minipage}[b]{\linewidth}\centering
%% Значение (на 100 г)
%% \end{minipage}} \\
%% \midrule\noalign{}
%% \endhead
%% \bottomrule\noalign{}
%% \endlastfoot
%% Калорийность (ккал) & 149 & 36 & 23 & 36 & 16 & 31 \\
%% Белки (г) & 6,4 & 3,0 & 3,15 & 3,0 & 0,7 & 1 \\
%% Жиры (г) & 0,5 & 0,5 & 0,64 & 0,8 & 0,2 & 0,3 \\
%% Углеводы (г) & 33,1 & 7,0 & 2,65 & 6,0 & 3,0 & 6 \\
%% Клетчатка (г) & 2,1 & 2,1 & 1,6 & 2,3 & 1,6 & 2,1 \\
%% \end{longtable}

\begin{multicols}{2}
Чеснок является богатым источником питательных веществ, в особенности
углеводов и клетчатки. При потреблении 100 грамм чеснока можно покрыть
значительную часть суточной потребности в углеводах - до 8,28\% от
максимальной нормы, а клетчатка в чесноке удовлетворяет до 7\% дневной
потребности. Хотя содержание белков и жиров в чесноке относительно
невелико, оно также вносит вклад в общую суточную норму этих
макроэлементов. В целом, чеснок может рассматриваться как полезный
компонент рациона, особенно для тех, кто стремится увеличить потребление
клетчатки и углеводов, улучшая пищеварение и поддерживая стабильный
уровень энергии.

Укроп (\emph{Anethum graveolens}) - это одно из самых популярных пряных
растений, широко используемое в кулинарии и медицине. Его характерный
аромат и освежающий вкус делают укроп незаменимым компонентом множества
блюд. Рекомендуемая суточная норма укропа составляет около 10-15 г
свежего или 2-4 г сушеного. Это количество позволит добавить аромат и
полезные свойства в еду. Укроп отличается низкой калорийностью и высоким
содержанием витаминов и минералов.

Укроп, благодаря своему богатому составу, является ценным дополнением к
любому рациону, несмотря на свою низкую калорийность. Хотя содержание
белков и жиров в укропе невелико, он покрывает значительную часть
суточной потребности в клетчатке --- до 8,4\% от минимальной нормы, что
способствует нормализации работы пищеварительной системы. Углеводы в
укропе представлены сложными сахарами, которые медленно усваиваются,
обеспечивая стабильный уровень энергии. Таким образом, укроп можно
рекомендовать как полезный ингредиент, особенно для тех, кто стремится
поддерживать баланс клетчатки и улучшить пищеварение.

Базилик (\emph{Ocimum basilicum}) - это ароматическое растение, которое
широко используется в кулинарии и медицине. Его сладкий, пряный вкус и
неповторимый аромат делают базилик популярным ингредиентом во многих
кухнях мира, особенно в итальянской и средиземноморской. Суточная
потребность в базилике составляет примерно 5-10 г свежего или 1-2 г
сушеного {[}6{]}.

Базилик, благодаря своему составу, является полезным дополнением к
рациону. Несмотря на низкое содержание калорий, он богат клетчаткой,
покрывая до 6,4\% минимальной суточной нормы, что способствует
поддержанию здорового пищеварения.

Белки, присутствующие в базилике, также вносят свой вклад в
удовлетворение потребности организма, покрывая до 6,3\% от минимальной
нормы. Хотя содержание жиров и углеводов в базилике незначительно, его
питательная ценность в первую очередь обусловлена высоким содержанием
клетчатки и белков, что делает его полезным компонентом для
сбалансированного питания\emph{.}

Петрушка (\emph{Petroselinum crispum})-является одним из наиболее
изучаемых представителей семейства сельдерейных (\emph{Apiaceae}) и
широко используется как пряное и лекарственное растение. Ее применение
охватывает не только кулинарную сферу, где она служит как приправой и
декоративным элементом, но и фитотерапию, благодаря богатому содержанию
витаминов, минералов и антиоксидантов. Петрушка обладает множеством
биологически активных соединений, способствующих улучшению здоровья
человека. Суточная потребность в петрушке составляет около 10-20 г
свежей зелени или 1-2 г сушеной {[}7{]}.

Петрушка является низкокалорийным, но питательным продуктом, особенно
богатым клетчаткой. Она покрывает до 9,2\% минимальной суточной нормы
клетчатки, что способствует улучшению пищеварения и поддержанию здоровья
кишечника. Хотя содержание белков и жиров в петрушке относительно
невелико, она всё же вносит вклад в суточную норму этих макроэлементов.
Углеводы в петрушке представлены в небольшом количестве, что делает её
отличным дополнением к рациону, особенно для тех, кто стремится
увеличить потребление клетчатки, не увеличивая калорийность.

Таким образом, петрушка - это полезный и лёгкий ингредиент для
сбалансированного питания.

Сельдерей (\emph{Apium graveolens}) - это одно из самых универсальных
овощных растений, широко используемое в кулинарии и традиционной
медицине. С древних времен сельдерей ценился не только за свой
уникальный вкус и аромат, но и за свои полезные свойства. Он принадлежит
к семейству сельдерейных (\emph{Apiaceae}) и бывает нескольких видов,
среди которых наиболее распространены стеблевой и корневой сельдерей.
Суточная потребность в сельдерее составляет примерно 1-2 стебля свежего
сельдерея {[}8{]}.

Сельдерей характеризуется низкой калорийностью и высоким содержанием
воды, что делает его популярным среди людей, следящих за своим весом.

Сельдерей является низкокалорийным продуктом, однако он богат
клетчаткой, покрывая до 6,4\% минимальной суточной нормы. Это делает его
полезным для поддержания здорового пищеварения и улучшения обмена
веществ. Хотя содержание белков, жиров и углеводов в сельдерее невелико,
его главная ценность заключается в высоком содержании пищевых волокон,
которые помогают нормализовать работу желудочно-кишечного тракта.

Таким образом, сельдерей - это отличный выбор для тех, кто стремится
улучшить пищеварение и добавить в рацион источник клетчатки, не
увеличивая калорийность блюд.

Красный перец (\emph{Capsicum annuum}) - это яркий и ароматный овощ,
который занимает важное место в кулинарии многих культур. Он известен не
только своим острым вкусом, но и богатым содержанием витаминов,
минералов и антиоксидантов. Красный перец является популярным
ингредиентом в различных блюдах, начиная от салатов и закусок и
заканчивая основными блюдами и соусами. Суточная потребность в красном
перце в виде порошка составляет примерно 1/2--1 чайную ложку.

Красный перец является питательным продуктом с низкой калорийностью,
который вносит разнообразие в рацион благодаря своим питательным
свойствам. Несмотря на то что содержание белков, жиров и углеводов в нем
невелико, красный перец все же покрывает от 1\% до 2,4\% суточной нормы,
что делает его полезным дополнением к другим продуктам. Особенно стоит
отметить высокое содержание клетчатки, которое составляет 7\% - 8,4\% от
суточной нормы. Клетчатка способствует нормализации пищеварения и
улучшению обмена веществ. Кроме того, красный перец богат углеводами, в
основном представленными сахарами, что придает ему сладковатый вкус и
делает его приятным в употреблении.

Таким образом, красный перец не только добавляет цвет и вкус в блюда, но
и является ценным источником питательных веществ, способствующих
поддержанию здоровья и общего благополучия.

Красный перец богат углеводами, среди которых преобладают сахара, что
придает ему сладковатый вкус {[}9{]}.

Использование пряно-ароматических растений Южного Казахстана в
производстве соусов-приправ представляет собой перспективное направление
для создания натуральных, безопасных и функциональных продуктов питания.
В ходе исследования, проведенного Орынбасаровой Б.А. и её коллегами,
было установлено, что разнообразие дикорастущих растений в регионе,
включая зверобой, чабрец и душицу, предоставляет уникальные возможности
для разработки фито композиций, обогащающих соусы и приправы вкусовыми и
полезными свойствами. Переход на жидкие экстракты и инкапсулированные
формы пряностей значительно упрощает процесс производства и улучшает
качество конечного продукта, гарантируя высокие органолептические
показатели и сохранение биологически активных веществ.

Таким образом, интеграция традиционных пряных трав в современное
производство соусов-приправ соответствует актуальным требованиям
здорового питания и способствует сохранению и развитию культурных
традиций региона {[}10{]}.

Изобретение, предложенное Кристиной Лутц и Феликсом Рихтерихом
(Швейцария), относится к кондитерским изделиям, изготовленным на основе
травяных смесей. Продукт содержит экстракты трав, включая мяту перечную,
шалфей, тысячелистник, тимьян и стевию ребаудиану. Дополнительно в
состав могут входить экстракты таких трав, как мята лимонная,
подорожник, алтей, манжетка, бузина, примула, бедренец, вероника, мальва
и шандра.

Изделия предлагаются в различных формах: леденцы, жевательные конфеты,
карамель, жевательная резинка или сиропы, которые можно использовать для
приготовления быстрорастворимых напитков. Технология изготовления
включает экстракцию травяных смесей, их сгущение и переработку в
кондитерские изделия. Продукты обладают освежающим вяжущим
медово-травяным вкусом с охлаждающим эффектом. Данное изобретение
нацелено на создание натуральных функциональных кондитерских изделий с
улучшенными вкусовыми характеристиками {[}11{]}.

Разработка пряно-ароматических фито композиций для соусов-приправ,
выполненная Ушаковой А.А. на основе анализа дикорастущих растений,
демонстрирует значительный потенциал для создания натуральных продуктов,
обогащенных биологически активными веществами. В результате исследований
была предложена методология экстракции и использования фито композиций,
которые позволяют существенно повысить содержание фито микронутриентов,
таких как флавоноиды, каротиноиды и антиоксиданты, в готовой продукции.
Использование этих фито композиций в рецептурах соусов и приправ
увеличивает содержание биологически активных компонентов от 5 до 13\%,
что позволяет относить данные продукты к категории функциональных и
обогащенных. Разработанные рецептуры демонстрируют высокую
конкурентоспособность на рынке благодаря их натуральному составу и
богатству полезных веществ {[}12{]}.

Корейскими учеными разработан травяной порошок для жарки, который
представляет полезный и питательный продукт, обогащённый витаминами и
минералами. В его состав входят травы с фармакологически активными
компонентами, которые придают порошку уникальные вкусовые качества и
полезные свойства. Этот порошок улучшает пищеварения, обладает
диуретическим, стерилизующим и антибактериальным эффектом. За счёт
комбинации различных видов муки, крахмала и специй продукт обеспечивает
хрустящую текстуру при жарке, что делает его удобным и полезным для
использования в кулинарии {[}13{]}.

Исследование доказало, что использование пряного растительного сырья, в
частности порошка черешков сельдерея, обработанного инфракрасной (ИК)
сушкой, в технологии производства заварного полуфабриката является
эффективным способом повышения его пищевой ценности. Добавление порошка
сельдерея в количестве 5\% от массы муки обеспечивает улучшение
нутриентного состава полуфабриката, увеличивая содержание витаминов
группы В и β-каротина, что способствует обогащению продукта важными для
здоровья микроэлементами. При этом вкусоароматические характеристики
готового изделия улучшаются за счет кислых и пряных оттенков, которые
привносит сельдерей. Введение свежего пряного растительного сырья
ухудшает органолептические свойства изделия, что делает применение
именно порошков ИК-сушки более целесообразным. Дальнейшие исследования
могут быть направлены на улучшение рецептур начинки для комплексного
улучшения нутриентного состава и маскировки нежелательных привкусов,
присущих сельдерею {[}14{]}.

Изобретение разработанное Баудауином ван Афферденом относится к
консервированной композиции на основе пряных трав.

Настоящее изобретение касается метода обработки травяной воды,
направленного на восстановление и обогащение здоровья посредством
использования географического биоразнообразия. Травяная вода,
обогащенная значительными питательными веществами и природными
антибиотиками, способствует активному и здоровому образу жизни. Эти
компоненты обеспечивают оптимальный рост и развитие организма, защищая
его от неблагоприятных патогенов и токсинов. Метод обработки травяной
воды включает использование различных трав, таких как индийский
крыжовник (\emph{Emblica officinalis}), священный базилик (\emph{Ocimum
sanctum}), ним (\emph{Azadirachta indica}), а также других растений,
доступных в соответствующем географическом регионе. Выбор конкретных
трав или их частей зависит от области применения и желаемого эффекта.

Обогащенная травяная вода, содержащая питательные вещества и природные
антибиотики, может быть использована в производстве продуктов для
поддержания здоровья и профилактики заболеваний. Она также находит
применение в научных исследованиях, направленных на изучение
биологических процессов и разработку биотехнологий {[}15{]}.

{\bfseries Материалы и методы.} Объектами исследования являются перец,
чеснок, петрушка, укроп, сельдерей, базилик.

При выполнении проекта будут использованы общепринятые и специальные
методы определения физико-химических свойств, пищевая и биологическая
ценность сырья.

Для исследования будут использованы следующие методики из стандартов:

ГОСТ 33271-2015 Пряности сухие, травы и приправы овощные.

ГОСТ 28875-90 Пряности. Приемка и методы анализа (определение влаги
методом отгонки).

ГОСТ ISO 927-2014 Пряности и приправы. Определение содержания примесей и
посторонних веществ {[}18{]}.

ГОСТ 28876-90 Пряности и приправы. Отбор проб.

ГОСТ ISO 6571 -- 2016 Определение содержания эфирных масел.

ГОСТ EN 12823-2-2014 Продукты пищевые. определение содержания витамина А
методом высокоэффективной жидкостной хроматографии.

ГОСТ EN 12822---2014 продукты пищевые Определение содержания витамина Е
(а-, р-, у- и 5- токоферолов) методом высокоэффективной жидкостной
хроматографии.

ГОСТ 34151-2017 Продукты пищевые Определение витамина C с помощью
высокоэффективной жидкостной хроматографии.

ГОСТ 26928-86 Продукты пищевые. Метод определения железа.

ГОСТ 26931-86 Сырье и продукты пищевые методы определения меди.

ГОСТ 26934-86 Сырье и продукты пищевые метод определения цинка.

ГОСТ 27076-86 Продукты пищевые и пищевое сырье. Метод определения
кальция.

ГОСТ 30615-99 Сырье и продукты пищевые. Метод определения фосфора.

ГОСТ ISO 928 - 2015 Определение общего содержания золы.

Определение массовой доли влаги в пряных травах проводили с
использованием сушильного шкафа СЭШ-ЗМК при температуре 105 ± 2°C.
Навеску сырья массой 2--5 г помещали в предварительно прокаленные и
взвешенные бюксы, после чего сушили в течение 3 часов до постоянной
массы, с последующим охлаждением в эксикаторе и взвешиванием. Массовую
долю влаги (\%) рассчитывали по формуле: W = (m₁ -- m₂)/(m₁ -- m₀) ×
100, где m₀ -- масса пустой бюксы, m₁ -- масса бюксы с влажным образцом,
m₂ -- масса бюксы с высушенным образцом. Разность между двумя
взвешиваниями после повторной сушки не должна превышать 0,001 г.
Определение проводили в двух параллельных повторностях, результаты
фиксировали с точностью до 0,01\%.

Определение золы в растительном сырье проводится путем прокаливания
навески образца до полного выгорания органических веществ. Для этого в
фарфоровую чашку помещают навеску измельченного и высушенного сырья
(обычно 2--5 г), затем обугливают на открытом пламени, чтобы
предотвратить разбрызгивание, и помещают в муфельную печь, где
прокаливают при температуре 500--550\,°C в течение 4--6 часов до
получения светло-серой или белой золы. После охлаждения в эксикаторе
чашку взвешивают. Процентное содержание золы рассчитывают по формуле:

W = (m1 -- m0) / m × 100,

где m1 --- масса чашки с золой, m0 --- масса пустой чашки, m --- масса
навески. Этот метод позволяет определить общее содержание минеральных
веществ в растении.

Определение содержания эфирных масел в растительном сырье обычно
проводят методом гидродистилляции с использованием прибора Клевенджера.
В колбу помещают измельчённое сырьё (например, 10--30 г) и заливают
водой в соотношении 1:10 или 1:20. Смесь нагревают, при кипении пары
эфирных масел и воды конденсируются и стекают в градуированный приёмник,
где масло отделяется от воды. Процесс длится 2--3 часа. После окончания
дистилляции объем эфирного масла измеряют непосредственно в приёмнике, и
содержание выражают в процентах по формуле:

X = V / m × 100,

где V --- объем эфирного масла (мл), m --- масса сырья (г). Метод
позволяет точно оценить количество летучих ароматических компонентов в
пряных травах и других эфиромасличных растениях.

{\bfseries Результаты и обсуждение.} Купили все образцы по 300 граммов.
После предварительной мойки их поместили на сушку. Сушили в специальном
сушильном шкафу при температуре 40--45 градусов, чтобы сохранить все
полезные вещества. Процесс сушки длился 19--20 часов. После определяли
физико-химические свойства, пищевую и биологическую ценность выбранных
пряных трав в Испытательной лаборатории ТОО «Нутритест» (таблица 1,2,3).
\end{multicols}

{\bfseries Таблица 3 - Физико-химические показатели пряных трав}

%% \begin{longtable}[]{@{}
%%   >{\raggedright\arraybackslash}p{(\linewidth - 12\tabcolsep) * \real{0.2303}}
%%   >{\centering\arraybackslash}p{(\linewidth - 12\tabcolsep) * \real{0.1333}}
%%   >{\centering\arraybackslash}p{(\linewidth - 12\tabcolsep) * \real{0.1212}}
%%   >{\centering\arraybackslash}p{(\linewidth - 12\tabcolsep) * \real{0.1364}}
%%   >{\centering\arraybackslash}p{(\linewidth - 12\tabcolsep) * \real{0.1243}}
%%   >{\centering\arraybackslash}p{(\linewidth - 12\tabcolsep) * \real{0.1333}}
%%   >{\centering\arraybackslash}p{(\linewidth - 12\tabcolsep) * \real{0.1212}}@{}}
%% \toprule\noalign{}
%% \multirow{2}{=}{\begin{minipage}[b]{\linewidth}\centering
%% Наименование показателей (\%)
%% \end{minipage}} &
%% \multicolumn{6}{>{\centering\arraybackslash}p{(\linewidth - 12\tabcolsep) * \real{0.7697} + 10\tabcolsep}@{}}{%
%% \begin{minipage}[b]{\linewidth}\centering
%% Наименование пряных трав
%% \end{minipage}} \\
%% & \begin{minipage}[b]{\linewidth}\centering
%% укроп
%% \end{minipage} & \begin{minipage}[b]{\linewidth}\centering
%% чеснок
%% \end{minipage} & \begin{minipage}[b]{\linewidth}\centering
%% красный перец
%% \end{minipage} & \begin{minipage}[b]{\linewidth}\centering
%% сельдерей
%% \end{minipage} & \begin{minipage}[b]{\linewidth}\centering
%% петрушка
%% \end{minipage} & \begin{minipage}[b]{\linewidth}\centering
%% базилик
%% \end{minipage} \\
%% \midrule\noalign{}
%% \endhead
%% \bottomrule\noalign{}
%% \endlastfoot
%% Масоовая доля белка(протеина) & 21,02±0,22 & 15,08±0,21 & 12,90±0,13 &
%% 12,75±0,13 & 28,31±0,30 & 25,63±0,33 \\
%% Массовая доля жира и экстрактивных веществ & 3,68±0,08 & 0,51±0,007 &
%% 0,11±0,002 & 2,42±0,05 & 3,90±0,07 & 3,96±0,04 \\
%% Массовая доля влаги & 20,6±0,23 & 57,12±0,75 & 90,1±1,2 & 17,2±0,3 &
%% 19,14±0,28 & 13,15±0,21 \\
%% Зольность & 15,36±1,2 & 0,94±0,02 & 0,48±0,008 & 13,25±0,14 & 12,63±0,15
%% & 15,41±0,23 \\
%% Посторонние примеси &
%% \multicolumn{6}{>{\centering\arraybackslash}p{(\linewidth - 12\tabcolsep) * \real{0.7697} + 10\tabcolsep}@{}}{%
%% не обнаружены} \\
%% \end{longtable}

\begin{multicols}{2}
Наибольшее содержание белка отмечается у петрушки - 28,31\%, а
наименьшее - у селдерея 12,75\%. Это говорит о том, что петрушка
является богатым источником белка по сравнению с другими травами. Также
петрушка и базилик имеют примерно одинаковое и самое высокое содержание
жиров - 3,90 и 3,96 \%. Это может быть связано с наличием в этих травах
эфирных масел. Максимальная влажность у красного перца - 90,11\%, что
указывает на его высокую сочность, тогда как у сельдерея -17,2\% и
базилика - 13,15 самая низкая влажность, что делает их более сухими
травами. Наибольшее содержание золы обнаружено в базилике 15,41\%, что
может свидетельствовать о большом содержании минеральных веществ. Данные
о посторонних примесях отсутствуют для всех пряных трав, что говорит о
их чистоте.
\end{multicols}

{\bfseries Таблица 4-Показатели минералов и витаминов в составе пряных
трав.}

%% \begin{longtable}[]{@{}
%%   >{\centering\arraybackslash}p{(\linewidth - 12\tabcolsep) * \real{0.2122}}
%%   >{\centering\arraybackslash}p{(\linewidth - 12\tabcolsep) * \real{0.1212}}
%%   >{\centering\arraybackslash}p{(\linewidth - 12\tabcolsep) * \real{0.1364}}
%%   >{\centering\arraybackslash}p{(\linewidth - 12\tabcolsep) * \real{0.1363}}
%%   >{\centering\arraybackslash}p{(\linewidth - 12\tabcolsep) * \real{0.1212}}
%%   >{\centering\arraybackslash}p{(\linewidth - 12\tabcolsep) * \real{0.1516}}
%%   >{\centering\arraybackslash}p{(\linewidth - 12\tabcolsep) * \real{0.1212}}@{}}
%% \toprule\noalign{}
%% \multirow{2}{=}{\begin{minipage}[b]{\linewidth}\centering
%% Наименование показателей
%% \end{minipage}} &
%% \multicolumn{6}{>{\centering\arraybackslash}p{(\linewidth - 12\tabcolsep) * \real{0.7878} + 10\tabcolsep}@{}}{%
%% \begin{minipage}[b]{\linewidth}\centering
%% Наименование пряных трав
%% \end{minipage}} \\
%% & \begin{minipage}[b]{\linewidth}\centering
%% укроп
%% \end{minipage} & \begin{minipage}[b]{\linewidth}\centering
%% чеснок
%% \end{minipage} & \begin{minipage}[b]{\linewidth}\centering
%% красный перец
%% \end{minipage} & \begin{minipage}[b]{\linewidth}\centering
%% сельдерей
%% \end{minipage} & \begin{minipage}[b]{\linewidth}\centering
%% петрушка
%% \end{minipage} & \begin{minipage}[b]{\linewidth}\centering
%% базилик
%% \end{minipage} \\
%% \midrule\noalign{}
%% \endhead
%% \bottomrule\noalign{}
%% \endlastfoot
%% \multicolumn{7}{@{}>{\centering\arraybackslash}p{(\linewidth - 12\tabcolsep) * \real{1.0000} + 12\tabcolsep}@{}}{%
%% Витаминный состав} \\
%% Витамин А мг/кг & 34,0±1,7 & - & 665,20±33,26 & 241±12,05 & 199,1±9,96 &
%% - \\
%% Витамин Е, г/кг & 1,65±0,08 & - & - & 4,1±0,21 & 3,23±0,16 & 8,6±0,43 \\
%% Витамин С, мг/100 г & 43,5±0,62 & 10,98±0,18 & 250±41,36 & 75,1±0,87 &
%% 103,2±17,98 & 0,78±0,14 \\
%% \multicolumn{7}{@{}>{\centering\arraybackslash}p{(\linewidth - 12\tabcolsep) * \real{1.0000} + 12\tabcolsep}@{}}{%
%% Минеральные вещества, мг} \\
%% Железо (мг/100 г) & 58,36±0,75 & 0,96±0,01 & 0,56±0,008 & 17,23±0,15 &
%% 23,98±0,36 & 99,36±2,16 \\
%% Медь (мг/100 г) & 0,65±0,008 & 0,095±0,003 & 0,98±0,003 & 0,56±0,01 &
%% 0,986±0,04 & 3,21±0,06 \\
%% Цинк (мг/100 г) & 4,23±0,09 & 0,65±0,006 & 0,56±0,008 & 3,41±0,06 &
%% 6,25±0,1 & 7,78±0,12 \\
%% Магний (мг/100 г) & 530±6,11 & 17,17±0,18 & 6,3±0,07 & 195,76±3,14 &
%% 437,63±6,14 & 724,82±9,52 \\
%% Кальций (мг/100 г) & 2100,96±26,3 & 101,65±1,29 & 6,98±0,1 & 573,29±7,49
%% & 1264,8±15,12 & 2279,56±26,75 \\
%% Йод (мг/100 г) & - & 0,0075±0,00001 & 0,006±0,00001 & 0,084±0,0007 &
%% 0,054±0,0006 & - \\
%% Селен (мг/100 г) & - & 0,096±0,00001 & - & 0,0057±0,00001 & 0,035±0,0003
%% & 0,005±0,0001 \\
%% \end{longtable}

\begin{multicols}{2}
На основе данных таблицы 4 видно, что показатели содержания минералов и
витаминов в составе различных пряных трав, можно сделать следующие
выводы:

Красный перец является лидером по содержанию витаминов А -665,20 мг/100
г и С -- 250 мг/100 г, что делает его отличным источником антиоксидантов
и полезным для зрения и иммунной системы веществ. Сельдерей также
содержит большое количество витамина С -75,1мг/100 г и петрушка -103,2
мг/100 г, что подчеркивает их полезные свойства для укрепления
иммунитета. Витамин Е в наибольшем количестве содержится в базилике -
8,6 мг/100 г, что делает его важным источником антиоксидантов. Также по
содержанию железа лидирует базилик - 99,36 мг/100 г и укроп - 58,36
мг/100 г заметно превосходят другие травы, что может быть полезно для
поддержания нормального уровня гемоглобина в крови. Укроп больше
содержать магния - 530 мг/100 г по сравнению с другими травами, что
способствует поддержанию нормального функционирования нервной и мышечной
систем. Базилик выделяется высоким содержанием кальция 2279,56 мг/100 г,
что делает ее полезной для здоровья костей. Йод в наибольшем количестве
содержится в сельдерее - 0,084 мг/100 г, что может способствовать
поддержанию нормального функционирования щитовидной железы. Содержание
селена максимально в чесноке - 0,096\%, что важно для антиоксидантной
защиты организма.

В таблице 5 приведены аминокислотный состав пряных трав.
\end{multicols}

{\bfseries Таблица 5 - Аминокислотный состав пряных трав}

%% \begin{longtable}[]{@{}
%%   >{\raggedright\arraybackslash}p{(\linewidth - 12\tabcolsep) * \real{0.1849}}
%%   >{\centering\arraybackslash}p{(\linewidth - 12\tabcolsep) * \real{0.1333}}
%%   >{\centering\arraybackslash}p{(\linewidth - 12\tabcolsep) * \real{0.1364}}
%%   >{\centering\arraybackslash}p{(\linewidth - 12\tabcolsep) * \real{0.1212}}
%%   >{\centering\arraybackslash}p{(\linewidth - 12\tabcolsep) * \real{0.1364}}
%%   >{\centering\arraybackslash}p{(\linewidth - 12\tabcolsep) * \real{0.1363}}
%%   >{\centering\arraybackslash}p{(\linewidth - 12\tabcolsep) * \real{0.1516}}@{}}
%% \toprule\noalign{}
%% \multirow{3}{=}{\begin{minipage}[b]{\linewidth}\centering
%% Наименование показателей
%% \end{minipage}} &
%% \multicolumn{6}{>{\centering\arraybackslash}p{(\linewidth - 12\tabcolsep) * \real{0.8151} + 10\tabcolsep}@{}}{%
%% \begin{minipage}[b]{\linewidth}\centering
%% Наименование пряных трав
%% \end{minipage}} \\
%% & \begin{minipage}[b]{\linewidth}\centering
%% укроп
%% \end{minipage} & \begin{minipage}[b]{\linewidth}\centering
%% чеснок
%% \end{minipage} & \begin{minipage}[b]{\linewidth}\centering
%% красный перец
%% \end{minipage} & \begin{minipage}[b]{\linewidth}\centering
%% сельдерей
%% \end{minipage} & \begin{minipage}[b]{\linewidth}\centering
%% петрушка
%% \end{minipage} & \begin{minipage}[b]{\linewidth}\centering
%% базилик
%% \end{minipage} \\
%% &
%% \multicolumn{6}{>{\centering\arraybackslash}p{(\linewidth - 12\tabcolsep) * \real{0.8151} + 10\tabcolsep}@{}}{%
%% \begin{minipage}[b]{\linewidth}\centering
%% Содержание аминокислотного состава (мг/100г)
%% \end{minipage}} \\
%% \midrule\noalign{}
%% \endhead
%% \bottomrule\noalign{}
%% \endlastfoot
%% Валин & 156,0±75,0 & 285,0±91,0 & 32,0±0,8 & 25,0±11,0 & 172,0±25,0 &
%% 133,0±21,0 \\
%% Лейцин & 146,0±34,0 & 308,0±26,0 & 40,0±21,0 & 33,0±13,0 & 197,0±32,0 &
%% 185,0±42,0 \\
%% Изолейцин & 201,0±25,0 & 205,0±62,0 & 23,0±9,0 & 20,0±4,0 & 203,0±24,0 &
%% 99,0±14,0 \\
%% Лизин & 238,0±67,0 & 269,0±32,0 & 32,0±5,0 & 21,0±0,7 & 175,0±28,0 &
%% 102,0±36,0 \\
%% Метионин & 15,±0,04 & 71,0±13,0 & 6,0±0,1,0 & 4,0±0,12 & 38,0±4,0 &
%% 31,0±7,0 \\
%% Треонин & 52,0±7,0 & 151,0±24,0 & 42,0±17,0 & 18,0±3,0 & 114,0±14,0 &
%% 95,0±16,0 \\
%% Триптофан & 14,0±3,0 & 60,0±12,0 & 15,0±3,0 & 8,0±0,6 & 42,0±8,0 &
%% 31,0±17,0 \\
%% Фенилаланин & 68,0±16,0 & 173,0±14,0 & 88,0±17,0 & 23,0±6,0 & 147,0±18,0
%% & 137,0±13,0 \\
%% \end{longtable}

\begin{multicols}{2}
Содержание (мг/100 г): валин содержится в укропе, чесноке, красном
перце, сельдерее, петрушке и базилике соответственно: 156,0±75,0;
285,0±91,0; 32,0±0,8; 25,0±11,0; 172,0±25,0; 133,0±21,0; лейцин --
146,0±34,0; 308,0±26,0; 40,0±21,0; 33,0±13,0; 197,0±32,0; 185,0±42,0;
изолейцин -- 201,0±25,0; 205,0±62,0; 23,0±9,0; 20,0±4,0; 203,0±24,0;
99,0±14,0; лизин -- 238,0±67,0; 269,0±32,0; 32,0±5,0; 21,0±0,7;
175,0±28,0; 102,0±36,0; метионин -- 15,0±0,04; 71,0±13,0; 6,0±0,1;
4,0±0,12; 38,0±4,0; 31,0±7,0; треонин- 52,0±7,0 ; 151,0±24,0; 42,0±17,0;
18,0±3,0; 114,0±14,0; 95,0±16,0; триптофан -- 14,0±3,0; 60,0±12,0;
15,0±3,0; 8,0±0,6; 42,0±8,0; 31,0±17,0 и Фенилаланин -- 68,0±16,0;
173,0±14,0; 88,0±17,0; 23,0±6,0; 147,0±18,0; 137,0±13,0.

Эти растения могут быть полезными источниками аминокислот, включая
незаменимые аминокислоты, которые необходимо получать с пищей, так как
организм не может их синтезировать самостоятельно.

Каждая аминокислота играет свою уникальную роль: валин, лейцин и
изолейцин, которые высоко представлены в чесноке и петрушке,
способствуют восстановлению мышц и энергии, что особенно важно для
людей, ведущих активный образ жизни. Лизин, особенно заметный в чесноке
и укропе, необходим для роста тканей и синтеза коллагена. Метионин и
треонин, встречающиеся в большом количестве в чесноке и петрушке, играют
важную роль в обмене веществ и поддержании здоровья кожи и волос.

{\bfseries Выводы.} На основе анализа физико-химических и биохимических
свойств различных пряных трав и специй можно сделать вывод о высокой
пищевой и биологической ценности данных растений. Среди исследованных
образцов петрушка продемонстрировала наибольшее содержание белка ---
28,31\%, что позволяет рассматривать её как ценный источник
растительного белка. Сельдерей, напротив, содержит наименьшее количество
белка --- 12,75\%, но отличается высоким содержанием витамина C (75,1
мг/100 г) и йода (0,084 мг/100 г), важного для поддержания функций
щитовидной железы. По содержанию жиров лидируют базилик (3,96\%) и
петрушка (3,90\%), что может быть связано с присутствием эфирных масел.
Самую высокую влажность имеет красный перец --- 90,11\%, что указывает
на его сочность, тогда как базилик (13,15\%) и сельдерей (17,2\%)
являются более сухими. Важным показателем является содержание золы,
свидетельствующее о минеральной насыщенности. Максимальное значение
отмечено у базилика --- 15,41\%, что коррелирует с его высоким уровнем
кальция (2279,56 мг/100 г) и железа (99,36 мг/100 г). Укроп выделяется
наибольшим содержанием магния --- 530 мг/100 г, необходимого для нервной
и мышечной системы. По содержанию витаминов красный перец является
абсолютным лидером: 665,20 мг/100 г витамина А и 250 мг/100 г витамина
С. Чеснок показал наивысшее содержание селена --- 0,096\%,
обеспечивающего антиоксидантную защиту организма. Анализ аминокислотного
состава подтверждает, что чеснок и петрушка богаты валином (285,0 мг/100
г и 172,0 мг/100 г соответственно), лейцином (308,0 мг/100 г и 197,0
мг/100 г) и треонином (151,0 мг/100 г и 114,0 мг/100 г), что особенно
ценно для восстановления мышц, обмена веществ и поддержания иммунитета.
Таким образом, пряные травы представляют собой важные функциональные
ингредиенты с высоким содержанием белка, витаминов, минералов и
незаменимых аминокислот, что делает их актуальными в диетическом и
лечебно-профилактическом питании.

\emph{{\bfseries Финансирование.} Материалы подготовлены в рамках
научно-технической программы: BR24993031 «Разработка технологии
приготовления полезных продуктов питания для ежедневного рациона,
обогащенных природными антиоксидантами и биологически активными
веществами» по проекту (мероприятие №5) «Технология производства
порошков и экстрактов из пряных трав для кулинарных изделий и блюд» по
бюджетной программе 217 «Развитие науки» по подпрограмме 101
«Программно-целевое финансирование субъектов научной и/или
научно-технической деятельности» Министерства науки и высшего
образования Республики Казахстан на 2024-2026 годы.}
\end{multicols}

\begin{center}
{\bfseries Литература}
\end{center}

\begin{references}
1. Пряные травы и специи. Интернет ресурс. - 2024.
URL: \href{https://universityagro.ru/овощеводство/пряные-травы-и-специи/?ysclid=lw61pt05hp351265197}{https://universityagro.ru} - Дата обращения
10.05.2024

2. Кучменко Т.А., Абрамян М.К. Изучение состава экстрактов пряных трав в
процессе сушки //Вестник Воронежского государственного университета
инженерных технологий. -2022.-Т.84, № 1(91).- С.93--98. DOI
\href{http://doi.org/10.20914/2310-1202-2022-1-93-98}{10.20914/2310-1202-2022-1-93-98}

3. Сунцова Н.Ю., Попова Е.В. Пряные растения в культуре бесермян //
Ежегодник финно-угорских исследований.- 2022.-Т.16(4).- С.667--680 DOI
10.35634/2224-9443-2022-16-4-667-680

4. Рекомендуемая суточная норма потребления {[}Электронный
ресурс{]} // Википедия.-URL:\\
\href{https://ru.wikipedia.org/wiki/Рекомендуемая\_суточная\_норма\_потребления}{https://ru.wikipedia.org} -
Дата обращения 29.01.2025.

5. Таблица калорийности продуктов {[}Электронный
ресурс{]} // Health-Diet.ru.- URL:\\
\href{https://health-diet.ru/table_calorie_users/2314523/}{https://health-diet.ru} - Дата
обращения 29.01.2025.

6. Рыкова Н.Д. Пряные травы: распространение и применение в кулинарии
//Проблемы интенсивного развития животноводства и их решение: материалы
конференции, Брянск, 25-26 марта 2021 года. - Брянск: Брянский
государственный аграрный университет, 2021. -С.65-69.

7. Мартынова Е.В., Старовойтова Н.П. Биохимические характеристики
пряностей и пряных трав//Агроэкологические аспекты устойчивого развития
АПК: материалы XXI международной научной конференции, Брянск, 18 марта
2024 года.- С.45--50.

8. Кароматов И.Д., Ганиев Р. Эффективное лечебное
средство-сельдерей//Биология и интегративная медицина. -2018.- № 6(23).-
С.188--201.

9. Давыдова Р. Красный перец для мясопродуктов //Мясные технологии.
-2015.- № 3(147). -С.50--57.

10. Орынбасарова Б. А., Тасыбаева Ш. Б., Оралбекова Ж., Баимбетова Ж.,
Бекетова А. Использование пряно-ароматических растений южного Казахстана
в производстве соусов-приправ//WORLD SCIENCE.- 2018.- Vol.1.- № 2(30). -
С.73-78.

11. Пат.2323584 Российская Федерация, МПК A23G 3/48. Кондитерские
изделия на основе травяных смесей / Лутц Кристина (CH), Рихтерих Феликс
(CH); заявитель и патентообладатель \\PCT/CH02/00418.- № 2005105074/13;
заявл.24.07.2003; приоритет 25.07.2002 (CH).- Опубл. \\10.05.2008, Бюл. №
C2.

12. Ушакова, А. А. Разработка фитокомпозиции и соусов-приправ с
биологически активными веществами пряно-ароматических растений:автореф
канд. техн. наук. -. Санкт-Петербург, 2014. - С.1-18.

13. Пат. KR10-2019-0047933 Республика Корея, МПК A23L 7/157, A23L 5/43.
Травяной порошок для жарки/Ли Хёнгён (Lee Hyunkyung); заявитель и
патентообладатель Ли Хёнгён. - № 10-2017-0142337; заявл.30.10.2017.-
Опубл.09.05.2019.

14. Копылова, А. В., Давыденко, Н. И., Сапожников, А. Н., Ульянова, Г. С.
Использование пряного растительного сырья в технологии заварного
полуфабриката // Техника и технология пищевых производств. - 2021. - Т.
51, № 4. -С.701-711. DOI 10.21603/2074-9414-2021-4-701-711

15. Пат. WO 2004/092078 A1 Всемирная организация интеллектуальной
собственности, МПК C02F 3/32. Метод обработки травяной воды для
процветания биологических существ /Devaraj K.; заявитель и
патентообладатель PCT/IN2003/000242.- заявл.15.07.2003; приоритет
16.04.2003 (IN).- \\Опубл. 28.10.2004.

16. ГОСТ 33271-2015. Пряности сухие, травы и приправы овощные.-М.:
Стандартинформ, 2016.- С.1-7.

17. ГОСТ 28875-90. Пряности. Приемка и методы анализа (определение влаги
методом отгонки). -М.: Стандартинформ, 2011.- С.104-111.

18. ГОСТ ISO 927-2014. Пряности и приправы. Определение содержания
примесей и посторонних веществ.- М.: Стандартинформ, 2015.- С.1-5.

19. ГОСТ 28876-90. Пряности и приправы. Отбор проб.- М.: Стандартинформ,
2011. -С.118-120.

20. ГОСТ ISO 6571-2016. Определение содержания эфирных масел.- М.:
Стандартинформ, 2019. - С.1-11.

21. ГОСТ EN 12823-2-2014. Продукты пищевые. Определение содержания
витамина А методом высокоэффективной жидкостной хроматографии. -М.:
Стандартинформ, 2014.-С.1-10.

22. ГОСТ EN 12822-2014. Продукты пищевые. Определение содержания витамина
Е (α-, β-, γ- и δ-токоферолов) методом высокоэффективной жидкостной
хроматографии.-М.: Стандартинформ, 2013. -С.1-20.

23. ГОСТ 34151-2017. Продукты пищевые. Определение витамина C с помощью
высокоэффективной жидкостной хроматографии.-М.: Стандартинформ, 2017.-
С.1-10.

24. ГОСТ 26928-86. Продукты пищевые.Метод определения железа.-М.:
Стандартинформ, 2010.-С.107-110.

25. ГОСТ 26931-86. Сырье и продукты пищевые. Методы определения меди.-М.:
Стандартинформ, 2010.- С.133-137.

26. ГОСТ 26934-86. Сырье и продукты пищевые. Метод определения цинка.-
М.: Стандартинформ, 2010. -С.173-178.

27. ГОСТ 27076-86. Продукты пищевые и пищевое сырье. Метод определения
кальция. -М.: Стандартинформ, 1993.- С.6-9.

28. ГОСТ 30615-99. Сырье и продукты пищевые. Метод определения
фосфора.-М.: Стандартинформ, 2003.- С.1-4.

29. ГОСТ ISO 928-2015. Определение общего содержания золы.- М.:
Стандартинформ, 2018. - С.1-3.
\end{references}

\begin{center}
{\bfseries References}
\end{center}

\begin{references}
1. Prjanye travy i specii. Internet resurs. - 2024.
URL:https://universityagro.ru/ovoshhevods-

tvo/prjanye-travy-i-specii/?ysclid=lw61pt05hp351265197- Data
obrashhenija 10.05.2024. {[}in Russian{]}

2. Kuchmenko T.A., Abramjan M.K. Izuchenie sostava jekstraktov prjanyh
trav v processe sushki //Vestnik Voronezhskogo gosudarstvennogo
universiteta inzhenernyh tehnologij. -2022.-T.84, № 1(91).- S.93--98.
DOI 10.20914/2310-1202-2022-1-93-98. {[}in Russian{]}

3. Suncova N.Ju., Popova E.V. Prjanye rastenija v
kul' ture besermjan // Ezhegodnik finno-ugorskih
issledovanij.- 2022.-T.16(4).- S.667--680 DOI
10.35634/2224-9443-2022-16-4-667-680. {[}in Russian{]}

4. Rekomenduemaja sutochnaja norma potreblenija {[}Jelektronnyj
resurs{]} // Vikipedija.- URL:\\
\href{https://ru.wikipedia.org/wiki/Rekomenduemaja\_sutochnaja\_norma\_potreblenija}{https://ru.wikipedia.org} -
Data obrashhenija 29.01.2025. {[}in Russian{]}

5. Tablica kalorijnosti produktov {[}Jelektronnyj
resurs{]}//Health-Diet.ru.-URL:
\href{https://health-diet.ru/table\_calorie\_users/2314523/}{https://health-diet.ru}.- Data
obrashhenija 29.01.2025. {[}in Russian{]}

6. Rykova N.D. Prjanye travy: rasprostranenie i primenenie v kulinarii
//Problemy intensivnogo razvitija zhivotnovodstva i ih reshenie:
materialy konferencii, Brjansk, 25-26 marta 2021 goda. - Brjansk:
Brjanskij gosudarstvennyj agrarnyj universitet, 2021. -S.
65-69. {[}in Russian{]}

7. Martynova E.V., Starovojtova N.P. Biohimicheskie harakteristiki
prjanostej i prjanyh trav // \\Agrojekologicheskie aspekty ustojchivogo
razvitija APK: materialy XXI mezhdunarodnoj nauchnoj \\konferencii,
Brjansk, 18 marta 2024 goda.- S.45--50. {[}in Russian{]}

8. Karomatov I.D., Ganiev R. Jeffektivnoe lechebnoe
sredstvo-sel' derej//Biologija i integrativnaja \\medicina.
-2018.- № 6(23).- S.188--201. {[}in Russian{]}

9. Davydova R. Krasnyj perec dlja mjasoproduktov //Mjasnye tehnologii.
-2015.- № 3(147). -S.50--57. {[}in Russian{]}

10. Orynbasarova B. A., Tasybaeva Sh. B., Oralbekova Zh., Baimbetova Zh.,
Beketova A. Ispol' zovanie prjano-aromaticheskih rastenij
juzhnogo Kazahstana v proizvodstve sousov-priprav//WORLD SCIENCE.-
2018. - Vol.1.- № 2(30). - S.73-78. {[}in Russian{]}

11. Pat.2323584 Rossijskaja Federacija, MPK A23G 3/48. Konditerskie
izdelija na osnove travjanyh smesej / Lutc Kristina (CH), Rihterih
Feliks (CH); zajavitel'{} i
patentoobladatel'{} PCT/CH02/00418.- № 2005105074/13;
zajavl.24.07.2003; prioritet 25.07.2002 (CH).- Opubl.10.05.2008, Bjul.
№ C2. {[}in \\Russian{]}

12. Ushakova, A. A. Razrabotka fitokompozicii i sousov-priprav s
biologicheski aktivnymi veshhestvami prjano-aromaticheskih
rastenij:avtoref kand. tehn. nauk. -. Sankt-Peterburg, 2014. - S.1-18.

13. Pat. KR10-2019-0047933 Respublika Koreja, MPK A23L 7/157, A23L 5/43.
Travjanoj poroshok dlja zharki/Li Hjongjon (Lee Hyunkyung);
zajavitel'{} i patentoobladatel'{} Li
Hjongjon. - № 10-2017-0142337; zajavl.30.10.2017.- Opubl.09.05.2019.
{[}in Russian{]}

14. Kopylova, A. V., Davydenko, N. I., Sapozhnikov, A. N.,
Ul' janova, G. S. Ispol' zovanie prjanogo
rastitel' nogo syr' ja v tehnologii
zavarnogo polufabrikata // Tehnika i tehnologija pishhevyh proizvodstv.
- 2021. - T.51, № 4. -S.701-711. DOI
10.21603/2074-9414-2021-4-701-711. {[}in Russian{]}

15. Pat. WO 2004/092078 A1 Vsemirnaja organizacija
intellektual' noj sobstvennosti, MPK C02F 3/32. Metod
obrabotki travjanoj vody dlja procvetanija biologicheskih sushhestv
/Devaraj K.; zajavitel'{} i\\
patentoobladatel'{} PCT/IN2003/000242.-
zajavl.15.07.2003; prioritet 16.04.2003 (IN).- Opubl.28.10.2004. {[}in
Russian{]}

16. GOST 33271-2015. Prjanosti suhie, travy i pripravy ovoshhnye.-M.:
Standartinform, 2016.- S.1-7. {[}in Russian{]}

17. GOST 28875-90. Prjanosti. Priemka i metody analiza (opredelenie vlagi
metodom otgonki). -M.: Standartinform, 2011.- S.104-111. {[}in
Russian{]}

18. GOST ISO 927-2014. Prjanosti i pripravy. Opredelenie soderzhanija
primesej i postoronnih veshhestv.- M.: Standartinform, 2015.- S.1-5.
{[}in Russian{]}

19. GOST 28876-90. Prjanosti i pripravy. Otbor prob.- M.: Standartinform,
2011. -S.118-120.

20. GOST ISO 6571-2016. Opredelenie soderzhanija jefirnyh masel.- M.:
Standartinform, 2019. - S.1-11. {[}in Russian{]}

21. GOST EN 12823-2-2014. Produkty pishhevye. Opredelenie soderzhanija
vitamina A metodom \\vysokojeffektivnoj zhidkostnoj hromatografii. -M.:
Standartinform, 2014.-S.1-10. {[}in Russian{]}

22. GOST EN 12822-2014. Produkty pishhevye. Opredelenie soderzhanija
vitamina E (α-, β-, γ- i δ-tokoferolov) metodom vysokojeffektivnoj
zhidkostnoj hromatografii.-M.: Standartinform, 2013. -S.1-20. {[}in
Russian{]}

23. GOST 34151-2017. Produkty pishhevye. Opredelenie vitamina C s
pomoshh' ju vysokojeffektivnoj zhidkostnoj
hromatografii.-M.: Standartinform, 2017.- S.1-10. {[}in Russian{]}

24. GOST 26928-86. Produkty pishhevye.Metod opredelenija zheleza.-M.:
Standartinform, 2010.-S.107-110. {[}in Russian{]}

25. GOST 26931-86. Syr' e i produkty pishhevye. Metody
opredelenija medi.-M.: Standartinform, 2010.- S.133-137. {[}in
Russian{]}

26. GOST 26934-86. Syr' e i produkty pishhevye. Metod
opredelenija cinka.- M.: Standartinform, 2010. -S.173-178. {[}in
Russian{]}

27. GOST 27076-86. Produkty pishhevye i pishhevoe syr' e.
Metod opredelenija kal' cija. -M.: \\Standartinform, 1993.-
S.6-9. {[}in Russian{]}

28. GOST 30615-99. Syr' e i produkty pishhevye. Metod
opredelenija fosfora.-M.: Standartinform, 2003.- S.1-4. {[}in
Russian{]}

29. GOST ISO 928---2015. Opredelenie obshhego soderzhanija zoly.- M.:
Standartinform, 2018. - S.1-3. {[}in Russian{]}
\end{references}

\begin{authorinfo}
\emph{{\bfseries Сведения об авторах}}

Чоманов У.Ч. - доктор технических наук, ТОО «Казахский
научно-исследовательский институт перерабатывающей и пищевой
промышленности»,Алматы, Казахстан е-mail:
\href{mailto:chomanov\_u@mail.ru}{\nolinkurl{chomanov\_u@mail.ru}};
\url{https://orcid.org/0000-0002-5594-8216}

Жумалиева Г.Е. - кандидат технических наук, ТОО «Казахский
научно-исследовательский институт перерабатывающей и пищевой
промышленности» Алматы, Казахстан, е-mail:
\href{mailto:g.zhumalieva@rpf.kz}{\nolinkurl{g.zhumalieva@rpf.kz}};
\url{https://orcid.org/0000-0002-5028-465X}

Асан А.Н. - магистр технических наук, ТОО «Казахский
научно-исследовательский институт перерабатывающей и пищевой
промышленности» Алматы, Казахстан, е-mail:
\href{mailto:arailym\_178@mail.ru}{\nolinkurl{arailym\_178@mail.ru}}
\url{https://orcid.org/0009-0007-4260-7221}

\emph{{\bfseries Information about authors}}

Chomanov U.Ch. - Doctor of Technical Sciences, LTD ``Kazakh Research
Institute of Processing and Food Industry'', Almaty, Kazakhstan е-mail:
\href{mailto:chomanov\_u@mail.ru}{\nolinkurl{chomanov\_u@mail.ru}};

Zhumalieva G.E.-Candidate of Technical Sciences, LTD ``Kazakh Research
Institute of Processing and Food Industry'', Almaty, Kazakhstan е-mail:
\href{mailto:g.zhumalieva@rpf.kz}{\nolinkurl{g.zhumalieva@rpf.kz}};

Asan A.N. -- Master of Technical Sciences, LTD ``Kazakh Research
Institute of Processing and Food Industry'', Almaty, \\Kazakhstan
\href{mailto:arailym\_178@mail.ru}{\nolinkurl{arailym\_178@mail.ru}}\
\end{authorinfo}
