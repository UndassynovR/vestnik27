\newpage
\let\cleardoublepage\clearpage
\part{Производственные и обрабатывающие отрасли}
\chapter{Пищевая технология}
\ID{МРНТИ 65.63.03}{}

\begin{articleheader}
\sectionwithauthors{А.А. Бектурганова, А.Ж. Хастаева, А.М. Омаралиева, Н.С. Машанова, С.Л. Гаптар}{ТОЛТЫРҒЫШТАРДЫҢ СҮЗБЕ ӨНІМІНІҢ ҚҰРЫЛЫМДЫҚ-МЕХАНИКАЛЫҚ ҚАСИЕТТЕРІНЕ ӘСЕРІН ЗЕРТТЕУ}

{\bfseries
\textsuperscript{1}А.А. Бектурганова\alink{https://orcid.org/0000-0002-0906-2027}\textsuperscript{\envelope },
\textsuperscript{1}А.Ж. Хастаева\alink{https://orcid.org/0000-0002-2679-0210},
\textsuperscript{1}А.М. Омаралиева\alink{https://orcid.org/0000-0003-4432-8828},
\textsuperscript{2}Н.С. Машанова\alink{https://orcid.org/0000-0001-8664-5173},
\textsuperscript{3}С.Л. Гаптар\alink{https://orcid.org/0000-0002-7823-6369}
}
\end{articleheader}

\begin{affiliation}
\textsuperscript{1}Қ. Құлажанов атындағы Қазақ технология және бизнес университеті, Астана, Қазақстан,

\textsuperscript{2}Сейфуллин атындағы Қазақ агротехникалық зерттеу университеті, Астана, Қазақстан,

\textsuperscript{3}Новосибирск аграрлық мемлекеттік университеті, Новосибирск, Ресей Федерациясы

\raggedright \textsuperscript{\envelope }Корреспондент-автор: 1968al1@mail.ru
\end{affiliation}

Дұрыс тамақтану адамның қалыпты өсуі мен дамуын қамтамасыз етеді,
аурудың алдын алуға, өмірді ұзартуға, өнімділікті арттыруға ықпал етеді
және адамдардың қоршаған ортаға адекватты бейімделуіне жағдай жасайды.
Қазақстан халқының көпшілігінде дәрумендердің, минералдардың, толыққанды
ақуыздардың жеткіліксіз тұтынылуына және олардың ұтымсыз арақатынасына
байланысты тамақтану жүйесінің бұзылуы анықталады. Сондықтан, әр түрлі
жастағы адамдардың тамақтануы үшін байытылған өнімдердің қолданыстағы
технологияларын жетілдіру және жаңа технологияларын әзірлеу, оның ішінде
өсімдік компоненттері есебінен макро - және микроэлементтермен байыту
басым бағыт болып табылады.

Жұмыстың осы кезеңінде клесі міндет қойылды -- желе түзу қасиеті бар
ингредиенттерді қолданатын сүзбе өнімі түріндегі ашытылған сүт өнімін
алу. Зерттеулерде жүгері ұнын ашытылған сүт ақуызы өнімін өндіруде
өсімдік қоспасы ретінде пайдалану мүмкін. Химиялық құрамды салыстырмалы
талдау көрсеткендей, алдын-ала термиялық өңдеуден өткен жүгері ұны
дәрумендердің, минералдардың, аминқышқылдарының, диеталық талшықтардың
ең құнды көзі болып табылады, сондықтан оларды майсыз сүзбемен біріктіру
нәтижесінде оның құрамы қоректік заттардың бүкіл кешенімен едәуір
байытылады. Сонымен қатар, олар оңай өңделеді, жергілікті өсімдік
шикізаты, яғни бұл оларды өндіріске жеткізу шығындарын азайтады.

{\bfseries Түйін сөздер:} технология, ашытылған сүт өнімі, жүгері ұны,
майсыз сүзбе, аралас өнім, астық өнімі.

\begin{articleheader}
{\bfseries ИССЛЕДОВАНИЕ ВЛИЯНИЯ НАПОЛНИТЕЛЕЙ НА СТРУКТУРНО-МЕХАНИЧЕСКИЕ СВОЙСТВА ТВОРОЖНОГО ПРОДУКТА}

{\bfseries
\textsuperscript{1}А.А. Бектурганова\textsuperscript{\envelope },
\textsuperscript{1}А.Ж. Хастаева,
\textsuperscript{1}А.М. Омаралиева,
\textsuperscript{2}Н.С. Машанова,
\textsuperscript{3}С.Л. Гаптар
}
\end{articleheader}

\begin{affiliation}
\textsuperscript{1}Казахский университет технологии и бизнеса им К.Кулажанова, Астана, Казахстан,

\textsuperscript{2}Казахский агротехнический исследовательский университет имени С.Сейфуллина Астана, Казахстан,

\textsuperscript{3}Новосибирский аграрный государственный университет, Новосибирск, Российская Федерация,

e-mail: 1968al1@mail.ru
\end{affiliation}

Правильное питание обеспечивает нормальный рост и развитие человека,
способствует профилактике заболеваний, продлению жизни, повышению
работоспособности и создает условия для адекватной адаптации людей к
окружающей среде. У большинства населения Казахстана выявляются
нарушения системы питания, обусловленные недостаточным потреблением
витаминов, минеральных веществ, полноценных белков и нерациональным их
соотношением. Поэтому, приоритетным направлением является
совершенствование существующих и разработка новых технологий обогащенных
продуктов для питания людей разного возраста, в том числе обогащение
макро- и~микронутриентами за счет растительных компонентов.

На данном этапе работы была поставлена задача -- получить
комбинированный кисломолочный продукт типа творожного продукта, для чего
использовали ингредиенты, обладающие свойством желеобразования. В
исследованиях, использование кукурузной муки в качестве растительной
добавки при производстве кисломолочного белкового продукта является
возможным. Сравнительный анализ химического состава показал, что
кукурузная мука, подвергнутые предварительной термической обработке,
являются ценнейшим источником витаминов, минеральных веществ,
амино­кислот, пищевых волокон, поэтому, в результате комбинировании их с
нежир­ным творогом будут существенно обогащать его состав целым
комплексом полезных веществ. Кроме того, они легко поддаются
технологической обра­ботке, являются местным растительным сырьем, что
снижает затраты по их доставке на производство.

{\bfseries Ключевые слова:} технология, кисломолочный продукт, кукурузная
мука, творог обезжиренный, комбинированный продукт, зерновой продукт.

\begin{articleheader}
{\bfseries STUDY OF THE INFLUENCE OF FILLERS ON THE STRUCTURAL AND MECHANICAL PROPERTIES OF A COTTAGE CHEESE PRODUCT}

{\bfseries
\textsuperscript{1}A.A. Bekturganova\textsuperscript{\envelope },
\textsuperscript{1}A.M. Omaraliyeva,
\textsuperscript{1}A.Zh. Khastayeva,
\textsuperscript{1}N.S. Mashanova,
\textsuperscript{3}S.L. Gaptar
}
\end{articleheader}

\begin{affiliation}
\textsuperscript{1}Kazakh University of Technology and Business named after K.Kulazhan, Аstanа, Kazakhstan,

\textsuperscript{2}Kazakh Agrotechnical Research University named after S.Seifullin Аstanа, Kazakhstan,

\textsuperscript{3}Novosibirsk State Agrarian University, Novosibirsk, Russian Federation,

e-mail: 1968al1@mail.ru
\end{affiliation}

Proper nutrition ensures normal human growth and development,
contributes to the prevention of diseases, prolongation of life,
improvement of working capacity and creates conditions for adequate
adaptation of people to the environment. The majority of the population
of Kazakhstan has nutritional disorders due to insufficient intake of
vitamins, minerals, full-fledged proteins and their irrational ratio.
Therefore, the priority is to improve existing and develop new
technologies for fortified foods for people of different ages, including
the enrichment of macro- and micronutrients from plant components.

At this stage of the work, the task was set to obtain a combined
fermented milk product such as a curd product, for which ingredients
with the property of gelling were used. In research, the use of corn
flour as a vegetable additive in the production of a fermented milk
protein product is possible. A comparative analysis of the chemical
composition has shown that corn flour, subjected to preliminary heat
treatment, is the most valuable source of vitamins, minerals, amino
acids, dietary fiber, therefore, as a result of combining them with
low-fat cottage cheese, they will significantly enrich its composition
with a whole complex of useful substances. In addition, they are easy to
process, they are local plant raw materials, which reduces the cost of
their delivery to production.

{\bfseries Keywords:} technology, fermented milk product, corn flour,
low-fat cottage cheese, combined product, grain product.

\begin{multicols}{2}
{\bfseries Кіріспе.} Соңғы жылдары тамақтанудың өзгеруіне байланысты
адамдардың көпшілігінде негізгі қоректік заттар мен энергияны тұтынудың
біртіндеп төмендеуі байқалады. Себептердің бірі-өнімнің құнын арзандату
үшін тамақ өнімдерін өндірушілер пайдаланатын шикізаттың өзара
алмастырылуы. Рационда өсімдік және жануар тектес ақуыздар мен ақуыздық
заттардың азаюы анықталды, бұл халықтың жекелеген санаттарында ақуыз
жеткіліксіздігінің белгілерін қалыптастыру үшін алғышарттар жасайды. Бұл
мәселені шешудің ең тиімді және үнемді жолы-азық-түлікті байыту немесе
маңызды қоректік факторларға бай тағамдық қоспаларды пайдалану
{[}1-3{]}.

Ағзаны өмірлік маңызды дәрумендермен және микроэлементтермен қамтамасыз
ету мәселесі ерекше қызығушылық тудырады.

Комбинирленген сүт өнімдерін өндіруде өсімдік компоненттерін пайдалану
өнімдердің тағамдық және биологиялық құндылығын арттырып қана қоймайды,
сонымен қатар стартерлік микроорганизмдердің дамуына басқаша әсер етеді.
Зерттеу жұмысында ашытылған сүт өнімін өсімдік ақуыздарымен,
көмірсулармен, майлармен, минералды заттармен байыту үшін жүгері ұнын
пайдаланамыз {[}4-5{]}.

Әдеби деректерге аналитикалық шолу жасау негізінде зерттеудің бағыты,
мақсаты мен міндеттері негізделді. Зерттеудің бұл кезеңінің мақсаты
біріктірілген ашытылған сүт өнімінің құрылымы мен консистенциясына әсер
ететін толтырғыштардың мөлшерін зерттеу болды. Жұмыстың осы кезеңінде
сүзбе өнімінің түрінің құрама ашытылған сүт өнімін алу міндеті қойылды,
ол үшін гельдік қасиеті бар ингредиенттер пайдаланылды.

Ақуыз өнімдерін өндіру ашыту, құрылымды қалыптастыру және араластыру
процесіне негізделген, оның барысында енгізілген компоненттер бір
массаға біріктіріледі. Тағамдық қоспаның құрылымдық-механикалық
қасиеттері үлкен технологиялық маңызға ие және көптеген факторларға
байланысты, олардың ішіндегі ең маңыздысы ингредиенттер балансы болып
табылады {[}6{]}. Бұл зерттеуде біртекті, жұмсақ, нәзік консистенциялы
өнімді алу қажет.

{\bfseries Материалдар мен әдісдер.} Қ.Құлажанов атындағы Қазақ технология
және бизнес университетінің «Технология және стандарттау» кафедрасының
зертханаларында эксперименталды зерттеулер жүргізілді.

Әдеби дереккөздерді талдау және тұжырымдалған міндеттер негізінде
зерттеу нысаны ретінде келесілер таңдалды:

- ҚР СТ 94-95 - Сүзбе. Техникалық шарттар;

- МемСТ 327342 «Жартылай фабрикаттар. Жеміс-көкөніс пюресі,
консервіленген. Техникалық шарттар»;

- МЕМСТ14176-69 - «Жүгері ұны. Техникалық шарттар».

Зерттеу барысында қойылған міндеттерді шешу үшін жалпы қабылданған
стандартты әдістер, оның ішінде органолептикалық және физика-химиялық
талдау әдістері қолданылды. Дәнді қоспалары бар қышқыл сүтті ақуыз
өнімдерінің сынамаларын іріктеу және оларды зерттеуге дайындау МЕМСТ
26809-86 талаптарына сәйкес жүргізілді. Ал дайын өнімдердің титрленетін
қышқылдығы МЕМСТ 3624-92 «Сүт және сүт өнімдері. Қышқылдықты анықтаудың
титриметриялық әдістері» стандарты бойынша анықталды.

Жүгері ұны қосылған дәнді сүзбе массалары мен майсыз сүзбенің
микробиологиялық зерттеулерге арналған сынамаларын дайындау МЕМСТ
26669-85 «Тағамдық және дәмдік өнімдер. Микробиологиялық талдауға
арналған сынамаларды дайындау» стандартына сәйкес жүргізілді. Ал
өнімдердің белсенді қышқылдығы (рН) МЕМСТ 32892-2014 талаптарына сай
анықталды.

{\bfseries Нәтижелер мен талқылау.} Тағамдық қоспаның құрылым түзу процесі
оның дисперсия дәрежесіне, құрамындағы заттардың табиғатына және ақуыз
жүйесінің күйіне тікелей байланысты. Зерттеу барысында негізгі мақсат --
құрылымы тұрақты, үйлестірілген құрамға ие қышқыл сүтті өнім, яғни сүзбе
тәрізді өнім алу болды. Осы мақсатта желе түзуші қасиеттері бар
ингредиенттер қолданылды, олар өнімнің қажетті консистенциясы мен
сапалық көрсеткіштерін қамтамасыз етуге мүмкіндік берді {[}7{]}.

Құрамдас компонент ретінде жүгері ұнын қолдану арқылы дайындалған қышқыл
сүтті ақуыз өнімінің санитарлық-гигиеналық көрсеткіштерін мұқият
зерттеу, сондай-ақ өсімдік шикізатын алдын ала термиялық өңдеу барысында
олардың өзгеру динамикасын талдау қажет. Дәнді қоспа ретінде қуырудан
өткен жүгері ұнын пайдалану дайын өнімнің сапасын арттырып, оның
қауіпсіздігі мен тұрақтылығын қамтамасыз етеді. Ашыған сүтке қосар
алдында жүгері ұны №29-32 електен өткізіліп, содан кейін 20 °С
температурада майсыз сүтте ерітіліп, негізгі сүтті негізге енгізіледі.
Бұл оның өнім құрылымында біркелкі таралуына және сапалық
сипаттамаларының жақсаруына ықпал етеді.

Әрі қарай тәжірибе барысында енгізілген қоспаның дозасы, ашыту ұзақтығы,
өсімдік құрамдас бөлігінің тромбтың құрылымына әсері және
органолептикалық көрсеткіштер сияқты көрсеткіштерді зерттейміз.

Сондай-ақ, жоғарыда айтылғандардың барлығынан басқа, жүгері ұны майы аз
сүзбемен үйлескенде құрылымдаушы және тұрақтандырушы агент ретінде
әрекет етеді, өйткені дәнді дақылдардың полисахаридтері мен ақуыздары
казеинаттармен ақуызды-полисахаридті кешендерді түзуге қабілетті,
осылайша сүт протеинінің қоспасының эмульгаторлық, тұрақтандырушы және
суды сақтайтын қасиеттерін арттырады. Алынған композицияның құндылығы
сонымен қатар жануарлардан алынатын ақуыздар мен майлардың бір бөлігін
өсімдік тектестерге ауыстыруда болады.

Жұмыстың бірінші кезеңінде біз енгізілетін жарма қоспасының дозасын
анықтаймыз. Жүгері ұнтағының концентрациясын 0,5\%, 1\%, 1,5\%, 2\%
(«Сүт және сүт өнімдерінің қауіпсіздігі туралы» Кеден одағының
техникалық регламенті бойынша) таңдап аламыз және салыстыру үшін
толтырғыш қоспай бақылау экспериментін жүргіземіз {[}8{]}. Зерттеу
нәтижелері 1-кестеде берілген.
\end{multicols}

\begin{table}[H]
\caption*{1-кесте. Жүгері ұнының мөлшеріне байланысты қышқыл сүт негізінің органолептикалық көрсеткіштері}
\centering
\begin{tblr}{
  width = \linewidth,
  colspec = {Q[180]Q[108]Q[108]Q[146]Q[210]Q[215]},
  cells = {c},
  cell{1}{1} = {r=2}{},
  cell{1}{2} = {c=5}{0.786\linewidth},
  vlines,
  hline{1,3-7} = {-}{},
  hline{2} = {2-6}{},
}
Көрсеткіш                       & Енгізілген жарма қоспасының мөлшері, \% &                      &                                   &                                                 &                                                    \\
                                & 0                                       & 0,5                  & 1                                 & 1,5                                             & 2                                                  \\
Сыртқы түрі және консистенциясы & біртекті, тығыз                         & біртекті, тығыз      & біртекті, тығыз, тұтқырлығы төмен & біртекті, тығыз, тұтқырлығы төмен               & біртекті, тығыз, аздап ұнды                        \\
Дәмі мен иісі                   & жағымды қышқыл сүтті                    & жағымды қышқыл сүтті & жағымды қышқыл сүтті              & қоспалардың аздап дәмі бар жағымды қышқыл сүтті & жүгері ұнтағының тән дәмі бар жағымды қышқыл сүтті \\
Түсі                            & ақ                                      & ақ                   & сәл сарғыш                        & сәл сарғыш                                      & сарғыш                                             \\
Ашыту ұзақтығы, сағ             & 4,0-4,5                                 & 4,0                  & 3- 3,5                            & 3,5                                             & 3,0                                                
\end{tblr}
\end{table}

\begin{table}[H]
\caption*{2-кесте. Жеміс-көкөніс толтырғышының мөлшерінің қышқыл сүт өнімінің органолептикалық көрсеткіштеріне әсері}
\centering
\begin{tblr}{
  width = \linewidth,
  colspec = {Q[171]Q[213]Q[83]Q[387]Q[83]},
  cells = {c},
  cell{1}{1} = {r=2}{},
  cell{1}{2} = {c=4}{0.766\linewidth},
  vlines,
  hline{1,3-6} = {-}{},
  hline{2} = {2-5}{},
}
Толтырғыш дозасы & Органолептикалық бағалау &         &                                                &         \\
                 & консистенциясы           & ұпайлар & дәмі мен иісі                                  & ұпайлар \\
10\%             & орташа тығыз, сәл сұйық  & 4,5     & қышқыл сүтті, айқын білінетін                  & 3,5     \\
15\%             & тығыз, сарысу бөлінбеген & 5,0     & қышқыл сүтті, сәл алма-сәбіз пюресінің дәмімен & 5       \\
20\%             & тығыз, сарысу бөлінбеген & 4,8     & қышқыл сүтті, алма-сәбіз пюресіне тән дәмімен  & 4,5     
\end{tblr}
\end{table}

\begin{multicols}{2}
Зерттеу нәтижесінде өнімдегі жарма толтырғышының ең оңтайлы
концентрациясы 1\% құрайтыны анықталды, бұл басқа үлгілермен
салыстырғанда ең жақсы органолептикалық көрсеткіштерді береді. Жүгері
ұнтағын 0,5\% мөлшерде пайдаланғанда ашыту уақыты 4,5 сағатқа дейін
артады, толтырғыштың 1\%-дан астамын қосқанда өнімнің органолептикалық
қасиеттері нашарлайды, бірақ ашыту процесі қысқарады, бірақ өнім жүгері
ұнтағының дәмімен шығады. Кестедегі мәліметтерді талдай отырып, 1\%
мөлшерінде қосылған өсімдік қоспасы ашытылған сүт өнімінің ашыту
ұзақтығына және органолептикалық қасиеттеріне оң әсер етеді деп
қорытынды жасауға болады.

Ашытқы микроорганизмдерінің қышқыл түзу қабілетіне жеміс-көкөніс пюресі
дозасы. Бақылау үлгісі ретінде жеміс-көкөніс пюресі қосылмаған сүзбе
массасы болды.

1-нұсқа - 10\%, 2-нұсқа - 15\%, 3-нұсқа - 20\%. Жүргізілген зерттеу
нәтижелері суретте көрсетілген.

Мәліметтерді талдай отырып, сүзбе массасына өсімдік толтырғыштарын
енгізу қышқыл түзілу процесін баяулатады деген қорытындыға келуге
болады. Алма-сәбіз пюресі оңтайлы дозасын анықтау үшін өнімнің
органолептикалық көрсеткіштері анықталды (2-кесте).

Жеміс-көкөніс толтырғышының мөлшерінің өнімнің консистенциясына әсерін
талдау нәтижесінде, өсімдік толтырғыштарының 15\% мөлшерінде қосылған
өнімдер ең жоғары органолептикалық көрсеткіштерге ие екені анықталды.

Келесі зерттеу кезеңінде сүзбе өнімінің ашыту процесіне полизакваска
мөлшерінің әсері зерттелді.

Ашыту процесіне енгізілетін ашытқы қоспасының құрамы ғана емес, сонымен
қатар оның мөлшері де үлкен әсер етеді. Өнімнің жоғары органолептикалық
көрсеткіштерін қамтамасыз ететін полизакваска мөлшерінің оңтайлы
деңгейін анықтау мақсатында зерттеулер жүргізілді. Ізденіс
экспериментінің нәтижелері мен ғылыми әдебиеттерге сүйене отырып, сынақ
үшін үш түрлі закваска мөлшері таңдалды: 1-нұсқа -- 3\%, 2-нұсқа -- 5\%,
3-нұсқа -- 7\%. Зерттеу нәтижелері 3-кестеде келтірілген.
\end{multicols}

\begin{table}[H]
\caption*{3-кесте. Закваска мөлшерінің ашытылған өнімнің сапалық көрсеткіштеріне әсері}
\centering
\begin{tblr}{
  width = \linewidth,
  colspec = {Q[62]Q[123]Q[98]Q[213]Q[240]Q[65]Q[133]},
  cells = {c},
  cell{1}{1} = {r=2}{},
  cell{1}{2} = {r=2}{},
  cell{1}{3} = {r=2}{},
  cell{1}{4} = {c=3}{0.518\linewidth},
  cell{1}{7} = {r=2}{},
  vlines,
  hline{1,3-6} = {-}{},
  hline{2} = {4-6}{},
}
Өнім    & Ашытқы дозасы, \% & Қышқыл\-дық, ºТ & Органолептикалық көрсеткіштер        &                                           &         & Тұтқырлық, Па*с*10-3 \\
        &                   &              & консистенциясы                       & дәмі мен иісі                             & ұпайлар &                      \\
Нұсқа 1 & 3                 & 98           & тығыз, сәл тұтқыр, сарысу бөлінбеген & қышқыл сүтті, әлсіз білінетін             & 4,5     & 3,1                  \\
Нұсқа 2 & 5                 & 110          & біртекті, серпімді, сарысу бөлінген  & қышқыл сүтті, айқын білінетін             & 5,0     & 3,2                  \\
Нұсқа 3 & 7                 & 120          & біртекті, серпімді, сарысу бөлінген  & қышқыл сүтті, айқын білінетін, тым қышқыл & 4,3     & 3,2                  
\end{tblr}
\end{table}

\begin{multicols}{2}
3-кестеден ашытқы мөлшерін 7\% - ға дейін арттыру (3-нұсқа) өнімнің
консистенциясына, қышқылдығына және органолептикалық көрсеткіштеріне
теріс әсер ететінін көруге болады. Атап айтқанда, ашыту дозасы
жоғарылаған кезде ашыту процесі қарқынды жүретіні, ашыту уақыты 2,5-3
сағатқа дейін азаятыны анықталды. Бұл жағдайда қоспаның қышқылдығы өте
тез өседі, ал қоспаның ақуыз құрылымы әлі қалыптасуға уақыт жоқ
{[}9,10{]}. Зерттеу нәтижелері бойынша өнімге жоғары органолептикалық
көрсеткіштерді беретін ашытқы мөлшері анықталды - 5\%. Ашыту
температурасы ашытқы құрамына кіретін микроорганизмдердің түрімен
анықталды-барлық микроорганизмдер 38-40\textsuperscript{º}С
температурада оңтайлы дамуға ие.

Келесі кезеңде сүзбе өнімінің органолептикалық, физика-химиялық
көрсеткіштері, химиялық құрамы зерттелді.

Сүзбе өнімінің органолептикалық көрсеткіштерінің сипаттамасы 4-кестеде
келтірілген.

Сүт шикізатын өсімдік тектес толтырғыштармен байыта отырып, әр түрлі
жастағы сүзбе өнімінің дәмдік қасиеттерін бағалау кезінде өнімнің барлық
компоненттердің жақсы дәмдік үйлесімділігі бар екендігі анықталды. Сүзбе
өнімінің физика-химиялық көрсеткіштері мен энергетикалық құндылығы
5-кестеде келтірілген.
\end{multicols}

\begin{table}[H]
\caption*{4-кесте. Сүзбе өнімінің органолептикалық көрсеткіштері}
\centering
\begin{tblr}{
  width = \linewidth,
  colspec = {Q[63]Q[181]Q[475]Q[221]},
  row{1} = {c},
  row{2} = {c},
  cell{1}{1} = {r=2}{},
  cell{1}{2} = {c=3}{0.876\linewidth},
  cell{3}{1} = {c},
  vlines,
  hline{1,3-4} = {-}{},
  hline{2} = {2-4}{},
}
Өнім        & Көрсеткіштің атауы                                  &                                                                                                                                              &                                                             \\
            & сыртқы түрі, консистенциясы                         & дәмі, иісі                                                                                                                                   & түсі                                                        \\
Сүзбе өнімі & біртекті, серпімді, біркелкі ұйыған, газ түзілмеген & таза, қышқыл сүтті, бөгде иістерсіз, қалыпты тәттілікке ие, қосылған өсімдік толтырғыштарының (жүгері ұны, алма-сәбіз пюресі) дәмі сезілетін & қосылған толтырғышқа тән түс, бүкіл масса бойынша біркелкі. 
\end{tblr}
\end{table}

\begin{table}[H]
\caption*{5-кесте -- Сүзбе өнімінің физика-химиялық көрсеткіштері және энергетикалық құндылығы}
\centering
\begin{tblr}{
  row{1} = {c},
  cell{2}{2} = {c},
  cell{3}{2} = {c},
  cell{4}{2} = {c},
  cell{5}{2} = {c},
  cell{6}{2} = {c},
  cell{7}{2} = {c},
  cell{8}{2} = {c},
  cell{9}{2} = {c},
  cell{10}{2} = {c},
  cell{11}{2} = {c},
  hlines,
  vlines,
}
Көрсеткіштер                                               & Сүзбе өнімі \\
Массалық үлес, \%                                          &             \\
Май                                                        & 0,06        \\
Ақуыз                                                      & 3,0         \\
Көмірсулар                                                 & 9,6         \\
Құрғақ заттар                                              & 15,8        \\
Белсенді қышқылдық, рН                                     & 4,7         \\
Титрленетін қышқылдық, ºТ                                  & 105         \\
Кәсіпорыннан шығару кезіндегі температура, ºС, жоғары емес & 2-4         \\
Сақтау мерзімі, сағат.                                     & 72          \\
Энергетикалық құндылығы, кДж                               & 334,7       
\end{tblr}
\end{table}

\begin{multicols}{2}
{\bfseries Қорытынды.} Жұмыстың осы кезеңінде бастапқы шикізаттың барлық
биологиялық құнды заттарын жоғалтпай, өнімге жақсы дәм беретін бұрын
алынған зерттеу нәтижелері негізінде жүгері ұнынан астық қоспасын алудың
технологиялық параметрлерін пысықтау міндеті қойылды.

Зерттеу нәтижесінде жүгері ұнының технологиялық өңдеуге жақсы
бейімделетіні және оны өңдеу үшін үлкен энергетикалық шығындар талап
етілмейтіні анықталды. Жергілікті шикізаттан алынатын бұл дәнді дақыл
құнды химиялық құрамға ие, оның арқасында қышқыл сүтті ақуызды негізді
(майсыз сүзбе) витаминдермен, минералды заттармен, тағамдық талшықтармен
және аминқышқылдарымен байытуға мүмкіндік береді. Сонымен қатар, жүгері
ұны салыстырмалы түрде арзан және қолжетімді.

Осылайша, жүгері ұны күрделі шикізат құрамына ие үйлестірілген қышқыл
сүтті өнімдерді өндіруде пайдаланылатын өсімдік шикізатына қойылатын
барлық талаптарға толық сәйкес келеді.

Алдыңғы зерттеулер нәтижесінде термиялық өңдеуден өткен бұл дәнді
дақылдың жоғары ісіну қабілетіне, ылғал сіңіру қасиетіне және ылғалды
жылдам сіңіру жылдамдығына ие екендігі анықталды. Жүгері ұнынан
дайындалатын өсімдік қоспасын өндіру технологиясы бірнеше кезеңнен
тұрады: қуыру, салқындату, қайнату және оны қышқыл сүтті ақуызды негізге
енгізу.

Жоғарыда айтылғандарды ескере отырып, әртүрлі жастағы тұтынушылар үшін
пайдалы нутриенттермен байытылған арнайы өнімдерді әзірлеу мен олардың
технологиясын жасау өзекті мәселе болып табылады.
\end{multicols}

\begin{center}
{\bfseries Әдебиеттер}
\end{center}

\begin{references}
1. Пастушкова Е.В., Мысаков Д.С., Чугунова О.В. Некоторые аспекты фактора
питания и здоровья человека // Медико-фармацевтический журнал «Пульс».
-2016. -№ 4. --С.67-71.

2. Погосян Д.Г. Функциональные пищевые ингредиенты в молочных продуктах /
Д.Г. Погосян, И.В. Гаврюшина // Переработка молока: технология,
оборудование, продукция: отраслевой специализированный журнал. --2013.
-- № 3(161). --С.24-26.

3. Банникова, А. В. Новые технологические решения по созданию йогуртов с
пищевыми волокнами // Техника и технология пищевых производств. --
5014. -- № 3. -- С.5--9.

4. Надирова С.А., Мамунова А.М. Использование растительных добавок для
получения молочных продуктов // Вестник Алматинского технологического
университета. -2018. -№ 2. --С.54-58.

5. Федотова ОБ., Макаркин Д.В., Соколова О.В., Дунченко Н.И. Разработка и
исследования пищевой и биологической ценности и потребительских
свойств кисломолочного продукта с мукой, не содержащего глютен //
Вопр. питания. -2019. -Т.88, -№ 2. -С.101-110. DOI
70.24411/0042-8833-2019-10023.

6. Громов Егор Сергеевич. Разработка новых белковых продуктов на основе
исследования особенностей сычужной коагуляции молока: Дис. ... канд.
техн. наук: 05.18.04. -Кемерово, 2004. -148 c.

7. Казанцева И.Л. Научно-практическое обоснование и совершенствование
технологии комплексной переработки зерна нута с получением
ингредиентов для создания продуктов здорового питания: дис. ... канд.
техн. наук: 05.18.01. -- Саратов, 2017. -391 с.

8. Технический регламент Таможенного союза "О безопасности молока и
молочной продукции" (с изменениями на 19 декабря 2019 года).
{[}Электронный ресурс{]}. URL:  --
Дата обращения: 15.01.2025.

9. Каленик Т.К., Медведева Е.В., Моткина Е.В., Медведев Г.В.
Кисломолочный биопродукт с повышенным содержанием белка // Молочная
промышленность. -- 2020. -- № 12. --С.12-13.

10. Зобкова З.С. Разработка технологии творожного продукта, обогащенного
функциональными ингредиентами / З.С. Зобкова, Т.П. Фурсова, Д.В.
Зенина, А.Д. Гаврилина, И.Р. Шелагинова // Молочная промышленность.
--2019. --№ 5. --С.44--46.
\end{references}

\begin{center}
{\bfseries References}
\end{center}

\begin{references}
1. Pastushkova E.V., Mysakov D.S., Chugunova O.V. Nekotorye aspekty
faktora pitanija i zdorov' ja cheloveka //
Mediko-farmacevticheskij zhurnal «Pul' s». -2016. -№ 4.
--S.67-71.{[}in Russian{]}

2. Pogosjan D.G. Funkcional' nye pishhevye ingredienty v
molochnyh produktah / D.G. Pogosjan, I.V. Gavrjushina // Pererabotka
moloka: tehnologija, oborudovanie, produkcija: otraslevoj
specializirovannyj zhurnal. --2013. -- № 3(161). --S.24-26. .{[}in
Russian{]}

3. Bannikova, A. V. Novye tehnologicheskie reshenija po sozdaniju
jogurtov s pishhevymi voloknami // Tehnika i tehnologija pishhevyh
proizvodstv. -- 2014. -- № 3. -- S.5--9. .{[}in Russian{]}

4. Nadirova S.A., Mamunova A.M. Ispol' zovanie
rastitel' nyh dobavok dlja poluchenija \\molochnyh
produktov // Vestnik Almatinskogo tehnologicheskogo universiteta. -2018.
-№ 2. --S.54-58. .{[}in Russian{]}

5. Fedotova OB., Makarkin D.V., Sokolova O.V., Dunchenko N.I. Razrabotka
i issledovanija pishhevoj i biologicheskoj cennosti i
potrebitel' skih svojstv kislomolochnogo produkta s
mukoj, ne soderzhashhego gljuten // Vopr. pitanija. -2019. -T.88, -№ 2.
-S.101-110. DOI 10.24411/0042-8833-2019-10023. {[}in Russian{]}

6. Gromov Egor Sergeevich. Razrabotka novyh belkovyh produktov na osnove
issledovanija osobennostej sychuzhnoj koaguljacii moloka: Dis. ... kand.
tehn. nauk: 05.18.04. -Kemerovo, 2004. -148 c. {[}in Russian{]}

7. Kazanceva I.L. Nauchno-prakticheskoe obosnovanie i sovershenstvovanie
tehnologii kompleksnoj \\pererabotki zerna nuta s polucheniem ingredientov
dlja sozdanija produktov zdorovogo pitanija: dis. ... kand. tehn. nauk:
05.18.01. -- Saratov, 2017. -391 s. .{[}in Russian{]}

8. Tehnicheskij reglament Tamozhennogo sojuza "O bezopasnosti moloka i
molochnoj produkcii" (s \\izmenenijami na 19 dekabrja 2019 goda).
{[}Jelektronnyj resurs{]}. URL: http://docs.cntd.ru/. -- Data\\
obrashhenija: 15.01.2025. {[}in Russian{]}

9. Kalenik T.K., Medvedeva E.V., Motkina E.V., Medvedev G.V.
Kislomolochnyj bioprodukt s \\povyshennym soderzhaniem belka // Molochnaja
promyshlennost'. -- 2020. -- № 12. --S.12-13. {[}in
Russian{]}

10. Zobkova Z.S. Razrabotka tehnologii tvorozhnogo produkta,
obogashhennogo funkcional' nymi \\ingredientami / Z.S.
Zobkova, T.P. Fursova, D.V. Zenina, A.D. Gavrilina, I.R. Shelaginova //
Molochnaja promyshlennost'. --2019. --№ 5. --S.44--46.
{[}in Russian{]}
\end{references}

\begin{authorinfo}
\emph{{\bfseries Сведения об авторах}}

Бектурганова А.А. -- кандидат технических наук, асс.профессор, Казахский
университет технологии и бизнеса имени К. Кулажанова, Астана, Казахстан,
е-mail: 1968al1@mail.ru;

Хастаева А.Ж. -- PhD, Казахский университет технологии и бизнеса им
К.Кулажанова, Астана Казахстан, еmail: \\gera\_or@mail.ru;

Омаралиева А.М.- кандидат технических наук, асс.профессор, Казахский
университет технологии и бизнеса им К. Кулажанова, Казахстан, Астана,
е-mail: aigul-omar@mail.ru;

Машанова Н.С.- - доктор технических наук, Казахский агротехнический
исследовательский университет имени \\С. Сейфуллина, Астана, Казахстан,
е-mail: nurmashanova@gmail.com;

Гаптар С.Л. - кандидат технических наук, Новосибирский аграрный
государственный университет, Новосибирск, Российская Федерация, е-mail:
466485@mail.ru;
\end{authorinfo}
