\id{ҒТАМР 65.65.03}{}

\begin{articleheader}
\sectionwithauthors{}{ЭМУЛЬСИЯЛЫҚ ӨНІМДЕРДІ АЛУДА КҮНБАҒЫС ЖӘНЕ ЗЫҒЫР МАЙЛАРЫН ЗЕРТТЕУ}

{\bfseries
М.Е. Смагулова,
Ш.Ж. Жасқайрат\textsuperscript{\envelope },
С.К. Таджибаева,
С.Б. Мағауия
}
\end{articleheader}

\begin{affiliation}
Астана медицина университеті КеАҚ, Астана, Қазақстан,

\raggedright \textsuperscript{\envelope }Автор-корреспондент:\href{mailto:shynarai_92@mail.ru}{\nolinkurl{shynarai\_92@mail.ru}}
\end{affiliation}

Бұл мақалада күнбағыс және зығыр майларының органолептикалық және
физико-химиялық көрсеткіштері, сонымен қатар құрамындағы Е дәруменінің
мөлшері, күнбағыс және зығыр майларының май қышқылдық құрамының талдау
нәтижелері көрсетілген. Талдау нәтижелері бойынша күнбағыс майында
қышқылдық саны 0,3 мг КОН/г болса, зығыр майында 0,7 мг КОН/г мөлшерге
ие. Ылғал мен ұшпа заттардың массалық үлесі мен белсенді көміртегінің
тотығу сандары бірдей. Өсімдік майларындағы Е дәруменінің мөлшері
күнбағыс майында 20,05 мг құраса, ал зығыр майында 38,08 мг мөлшерге ие.
Сонымен қатар күнбағыс майында зығыр майымен салыстырғанда қаныққан май
қышқылдары да (күнбағыс майында С16:0 -- 5,9, С18:0 -- 3,2, С22:0 --
0,6; зығыр майында С16:0 -- 4,4, С18:0 -- 3,2, С22:0 -- жоқ)
моноқанықпаған май қышқылдары да (күнбағыс майында С16:1 -- 0,1, С18:1
-- 23,8, С20:1 - жоқ, зығыр майында С16:1 -- жоқ, С18:1 -- 14,4, С20:1
-- жоқ) қанықпаған май қышқылдары да (күнбағыс майында С18:2- 66,4,
С18:3 - жоқ зығыр майында С18:2- 15,1, С18:3 -- 62,9) едәуір жоғары
мөлшерде. Алынған нәтижелер өсімдік шикізаты негізінде эмульсиялық
өнімдерді алуда пайдалы болады.

{\bfseries Түйін сөздер}: күнбағыс майы, зығыр майы, физико-химиялық
көрсеткіштер, органолептикалық көрсеткіштер, Е дәрумені мөлшері, май
қышқылдық құрамы

\begin{articleheader}
{\bfseries ИССЛЕДОВАНИЕ ПОДСОЛНЕЧНОГО И ЛЬНЯНОГО МАСЕЛ ПРИ ПОЛУЧЕНИИ
ЭМУЛЬСИОННЫХ ПРОДУКТОВ}

{\bfseries
М.Е. Смагулова,
Ш.Ж. Жасқайрат\textsuperscript{\envelope },
С.К. Таджибаева,
С.Б. Мағауия
}
\end{articleheader}

\begin{affiliation}
НАО ``Медицинский университет Астана'', Астана, Казахстан,

e-mail:
\href{mailto:shynarai_92@mail.ru}{\nolinkurl{shynarai\_92@mail.ru}}
\end{affiliation}

В данной статье представлены органолептические и физико-химические
показатели подсолнечного и льняного масел, а также содержание в них
витамина Е, результаты анализа жирно-кислотного состава подсолнечного и
льняного масел. По результатам анализа, подсолнечное масло имеет
кислотное количество 0,3 мг КОН/г, льняное масло имеет содержание 0,7 мг
КОН/г. Массовая доля влаги и летучих веществ и число окисления активного
углерода одинаковы. Содержание витамина Е в растительных маслах
составляет 20,05 мг в подсолнечном масле и 38,08 мг в льняном масле.
Кроме того, подсолнечное масло содержит как насыщенные жирные кислоты по
сравнению с льняным маслом (подсолнечное масло содержит С16:0 -- 5,9,
С18:0 -- 3,2, С22:0 -- 0,6; льняное масло содержит с16:0 -- 4,4, С18:0
-- 3,2, С22:0 -- нет), так и мононенасыщенные жирные кислоты
(подсолнечное масло содержит с16:1 -- 0,1, С18:1 - 23,8, С20:1 -- нет, в
льняном масле 16:1 -- нет, С18:1 -- 14,4, С20:1 - нет) также
ненасыщенные жирные кислоты (в подсолнечном масле С18:2-66,4, С18:3-нет;
в льняном масле С18: 2-- 15,1, С18:3-62,9) в значительно более высоких
количествах. Полученные результаты будут полезны при получении
эмульсионных продуктов на основе растительного сырья.

{\bfseries Ключевые слова}: подсолнечное масло, льняное масло,
физико-химические показатели, органолептические показатели, содержание
витамина Е, жирно-кислотный состав

\begin{articleheader}
{\bfseries INVESTIGATION OF SUNFLOWER AND LINSEED OILS IN THE PRODUCTION OF
EMULSION PRODUCTS}

{\bfseries
M.E. Smagulova,
Sh.Zh. Zhaskairat\textsuperscript{\envelope },
S.K. Tadjibayeva,
S.B. Magauiya
}
\end{articleheader}

\begin{affiliation}
NAO ``Astana Medical University", Republic of Kazakhstan, Astana

e-mail: \href{mailto:shynarai_92@mail.ru}{\nolinkurl{shynarai\_92@mail.ru}}
\end{affiliation}

This article presents the organoleptic and physico-chemical parameters
of sunflower and linseed oils, as well as the content of vitamin E in
them, the results of the analysis of the fatty acid composition of
sunflower and linseed oils. According to the analysis results, sunflower
oil has an acid content of 0.3 mg KOH/g, linseed oil has a content of
0.7 mg KOH/g. The mass fraction of moisture and volatile substances and
the oxidation number of active carbon are the same. The vitamin E
content in vegetable oils is 20.05 mg in sunflower oil and 38.08 mg in
linseed oil. In addition, sunflower oil contains both saturated fatty
acids compared to linseed oil (sunflower oil contains C16:0 -- 5.9,
C18:0 -- 3.2, C22:0 -- 0.6; linseed oil contains c16:0 -- 4.4, C18:0 --
3.2, C22:0 -- none), and and monounsaturated fatty acids (sunflower oil
contains c16:1 -- 0.1, C18:1 - 23.8, C20:1 -- none, in linseed oil 16:1
-- none, C18:1 -- 14.4, C20:1 - none) also unsaturated fatty acids (in
sunflower oil C18:2-66.4, C18:3-none; in linseed oil C18: 2-- 15.1,
C18:3-62.9) in significantly higher quantities. The results obtained
will be useful in the production of emulsion products based on vegetable
raw materials.

{\bfseries Keywords:} sunflower oil, linseed oil, physico-chemical
parameters, organoleptic parameters, vitamin E content, fatty acid
composition

\begin{multicols}{2}
{\bfseries Кіріспе.} Нақты тамақтануды талдау және бағалау Қазақстанның
әртүрлі өңірлеріндегі халықтың тамақтану рационы жануарлардан алынатын
майларды және оңай сіңірілетін көмірсуларды шамадан тыс тұтынумен
сипатталады, сонымен қатар халықтың көпшілігі үшін тамақтану рационы
полиқанықпаған май қышқылдарына (омега-3 және омега-6) еритін және
еримейтін диеталық талшықтар (пектин, целлюлоза және т. б.), дәрумендер
(В, Е топтары), табиғи витаминге тәрізді заттардың кең спектрі
(L-карнитин, убихинин, холин, метилметионинульфоний, липой қышқылы және
т.б.), макроэлементтер (кальций және т.б.), микроэлементтерге (йод,
темір, селен, мырыш және т.б.) тапшы екенін көрсетеді {[}1{]}.

Қазіргі уақытта Дүниежүзілік денсаулық сақтау ұйымының ақпараттары
бойынша майдың тәуліктік қолданылуы адамның 1 кг салмағына шаққанда
1,4-2,2 г құрайды. Яғни 63-158 кг, бұл көрсеткіш адамның жынысына,
жасына, еңбек жасау түріне және тұрғылықты жердің ауа-райына байланысты.
Бұлардың ішіндегі жануар тектес май 70\%, ал өсімдік тектес 30\% құрайды
{[}2{]}.

Тамақпен қолданылатын майлардың түрі және саны адам денсаулығын сақтау
үшін және жүрек-тамыры ауруларының профилактикасына әсері бар. Тамақпен
бірге көп мөлшердегі холестеринді және қанықпаған май қышқылын (жануар
тектес) пайдалану атеросклероздың пайда болуына әсер етеді. Тамақтағы
артық майлардың әсерінен семіру, жүрек-тамыры аурулары және қатерлі ісік
аурулары болады {[}3{]}.

Қазіргі таңда эмульсиялық өнімдерді пайдалану қарқынды дамып келеді.
Бағытталған өсімдік майы негізінде майонез, тұздық, спредтер мен
маргарин тәрізді әртүрлі майлы және калориялы өнімдер ассортиментін
өндіру мүмкін {[}4{]}.

Майонезді тұздықтардың ғылыми негіздерін қалыптастыруда С.А. Королев,
А.Ф.Войткевич, Д.А.Граников, В.М.Богданов, И.И.Климовский, М.Р.Гибшман,
А.В.Гудковтың, 3.X.Диланян және басқа да отандық және шетелдік ғалымдар
жұмыстары маңызды рөлге ие болды. Майонез құрамының көп компонентті
болуын және эмульгаторлық компонент ретінде шикізаттың түрлі
ассортиментін пайдалану мүмкіндігін ескерсек, өңдірілетін өнім
ассортиментін кеңейту перспективалары өте әсерлі болып табылады.
Азық-түлік технологиясын дамытудағы негізгі үрдіс физиологиялық
функционалды тамақтануға арналған өнімдерді өндіру болып табылады
{[}5{]}.

Эмульсиялық өнімдер құруда жаңа бағыт -- рецептураға байыту деңгейін
микронутриенттер, биологиялық активті заттарды жеткілікті мөлшерде
түсуді қамтамасыз ететіндей дәрежеге жеткізу және шектеулі мөлшерде
эссенциальді тағамдық заттармен байыту {[}6{]}.

Азық-түлік эмульсиялары жаңа функционалды өнімдерді жасау үшін
перспективті болып табылады. Алайда, бұл салада суда және майда еритін
биологиялық белсенді заттарды да байыту әдістерін және осы қосылыстардың
алынған өнімдердің сапасына әсерін зерттеу қажет {[}7{]}.

Жаңа тамақ өнімдерін құрудың отандық және шет ел ғылыми-техникалық
ақпараттардың аналитикалық шолуы мен теориялық және тәжірибелік
зерттеулері келесідей қорытындыға келген, адам организміне маңызды
өсімдік майларының заттары: фосфолипидтер, полиқанықпаған май
қышқылдары, майда еритін дәрумендер және микроэлементтер коп болып
табылады {[}8{]}. Сондықтан функциональдық бағыттағы эмульсиялық өнімдер
рецептурасын құрастыруда сапалы өсімдік майларын қолдану қажет.

{\bfseries Материалдар мен әдістер.} Зерттеу нысаны ретінде С.Сейфуллин
атындағы Қазақ агротехникалық университетінің «Тамақ және қайта өңдеу
өндірісінің технологиясы» кафедрасындағы «Майлы дақылдарды қайта өңдеуге
арналған тәжірибелік-өндірістік цехында» өндірілген зығыр және күнбағыс
майлары алынды. Жұмыс барысында қолданылатын барлық компоненттер
қолданыстағы нормативтік-техникалық құжаттаманың талаптарына сәйкес.

Өсімдік майларын зерттеу әдістері. Ылғал мен ұшпа заттардың массалық
үлесін МЕМСТ 11812-2022 сәйкес анықталды. Бұл әдістің мәні - талданатын
үлгіні 103±2°С температурада ылғал мен ұшқыш заттар толық жойылғанға
дейін қыздыру және оның массалық шығынын анықтау.

Майдың түстік көрсеткішін Лобивонд шкаласы арқылы анықтау МЕМСТ
5477-2015. Бұл әдіс белгілі оптикалық жол ұзындығы бар сұйық май қабаты
арқылы жарық өткенде алынған түс сипаттамасын бір көзден келетін жарық
стандартты түсті көзілдірік арқылы өткенде алынған түс сипаттамасымен
салыстыруға негізделген. Нәтижелер шартты Ловибонд бірліктерімен
көрсетіледі.

Майдың қышқылдығын анықтау МЕМСТ 31933-2012. Майдың қышқылдығын анықтау
үшін титриметриялық әдіс қолданылды. Бұл әдіс үлгіні аралас еріткіште
ерітуден, калий гидроксиді ерітіндісіндегі бос май қышқылдарын
титрлеуден тұрады.

Өсімдік майының тотығу санын анықтау МЕМСТ 24104-2001 сәйкес орындалды.
Бұл әдіс өсімдік майлары тотығу өнімдерінің сірке қышқылы мен хлороформ
ерітіндісіндегі калий йодидімен әрекеттесу реакциясына және кейіннен
натрий тиосульфаты ерітіндісімен бөлінетін йодты титриметриялық әдіспен
сандық анықтауға негізделген.

Май құрамындағы сабынды анықтау МЕМСТ ГОСТ 5480-2023 сай жүргізілді.
Сапалы әдіс - табиғи май қышқылдарынан сілтілік тазартудан кейін
тазартылған майларда сабынның (бос май қышқылдарының натрий тұздары)
болмауын анықтайды.

Өсімдік майларындағы Е дәруменінің мөлшері

МЕМСТ EN 12822-2020 сәйкес анықталды. Бұл әдіс токоферолдарды өнімділігі
жоғары сұйық хроматографиямен (HPLC) сынаманың ерітіндісінде
фотометриялық (ультракүлгін аймақта) немесе флуорометриялық анықтаумен
негізделген. Сынама ерітіндісін дайындау үшін көп жағдайда сынама
материалын сабындандыру, кейіннен талдауларды экстракциялау қажет.

Майлардың теңдестірілген қоспасының компоненттік құрамын есептеу әдісі.
Эксперименттік талдаудың сандық мәліметтерін өңдеу техникасы өсімдік
майлары теңдеулер жүйесін шешуге негізделген. Бастапқы деректер линол
және линолен қышқылдарының қатынасы болып табылады аралас жүйе, ал шығыс
-- пайыз теңдестірілген қоспадағы өсімдік майлары.

Өсімдік майларының екі компонентті қоспаларының құрамын есептеу бір
қадаммен жүзеге асырылады. Кезеңнің мақсаты -- екі негізгі компоненттің
қатынасын анықтау.

Екі компонентті май қоспадасындағы өсімдік майларының массалық үлесін
есептеу келесі формула бойынша анықтайды:

\(\frac{m_{a} \times c_{a}^{1} + m_{b} \times c_{b}^{1}}{m_{a} \times c_{a}^{2} + m_{b} \times c_{b}^{2}}\)
(1)

m\textsubscript{a}+m\textsubscript{b}=1

Мұндағы \(m_{a}\) - бірінші өсімдік майының массасы, кг;

\({\ \ \ \ \ \ \ \ \ \ \ m}_{b}\) - екінші өсімдік майының массасы, кг;

\begin{quote}
\(c_{a}^{1}\) - өсімдік майындағы линол қышқылының концентрациясы, \%;
\end{quote}

\({\ \ \ \ \ \ \ \ \ \ c}_{a}^{2}\) - өсімдік майындағы линол қышқылының
концентрациясы, \%;

\(c_{b}^{1}\) - өсімдік майындағы линолен қышқылының концентрациясы, \%;

\(\ \ \ \ \ \ \ \ \ \ c_{b}^{2}\) - өсімдік майындағы линолен қышқылының
концентрациясы, \%;

Теңдеулер жүйесі \(m_{a}\) және \(m_{b}\) қатысты шешіледі. Нәтижесінде
май қоспалары берілген оптималды ω-3 және ω-6 май қышқылдарын құрайды.

{\bfseries Нәтижелер мен талқылау.} 1-кестеде кестеде күнбағыс және зығыр
майларының органолептикалық және физико-химиялық көрсеткіштері
көрсетілген.
\end{multicols}

{\bfseries 1 - кесте. Өсімдік майларының органолептикалық және физико-химиялық көрсеткіштері}

%% \begin{longtable}[]{@{}
%%   >{\raggedright\arraybackslash}p{(\linewidth - 4\tabcolsep) * \real{0.3267}}
%%   >{\centering\arraybackslash}p{(\linewidth - 4\tabcolsep) * \real{0.3368}}
%%   >{\centering\arraybackslash}p{(\linewidth - 4\tabcolsep) * \real{0.3365}}@{}}
%% \toprule\noalign{}
%% \begin{minipage}[b]{\linewidth}\raggedright
%% Көрсеткіш атауы
%% \end{minipage} & \begin{minipage}[b]{\linewidth}\centering
%% Күнбағыс майы
%% \end{minipage} & \begin{minipage}[b]{\linewidth}\centering
%% \begin{quote}
%% Зығыр майы
%% \end{quote}
%% \end{minipage} \\
%% \midrule\noalign{}
%% \endhead
%% \bottomrule\noalign{}
%% \endlastfoot
%% Иісі мен дәмі & Бөгде иіс және ащы дәм жоқ, күнбағыс майына тән &
%% \begin{minipage}[t]{\linewidth}\centering
%% \begin{quote}
%% Бөгде иіс және ащы дәм жоқ, зығыр майына тән
%% \end{quote}
%% \end{minipage} \\
%% Түсі & Ашық-сары & \begin{minipage}[t]{\linewidth}\centering
%% \begin{quote}
%% Алтын түстес
%% \end{quote}
%% \end{minipage} \\
%% Түстік саны, мг йод & 6 & \begin{minipage}[t]{\linewidth}\centering
%% \begin{quote}
%% 20
%% \end{quote}
%% \end{minipage} \\
%% Қышқылдық саны, мг КОН/г, көп емес & 0,3 &
%% \begin{minipage}[t]{\linewidth}\centering
%% \begin{quote}
%% 0,7
%% \end{quote}
%% \end{minipage} \\
%% Тотығу саны, моль белсенді көміртегінің/кг & 4,0 &
%% \begin{minipage}[t]{\linewidth}\centering
%% \begin{quote}
%% 4,0
%% \end{quote}
%% \end{minipage} \\
%% Ылғал мен ұшпа заттардың массалық үлесі, \% & 0,10 &
%% \begin{minipage}[t]{\linewidth}\centering
%% \begin{quote}
%% 0,10
%% \end{quote}
%% \end{minipage} \\
%% Сабын(сапалы сынама) & Жоқ & \begin{minipage}[t]{\linewidth}\centering
%% \begin{quote}
%% Жоқ
%% \end{quote}
%% \end{minipage} \\
%% \end{longtable}

\begin{multicols}{2}
Талдау нәтижесінде барлық майлар сапалы, өйткені олар белгіленген
талаптарға сәйкес келеді және әрі қарай жұмыста қолданыста болуы мүмкін.
Бұл мәндер МЕМСТ - та белгіленген стандарттарға сәйкес келеді:

- күнбағыс майы үшін ҚР СТ МЕМСТ Р 52465-2010;

- зығыр майы үшін ҚР СТ 2645-2015.

Мухаметов А. Е., Аскарбеков Э.Б. және т.б. ғалымдар өздерінің «Өсімдік
майларының қоспасынан дайындалатын майлы өнімдердің сапалық
көрсеткіштерін зерттеу» тақырыбындағы жариялымдарында күнбағыс және
зығыр майларының органолептикалық және физико-химиялық көрсеткіштерін
көрсеткен {[}9{]}.2-кестеде кестеде күнбағыс және зығыр майларының Е
дәруменінің мөлшері көрсетілген.
\end{multicols}

{\bfseries 2 - кесте. Өсімдік майларындағы Е дәруменінің мөлшері}

%% \begin{longtable}[]{@{}
%%   >{\raggedright\arraybackslash}p{(\linewidth - 2\tabcolsep) * \real{0.4983}}
%%   >{\centering\arraybackslash}p{(\linewidth - 2\tabcolsep) * \real{0.5017}}@{}}
%% \toprule\noalign{}
%% \begin{minipage}[b]{\linewidth}\raggedright
%% Өсімдік майының түрі
%% \end{minipage} & \begin{minipage}[b]{\linewidth}\centering
%% Мөлшері (100 мл.), мг
%% \end{minipage} \\
%% \midrule\noalign{}
%% \endhead
%% \bottomrule\noalign{}
%% \endlastfoot
%% Күнбағыс майы & 20,05 \\
%% Зығыр майы & 38,08 \\
%% \end{longtable}

\begin{multicols}{2}
Майлардың құрамындағы токоферолдардың, яғни Е витаминінің болуын атап
өткен жөн - бұл табиғи антиоксидант болып саналады. Алимарданова М.,
Матибаева А., Джетписбаева Б. өздерінің оқу құралдарында өсімдік
майларындағы Е дәруменінің мөлшерін толыққанды көрсеткен {[}10{]}.
3-кестеде кестеде күнбағыс және зығыр майларының май қышқылдық құрамының
көрсеткіштерінің талдау нәтижелері көрсетілген.
\end{multicols}

{\bfseries 3-кесте. Өсімдік майларының май қышқылдық құрамы, \%}

%% \begin{longtable}[]{@{}
%%   >{\raggedright\arraybackslash}p{(\linewidth - 6\tabcolsep) * \real{0.2417}}
%%   >{\raggedright\arraybackslash}p{(\linewidth - 6\tabcolsep) * \real{0.1737}}
%%   >{\centering\arraybackslash}p{(\linewidth - 6\tabcolsep) * \real{0.3133}}
%%   >{\centering\arraybackslash}p{(\linewidth - 6\tabcolsep) * \real{0.2712}}@{}}
%% \toprule\noalign{}
%% \begin{minipage}[b]{\linewidth}\raggedright
%% Май қышқылының атауы
%% \end{minipage} & \begin{minipage}[b]{\linewidth}\raggedright
%% Белгіленуі
%% \end{minipage} & \begin{minipage}[b]{\linewidth}\raggedright
%% Күнбағыс майының май қышқылды құрамы (май қышқылы жиынтығына)
%% \end{minipage} & \begin{minipage}[b]{\linewidth}\raggedright
%% Зығыр майының май қышқылды құрамы (май қышқылы жиынтығына)
%% \end{minipage} \\
%% \midrule\noalign{}
%% \endhead
%% \bottomrule\noalign{}
%% \endlastfoot
%% \multirow{3}{=}{Қаныққан май қышқылдары} & С16:0 & 5,9 & 4,4 \\
%% & С18:0 & 3,2 & 3,2 \\
%% & С22:0 & 0,6 & - \\
%% \multirow{3}{=}{Моноқанықпаған май қышқылдары} & С16:1 & 0,1 & - \\
%% & С18:1 & 23,8 & 14,4 \\
%% & С20:1 & - & - \\
%% \multirow{3}{=}{Полиқанықпаған май қышқылдары} & С18:2 & 66,4 & 15,1 \\
%% & С18:3 & - & 62,9 \\
%% & С20:2 & - & - \\
%% \end{longtable}

\begin{multicols}{2}
ω-3: ω-6 қышқылдарының қатынасы бойынша теңдестірілген өсімдік
майларының қоспасын жасау үшін, өсімдік майларының май қышқылды құрамы
эксперименталды түрде зерттелді. Сызықтық программалау әдісін қолдану
дұрыс тамақтану үшін ұсынылған 1:4 қатынасында ω-3 және ω-6 май
қышқылдарының құрамы бар өсімдік майларының қоспасын жасауға мүмкіндік
берді. Өсімдік майларының қоспасы келесідей қатынаста алынды: күнбағыс:
зығыр = 75\%:25\%.

Айдарханова Г.С., Сатаева Ж.И. және т.б. ғалымдар өсімдік майлары туралы
биологиялық белсенді органикалық компоненттер мен минералдардың құрамына
байланысты тамақ және фармацевтика салалары үшін құнды мультивитаминдік
өнімдер ретінде өсімдік майлары туралы ақпаратқа қысқаша талдау жасаған
{[}11{]}. Далабаев А. Б. және т.б. ғалымдар да өсімдік майларының май
қышқылдық құрамына тереңірек тоқталып өткен {[}12{]}.

{\bfseries Қорытынды.} Жаңа эмульсиялық өнім үшін майлы фазаны жасау
кезінде оның май қышқылының құрамын ω-6 және ω-3 қышқылдары бойынша
теңдестіруге қол жеткізу мақсаты қойылған. Ол үшін әр түрлі май қышқылы
топтарына жататын өсімдік майлары қосылуы керек. Жоғарыда аталған
талаптарға сәйкес келетін май негізін құрудың ең жақсы тәсілі-май
қышқылының әртүрлі құрамын таңдау. Алдын-ала әдеби талдау негізінде
келесідей майлар таңдалды: күнбағыс және зығыр майы.

Талдау нәтижесінде екі майда сапалы, өйткені олар белгіленген талаптарға
сәйкес келді. Қышқылдық саны күнбағыс майында 0,3 мг КОН/г болса, зығыр
майында 0,7 мг КОН/. Тотығу саны мен ылғал және ұшпа заттардың массалық
үлесі екі майда да бірдей көрсеткіш көрсетті: 4,0 моль, 0,10 \%

Өсімдік майларындағы Е дәруменінің мөлшері күнбағыс майында 20,05 мг, ал
зығыр майында 38,08 мг көрсетті.

Күнбағыс майының май қышқылды құрамы зығыр майымен салыстырғанда жоғары
көрсеткішке ие болды.

Жаңа өнімді әзірлеу үшін шикізат жоғары сапалы болуы керек болғандықтан,
барлық майлар органолептикалық және физика-химиялық көрсеткіштері
көрсеткіштері бойынша тексерілді. Нәтижелері кестелерге салынды.
\end{multicols}

\begin{center}
{\bfseries Әдебиеттер}
\end{center}

\begin{references}
1. Мухаметов А.Е. Ассортимент, качество и востребованность майонезной
продукции в Казахстане// Проблемы агрорынка. -2022. -№ 1. --С.
144-152. \href{https://doi.org/10.46666/2022-1.2708-9991.17}{DOI
10.46666/2022-1.2708-9991.17}

2. Шендеров, Б.А. Современное состояние и перспективы развития концепции
«Функциональное питание» // Пищевая промышленность. -- 2013. - №5.
-С.4-5.

3. Ипатова Л. Г. и др. Жировые продукты для здорового питания.
Современный взгляд //М.: ДеЛи принт. - 2009. - С.368-391. ISBN
978-5-94343-206-4

4. Анализ рынка майонеза в Казахстане / Маркетинговые исследования. --
Алматы: Tebiz Group, 2021. -- 82 с. -- URL:
\url{https://tebiz.ru/mi/analiz-rynka-majoneza-v-kazakhstane}. Дата
обращения: 25.02.2025.

5. Журавко Е.В., Грузинов Е.В. «Майонез Диабетический с экстрактом
стевии» // Масложировая промышленность -- 2004- №.2.- С.41-42.

6. Волкова, Н. Н. Разработка способа получения низкокалорийных
эмульсионных соусов на основе натуральных ингредиентов: дис. \ldots{}
канд. техн. наук: 05.18.06 / Волкова Наталия Николаевна. -- Москва,
2008. -- 131 с.
7. Доронин, А.Ф. Функциональные пищевые продукты. Введение в технологии /
А.Ф. Доронин, Л.В. Ипатова, А.А. Кочеткова, А.П. Нечаев, С.А. Хуршудян,
О.Г. Шубина. -- М.: ДеЛи принт.2009. -- 286 с.- С.89-93

8. Гаврилова, Д. В. Разработка и товароведная оценка майонеза и
майонезного соуса для здорового питания с пектином: дис. \ldots{}
канд. техн. наук / Гаврилова Дарья Викторовна. -- Москва, 2014. -- 147
с.

9. Мухаметов, А.Е., Аскарбеков, Э.Б., Ербулекова, М.Т., Сейсеналы, М.Е.
Өсімдік майларының қоспасынан дайындалатын майлы өнімдердің сапалық
көрсеткіштерін зерттеу // Алматы технологиялық университетінің
хабаршысы. -- 2022. -- № 4. -- С.61--68. DOI
10.48184/2304-568X-2022-4-61-68.

10. Алимарданова М., Матибаева А., Джетписбаева Б. Тағамдық майлардың, сүт
және сүт өнімдерінің тауартануы және сараптау: оқу құралы -- 2019.
-256 б. ISBN 978-601-302-947-4

11. Aidarkhanova, G. S., Satayeva, Zh. I., Jakanova, M. T., Seilkhanov, T.
M. Assessment of quality and food safety of vegetable oils produced in
various regions of Kazakhstan // Reports of the National Academy of
Sciences of the Republic of Kazakhstan. -- 2021. -- Vol.3(337). -- P.
5--11. DOI 10.32014/2021.2518-1483.41.

12. Далабаев А.Б., Жүнісова Қ.З., Альжаксина Н.Е. Түрлі өсімдік
майларындағы глицидил эфирлерінің мөлшерін анықтау // С.Сейфуллин
атындағы Қазақ агротехникалық университетінің Ғылым жаршысы. -2022.
-№3 (114). -Ч.1. -С.36--45. DOI 10.51452/kazatu.2022.3(114).1107
\end{references}

\begin{center}
{\bfseries References}
\end{center}

\begin{references}
1. Muhametov A.E. Assortiment, kachestvo i
vostrebovannost'{} majoneznoj produkcii v Kazahstane//\\
Problemy agrorynka. -2022. -№ 1. --S.144-152. DOI
10.46666/2022-1.2708-9991.17. {[}in Russian{]}

2. Shenderov, B.A. Sovremennoe sostojanie i perspektivy razvitija
koncepcii «Funkcional' noe pitanie» // Pishhevaja
promyshlennost'. -- 2013. - №5. -S.4-5. {[}in Russian{]}

3. Ipatova L. G. i dr. Zhirovye produkty dlja zdorovogo pitanija.
Sovremennyj vzgljad //M.: DeLi print. - 2009. - S.368-391. ISBN
978-5-94343-206-4. {[}in Russian{]}

4. Analiz rynka majoneza v Kazahstane / Marketingovye issledovanija. --
Almaty: Tebiz Group, 2021. -- 82 s. -- URL:
\url{https://tebiz.ru/mi/analiz-rynka-majoneza-v-kazakhstane}. Data
obrashhenija: 25.02.2025. {[}in Russian{]}

5. Zhuravko E.V., Gruzinov E.V. «Majonez Diabeticheskij s jekstraktom
stevii» // Maslozhirovaja \\promyshlennost'{} -- 2004- №.
2. - S.41-42. {[}in Russian{]}

6. Volkova, N. N. Razrabotka sposoba poluchenija nizkokalorijnyh
jemul' sionnyh sousov na osnove
natural' nyh ingredientov: dis. \ldots{} kand. tehn.
nauk: 05.18.06 / Volkova Natalija Nikolaevna. -- Moskva, 2008. -- 131 s.
{[}in Russian{]}

7. Doronin, A.F. Funkcional' nye pishhevye produkty.
Vvedenie v tehnologii / A.F. Doronin, L.V. Ipatova, A.A. Kochetkova,
A.P. Nechaev, S.A. Hurshudjan, O.G. Shubina. -- M.: DeLi print.2009. --
286 s.- S.89-93. {[}in Russian{]}

8. Gavrilova, D. V. Razrabotka i tovarovednaja ocenka majoneza i
majoneznogo sousa dlja zdorovogo pitanija s pektinom: dis. \ldots{}
kand. tehn. nauk / Gavrilova Dar' ja Viktorovna. --
Moskva, 2014. -- 147 s. {[}in Russian{]}

9. Muhametov, A.E., Askarbekov, Je.B., Erbulekova, M.T., Sejsenaly, M.E.
Osіmdіk majlarynyn \\kospasynan dajyndalatyn majly өnіmderdің sapalyқ
kөrsetkіshterіn zertteu // Almaty tehnologijalyk \\universitetіnin
habarshysy. -- 2022. -- № 4. -- S.61--68. DOI
10.48184/2304-568X-2022-4-61-68.{[}in Kazakh{]}

10. Alimardanova M., Matibaeva A., Dzhetpisbaeva B. Taғamdyқ majlardyң,
sүt zhәne sүt өnіmderіnің tauartanuy zhәne saraptau: oқu құraly -- 2019.
-256 b. ISBN 978-601-302-947-4.{[}in Kazakh{]}

11. Aidarkhanova, G. S., Satayeva, Zh. I., Jakanova, M. T., Seilkhanov,
T. M. Assessment of quality and food safety of vegetable oils produced
in various regions of Kazakhstan // Reports of the National Academy of
Sciences of the Republic of Kazakhstan.-2021.-Vol.3(337). -P.5--11. DOI
10.32014/2021.2518-1483.41.

12. Dalabaev A.B., Zhүnіsova Қ.Z., Al' zhaksina N.E. Tүrlі
өsіmdіk majlaryndaғy glicidil jefirlerіnің mөlsherіn anyқtau //
S.Sejfullin atyndaғy Қazaқ agrotehnikalyқ universitetіnің Ғylym
zharshysy. -2022. -№3 (114). -Ch.1. -S.36--45. DOI
10.51452/kazatu.2022.3(114).1107. {[}in Kazakh{]}
\end{references}

\begin{authorinfo}
\emph{{\bfseries Авторлар туралы мәліметтер}}

Смагулова М.Е. - химия ғылымдарының кандидаты, Астана медицина
университеті, Академик Е.Д. Даленов атындағы профилактикалық медицина
ғылыми-зерттеу институты «Денсаулық Диагностикасы» зертханасының
меңгерушісі, Астана, Қазақстан қаласы, e-mail:
\href{mailto:mirgul.smagulova@bk.ru}{\nolinkurl{mirgul.smagulova@bk.ru}};
ORCID: \url{https://orcid.org/0000-0001-5793-0813}

Жасқайрат Ш.Ж. - т.ғ.м., жетекші ғылыми қызметкер, Астана медицина
университеті, Академик Е.Д. Даленов атындағы профилактикалық медицина
ғылыми-зерттеу институты, Астана, Қазақстан, e-mail:
\href{mailto:shynarai_92@mail.ru}{\nolinkurl{shynarai\_92@mail.ru}};ORCID:
\url{https://orcid.org/0009-0008-2472-4210}

Таджибаева С.К. - м.ғ.к., жетекші ғылыми қызметкер, Астана медицина
университеті, Академик Е.Д. Даленов атындағы профилактикалық медицина
ғылыми-зерттеу институты, Астана, Қазақстан, e-mail:
\href{mailto:tardzhibaeva.s@amu.kz}{\nolinkurl{tardzhibaeva.s@amu.kz}};
ORCID: \url{https://orcid.org/0000-0002-4150-7997}

Мағауия С.Б. - кіші қызметкер, Астана медицина университеті, Академик
Е.Д. Даленов атындағы профилактикалық медицина ғылыми-зерттеу институты,
Астана, Қазақстан, e-mail:
\href{mailto:htnhto@mail.ru}{\nolinkurl{htnhto@mail.ru}}

ORCID : \url{https://orcid.org/0000-0002-6431-7248}

\emph{{\bfseries Information about the authors}}

Smagulova M. E. - candidate of chemical sciences, Astana Medical
University, Research Institute of Preventive Medicine named after
academician E.D. Dalenov, head of the laboratory "Diagnostics of
Health", Astana, Kazakhstan, e-mail:\\
\href{mailto:mirgul.smagulova@bk.ru}{\nolinkurl{mirgul.smagulova@bk.ru}};

Zhaskairat Sh. Zh. - leading researcher, Astana Medical University,
Research Institute of Preventive Medicine named after academician E.D.
Dalenov, Astana, Kazakhstan, e-mail:
\href{mailto:shynarai_92@mail.ru}{\nolinkurl{shynarai\_92@mail.ru}};

Tajibayeva S. K. - Ph.D, leading researcher Astana Medical University,
Research Institute of Preventive Medicine named after academician E.D.
Dalenov, Astana, Kazakhstan, e-mail:
\href{mailto:tardzhibaeva.s@amu.kz}{\nolinkurl{tardzhibaeva.s@amu.kz}};

Magauiya S. B.- junior employee, Astana Medical University, Research
Institute of Preventive Medicine named after academician E.D. Dalenov,
Astana, Kazakhstan, e-mail:
\href{mailto:htnhto@mail.ru}{\nolinkurl{htnhto@mail.ru}}\
\end{authorinfo}
