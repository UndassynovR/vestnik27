\id{МРНТИ 65.09.03}{}

\begin{articleheader}
\sectionwithauthors{Г.Е. Жумалиева, У.Ч. Чоманов, Ә.К. Шоман, А.Ғ. Оғазова}{ИССЛЕДОВАНИЕ СЫРЬЯ ДЛЯ РАЗРАБОТКИ БИОЛОГИЧЕСКИ АКТИВНЫХ ДОБАВОК
ИММУНОМОДУЛИРУЮЩЕЙ НАПРАВЛЕННОСТИ}

{\bfseries
Г.Е. Жумалиева\textsuperscript{\envelope },
У.Ч. Чоманов,
Ә.К. Шоман,
А.Ғ. Оғазова\textsuperscript{\envelope }
}
\end{articleheader}

\begin{affiliation}
ТОО «Казахский научно-исследовательский институт перерабатывающей и
пищевой

промышленности», Алматы, Казахстан

\raggedright \textsuperscript{\envelope }Корреспондент-автор: guljan\_7171@mail.ru, o.aidana\_01@mail.ru
\end{affiliation}

Статья посвящена оценке биологически активных растительных компонентов
для разработки добавок, способствующих укреплению иммунной системы. В
условиях роста интереса к натуральным продуктам, использование
пророщенных овса, ячменя, корня солодки и топинамбура в качестве сырья
для создания БАД становится актуальной задачей в области питания и
здравоохранения. Цель исследования заключается в изучении
физико-химических, биологических свойств этих растений, а также их
аминокислотного состава, содержания витаминов, минералов и других
активных соединений, способствующих поддержанию иммунной функции
организма.

Материалы и методы исследования включают отбор пророщенного овса,
ячменя, корня солодки и топинамбура для анализа. Исследования
проводились в научно-исследовательской лаборатории АО «Алматинский
технологический университет», где был выполнен физико-химический и
микробиологический анализ образцов для оценки их безопасности и
качества. Этот этап исследования позволил выявить важные характеристики
растительных компонентов, которые могут быть использованы для разработки
новых функциональных продуктов.

Полученные результаты показали, что пророщенные овес и ячмень обладают
улучшенными питательными свойствами. Ячмень прорастал быстрее, что может
способствовать увеличению его биологической активности, а овес содержал
больше белка, что делает его ценным источником для добавок. Топинамбур и
корень солодки богаты витаминами, минералами, клетчаткой и незаменимыми
аминокислотами, такими как лизин и метионин. Микробиологические
исследования подтверждают безопасность этих продуктов, что позволяет
рекомендовать их использование в составе биологически активных добавок.

Заключение исследования подчеркивает, что пророщенные овес и ячмень,
топинамбур и корень солодки могут стать перспективными компонентами для
разработки БАД, способствующих улучшению иммунной функции и общего
состояния здоровья. Эти продукты обладают высокой питательной ценностью,
содержат важные нутриенты, что делает их полезными для включения в
рацион с целью профилактики заболеваний и укрепления организма.

{\bfseries Ключевые слова:} биологически активная добавка, корень солодки,
овес, пророщенное семя, топинамбур, ячмень.

\begin{articleheader}
{\bfseries ИММУНОМОДУЛЯТОРЛЫҚ БАҒДАРДАҒЫ БИОЛОГИЯЛЫҚ БЕЛСЕНДІ ҚОСПАЛАРДЫ ӘЗІРЛЕУГЕ АРНАЛҒАН ШИКІЗАТТЫ ЗЕРТТЕУ}

{\bfseries
Г.Е. Жумалиева\textsuperscript{\envelope },
У.Ч. Чоманов,
Ә.К. Шоман,
А.Ғ. Оғазова\textsuperscript{\envelope }
}
\end{articleheader}

\begin{affiliation}
«Қазақ қайта өңдеу және тағам өнеркәсіптері ғылыми-зерттеу институты»
ЖШС, Алматы, Казақстан,

e-mail:
\href{mailto:guljan\_7171@mail.ru}{\nolinkurl{guljan\_7171@mail.ru}},
\href{mailto:o.aidana\_01@mail.ru}{\nolinkurl{o.aidana\_01@mail.ru}}
\end{affiliation}

Мақала иммундық жүйені нығайтуға көмектесетін қоспаларды әзірлеу үшін
биологиялық белсенді өсімдік компоненттерін бағалауға арналған. Табиғи
өнімдерге деген қызығушылықтың артуы жағдайында өсірілген сұлы, арпа,
мия тамыры және топинамбурды тағамдық қоспалар жасау үшін шикізат
ретінде пайдалану тамақтану және денсаулық сақтау саласындағы өзекті
мәселеге айналуда. Зерттеудің мақсаты - бұл өсімдіктердің
физика-химиялық және биологиялық қасиеттерін, сондай-ақ олардың
аминқышқылдарының құрамын, витаминдердің, минералдардың және организмнің
иммундық қызметін сақтауға көмектесетін басқа да белсенді қосылыстардың
мазмұнын зерттеу.

Зерттеудің материалдары мен әдістеріне талдау үшін өсірілген сұлы, арпа,
мия тамыры және топинамбурды таңдау кіреді. Зерттеулер «Алматы
технологиялық университеті» АҚ ғылыми-зерттеу зертханасында жүргізілді,
онда үлгілердің қауіпсіздігі мен сапасын бағалау үшін физикалық - химиялық
және микробиологиялық талдаулар жүргізілді. Зерттеудің бұл кезеңі жаңа
функционалды өнімдерді жасау үшін қолданылатын өсімдік компоненттерінің
маңызды сипаттамаларын анықтауға мүмкіндік берді.

Алынған нәтижелер көктеп шыққан сұлы мен арпаның қоректік қасиеттерінің
жақсарғанын көрсетті. Арпа тезірек өсіп шықты, бұл оның биологиялық
белсенділігін арттыруға ықпал етуі мүмкін, ал сұлы құрамында көбірек
ақуыз болды, бұл оларды қоспалар үшін құнды көз етеді. Топинамбур мен
мия тамыры витаминдерге, минералдарға, талшықтарға және лизин мен
метионин сияқты маңызды аминқышқылдарына бай. Микробиологиялық
зерттеулер бұл өнімдердің қауіпсіздігін растайды, бұл бізге оларды
диеталық қоспаларда пайдалануды ұсынуға мүмкіндік береді.

Зерттеу қорытындысында өсірілген сұлы мен арпа, топинамбур және мия
тамыры иммундық функцияны және жалпы денсаулықты жақсартатын тағамдық
қоспаларды әзірлеу үшін перспективалы компоненттер болуы мүмкін екенін
көрсетеді. Бұл тағамдардың тағамдық құндылығы жоғары және құрамында
маңызды қоректік заттар бар, оларды аурулардың алдын алу және денені
нығайту үшін рационға қосу пайдалы.

{\bfseries Түйін сөздер}: биологиялық белсенді қоспа; мия тамыры; сұлы;
өскен тұқым; топинамбур; арпа.

\begin{articleheader}
{\bfseries RESEARCH OF RAW MATERIALS FOR THE DEVELOPMENT OF BIOLOGICALLY
ACTIVE ADDITIVES WITH IMMUNOMODULATORY FOCUS}

{\bfseries
G.E. Zhumalieva\textsuperscript{\envelope },
U.Ch. Chomanov,
A.K. Shoman,
A.G. Ogazova\textsuperscript{\envelope }
}
\end{articleheader}

\begin{affiliation}
LLC "Kazakh Research Institute of Processing and Food Industry", Almaty,
Kazakhstan

e-mail\emph{:}
\href{mailto:guljan\_7171@mail.ru}{\nolinkurl{guljan\_7171@mail.ru}},
o.aidana\_01@mail.ru
\end{affiliation}

The article is devoted to the evaluation of biologically active plant
components for the development of supplements that help strengthen the
immune system. In the context of growing interest in natural products,
the use of sprouted oats, barley, licorice root and Jerusalem artichoke
as raw materials for the creation of dietary supplements is becoming an
urgent task in the field of nutrition and health care. The purpose of
the study is to study the physicochemical, biological properties of
these plants, as well as their amino acid composition, content of
vitamins, minerals and other active compounds that help maintain the
body' s immune function.

Materials and methods of the study include the selection of sprouted
oats, barley, licorice root and Jerusalem artichoke for analysis. The
studies were conducted in the research laboratory of JSC "Almaty
Technological University", where physicochemical and microbiological
analysis of the samples was \\performed to assess their safety and
quality. This stage of the study revealed important characteristics of
plant components that can be used to develop new functional products.

The results showed that sprouted oats and barley have improved
nutritional properties. Barley sprouted faster, which may increase its
biological activity, and oats contained more protein, making them a
valuable source for supplements. Jerusalem artichoke and licorice root
are rich in vitamins, minerals, fiber, and essential amino acids such as
lysine and methionine. Microbiological studies confirm the safety of
these products, which allows us to recommend their use in dietary
supplements. The conclusion of the study emphasizes that sprouted oats
and barley, Jerusalem artichoke and licorice root may be promising\\
components for the development of dietary supplements that help improve
immune function and overall health. These products have high nutritional
value, contain important nutrients, which makes them useful for
inclusion in the diet for the purpose of disease prevention and
strengthening the body.

{\bfseries Keywords} dietary supplement, licorice root, oats, sprouted
seed, Jerusalem artichoke; sprouted seeds, barley.

\begin{multicols}{2}
{\bfseries Введение.} Задачей любой страны является корректировка питания
ее жителей согласно требованиям времени. По этой причине в различных
странах мира по инициативе Всемирной организации здравоохранения
начались разработки пищевых добавок, которое стремительно получили
широкое распространение. В состав данных БАД входят натуральные
биологические активные компоненты, необходимые для саморегулирования
организма {[}1{]}.

В идеале все необходимые для здоровья вещества следует получать из пищи:
овощей и фруктов, мяса и рыбы, молочных продуктов. Для этого питание
должно быть разнообразным и сбалансированным. Но подобным может
похвастаться менее 1\% взрослого населения. Поэтому не будет
преувеличением утверждение, что БАД нужны всем, кроме грудных младенцев
{[}2{]}.

В условиях современного мира наблюдается рост числа факторов, негативно
влияющих на здоровье человека --- ухудшение экологической обстановки,
учащающиеся природные катаклизмы, хронические стрессы и ослабление
иммунной защиты. Особенно остро эта проблема проявляется у людей с
нарушениями обмена веществ, в частности, при сахарном диабете, когда
иммунитет ослаблен, а организм подвержен частым воспалительным и
инфекционным заболеваниям. В связи с этим возрастает интерес к
разработке эффективных и безопасных средств, способствующих укреплению
защитных функций организма.

Одним из перспективных направлений является создание комплексных
биологически активных добавок (БАД) на основе натурального растительного
сырья с иммуномодулирующим действием. Использование пророщенных зерновых
культур, топинамбура и корня солодки позволяет получить композиции,
богатые белками, витаминами, микроэлементами и природными
антиоксидантами. Это обосновывает актуальность исследования,
направленного на разработку технологии и изучение свойств комплексной
БАД, способствующей поддержанию иммунного статуса и улучшению общего
состояния здоровья.

В развитых странах, таких как Япония, США и ЕС, уже активно используют
БАДы для коррекции питания и предотвращения заболеваний, вызванных
экологическими, географическими, эмоциональными и другими факторами
{[}3{]}.

Интерес к биологически активным добавкам (БАД) продолжает расти, что
связано с необходимостью профилактики заболеваний и поддержания здоровья
естественным путём. В последние годы этот интерес активно развивается,
особенно в контексте улучшения качества жизни и усиления иммунной
системы {[}4{]}.

Отсутствует современное аналитическое сопровождение производственного
процесса, что не позволяет быть уверенным в безопасности и отсутствии
фальсификации в отношении этой продукции. Формы приема отечественных БАД
пока недостаточно разработаны: не всегда учитывается такой показатель,
как комфортность приема пищи {[}5{]}.

Биологически активные добавки (БАД) играют ключевую роль в улучшении
здоровья и коррекции питания, поддерживая нормальное функционирование
организма и предотвращая заболевания, вызванные экологическими,
географическими и эмоциональными факторами. {[}6{]}.

Однако несмотря на растущий интерес, остаются нерешёнными важные
вопросы, включая методологические подходы к исследованиям,
технологические сложности и отсутствие разработанных схем производства.
В частности, недостаточно обоснованы алгоритмы составления композитных
БАД, которые постепенно вытесняют моносоставы {[}7{]}.

Особое внимание уделяется БАДам с иммуномодулирующим и гликемическим
эффектом, которые поддерживают защитные функции организма и регулируют
уровень сахара в крови. Эти добавки особенно актуальны в условиях
изменений экологии, природных катастроф и стресса, когда важно укреплять
иммунную систему и контролировать уровень глюкозы.

Топинамбур, благодаря своему высокому содержанию инулина и низкому
гликемическому индексу, идеально подходит как компонент таких добавок.
Он не только способствует нормализации уровня сахара в крови у людей с
диабетом, но и помогает укреплять иммунную систему благодаря
антиоксидантным и противовоспалительным свойствам. Ведущими странами в
области разработки БАДов являются Япония, США и страны Европейского
Союза, где активно применяются добавки для коррекции питания,
поддержания здоровья и улучшения качества жизни, в том числе для людей с
сахарным диабетом.

{\bfseries Материалы и методы.} Исследования проведены в АО «Алматинский
технологический университет» Научно-исследовательской лаборатории по
оценке качества и безопасности продовольственных продуктов и
использовались стандартные методы анализа и определения показателей
качества в соответствии с действующими ГОСТами. Все анализы выполнены в
лабораторных условиях с использованием современного оборудования и
точных методик, что гарантирует высокую степень достоверности полученных
данных.

1. Энергию прорастания и способность к прорастанию определяли по ГОСТ
10968-88, метод основан на проращивании определённого количества семян
во влажной среде при заданной температуре в течение установленного
времени.

2. Влажность зерна определяли по ГОСТ 13586.5-2015, суть метода
заключается в высушивании навески зерна при температуре 130\,±\,2\,°C
до постоянной массы. Результаты выражались в процентах.

3. Кислотность зерна определяли по ГОСТ 27493-87. Метод основан на водной
экстракции органических кислот из зерна и последующем титровании
экстракта раствором щёлочи до нейтральной реакции.

4. Микробиологические показатели определяли по ГОСТ ISO 7218-2015. Метод
предусматривает приготовление серийных разведений исследуемого
образца, высев на питательные среды и инкубацию при оптимальных
условиях для роста микроорганизмов.

5. Влажность различных образцов определяли по ГОСТ 24027.2-80. Метод
основан на высушивании навески образца в сушильном шкафу при
температуре 100--105\,°C до постоянной массы. Потеря массы при сушке
соответствовала содержанию влаги, результаты выражались в процентах.

6. Содержание жира в муке определяли по ГОСТ 23042-86. Метод основан на
экстракции жира из навески муки органическим растворителем (обычно
эфиром или гексаном) с последующим удалением растворителя и
взвешиванием оставшегося жира.

7. Содержание азота и белка определяли по ГОСТ 25011-81 и ГОСТ Р 50453-92
(ИСО 937-78). Метод основан на минерализации пробы, последующем
выделении аммонийного азота и его количественном определении с
последующим пересчётом на белок с использованием азотного
коэффициента. (ИСО 937-78).

8. Содержание минеральных веществ (золы) определяли по ГОСТ Р
53642-2009. Метод основан на прокаливании навески образца в муфельной
печи при температуре 550\,±\,25\,°C до полного удаления органических
веществ.

9. Содержание магния и кальция определяли по ГОСТ EN 15505-2013. Метод
основан на минерализации пробы с последующим определением элементов с
использованием атомно-абсорбционной спектрометрии.

10. Содержание фосфора определяли по ГОСТ Р 51482-99 (ИСО 13730-96). Метод
основан на сухом озолении навески с последующим растворением золы в
кислоте и фотометрическом определении фосфора в виде
фосфорномолибденовой кислоты.

11. Содержание тяжёлых металлов (меди, железа, свинца, кадмия) определяли
в соответствии с ГОСТ 26931-86, ГОСТ 26928-86, ГОСТ 26932-86 и ГОСТ
26933-86. Методика включает минеральное разложение пробы и последующее
количественное определение элементов с использованием
атомно-абсорбционной спектрометрии.
{\bfseries Результаты и обсуждение.} Проращивание является одним из
наиболее эффективных методов изменения пищевой ценности зернового сырья,
который был подтвержден в ряде научных исследований. Этот процесс
представляет собой переход семени из состояния покоя в фазу активного
роста зародыша {[}8{]}.

В качестве объектов исследования выбраны пророщенный овес и пророщенный
ячмень. В лабораторных условиях проращивали овес и ячмень в специальном
приборе для проращивания, внутри оборудования зерно равномерно
опрыскивали теплой водой. В ходе работы установлена зависимость времени
проращивания от температуры, проращивание происходило при температуре
16-18\textsuperscript{0}С в течение 36-48 часов. Был проведен
физико-химический анализ пророщенных зерен овса и ячменя (Таблица 1).
\end{multicols}

\begin{longtblr}[
  caption = {\bfseries Таблица 1 - Физико-химические показатели пророщенных зерен овса и ячменя},
  label = none,
  entry = none,
]{
  width = \linewidth,
  colspec = {Q[179]Q[60]Q[83]Q[67]Q[94]Q[73]Q[106]Q[110]Q[158]},
  cells = {c},
  cell{1}{1} = {r=2}{},
  cell{1}{2} = {c=2}{0.143\linewidth},
  cell{1}{4} = {c=2}{0.161\linewidth},
  cell{1}{6} = {c=2}{0.179\linewidth},
  cell{1}{8} = {c=2}{0.268\linewidth},
  vlines,
  hline{1,3-6} = {-}{},
  hline{2} = {2-9}{},
}
Проращивание ,час & Влажность, \% &        & Кислотность, \textsuperscript{0}С &        & Длина ростков, мм &        & Количество всхожих зерен, \% &        \\
                  & овес          & ячмень & овес                              & ячмень & овес              & ячмень & овес                         & ячмень \\
ч/з 24            & 45,5          & 45,4   & 0,28                              & 0,32   & 2                 & 4      & 20                           & 40     \\
ч/з 36            & 47,5          & 48,1   & 0,32                              & 0,36   & 6                 & 8      & 40                           & 80     \\
ч/з 48            & 47,8          & 48,3   & 0,34                              & 0,43   & 8                 & 10     & 75                           & 95     
\end{longtblr}

Исследования показали, что у пророщенных зерен ростки достигают 2-10 мм,
а всхожесть увеличивается с 20 до 95 \% за 24-48 часов. Процесс
проращивания у ячменя быстрее, чем у овса. Для предотвращения порчи,
зерно высушивали в камере "Home Station 2" при температуре 43-45°C в
течение 12-14 часов до влажности 10-12\%. Основные показатели
пророщенных зерен овса и ячменя приведены в Таблице 2.

\begin{longtblr}[
  caption = {\bfseries Таблица 2 - Характеристики пророщенных зерен},
  label = none,
  entry = none,
]{
  cells = {c},
  hlines,
  vlines,
}
Показатели                                  & Овес  & Ячмень \\
Влажность, \%                               & 10,6  & 10,3   \\
Титруемая кислотность, \textsuperscript{0}Т & 0,32  & 0,41   \\
Содержание белка, \% СВ                     & 12,64 & 11,87  \\
Способность прорастания, \%                 & 75    & 95     
\end{longtblr}

\begin{multicols}{2}
По результатам, влажность пророщенных зерен овса и ячменя была близка:
10,6\% у овса и 10,3\% у ячменя, что указывает на оптимальные условия
проращивания для обоих видов зерна. Эти данные показывают, что процесс
сушки зерен после проращивания был успешным и обеспечил стабильность их
структуры. овес имеет более высокое содержание белковых веществ, ячмень
имеет более высокую способность к прорастанию.

Титруемая кислотность пророщенных зерен овса (0,32°Т) ниже, чем у ячменя
(0,41°Т). Это может свидетельствовать о более мягком характере овса в
отношении кислотности, что, в свою очередь, может влиять на его
восприимчивость к различным условиям хранения и обработки.

Содержание белка в пророщенных зернах овса (12,64\%) оказалось выше, чем
в ячмене (11,87\%). Это подтверждает, что овес является более белковым
зерном, что имеет важное значение для его использования в пищевых
добавках, особенно в диетическом питании и для людей с высокими
требованиями к белковым компонентам.

Способность прорастания у ячменя значительно выше (95\%) по сравнению с
овсом (75\%), что подчеркивает более высокую всхожесть ячменя. Это может
свидетельствовать о большей устойчивости ячменя к процессу проращивания
и его более быстрых темпах роста, что важно для эффективности
использования зерна в процессе получения проращенных продуктов.

Проведено исследование качественных характеристик измельченного
топинамбура, пророщенного овса, пророщенного ячменя и корня солодки. В
рамках исследования были определены физико-химические параметры,
содержание витаминов, макро- и микроэлементов, а также аминокислотный
состав данных материалов (Таблица 3).
\end{multicols}

\begin{longtblr}[
  caption = {\bfseries Таблица 3 - Качественные показатели сырья},
  label = none,
  entry = none,
]{
  width = \linewidth,
  colspec = {Q[150]Q[123]Q[129]Q[121]Q[113]Q[121]Q[113]},
  cells = {c},
  cell{2}{1} = {c=7}{},
  cell{8}{2} = {c=6}{0.72\linewidth},
  cell{9}{1} = {c=7}{0.931\linewidth},
  cell{16}{1} = {c=7}{0.931\linewidth},
  hlines,
  vlines,
}
Наименование показателя      & {Измельчен\-ный\\топинамбур} & Овес           & Пророщен\-ный овес & Ячмень       & Пророщен\-ный ячмень & {Корень\\солодки} \\
Физико-химические показатели &                            &                &                     &              &                       &                   \\
Массовая доля белка, \%      & 8,32 ± 0,09                  & 11,95 ± 0,16     & 12,64 ± 0,15          & 10,83 ± 0,14   & 11,87 ± 0,14            & 10,48 ± 0,12        \\
Массовая доля влаги, \%      & 9,63 ± 0,13                  & 9,89 ± 0,17      & 6,59 ± 0,10           & 10,21 ± 0,17   & 8,58 ± 0.13             & 7,1 ± 0,23          \\
Массовая доля золы, \%       & 5,17 ± 0,02                  & 3,37 ± 0,05      & 3,35 ± 0,04           & 2,77 ± 0,03    & 1,93 ± 0,04             & 6,245 ± 0,005       \\
Массовая доля клетчатки,\%   & 24,65 ± 0,29                 & 11,31 ± 0,13     & 10,03 ± 0,12          & 12,09 ± 0,16   & 8,62 ± 0,10             & 3,47 ± 0,04         \\
Титруемая кислотность, 0Т    & 0,86 ± 0,019                 & 0,422 ± 0,005    & 0,364 ± 0,008         & 0,508 ± 0,006  & 0,463 ± 0,011           & 1,521 ± 0.034       \\
Посторонние примеси          & Не обнаружено              &                &                     &              &                       &                   \\
Витаминный состав сырья      &                            &                &                     &              &                       &                   \\
Витамин Е                    & 0,912 ± 0,01                 & 1,554 ± 0,02     & 2,61 ± 0,03           & 1,398 ± 0,019  & 1,01 ± 0,01             & 0,08 ± 0,001        \\
Витамин В1                   & 0,365 ± 0,073                & 0,38 ± 0,075     & 0,472 ± 0,094         & 0,361 ± 0,07   & 0,294 ± 0.058           & 0,413 ± 0,082       \\
Витамин В2                   & 0,274 ± 0,115                & -              & 0,093 ± 0,039         & -            & 0,114 ± 0,047           & 0,77 ± 0,032        \\
Витамин В3                   & 8,21 ± 1,64                  & 4,28 ± 0,86      & 1,38 ± 0,27           & 5,24 ± 1,05    & 2,69 ± 0,54             & 2,15 ± 0,43         \\
Витамин В6                   & -                          & 2,46 ± 0,049     & 0,26 ± 0,05           & 0,317 ± 0,063  & 0,276 ± 0,055           & 0,103 ± 0,02        \\
Витамин С                    & 31,95 ± 5,7                  & -              & -                   & -            & -                     & 2,57 ± 0,46         \\
Минеральные элементы,мг/100  &                            &                &                     &              &                       &                   \\
Железо                       & 1,83 ± 0,021                 & 5,61 ± 0,07      & 10,27 ± 0,12          & 5,95 ± 0,08    & 4,64 ± 0,05             & 2,75 ± 0,03         \\
Магний                       & 63,89 ± 0,76                 & 110,32 ± 1,54    & 126,10 ± 1,51         & 142,01 ± 1,98  & 89,61 ± 1,07            & 27,46 ± 0,33        \\
Кальций                      & 98,77 ± 1,18                 & 120,07 ± 1,68    & 109,29 ± 1,31         & 105,16 ± 1,47  & 55,48 ± 0,66            & 80,25 ± 0,96        \\
Калий                        & 1007,44 ± 12,09              & 463,24 ± 6,48    & 393,25 ± 4,72         & 459,72 ± 7,43  & 260,97 ± 3,13           & 315,02 ± 3,78       \\
Фосфор                       & 424,46 ± 5,09                & 248,16 ± 3,47    & 337,21 ± 4,04         & 292,30 ± 4,09  & 210,79 ± 2,59           & 56,17 ± 0,67        \\
Йод                          & -                          & 0,0033 ± 0,00001 & 0,008 ± 0,0001        & 0,004 ± 0,0001 & 0,0059 ± 0,0001         & 0,013 ± 0,0001      
\end{longtblr}

\begin{multicols}{2}
Таблица 3 показывает состав измельченного топинамбура, пророщенного
овса, ячменя и корня солодки. Пророщенный овес (12,64\%) и ячмень
(11,87\%) содержат значительное количество белка по сравнению с другими
компонентами, такими как топинамбур (8,32\%). Это свидетельствует о
высоком качестве этих зерен как источников белка, что может быть полезно
для поддержания нормального обмена веществ, особенно для людей с
диабетом. Топинамбур, в свою очередь, выделяется высоким содержанием
клетчатки (24,65\%), что способствует улучшению работы пищеварительного
тракта и может быть полезно для детоксикации организма. Витамины,
содержащиеся в этих продуктах, также разнообразны: топинамбур богат
витамином C (31,95 мг/100 г), в то время как овес и ячмень содержат
витамины группы B, такие как витамин B1, B3 и B6, что делает эти
продукты ценными для поддержания общего состояния здоровья. Также стоит
отметить, что овес и ячмень содержат минералы, такие как магний, калий и
кальций, которые важны для нормального функционирования
сердечно-сосудистой системы и обмена веществ. Это подчеркивает их
потенциал для включения в диеты, направленные на улучшение метаболизма и
поддержание общего здоровья, особенно для людей, имеющих дефицит этих
микроэлементов..
\end{multicols}

\begin{longtblr}[
  caption = {\bfseries Таблица 4 - Аминокислотный состав},
  label = none,
  entry = none,
]{
  width = \linewidth,
  colspec = {Q[177]Q[171]Q[94]Q[133]Q[94]Q[146]Q[113]},
  cells = {c},
  hlines,
  vlines,
}
Наименование показателя, \% & Измельченный топинамбур & Овес        & Пророщен\-ный овес & Ячмень      & Пророщен\-ный ячмень & Корень солодки \\
Аргинин                     & 0,964 ± 0,385             & 1,570 ± 0,628 & 1,267 ± 0,507      & 1,363 ± 0,545 & 1,097 ± 0,439        & 1,138 ± 0,455    \\
Лизин                       & 0,450 ± 0,153             & 1,221 ± 0,415 & 1,429 ± 0,486      & 1,107 ± 0,376 & 1,426 ± 0,485        & 1,138 ± 0,387    \\
Тирозин                     & 0,418 ± 0,125             & 0,406 ± 0,122 & 0,747 ± 0,224      & 0,426 ± 0,128 & 0,695 ± 0,208        & 0,534 ± 0,160    \\
Фенилаланин                 & 0,610 ± 0,183             & 0,872 ± 0,262 & 0,747 ± 0,224      & 0,809 ± 0,243 & 0,621 ± 0,186        & 0,569 ± 0,171    \\
Гистидин                    & 0,308 ± 0,154             & 0,523 ± 0,262 & 0,845 ± 0,422      & 0,468 ± 0,234 & 0,512 ± 0,256        & 0,391 ± 0,196    \\
Лейцин + изолейцин            & 0,675 ± 0,175             & 1,090 ± 0,283 & 1,104 ± 0,287      & 1,022 ± 0,266 & 1,060 ± 0,276        & 0,889 ± 0,231    \\
Метионин                    & 0,145 ± 0,049             & 0,148 ± 0,05  & 0,552 ± 0,188      & 0,183 ± 0,062 & 0,548 ± 0,186        & 0,462 ± 0,157    \\
Валин                       & 0,707 ± 0,283             & 0,959 ± 0,384 & 0,975 ± 0,390      & 0,894 ± 0,358 & 0,951 ± 0,380        & 0,889 ± 0,356    \\
Пролин                      & 1,028 ± 0,267             & 0,828 ± 0,215 & 0,747 ± 0,194      & 0,766 ± 0,199 & 0,731 ± 0,190        & 0,747 ± 0,194    \\
Треонин                     & 0,482 ± 0,193             & 0,698 ± 0.279 & 0,845 ± 0,338      & 0,639 ± 0,255 & 0,877 ± 0,351        & 0,747 ± 0,299    \\
Серин                       & 0,803 ± 0,209             & 0,785 ± 0.204 & 0,715 ± 0,186      & 0,724 ± 0,188 & 0,695 ± 0,181        & 0,605 ± 0,157    \\
Аланин                      & 0,578 ± 0,150             & 0.741 ± 0,193 & 1,234 ± 0,321      & 0,681 ± 0,177 & 1,133 ± 0,295        & 1,032 ± 0,268    \\
Глицин                      & 0,675 ± 0,229             & 0,741 ± 0,252 & 0,780 ± 0,265      & 0,681 ± 0,232 & 0,768 ± 0,261        & 0,783 ± 0,266    
\end{longtblr}

\begin{multicols}{2}

Полученные данные подчеркивают значимость изученных продуктов как ценных
источников питательных веществ, которые могут быть использованы для
разработки новых БАДов. Процесс проращивания способствует повышению
биодоступности витаминов и минералов, улучшая их усвоение, а также
увеличивает антиоксидантную активность, что способствует защите клеток
от оксидативного стресса {[}9{]}.

Суточная потребность в клетчатке для диабетиков составляет 50 г.
Пророщенные овес, ячмень, корень солодки и топинамбур содержат клетчатку
(г/100 г) в количестве: 10,03; 8,62; 3,47 и 24,65, что покрывает
суточную норму (\%) на 20,06; 17,24; 6,94 и 49,3 соответственно. Таким
образом, пророщенные овес, ячмень и корень солодки являются хорошими
источниками клетчатки и аминокислот (таблица 4).

Аминокислотный состав, представленный в Таблице 4, демонстрирует
значительное содержание незаменимых аминокислот в пророщенных овсе и
ячмене, таких как лизин, валин, треонин и фенилаланин. Это делает эти
зерна ценными источниками аминокислот, которые играют ключевую роль в
поддержании иммунной системы, регенерации тканей и других жизненно
важных процессах. В частности, уровень лизина и фенилаланина в
пророщенных зернах овса и ячменя значительно выше по сравнению с
аналогичными показателями в обычных зернах овса и ячменя. Эти данные
согласуются с результатами других исследований, подтверждающих, что
проращивание увеличивает содержание незаменимых аминокислот в зерновых
культурах (например, в исследованиях, где показано улучшение
аминокислотного состава при проращивании ячменя и овса). Также стоит
отметить, что высокие уровни аргинина и метионина, наблюдаемые в этих
зернах, подтверждают их потенциал в поддержке метаболических и иммунных
функций организма. Полученные результаты могут быть полезны не только
для людей, следящих за своим рационом, но и для спортсменов, а также
тех, кто нуждается в дополнительном источнике аминокислот для
восстановления и улучшения общего состояния организма.

Топинамбур, в свою очередь, богат инулином и минеральными веществами,
что делает его полезным для поддержания работы пищеварительной системы и
метаболизма {[}10{]}.

Определены микробиологические показатели сырья (топинамбур молотый,
пророщенный овес, пророщенный ячмень, корень солодки, овес, ячмень)
(Таблица 5).
\end{multicols}

\begin{longtblr}[
  caption = {\bfseries Таблица 5 -- Микробиологические показатели сырья},
  label = none,
  entry = none,
]{
  width = \linewidth,
  colspec = {Q[273]Q[140]Q[148]Q[58]Q[165]Q[80]Q[121]},
  cells = {c},
  cell{3}{2} = {c=6}{},
  cell{5}{2} = {c=6}{},
  hlines,
  vlines,
}
Наименование показателей, ед.изм.   & Топинамбур молотый & Пророщен\-ный овес & Овес  & Пророщенный ячмень & Ячмень & Корень солодки \\
КМАФАнМ, КОЕ/г                      & 3*104                 & 3*104            & 4*103 & 7*103              & 7*103  & 4*104          \\
БГКП (колиформы) в 1.0 см3 продукта & не обнаружены         &                  &       &                    &        &                \\
Дрожжи, КОЕ/г                       & 5                     & 2                & 6     & 6                  & 5      & 1              \\
Плесени, КОЕ/г                      & не обнаружены         &                  &       &                    &        &                
\end{longtblr}

\begin{multicols}{2}
Микробиологические данные, представленные в Таблице 5, демонстрируют,
что все исследуемые образцы (топинамбур, овес, ячмень и корень солодки)
соответствуют санитарным нормам, поскольку уровень микробиологических
показателей (КМАФАнМ, БГКП, дрожжи и плесени) находится в пределах
допустимых значений. Это подтверждает, что сырье было правильно
обработано и соответствует стандартам безопасности для пищевых продуктов
и БАДов.

{\bfseries Выводы.} Результаты исследования показали, что пророщенные овес
и ячмень характеризуются высоким содержанием витаминов и незаменимых
аминокислот, что делает их ценными для питания, особенно в условиях
ограниченного рациона. Эти зерна являются отличным источником
необходимых веществ для поддержания общего здоровья и могут быть
рекомендованы в качестве дополнения к диетам. Топинамбур, в свою
очередь, богат инулином и минеральными веществами, что способствует
улучшению работы пищеварительной системы и обмена веществ. Также он
оказывает положительное влияние на регулирование уровня сахара в крови,
что делает его особенно полезным для людей с сахарным диабетом.

Кроме того, пророщенные семена каждой культуры обладают уникальным
составом, включающим аминокислоты, полисахариды и микроэлементы, что
позволяет им оказывать целенаправленное оздоровительное воздействие. Они
могут быть рекомендованы людям, страдающим различными заболеваниями,
включая расстройства обмена веществ.

Проращивание зерна способствует повышению биодоступности питательных
веществ, что способствует улучшению усвоения витаминов и минералов.
Пророщенные зерна, топинамбур и корень солодки обладают антиоксидантными
свойствами, защищая клетки от оксидативного стресса. В условиях стресса,
неправильного питания и неблагоприятных экологических факторов важность
использования БАДов с растительными компонентами возрастает, так как они
могут эффективно поддерживать здоровье и предотвращать заболевания,
такие как диабет и расстройства обмена веществ.

\emph{{\bfseries Финансирование:} Материалы подготовлены в рамках
выполнения проекта ИРН AP23485292 «Разработка технологии комплексной
биологически активной добавки (БАД) иммуномодулирующей направленности»
по бюджетной программе 217 «Развитие науки» по подпрограмме 102
«Грантовое финансирование научных исследований» Министерства науки и
высшего образования Республики Казахстан на 2024-2026 годы.}
\end{multicols}

\begin{center}
{\bfseries Литература}
\end{center}

\begin{references}
1. Донченко Л.В. Продукты питания в отечественной и зарубежной истории:
учеб. пособие для специальностей 311200 "Технология пр-ва и перераб.
с.-х. продукции" / Л.В. Донченко, В. Д. Надыкта. -Москва: ДеЛи принт,
2006. - 295 с. ISBN 5-94343-109-8

2. Чиркина Т.Ф. Биологически активные добавки и здоровое питание: матер.
всерос. научно-молодеж. конф. с междун. участием (ВСГПУ, 25-28 сент.
2001 г.) - Улан-Удэ: Изд-во ВСГПУ, 2001. - 125 с.

3. Белоусова О.В., Белоусов Е.А., Иващенкова Е.А. Изучение потребителя
биологически активных добавок // Молодой ученый. -2016. -№ 20 (124). -С.
66-69.

4. Чекмарева, Л. С., Иванова, М. М.Иммуномодуляторы растительного
происхождения и их роль в поддержании иммунной системы.- Вестник
иммунологии. -2016.- №5(2), 59-67.

5. Calder PC, Carr AC, Gombart AF, Eggersdorfer M. Optimal nutritional
status for a well-functioning immune system is an important factor to
protect against viral infections//Nutrients.- 2020.-Vol.12(4): 1181.
DOI~\href{https://doi.org/10.3390/nu12041181}{10.3390/nu12041181}

6. Кацерикова Н.В. Технология продуктов функционального питания: учебное
пособие. - Кемерово: Кемеровский технологический институт пищевой
промышленности. -2004. - С.15--17. ISBN 5-89289-311-1

7. Белоусова О.В., Белоусов Е.А., Иващенкова А.О. (2016). биологически
активные добавки как перспективное направление развития
фармацевтического рынка//Научные результаты биомедицинских
исследований.-2016.-№ 2 (4).- С.89-94. DOI
10.18413/2313-8955-2016-2-4-89-94

8. Осадченко И.М., Горлов И.Ф., Харченко О.В., Чурзин В.Н. Использование
электрохимически активированной воды при возделывании ярового ячменя
//Кормопроизводство. -2007. - №8. --С.26-28.

9. Науменко Н. В., Паймулина А. В., Велямов М. Т. Влияние размеров частиц
муки из пророщенного зерна на ее технологические свойства и качество
готовых изделий // Вестник Южно-Уральского государственного
университета. Серия: Пищевые и биотехнологии. -2019.- Т.7(1). - С.36 -
42. DOI 10.14529/food190105.

10. Жучкова М. А., Скрипников С. Г. Топинамбур -- растение XXI века
//Овощи России.2017.- № 1.- С.31- 33. DOI
10.18619/2072-9146-2017-1-31-33.
\end{references}

\begin{center}
{\bfseries References}
\end{center}

\begin{references}
1. Donchenko L.V. Produkty pitanija v otechestvennoj i zarubezhnoj
istorii: ucheb. posobie dlja special' nostej 311200
"Tehnologija pr-va i pererab. s.-h. produkcii" / L.V. Donchenko, V. D.
Nadykta. -Moskva: DeLi print, 2006.- 295 s. ISBN 5-94343-109-8. {[}in
Russian{]}

2. Chirkina T.F. Biologicheski aktivnye dobavki i zdorovoe pitanie:
mater. vseros. nauchno-molodezh. konf. s mezhdun. uchastiem (VSGPU,
25-28 sent.2001 g.) - Ulan-Udje: Izd-vo VSGPU, 2001. - 125 s. {[}in
Russian{]}

3. Belousova O.V., Belousov E.A., Ivashhenkova E.A. Izuchenie
potrebitelja biologicheski aktivnyh \\dobavok // Molodoj uchenyj. -2016.
-№ 20 (124). -S.66-69. {[}in Russian{]}

4. Chekmareva, L. S., Ivanova, M. M.Immunomoduljatory
rastitel' nogo proishozhdenija i ih rol'{}
v \\podderzhanii immunnoj sistemy.- Vestnik immunologii. -2016.- №5(2),
59-67. {[}in Russian{]}

5. Calder PC, Carr AC, Gombart AF, Eggersdorfer M. Optimal nutritional
status for a well-functioning immune system is an important factor to
protect against viral infections//Nutrients.- 2020.-Vol.12(4): 1181.
DOI~\href{https://doi.org/10.3390/nu12041181}{10.3390/nu12041181}

6. Kacerikova N.V. Tehnologija produktov funkcional' nogo
pitanija: uchebnoe posobie. - Kemerovo: Kemerovskij tehnologicheskij
institut pishhevoj promyshlennosti. -2004. - S.15--17. ISBN
5-89289-311-1. {[}in Russian{]}

7. Belousova O.V., Belousov E.A., Ivashhenkova A.O. (2016). biologicheski
aktivnye dobavki kak \\perspektivnoe napravlenie razvitija
farmacevticheskogo rynka//Nauchnye rezul' taty
biomedicinskih \\issledovanij.-2016.-№ 2 (4).- S.89-94. DOI
10.18413/2313-8955-2016-2-4-89-94. {[}in Russian{]}

8. Osadchenko I.M., Gorlov I.F., Harchenko O.V., Churzin V.N.
Ispol' zovanie jelektrohimicheski \\aktivirovannoj vody pri
vozdelyvanii jarovogo jachmenja //Kormoproizvodstvo. -2007. - №8.
--S.26-28. {[}in Russian{]}

9. Naumenko N. V., Pajmulina A. V., Veljamov M. T. Vlijanie razmerov
chastic muki iz proroshhennogo zerna na ee tehnologicheskie svojstva i
kachestvo gotovyh izdelij // Vestnik Juzhno-Ural' skogo
\\gosudarstvennogo universiteta. Serija: Pishhevye i biotehnologii.
-2019.- T.7(1). - S.36 - 42. DOI \\10.14529/food190105. {[}in Russian{]}

10. Zhuchkova M. A., Skripnikov S. G. Topinambur -- rastenie XXI veka
//Ovoshhi Rossii.2017.- № 1.- S.31- 33. DOI
10.18619/2072-9146-2017-1-31-33. {[}in Russian{]}
\end{references}

\emph{{\bfseries Сведения об авторах}}

Жумалиева Г.Е. - кандидат технических наук, ТОО «Казахский
научно-исследовательский институт перерабатывающей и пищевой
промышленности» Алматы, Казахстан, e-mail: g.zhumalieva@rpf.kz;

Чоманов У.Ч. - доктор технических наук, ТОО «Казахский
научно-исследовательский институт перерабатывающей и пищевой
промышленности»,Алматы, Казахстан, e-mail: chomanov\_u@mail.ru;

Шоман Ә.К.- магистр технических наук, ТОО «Казахский
научно-исследовательский институт перерабатывающей и пищевой
промышленности» Алматы, Казахстан, e-mail: shoman\_a@mail.ru;

Оғазова А.Ғ. - магистр технических наук, ТОО «Казахский
научно-исследовательский институт перерабатывающей и пищевой
промышленности» Алматы?Казахстан, e-mail: o.aidana\_01@mail.ru;

\emph{{\bfseries Information about the authors}}

Zhumalieva G.E. - Candidate of Technical Sciences, LLC «Kazakh research
institute of processing and food industry» Almaty, Kazahstan, e-mail:
g.zhumalieva@rpf.kz;

Chomanov U.Сh. - Doctor of Technical Sciences, LLC «Kazakh research
institute of processing and food industry» Almaty, Kazahstan, e-mail:
chomanov\_u@mail.ru;

Shoman A.K- Master of Engineering Sciences, LLC «Kazakh research
institute of processing and food industry» Almaty, Kazahstan, e-mail:
shoman\_a@mail.ru;\%20

Ogazova A.G. - Master of Engineering Sciences, LLC «Kazakh research
institute of processing and food industry» Almaty, Kazahstan, e-mail:
o.aidana\_01@mail.ru
