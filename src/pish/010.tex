\id{МРНТИ 65.35.03}{}

\begin{articleheader}
\sectionwithauthors{Б.Ж. Мулдабекова, А.Т. Жұмабекова, М.А. Якияева}{КОМПОЗИТТІ ҰНДЫ ҚОЛДАНУ АРҚЫЛЫ ЖҰМСАҚ ВАФЛИЛЕРДІҢ ТАҒАМДЫҚ
ҚҰНДЫЛЫҒЫН АРТТЫРУ}

{\bfseries
Б.Ж. Мулдабекова,
А.Т. Жұмабекова,
М.А. Якияева\textsuperscript{\envelope }
}
\end{articleheader}

\begin{affiliation}
\emph{Алматы технологиялық университеті, Алматы, Қазақстан,}

\raggedright \textsuperscript{\envelope }Корреспондент-автор:
\href{mailto:yamadina88@mail.ru}{\nolinkurl{yamadina88@mail.ru}}
\end{affiliation}

Салауатты тамақтану мен функционалдық өнімдерге деген қызығушылықтың
артуы аясында биологиялық құндылығы жоғары тағам өнімдерін әзірлеуге
ерекше көңіл бөлінуде. Қазіргі тағам өнеркәсібінің өзекті бағыттарының
бірі --- құрамында тағамдық талшықтар, дәрумендер мен минералдар бар
әртүрлі өсімдік текті шикізат қоспаларынан тұратын композитті ұнды
пайдалану. Бұл зерттеу жұмысы жұмсақ вафлилердің тағамдық құндылығын
арттыру мақсатында дәстүрлі бидай ұнын жүгері және қарақұмық ұнына
ішінара алмастыру мүмкіндіктерін зерттеуге арналған. Зерттеудің мақсаты
--- композитті ұн қолдану арқылы тағамдық көрсеткіштері жақсартылған
жұмсақ вафлилердің рецептурасын әзірлеу. Жұмыстың негізгі идеясы ---
жоғары дәмдік қасиеттер мен функционалдық мүмкіндіктерді біріктіретін,
салауатты тамақтану мәдениетін қалыптастыруға ықпал ететін өнім жасау.
Міндеттері --- композитті ұн құрамындағы компоненттердің оңтайлы
арақатынасын анықтау, алынған өнімнің органолептикалық, физико-химиялық
көрсеткіштерін және тағамдық құндылығын бағалау. Зерттеудің ғылыми
жаңалығы --- жұмсақ вафли дайындауда бидай, жүгері және қарақұмық
ұндарынан тұратын композитті ұнның нақты пропорцияларын қолдануда
көрініс табады. Жұмыстың практикалық маңыздылығы --- ұсынылған
рецептураны диеталық және функционалдық нан-тоқаш өнімдерін өндіруде
пайдалануға болатындығында. Зерттеу әдістемесі түрлі ұн түрлерін әртүрлі
пропорцияда араластыру, органолептикалық (дәм, иіс, құрылым, түс) және
физико-химиялық көрсеткіштер (ылғалдылық, құрылым) бойынша зертханалық
сынақтар жүргізу, сондай-ақ дайын өнімнің тағамдық құндылығын талдау
кезеңдерін қамтыды. Негізгі нәтижелер көрсеткендей, құрамында 30\%
жүгері ұны бар вафлилер оңтайлы сенсорлық сипаттамаларға ие болып,
тұтынушылар тарапынан жоғары бағаланған. Осы негізде 40\% бидай, 30\%
жүгері және 30\% қарақұмық ұнынан тұратын рецептура жасалды. Алынған
үлгілер құрамындағы тағамдық талшықтар мен ақуыз мөлшерінің жоғары
болуымен, жақсарған органолептикалық қасиеттерімен және теңгерімді
аминқышқылдық құрамымен ерекшеленді. Осылайша, жүргізілген зерттеу
композитті ұнды пайдалану жұмсақ вафлилердің тағамдық құндылығын
арттыруда тиімді екенін дәлелдеді. Бұл жұмыс функционалдық тағам
өнімдері технологияларын дамытуға үлес қосып, салауатты тағам түрлерінің
ассортиментін кеңейту мақсатында балама ұн түрлерін пайдаланудың
негізділігін ұсынады. Зерттеу нәтижелерінің практикалық мәні ---
әзірленген рецептураны өнеркәсіптік өндірісте бейімдеуге және оны
балаларға, егде жастағы адамдарға, сондай-ақ тағамдық талшыққа сұранысы
жоғары тұтынушыларға арналған өнімдер жасау үшін қолдануға
болатындығында.

{\bfseries Түйін сөздер:} жұмсақ вафли, композитті ұн, бидай ұны, жүгері
ұны, қарақұмық ұны, тағамдық құндылық, нутриенттік құрам, функционалдық
тағам өнімдері.

\begin{articleheader}
{\bfseries ПОВЫШЕНИЕ ПИТАТЕЛЬНОЙ ЦЕННОСТИ МЯГКИХ ВАФЕЛЬ С ИСПОЛЬЗОВАНИЕМ
КОМПОЗИТНОЙ МУКИ}

{\bfseries
Б.Ж. Мулдабекова,
А.Т. Жұмабекова,
М.А. Якияева\textsuperscript{\envelope }
}
\end{articleheader}

\begin{affiliation}
\emph{Алматинский технологический университет, Алматы, Казахстан,}

e-mail: \href{mailto:yamadina88@mail.ru}{\nolinkurl{yamadina88@mail.ru}}
\end{affiliation}

В условиях растущего интереса к здоровому питанию и функциональным
продуктам особое внимание уделяется разработке пищевых изделий с
повышенной биологической ценностью. Одним из актуальных направлений
современной пищевой промышленности является использование композитной
муки --- смесей различных видов растительного сырья, обогащённых
пищевыми волокнами, витаминами и минералами. Данная работа посвящена
исследованию возможностей повышения питательной ценности мягких вафель
путём частичной замены традиционной пшеничной муки на кукурузную и
гречневую муку. Целью исследования является разработка рецептуры мягких
вафель с улучшенными показателями пищевой ценности за счёт применения
композитной муки. Основной идеей работы стало создание продукта,
сочетающего в себе высокие вкусовые качества и функциональные свойства,
способствующие формированию культуры здорового питания. В качестве задач
поставлено выявление оптимального соотношения компонентов композитной
муки, оценка органолептических, физико-химических показателей и
питательной ценности полученного продукта. Научная новизна исследования
заключается в использовании определённых пропорций композитной муки на
основе пшеничной, кукурузной и гречневой муки для приготовления мягких
вафель. Практическая значимость работы обусловлена возможностью
внедрения предложенной рецептуры в производство диетических и
функциональных хлебобулочных изделий. Методология исследования включала
этапы подбора и смешивания муки различных видов в разных пропорциях,
лабораторные испытания по органолептическим и физико-химическим
показателям (влажность, структура, цвет, вкус, аромат), а также анализ
пищевой ценности конечного продукта. Основные результаты показывают, что
вафли с 30\% добавлением кукурузной муки обладают оптимальными
сенсорными характеристиками и высокой потребительской оценкой. Далее
была разработана рецептура, включающая 40\% пшеничной, 30\% кукурузной и
30\% гречневой муки. Полученные образцы отличались повышенным
содержанием пищевых волокон и белков, улучшенными органолептическими
показателями и сбалансированным аминокислотным составом. Таким образом,
проведённое исследование подтвердило эффективность использования
композитной муки для повышения питательной ценности мягких вафель.
Работа вносит вклад в развитие технологий функциональных продуктов
питания, обоснованно предлагая применение альтернативных видов муки для
расширения ассортимента изделий здорового питания. Практическое значение
результатов заключается в возможности адаптации разработанной рецептуры
в условиях промышленного производства и использовании её для создания
продуктов, ориентированных на улучшение рациона питания различных групп
населения, включая детей, пожилых людей и лиц с повышенной потребностью
в пищевых волокнах.

{\bfseries Ключевые слова:} мягкие вафли, композитная мука, пшеничная мука,
кукурузная мука, гречневая мука, пищевая ценность, нутриентный состав,
функциональные пищевые продукты

\begin{articleheader}
{\bfseries ENHANCING THE NUTRITIONAL VALUE OF SOFT WAFFLES THROUGH THE USE
OF COMPOSITE FLOUR}

{\bfseries
B.Zh. Muldabekova,
A.T. Zhumabekova,
M.A. Yakiyayeva\textsuperscript{\envelope }
}
\end{articleheader}

\begin{affiliation}
Almaty Technological University, Almaty, Kazakhstan,

e-mail: \href{mailto:yamadina88@mail.ru}{\nolinkurl{yamadina88@mail.ru}}
\end{affiliation}

In the context of increasing interest in healthy nutrition and
functional food products, special attention is being paid to the
development of food items with enhanced biological value. One of the
current trends in the modern food industry is the use of composite
flour---blends of various types of plant-based raw materials enriched
with dietary fibers, vitamins, and minerals. This study is devoted to
exploring the possibilities of improving the nutritional value of soft
waffles by partially replacing traditional wheat flour with corn and
buckwheat flour. The aim of the research is to develop a formulation for
soft waffles with improved nutritional characteristics through the use
of composite flour. The main idea of the work is to create a product
that combines high taste qualities with functional properties that
promote a culture of healthy eating. The objectives include identifying
the optimal ratio of composite flour components, evaluating organoleptic
and physicochemical properties, and assessing the nutritional value of
the resulting product. The scientific novelty of the study lies in the
use of specific proportions of composite flour based on wheat, corn, and
buckwheat flours for the preparation of soft waffles. The practical
significance of the work is determined by the possibility of
implementing the proposed formulation in the production of dietary and
functional bakery products. The methodology of the research involved the
selection and mixing of different types of flour in various proportions,
laboratory testing of organoleptic (taste, aroma, texture, color) and
physicochemical properties (moisture content, structure), as well as
analysis of the nutritional value of the final product. The main results
show that waffles with 30\% corn flour addition demonstrated optimal
sensory characteristics and received high consumer ratings. Based on
this, a formulation was developed using 40\% wheat flour, 30\% corn
flour, and 30\% buckwheat flour. The resulting samples exhibited
increased levels of dietary fiber and protein, improved organoleptic
qualities, and a balanced amino acid profile. Thus, the conducted study
confirmed the effectiveness of using composite flour to enhance the
nutritional value of soft waffles. The work contributes to the
advancement of functional food product technologies and reasonably
proposes the use of alternative flour types to diversify the range of
healthy food products. The practical relevance of the results lies in
the potential for adapting the developed formulation to industrial
production and using it to create products aimed at improving the diet
of various population groups, including children, the elderly, and
individuals with increased dietary fiber requirements.

{\bfseries Keywords:} soft waffles, composite flour, wheat flour, corn
flour, buckwheat flour, nutritional value, nutrient composition,
functional food products.

\begin{multicols}{2}
{\bfseries Кіріспе.} Бүгінгі күнде халықтың денсаулығын жақсарту, әртүрлі
аурулардың алдын алу мақсатында салауатты тамақтану мәдениетін
қалыптастыру -- қоғам үшін аса маңызды міндеттердің бірі болып табылады.
Себебі дұрыс және теңгерімді тамақтану адамның жалпы әл-ауқатын арттырып
қана қоймай, созылмалы аурулардың, оның ішінде жүрек-қан тамырлары,
семіздік, диабет сияқты кең таралған дерттердің алдын алуға септігін
тигізеді. Осы бағытта қазіргі таңда тағам өндірісі саласында көптеген
ғалымдар мен мамандар белсенді зерттеулер жүргізіп, халыққа пайдалы әрі
қауіпсіз өнім түрлерін ұсыну жолында еңбек етуде. Атап айтқанда, олар
азық-түлік өнімдерінің функционалдық құрамын байыту, тағамдық және
энергетикалық құндылығын жоғарылату, сондай-ақ биологиялық белсенді
компоненттерді кеңінен қолдану арқылы жаңа буындағы өнімдерді ойлап
тауып, тұтынушыларға ұсынуда {[}1, 2{]}.

Сәйкесінше, құрамында денсаулыққа пайдалы компоненттер -- дәрумендер,
антиоксиданттар, тағамдық талшықтар және минералды заттар бар өнімдерге
деген сұраныс жылдан жылға артып келеді. Бұл үрдіс әсіресе балалар мен
ересектер күнделікті жиі тұтынатын нан-тоқаш және ұннан жасалған
кондитерлік өнімдерге ерекше әсер етуде. Дегенмен, мұндай өнімдердің
тағамдық құндылығын арттыра отырып, олардың дәмдік сапасын, құрылымдық
қасиеттерін және сақтау мерзімін өзгеріссіз сақтау -- қазіргі заманғы
тағамтану ғылымының алдыңғы қатарлы мәселелерінің бірі болып қала беруде
{[}3, 4{]}.

Осы мәселені шешудің перспективалы жолдарының бірі -- дәстүрлі бидай
ұнын құрамында тағамдық талшықтар, микро- және макроэлементтер,
биологиялық белсенді заттар көп болатын баламалы өсімдік текті ұндармен
ішінара немесе толық алмастыру. Мұндай тәсіл тек өнімнің құнарлығын
арттырып қана қоймай, оны функционалдық тағам санатына қосуға да
мүмкіндік береді {[}5{]}. Сонымен қатар, бұл әдіс экологиялық тұрғыдан
да тиімді, себебі ауыл шаруашылығында кеңінен таралған дақылдардың қайта
өңделуін қамтамасыз етеді.

Композитті ұн қолдану қазіргі таңда тағам өндірісінде өнімнің тағамдық
құндылығын арттырудың тиімді әрі кең таралған тәсілдерінің бірі ретінде
қарастырылуда. Бұл әдіс арқылы алынған өнімдер тұтынушылардың
физиологиялық қажеттіліктерін неғұрлым толық қамтамасыз ете алады.
Мысалы, жүгері, қарақұмық, тары, сұлы, нут және басқа да дәнді-бұршақты
дақылдардан алынған ұн түрлерін қолдану өнімнің витаминдік және
минералдық құрамын едәуір байытады. Сонымен қатар, бұл қосымшалар
өнімнің органолептикалық қасиеттеріне -- түсіне, дәміне, иісіне және
құрылымына жағымды әсер ете отырып, технологиялық көрсеткіштерін де
жақсартады {[}6, 7{]}.

Зерттеулер көрсеткендей, жүгері мен қарақұмық ұны кондитерлік өнімдердің
амин қышқылдық құрамын теңгеріп, антиоксиданттық белсенділігін
арттырады. Бұл факторлар өнімнің функционалдық қасиеттерін күшейтіп, оны
салауатты өмір салтын ұстанатын адамдар үшін тартымды етеді {[}8--10{]}.
Сонымен бірге, мұндай композитті өнімдер балалар, қарт адамдар,
спортшылар және арнайы диета ұстанатын тұтынушылар үшін де ерекше
маңызға ие.

{\bfseries Материалдар мен әдісдер.} Композиттік ұн негізінде тағамдық
құндылығы жоғары жұмсақ вафли түрін әзірлеу мақсатында зерттеу жұмысы
жүргізілді.

Бұл үшін тәжірибенің алғашқы кезеңінде дәстүрлі бидай ұнына жүгері ұны
әртүрлі мөлшерде (10\%, 20\%, 30\%, 40\%, 50\%) қосылып, бес түрлі вафли
үлгісі дайындалды. Жүгері ұнының таңдалуы оның құрамында В тобы
дәрумендерінің (тиамин, ниацин, фолат), магний, мырыш, темір және табиғи
тағамдық талшықтардың болуымен байланысты. Бұл жүгері ұнын тек
энергетикалық емес, сонымен қатар биологиялық құндылығы жоғары өнім
ретінде пайдалануға мүмкіндік береді {[}11{]}.

Барлық сынамалар бірдей технологиялық жағдайда, төмендегі 1-кестеде
көрсетілген рецептура негізінде өндірілді.
\end{multicols}

\begin{longtblr}[
  caption = {\bfseries 1 - кесте. Бидай және жүгері ұндарын әртүрлі қатынаста қосып, вафли дайындаудың рецептурасы},
  label = none,
  entry = none,
]{
  width = \linewidth,
  colspec = {Q[160]Q[285]Q[98]Q[98]Q[98]Q[98]Q[98]},
  cells = {c},
  cell{1}{1} = {r=2}{},
  cell{1}{3} = {c=5}{0.49\linewidth},
  vlines,
  hline{1,3-11} = {-}{},
  hline{2} = {2-7}{},
}
\textbf{Құрамы} & \textbf{Дәстүрлі рецептура} & \textbf{Эксперименттік рецептура} &                 &                 &                 &                 \\
                & \textbf{Бақылау нұсқасы}    & \textbf{Үлгі 1}                   & \textbf{Үлгі 2} & \textbf{Үлгі 3} & \textbf{Үлгі 4} & \textbf{Үлгі 5} \\
Сары май        & 60 г.                       & 60 г.                             & 60 г.           & 60 г.           & 60 г.           & 60 г.           \\
Құмшекер        & 20 г.                       & 20 г.                             & 20 г.           & 20 г.           & 20 г.           & 20 г.           \\
Сүт             & 70 мл.                      & 70 мл.                            & 70 мл.          & 70 мл.          & 70 мл.          & 70 мл.          \\
Жұмыртқа        & 1 дана                      & 1 дана                            & 1 дана          & 1 дана          & 1 дана          & 1 дана          \\
Қопсытқыш       & 2 г.                        & 2 г.                              & 2 г.            & 2 г.            & 2 г.            & 2 г.            \\
Бидай ұны       & 100 г.                      & 90 г.                             & 80 г.           & 70 г.           & 60 г.           & 50 г.           \\
Жүгері ұны      & -                           & 10 г.                             & 20 г.           & 30 г.           & 40 г.           & 50 г.           \\
Барлығы         & 182 г.                      & 182 г.                            & 182 г.          & 182 г.          & 182 г.          & 182 г.          
\end{longtblr}

\begin{multicols}{2}
Дайын вафли өнімдері органолептикалық және физика-химиялық көрсеткіштері
бойынша салыстырмалы түрде бағаланды. Алынған нәтижелер негізінде ең
жақсы сапа көрсеткен үлгі анықталып, келесі кезеңде осы үлгі базасында
композиттік ұн (бидай -- 40\%, жүгері -- 30\%, қарақұмық -- 30\%)
қолданылып, жаңа вафли нұсқасы дайындалды. Бұл үлгі әрі қарай тағамдық
құндылығы мен тұтынушылық қасиеттері тұрғысынан терең зерттеуге
жіберілді.

{\bfseries Нәтижелер мен талқылау.} Органолептикалық және физика-химиялық
талдау нәтижелері вафли үлгілерінің сапасына қосылған жүгері ұны
мөлшерінің айтарлықтай әсер ететінін көрсетті.10\% және 20\% жүгері ұны
қосылған үлгілерде вафлидің құрылымы мен дәмі жағымды болғанымен, бидай
ұнына тым жақын болды.40\% және 50\% мөлшерінде қосылған үлгілерде
қаттылық пен құрғақтық байқалды, бұл органолептикалық бағалауға кері
әсер етті.

Ал 30\% жүгері ұны қосылған үлгі барлық көрсеткіштер бойынша оңтайлы
нәтижелер көрсетті:

- Орташа ылғалдылық мөлшері -- 17,2\% (бақылау үлгісіне қарағанда
5,5\%-ға жоғары);

- Тығыздығы -- 0,45 г/см³ (бақылаудан 8\% төмен, яғни жұмсақтық жоғары);

- Органолептикалық баға -- 4,7 балл (5 балдық жүйе бойынша), бұл ең
жоғары көрсеткіш болды.
Сонымен қатар, дәмі жағымды, құрылымы біртекті, қабыршақтануы жақсы үлгі
ретінде сарапшылар тарапынан мақұлданды. Салыстыру кезінде бұл үлгінің
жалпы сапасы бақылау үлгісінен (тек бидай ұнынан) шамамен 20--25\%-ға
жоғары деп бағаланды.

Келесі кезеңде бидай, жүгері және қарақұмық ұнын 40:30:30 қатынасында
қолданған үлгіде одан да жақсы нәтижелер тіркелді. Бұл үлгінің:

- Ақуыз мөлшері -- 7,5 г, бұл 70:30 үлгімен салыстырғанда 1,1 есе, ал
классикалық өніммен салыстырғанда 1,15 есе артық;

- Тағамдық талшық -- 2,5 г, бұл дәстүрлі вафлидегі мөлшерден шамамен 2,1
есе жоғары;

- Темір -- 2,8 мг, мырыш -- 1,7 мг, бұл да дәстүрлі өнімдермен
салыстырғанда 65\% жоғары;

- Энергетикалық құндылығы -- 375 ккал, бұл классикалық өніммен
салыстырғанда 4\% төмен.

- Органолептикалық баға -- 4,9 балл.

Бұл көрсеткіштер вафлидің тағамдық құндылығы мен сенсорлық қасиеттерінің
тиімділігін дәлелдеді.
\end{multicols}

\begin{longtblr}[
  caption = {\bfseries 2 - кесте. Дайын вафли өнімдерінің құрамының салыстырмалы көрсеткіштері},
  label = none,
  entry = none,
]{
  width = \linewidth,
  colspec = {Q[275]Q[181]Q[185]Q[300]},
  cells = {c},
  hlines,
  vlines,
}
\textbf{Көрсеткіш атауы (өлшем бірлігі)} & \textbf{Классикалық вафли} & \textbf{70:30 (бидай: жүгері)} & \textbf{40:30:30 (бидай: жүгері: қарақұмық)} \\
Ақуыз (г/100 г)                          & 6,5                        & 6,9                            & 7,5                                          \\
Май (г/100 г)                            & 23                         & 22,2                           & 21                                           \\
Көмірсу (г/100 г)                        & 38                         & 35,4                           & 36                                           \\
Тағамдық талшық (г/100 г)                & 1,2                        & 1,7                            & 2,5                                          \\
Энергетикалық құндылық (ккал)            & 390                        & 345                            & 375                                          \\
Кальций (мг/100 г)                       & 15-20                      & 21,8                           & 18-23                                        \\
Темір (мг/100 г)                         & 1,5-2,0                    & 2,1                            & 2,5-3,0                                      \\
Мырыш (мг/100 г)                         & 1,0-1,5                    & 0,64                           & 1,5-2,0                                      
\end{longtblr}

\begin{multicols}{2}
Жоғарыда ұсынылған эксперименттік нәтижелерге сүйене отырып, алынған
вафли үлгілерінің тағамдық және сапалық сипаттамалары кешенді түрде
салыстырылды. Алдымен 30\% жүгері ұны қосылған үлгінің органолептикалық
және физико-химиялық тұрғыдан тиімді екені байқалды. Алайда, тағамдық
құндылықты одан әрі арттыру мақсатында құрамға қарақұмық ұнын қосу
зерттелді, себебі оның құрамында толыққанды ақуыздар, тағамдық
талшықтар, В тобы дәрумендері, темір, мырыш, магний және антиоксиданттар
бар. Сонымен қатар, қарақұмық ұны глютенсіз болғандықтан, өнімнің
биологиялық құндылығын арттыра отырып, тұтынушылардың кең тобына қолайлы
етеді.

Жоғарыда айтылған үш түрлі нұсқадағы дайын вафли өнімдері төмендегі
2-кестеде тағамдық құндылығы жағынан нақты салыстырылып көрсетілген.

Органолептикалық тұрғыдан да 40:30:30 үлгісіндегі вафли өнімі
сарапшылардан жоғары бағаға ие болды. Өнімнің хош иісі мен дәмі
үйлесімді, құрылымы біркелкі әрі қытырлақ болды. Бұл көрсеткіштер МЕМСТ
талаптарына сәйкес келетіні анықталды.

Осылайша, қарақұмық және жүгері ұны қосылған композитті қоспа негізінде
дайындалған вафли өнімінің тағамдық құндылығы мен тұтынушылық қасиеттері
жақсаратыны дәлелденді. Мұндай өнімді функционалдық және диеталық
бағытта қолдануға толық негіз бар.

{\bfseries Қорытынды.} Жүргізілген ғылыми-зерттеу жұмысы композитті ұнды
пайдалану арқылы жұмсақ вафлидің тағамдық құндылығын арттыруға
болатындығын көрсетті. Алдымен әртүрлі мөлшерде жүгері ұны (10--50\%)
қосылып жасалған үлгілердің ішінде 70:30 (бидай:жүгері) арақатынасындағы
нұсқа органолептикалық және физико-химиялық көрсеткіштер бойынша оң
нәтижелер көрсетсе, кейін құрамына қарақұмық ұнын қосу арқылы 40:30:30
(бидай:жүгері:қарақұмық) үлгісі дайындалып, бұл нұсқаның барлық негізгі
көрсеткіштер бойынша тиімдірек екені дәлелденді. Атап айтқанда, бұл үлгі
ақуыз, тағамдық талшықтар, темір, мырыш, кальций секілді маңызды
қоректік заттар бойынша жоғары нәтижелер көрсетті. Сонымен қатар,
өнімнің органолептикалық сипаттамалары тұтынушылық талаптарға толық
сәйкес келді. Осыған орай, бидай ұнын жүгері және қарақұмық ұнымен
ішінара алмастыру жұмсақ вафлидің сапалық қасиеттерін жақсартумен қатар,
оның биологиялық құндылығын арттыруға мүмкіндік беретіні анықталды. Бұл
тәсіл қазіргі заманғы функционалдық тамақ өнімдерін жасау бағытындағы
перспективалы шешім болып табылады және өндірістік деңгейде қолдануға
ұсынылады.

\emph{{\bfseries Қаржыландыру.} Зерттеу Қазақстан Республикасы Ауыл
шаруашылығы министрлігінің № BR22886613 «Ауыл шаруашылығы дақылдарының
өнімдері мен шикізатын қайта өңдеу мен сақтаудың инновациялық
технологияларын әзірлеу» ғылыми-техникалық бағдарламасының шеңберінде
жүргізілді (№ 9-2024/2026 «Астық өңдеу өнеркәсібі үшін тритикаленің
әртүрлі сорттарын жоғары тиімді өнімге дейін сақтау және өңдеудің
инновациялық технологиясын әзірлеу» жобасы бойынша).}
\end{multicols}

\begin{center}
{\bfseries Әдебиеттер}
\end{center}

\begin{references}
1. Niazi, S., Pasha, I., Shoaib, M., Raza, H., Korma, S. A., Abed, S. M.
Nutritional, physiochemical, pasting and therapeutic attributes of
composite flour//Journal of Global Innovations in Agricultural and
Social Sciences.- 2016.-Vol.4(2).- P.95-105.
\href{https://doi.org/10.22194/jgiass/4.2.747}{DOI
10.22194/jgiass/4.2.747}

2. Nawaz, H., Aslam, M., Rehman, T., \& Mehmood, R. Modification of
Emulsifying Properties of Cereal Flours by Blending with Legume
Flours//Asian Journal of Dairy and Food Research.- 2021, Vol.40(3).-
P.315-320. \href{https://doi.org/10.18805/ajdfr.dr-223}{DOI
10.18805/ajdfr.dr-223}.

3. Arendt E. K., Zannini E. Cereal Grains for the Food and Beverage
Industries. -- Cambridge: Woodhead Publishing, 2013. - 485 p.
\href{https://doi.org/10.1533/9780857098924}{DOI 10.1533/9780857098924}

4. Eliseeva, L. G., Kokorina, D. S., Zhirkova, E. V., Nevskaya, E. V.,
Goncharenko, O. A., Othman, A. J. Using functional quinoa ingredients
for enhancing the nutritional value of bakery products. IOP Conference
Series: Earth and Environmental Science, 2021, 640(2), 022072.
\href{https://doi.org/10.1088/1755-1315/640/2/022072}{DOI\\
10.1088/1755-1315/640/2/022072}.

5. Adeleke, R. O., Odedeji, J. O. Functional Properties of Wheat and
Sweet Potato Flour Blends //Pakistan Journal of Nutrition.-
2010. -Vol9(6).- P.535--538.
\href{https://doi.org/10.3923/pjn.2010.535.538}{DOI
10.3923/pjn.2010.535.538}.

6. Ерғалиева Г. Ж., Байдилдина С. Ж. Композиттік ұн қолдану арқылы
вафлидің тағамдық қасиетін арттыру // Тағам қауіпсіздігі журналы. --
2022. -- №3. -- Б.45--49.

7. Жанқұлиева С. Т. Жүгері ұнының тағамдық құндылығы // Технология тағам
өнімдері. -- 2020. -- №2. -- Б.36--40.

8. Кенжебаева А. М., Мұратова Ж. Ж. Қарақұмық ұнының технологиялық
ерекшеліктері // Агроөнеркәсіп кешені. -- 2021. -- №4. -- Б.55--58.

9. Шаяхметова А. А. Тағам өнімдеріндегі функционалдық ингредиенттер //
Азық-түлік және денсаулық. -- 2019. -- №1. -- Б.18--22.

10. Оспанова А. М., Исмаилова Г. К. Ұнды кондитерлік өнімдердің тағамдық
құндылығын арттыру жолдары // Тамақ және өңдеу өнеркәсібі. -- 2023. --
№1. -- Б.30--34.

11. Arendt E. K., Zannini E. Cereal Grains for the Food and Beverage
Industries. -- Cambridge: Woodhead Publishing, 2013. -- 485 p.
\href{https://doi.org/10.1533/9780857098924}{DOI 10.1533/9780857098924}
\end{references}

\begin{center}
{\bfseries References}
\end{center}

\begin{references}
1. Niazi, S., Pasha, I., Shoaib, M., Raza, H., Korma, S. A., Abed, S. M.
Nutritional, physiochemical, pasting and therapeutic attributes of
composite flour//Journal of Global Innovations in Agricultural and
Social Sciences.- 2016.-Vol.4(2).- P.95-105.
\href{https://doi.org/10.22194/jgiass/4.2.747}{DOI
10.22194/jgiass/4.2.747}

2. Nawaz, H., Aslam, M., Rehman, T., \& Mehmood, R. Modification of
Emulsifying Properties of Cereal Flours by Blending with Legume
Flours//Asian Journal of Dairy and Food Research.- 2021, Vol.40(3).-
P.315-320. \href{https://doi.org/10.18805/ajdfr.dr-223}{DOI
10.18805/ajdfr.dr-223}.

3. Arendt E. K., Zannini E. Cereal Grains for the Food and Beverage
Industries. -- Cambridge: Woodhead Publishing, 2013. - 485 p.
\href{https://doi.org/10.1533/9780857098924}{DOI 10.1533/9780857098924}

4. Eliseeva, L. G., Kokorina, D. S., Zhirkova, E. V., Nevskaya, E. V.,
Goncharenko, O. A., Othman, A. J. Using functional quinoa ingredients
for enhancing the nutritional value of bakery products. IOP Conference
Series: Earth and Environmental Science, 2021, 640(2), 022072.
\href{https://doi.org/10.1088/1755-1315/640/2/022072}{DOI\\
10.1088/1755-1315/640/2/022072}.

5. Adeleke, R. O., Odedeji, J. O. Functional Properties of Wheat and
Sweet Potato Flour Blends //Pakistan Journal of Nutrition.-
2010. -Vol9(6).- P.535--538.
\href{https://doi.org/10.3923/pjn.2010.535.538}{DOI
10.3923/pjn.2010.535.538}.

6. Ерғалиева Г. Ж., Байдилдина С. Ж. Композиттік ұн қолдану арқылы
вафлидің тағамдық қасиетін арттыру // Тағам қауіпсіздігі журналы. --
2022. -- №3. -- Б.45--49.{[}in Kazakh{]}

7. Жанқұлиева С. Т. Жүгері ұнының тағамдық құндылығы // Технология тағам
өнімдері. -- 2020. -- №2. -- Б.36--40.{[}in Kazakh{]}

8. Кенжебаева А. М., Мұратова Ж. Ж. Қарақұмық ұнының технологиялық
ерекшеліктері // Агроөнеркәсіп кешені. -- 2021. -- №4. -- Б.55--58.
{[}in Kazakh{]}

9. Шаяхметова А. А. Тағам өнімдеріндегі функционалдық ингредиенттер //
Азық-түлік және денсаулық. -- 2019. -- №1. -- Б.18--22. {[}in Kazakh{]}

10. Оспанова А. М., Исмаилова Г. К. Ұнды кондитерлік өнімдердің тағамдық
құндылығын арттыру жолдары // Тамақ және өңдеу өнеркәсібі. -- 2023. --
№1. -- Б.30--34. {[}in Kazakh{]}

11. Arendt E. K., Zannini E. Cereal Grains for the Food and Beverage
Industries. -- Cambridge: Woodhead Publishing, 2013. -- 485 p.
\href{https://doi.org/10.1533/9780857098924}{DOI 10.1533/9780857098924}
\end{references}

\begin{authorinfo}
\emph{{\bfseries Авторлар туралы мәліметтер}}

Мулдабекова Б.Ж.- техника ғылымдарының докторы, профессор, «Алматы
технологиялық университеті» АҚ, e-mail:
\href{mailto:bayan\_1004@mail.ru}{\nolinkurl{bayan\_1004@mail.ru}};

Жумабекова А.- 2 курс магистранты, «Алматы технологиялық университеті»
АҚ, e-mail:
\href{mailto:zhumabekova.aikhanym@mail.ru}{\nolinkurl{zhumabekova.aikhanym@mail.ru}};

Якияева М.А.- философия ғылымдарының докторы (Ph.D), қауымдастырылған
профессор, «Алматы технологиялық университеті» АҚ, e-mail:
\href{mailto:yamadina88@mail.ru}{\nolinkurl{yamadina88@mail.ru}}

\emph{{\bfseries Information about authors}}

Muldabekova B.Zh. - Doctor of Technical Sciences, Professor, JSC "Almaty
Technological University", e-mail:\\
\href{mailto:bayan\_1004@mail.ru}{\nolinkurl{bayan\_1004@mail.ru}};

Zhumabekova А. - 2nd year master' s student, JSC "Almaty
Technological University", e-mail:
\href{mailto:zhumabekova.aikhanym@mail.ru}{\nolinkurl{zhumabekova.aikhanym@mail.ru}};

Yakiyayeva M.A. - Doctor of Philosophy (Ph.D), Associate Professor, JSC
"Almaty Technological University", e-mail:\\
\href{mailto:yamadina88@mail.ru}{\nolinkurl{yamadina88@mail.ru}}
\end{authorinfo}
